% Chapter 7 of Concrete Mathematics
% (c) Addison-Wesley, all rights reserved.
\input gkpmac
\refin bib
\pageno=320
\refin chap2
\refin chap4
\refin chap5
\refin chap6

\beginchapter 7 Generating Functions

THE MOST POWERFUL WAY to deal with sequences of numbers, as far as anybody
knows, is to manipulate infinite series that ``generate'' those sequences.
We've learned a lot of sequences and we've seen a few generating functions;
now we're ready to explore generating functions in depth, and to see how
remarkably useful they are.

\beginsection 7.1 Domino Theory and Change

"!Dimers and Dimes, \string\see Dominoes and Change"
"!Dominoes"
Generating functions are important enough, and for many of us new enough,
to justify a relaxed approach as we begin to look at them more closely.
So let's start this chapter with some fun and games as we try to develop our
intuitions about generating functions. We will study two applications
of the ideas, one involving dominoes and the other involving coins.

\unitlength=3pt
\def\domi{\beginpicture(0,2)(0,0) % identity
	\put(0,0){\line(0,1){2}}
	\endpicture}
\def\domI{\beginpicture(0,2)(0,0) % identity when isolated
	\put(0,-.1){\line(0,1){2.2}}
	\endpicture}
\def\domv{\beginpicture(1,2)(0,0) % vertical without left edge
	\put(1,0){\line(0,1){2}}
	\put(0,0){\line(1,0){1}}
	\put(0,2){\line(1,0){1}}
	\endpicture}
\def\domhh{\beginpicture(2,2)(0,0) % two horizontals without left edge
	\put(2,0){\line(0,1){2}}
	\put(0,0){\line(1,0){2}}
	\put(0,1){\line(1,0){2}}
	\put(0,2){\line(1,0){2}}
	\endpicture}
\def\Domh{\beginpicture(3,1)(-.5,0) % horizontal, stand-alone
	\put(2,0){\line(0,1){1}}
	\put(0,0){\line(0,1){1}}
	\put(0,0){\line(1,0){2}}
	\put(0,1){\line(1,0){2}}
	\endpicture}
\def\Domv{\beginpicture(2,2)(-.5,0) % vertical, stand-alone
	\put(0,0){\line(0,1){2}}
	\put(1,0){\line(0,1){2}}
	\put(0,0){\line(1,0){1}}
	\put(0,2){\line(1,0){1}}
	\endpicture}

How many ways $T_n$ are there to completely cover a $2\times n$
rectangle with $2\times1$ dominoes? We assume that the dominoes are
identical (either because they're face down, or
because someone has rendered them indistinguishable,
say by painting them all red);
thus only their orientations\dash---vertical or horizontal\dash---matter,
and we can imagine that we're working with domino-shaped tiles.
For example, there are three tilings of a $2 \times 3$ rectangle, namely
\domi\domv\domv\domv\kern1pt, \domi\domv\domhh\kern1pt, and~\domi\domhh\domv
\kern1pt; so $T_3 = 3$.

To find a closed form for general~$T_n$
\g\noindent\llap{``}Let me count the ways.''\par\hfill\dash---E.\thinspace B. "Browning"\g
we do our usual first thing,
look at small cases.
When $n=1$ there's obviously just one tiling,~\domi\domv\kern1pt;
and when $n=2$ there are two,
\domi\domv\domv~and~\domi\domhh\kern1pt.

How about when $n=0$;
how many tilings of a $2 \times 0$ rectangle are there?
It's not immediately clear what this question means, but we've seen
similar situations before: There is one permutation of zero objects
"!empty case" "!null case" "!basis of induction" "!thinking small" "!small cases"
(namely the empty permutation), so $0!=1$. There is one way to choose
zero things from $n$~things (namely to choose nothing), so ${n\choose0}
=1$. There is one way to partition the empty set into zero nonempty subsets,
but there are no such ways to partition a nonempty set; so ${n\brace0}=
\[n=0]$. By such reasoning we can conclude that there's just one way
to tile a $2\times0$ rectangle with dominoes, namely to use no dominoes;
therefore $T_0=1$. (This spoils the simple pattern $T_n=n$ that holds
when $n=1$, $2$, and~$3$; but that pattern was probably doomed anyway,
since $T_0$ wants to be~$1$ according to the logic of the situation.)
A proper understanding of the null case turns out to be useful whenever
we want to solve an enumeration problem.

Let's look at one more small case, $n=4$.
There are two possibilities for tiling the left edge of the rectangle\dash---%
we put either a vertical domino or two horizontal dominoes there.
If we choose a vertical one, the partial solution is
\domi\domv\beginpicture(3,2)(0,0)
 \put(0,0){\line(1,0){3}}
 \put(0,2){\line(1,0){3}}
 \put(3,0){\line(0,1){2}}\endpicture\
and the remaining $2\times3$ rectangle can be covered in $T_3$~ways.
If we choose two horizontals, the partial solution
\domi\domhh\beginpicture(2,2)(0,0)
 \put(0,0){\line(1,0){2}}
 \put(0,2){\line(1,0){2}}
 \put(2,0){\line(0,1){2}}\endpicture\
can be completed in $T_2$~ways.
Thus $T_4=T_3+T_2=5$. (The five tilings are
\domi\domv\domv\domv\domv\kern1pt,
\domi\domv\domv\domhh\kern1pt,
\domi\domv\domhh\domv\kern1pt,
\domi\domhh\domv\domv\kern1pt, and~%
\domi\domhh\domhh\kern1pt.)

We now know the first five values of~$T_n$:
\begindisplay \let\preamble=\tablepreamble
n&&0&1&2&3&4\cr
\noalign{\hrule}
T_n&&1&1&2&3&5\cr
\enddisplay
These look suspiciously like the "Fibonacci numbers",
and it's not hard to see why:
The reasoning we used to establish
$T_4 = T_3 + T_2$ easily generalizes to
$T_n = T_{n-1} + T_{n-2}$, for $n \geq 2$.
Thus we have the same recurrence here
as for the Fibonacci numbers,
except that the initial values $T_0 = 1$ and $T_1 = 1$
are a little different.
But these initial values are the consecutive Fibonacci numbers $F_1$~and~$F_2$,
so the $T$'s are just Fibonacci numbers
shifted up one place:
\begindisplay
 T_n
   = F_{n+1} \,, \qquad\hbox{for $n \geq 0$}.
\enddisplay
(We consider this to be a closed form for $T_n$,
because the Fibonacci numbers are important
enough to be considered ``known.\qback'' Also, $F_n$ itself has a
closed form \equ(6.|fib-sol|) in terms of algebraic operations.)
Notice that this equation confirms the wisdom of setting $T_0=1$.

But what does all this have to do with generating functions? Well, we're about
to get to that\dash---there's another way to figure out what $T_n$ is. This new
way is based on a bold idea.
\g To boldly go\par where no tiling has gone before.\g
 Let's consider the ``sum'' of all possible
$2\times n$ tilings, for all $n\ge0$, and call it~$T$:
\begindisplay
T	= \domI+
 \domi\domv+\domi\domv\domv + \domi\domhh
 + \domi\domv\domv\domv + \domi\domv\domhh + \domi\domhh\domv
 + \cdots\,.
\eqno\eqref|t-sum|
\enddisplay
(The first term `$\,\domI\,$' on the right
stands for the null tiling of a $2\times0$ rectangle.)
This sum~$T$ represents lots of information.
It's useful because it lets us prove things about $T$ as a whole
rather than forcing us to prove them (by induction)
about its individual terms.

The terms of this sum
stand for tilings, which are combinatorial objects. We won't be fussy
about what's considered legal when infinitely many tilings are added
together; everything can be made rigorous, but our goal right now is to
expand our consciousness beyond conventional algebraic formulas.

We've added the patterns together, and we can also multiply them\dash---%
by juxtaposition.
For example, we can multiply the tilings
\kern1pt\domi\domv\kern1pt\ and \kern1pt\domi\domhh\kern1pt\
to get the new tiling~\domi\domv\domhh\kern1pt.
But notice that multiplication is not "commutative";
that is, the order of multiplication counts:
\kern1pt\domi\domv\domhh\kern1pt\ is different from
\kern1pt\domi\domhh\domv\kern1pt.

Using this notion of multiplication
it's not hard to see that
the null tiling plays a special role\dash---%
it is the multiplicative identity.
For instance,
$\domI\times\domi\domhh=\domi\domhh\times\domI=\domi\domhh$\kern1pt.

Now we can use domino arithmetic to manipulate the infinite sum~$T$:
\begindisplay
T&=\domI+\domi\domv+\domi\domv\domv + \domi\domhh
 + \domi\domv\domv\domv + \domi\domv\domhh + \domi\domhh\domv
 + \cdots\cr
&=\domI+
 \domi\domv\,(\,\domI+\domi\domv+\domi\domv\domv + \domi\domhh+\cdots\,)+
 \domi\domhh\,(\,\domI+\domi\domv+\domi\domv\domv + \domi\domhh+\cdots\,)\cr
&=\domI+
 \domi\domv\,T+ \domi\domhh\,T\,.\eqno\cr
\enddisplay
Every valid tiling occurs
exactly once in each right side,
so what we've done is reasonable
even though we're ignoring the cautions in Chapter~2 about ``absolute
convergence.\qback''
\g I have a gut feeling that these sums
must converge, as long as the dominoes are small~enough.\g
The bottom line of this equation tells us that everything in~$T$
is either the null tiling,
or is a vertical tile followed by something else in~$T$,
or is two horizontal tiles followed by something else in~$T$.

So now let's try to solve the equation for $T$. Replacing the $T$ on the left
by $\domI\,T$ and subtracting
the last two terms on the right
from both sides of the equation, we get
\begindisplay
 (\,\domI-\domi\domv-\domi\domhh\,)T=\domI\,\,.
\eqno
\enddisplay
For a consistency check, here's an expanded version:
\begindisplay\def\preamble{&\strut$\hfil\,{}##{}\,\hfil$&$\hfil##\hfil$}
&\domI&+&\domi\domv&+&\domi\domv\domv&+&\domi\domhh&+&
 \domi\domv\domv\domv&+&\domi\domv\domhh&+&\domi\domhh\domv&+&
\cdots\cr
-&\domi\domv&-&\domi\domv\domv&-&\domi\domv\domv\domv&-&\domi\domv\domhh&-&
 \domi\domv\domv\domv\domv&-&\domi\domv\domv\domhh&-&\domi\domv\domhh\domv&-&
 \cdots\cr
-&\domi\domhh&-&\domi\domhh\domv&-&\domi\domhh\domv\domv&-&\domi\domhh\domhh&-&
 \domi\domhh\domv\domv\domv&-&\domi\domhh\domv\domhh&-&\domi\domhh\domhh\domv&-&
 \cdots\cr
\noalign{\kern2pt\hrule\kern2pt}
&\domI\cr
\enddisplay
Every term in the top row, except the first,
is cancelled by a term in either the second or third row,
so our equation is correct.

So far it's been fairly easy to make combinatorial sense
of the equations we've been working with.
Now, however, to get a compact expression for~$T$
we cross a combinatorial divide.
With a leap of algebraic faith
we divide both sides of equation \thiseq\ by
$\,\domI-\domi\domv-\domi\domhh\,$
to get
\begindisplay
 T
	= {\hbox{\domI}\over\domI-\domi\domv-\domi\domhh}\,.
\eqno\eqref|t-fraction|
\enddisplay
(Multiplication isn't commutative, so
we're on the verge of "cheating",
by not distinguishing between left and right division. In
our application it doesn't matter, because $\,\domI\,$ commutes with
everything. But let's not be picky, unless our wild ideas lead to
paradoxes.)

The next step is to expand this fraction as a power series,
using the rule
\begindisplay
 {1\over 1-z}
	= 1 + z + z^2 + z^3 + \cdots \,.
\enddisplay
The null tiling $\,\domI\,$,
which is the multiplicative identity for our combinatorial arithmetic,
plays the part of~$1$, the usual multiplicative identity;
and $\domi\domv+\domi\domhh$ plays~$z$.
So we get the expansion
\begindisplay
{\hbox{\domI}\over\domI-\domi\domv-\domi\domhh}
&=\domI+(\,\domi\domv+\domi\domhh\,)+
 (\,\domi\domv+\domi\domhh\,)^2+
 (\,\domi\domv+\domi\domhh\,)^3+\cdots\cr
&=\domI+(\,\domi\domv+\domi\domhh\,)+
 (\,\domi\domv\domv+\domi\domv\domhh+\domi\domhh\domv+\domi\domhh\domhh\,)\cr
&\qquad+(\,\domi\domv\domv\domv+\domi\domv\domv\domhh
 +\domi\domv\domhh\domv+\domi\domv\domhh\domhh
 +\domi\domhh\domv\domv+\domi\domhh\domv\domhh
 +\domi\domhh\domhh\domv+\domi\domhh\domhh\domhh\,)+\cdots\,.\cr
\enddisplay
This is $T$, but the tilings are arranged in a different order than
we had before. Every tiling appears exactly once in this sum;
for example, $\domi\domv\domhh\domhh\domv\domv\domhh\domv$
appears in the expansion of $(\,\domi\domv+\domi\domhh\,)^7$.

We can get useful information from this infinite sum by compressing it
down, ignoring details that are not of interest. For example, we
can imagine that the patterns become unglued and that the individual dominoes
commute with each other; then a term like
$\domi\domv\domhh\domhh\domv\domv\domhh\domv$ becomes $\Domv^4\Domh^6$,
because it contains four verticals and six horizontals. Collecting
like terms gives us the series
\begindisplay
 T = \domI+\Domv+\Domv^2+\Domh^2+\Domv^3+2\Domv\Domh^2+\Domv^4+3\Domv^2\Domh^2
 +\Domh^4+\cdots\,.
\enddisplay
The $2\Domv\Domh^2$ here represents
the two terms of the old expansion,
\domi\domv\domhh\ and~\domi\domhh\domv\kern1pt, that
have one vertical and two horizontal dominoes;
similarly $3\Domv^2\Domh^2$
represents the three terms
\domi\domv\domv\domhh\kern1pt, \domi\domv\domhh\domv\kern1pt, and 
\domi\domhh\domv\domv\kern1pt.
We're essentially treating \Domv\ and~\Domh\ as ordinary (commutative)
variables.

We can find a closed form for the coefficients in the commutative version
of~$T$ by using the binomial theorem:
\begindisplay
{\hbox{\domI}\over\domI-(\Domv+\Domh^2)}
&=\domI+(\Domv+\Domh^2) +(\Domv+\Domh^2)^2 +(\Domv+\Domh^2)^3+\cdots\cr
&=\sum_{k\ge0}(\Domv+\Domh^2)^k\cr
&=\sum_{j,k\ge0}{k\choose j}\Domv^j\Domh^{2k-2j}\cr
&=\sum_{j,m\ge0}{j+m\choose j}\Domv^j\Domh^{2m}\,.\eqno\cr
\enddisplay
(The last step replaces $k-j$ by $m$; this is legal because we have
\smash{\hbox{${k\choose j}=0$}}
when $0\le k<j$.) We conclude that $j+m\choose j$ is the number of ways
to tile a $2\times(j+2m)$ rectangle with $j$ vertical dominoes and $2m$
horizontal dominoes. For example, we recently looked at the $2\times10$ tiling
$\domi\domv\domhh\domhh\domv\domv\domhh\domv$\kern1pt,
which involves
four verticals and six horizontals; there are ${4+3\choose4}=35$ such
tilings in all, so one of the terms in the commutative version of~$T$
is $35@\Domv^4\Domh^6$.

We can suppress even more detail by ignoring the orientation of the dominoes.
Suppose we don't care about the horizontal/vertical breakdown; we only want
to know about the total number of $2\times n$ tilings. (This, in fact,
is the number~$T_n$ we started out trying to discover.)
We can collect the necessary information by simply substituting a
single quantity, $z$, for \Domv\ and~\Domh.
\g Now I'm dis\-oriented.\g
And we might as well also replace \kern1pt\domI\kern1pt\ by~$1$, getting
\begindisplay
T={1\over 1-z-z^2}\,.
\eqno
\enddisplay
This is the generating function \equ(6.|fib-gf|) for Fibonacci numbers,
except for a missing factor of~$z$ in the numerator; so we conclude
that the coefficient of $z^n$ in~$T$ is $F_{n+1}$.

The compact representations $\,\domI\,/(\,\domI-\domi\domv-\domi\domhh\,)$,
$\,\domI\,/(\,\domI-\Domv-\Domh^2)$, and $1/(1-z-z^2)$ that we have deduced
for~$T$ are called {\it"generating functions"}, because they generate
the coefficients of interest.

Incidentally, our derivation implies that the number of $2\times n$
domino tilings with exactly $m$ pairs of horizontal dominoes is
${n-m\choose m}$. (This follows because there are $j=n-2m$ vertical
dominoes, hence there are
\begindisplay 
{j+m\choose j}={j+m\choose m}={n-m\choose m}
\enddisplay % done for page break later
ways to do the tiling according to our formula.) We observed in Chapter~6 that
$n-m\choose m$ is the
number of "Morse code" sequences of length~$n$ that contain $m$~dashes;
in fact, it's easy to see that $2\times n$ domino tilings correspond
directly to Morse code sequences. (The tiling
$\,\domi\domv\domhh\domhh\domv\domv\domhh\domv\,$ corresponds to
\def\mdot{\beginpicture(1.5,2)(-.75,-1)\put(0,0){\disk{.7}}\endpicture}%
\def\mdash{\beginpicture(2.5,2)(-.5,-1)
 \put(0,0){\thicklines\line(1,0){1.5}}\endpicture}%
`$\mdot\mdash\mdash\mdot\mdot\mdash\mdot$'.)
Thus domino tilings are closely related to the continuant polynomials
we studied in Chapter~6. It's a small world. "!cliches"

We have solved the $T_n$ problem in two ways. The first way, guessing the
answer and proving it by induction, was easier; the second way, using
infinite sums of domino patterns and distilling out the coefficients
of interest, was fancier. But did we use the second method only because
it was amusing to play with dominoes as if they were algebraic variables?
No; the real reason for introducing the second way
was that the infinite-sum approach is a lot more powerful.
The second method applies to many more problems, because it doesn't require us
to make magic guesses.

\def\tomv#1#2#3{\beginpicture(0,2.7)(0,.5) % vertical segments (top-down)
 \if1#1\put(0,2){\line(0,1){1}}\fi
 \if0#2\else\put(0,1){\line(0,1){#2}}\fi
 \if0#3\else\put(0,0){\line(0,1){#3}}\fi \endpicture}
\def\tomh#1#2#3#4{\beginpicture(1,2)(0,.5) % horizontal segments (top-down)
 \if0#1\else\put(0,3){\line(1,0){#1}}\fi
 \if0#2\else\put(0,2){\line(1,0){#2}}\fi
 \if0#3\else\put(0,1){\line(1,0){#3}}\fi
 \if0#4\else\put(0,0){\line(1,0){#4}}\fi \endpicture}

Let's generalize up a notch, to a problem where guesswork
will be beyond us. How many ways $U_n$ are there to tile a
$3\times n$ rectangle with dominoes?

 The first few cases of this problem
tell us a little: The null tiling gives $U_0=1$. There is no valid
tiling when $n=1$, since a $2\times1$ domino doesn't fill a $3\times1$
rectangle, and since there isn't room for two. The next case, $n=2$, can
easily be done by hand; there are three tilings,
\tomv003\tomh2022\tomv020\tomh0000\tomv003\kern1pt,
\tomv003\tomh2202\tomv002\tomh0000\tomv003\kern1pt,
and
\tomv003\tomh2222\tomh0000\tomv003\kern1pt,
so $U_2=3$.
(Come to think of it we already knew this,
because the previous problem told us that $T_3 = 3$;
the number of ways to tile a $3 \times 2$ rectangle
is the same as the number to tile a~$2 \times 3$.)
When~$n=3$, as when~$n=1$, there are no tilings.
We can convince ourselves of this either
by making a quick exhaustive search or
by looking at the problem from a higher level:
The area of a $3 \times 3$ rectangle is odd,
so we can't possibly tile it with dominoes whose area is even.
(The same argument obviously applies to any odd~$n$.)
Finally, when $n=4$ there seem to be about a dozen tilings;
it's difficult to be sure about the exact number without spending a lot of time
to guarantee that the list is complete.

\begingroup \advance\medmuskip 3mu
So let's try the infinite-sum approach that worked last time:
\begindisplay
U=\tomv003
+\tomv003\tomh2022\tomv020\tomh0000\tomv003
+\tomv003\tomh2202\tomv002\tomh0000\tomv003
+\tomv003\tomh2222\tomh0000\tomv003
+\tomv003\tomh2022\tomv020\tomh0000\tomv003\tomh2022\tomv020\tomh0000\tomv003
+\tomv003\tomh2022\tomv020\tomh0000\tomv003\tomh2202\tomv002\tomh0000\tomv003
+\tomv003\tomh2022\tomv020\tomh0000\tomv003\tomh2222\tomh0000\tomv003
+\tomv003\tomh4044\tomv020\tomh0200\tomv001\tomh0000\tomv020\tomh0000\tomv003
+\tomv003\tomh2202\tomv002\tomh0000\tomv003\tomh2022\tomv020\tomh0000\tomv003
+\cdots\,.
\eqno
\enddisplay
Every non-null tiling begins with either
\tomv003\tomh1022\tomv020\tomh0000\tomv001\
or \tomv003\tomh2201\tomv002\tomh0000\tomv100
or \tomv003\tomh2222\tomh0000\tomv003\kern1pt;
but unfortunately the first two of these three possibilities don't
simply factor out and leave us with $U$ again. The sum of all terms in~$U$ that
begin with \tomv003\tomh1022\tomv020\tomh0000\tomv001\
can, however, be written as $\tomv003\tomh1022\tomv020\tomh0000\tomv001\,V$, where
\begindisplay
V=\tomv020\tomh1010\tomv020
+\tomv020\tomh3030\tomv003\tomh0002\tomv020\tomh0000\tomv003
+\tomv020\tomh3010\tomv003\tomh0202\tomv002\tomh0000\tomv003
+\tomv020\tomh3030\tomv003\tomh0202\tomh0000\tomv003
+\tomv020\tomh3230\tomv001\tomh0002\tomv020\tomh0000\tomv003 +\cdots
\enddisplay
is the sum of all domino tilings of a mutilated $3\times n$ rectangle that
has its lower left corner missing. Similarly, the terms of~$U$ that
begin with \tomv003\tomh2201\tomv002\tomh0000\tomv100\ can be written
$\tomv003\tomh2201\tomv002\tomh0000\tomv100\,\Lambda$, where
\begindisplay
\Lambda=\tomv002\tomh0101\tomv002
+\tomv002\tomh0303\tomv003\tomh2000\tomv002\tomh0000\tomv003
+\tomv002\tomh0103\tomv003\tomh2020\tomv020\tomh0000\tomv003
+\tomv002\tomh0303\tomv003\tomh2020\tomh0000\tomv003
+\tomv002\tomh0323\tomv100\tomh2000\tomv002\tomh0000\tomv003 +\cdots
\enddisplay
consists of all rectangular tilings lacking their upper left corner. The
series $\Lambda$ is a mirror image of $V$. These factorizations allow us to write
\begindisplay
U=\tomv003\,
+\tomv003\tomh1022\tomv020\tomh0000\tomv001\,V
+\tomv003\tomh2201\tomv002\tomh0000\tomv100\,\Lambda
+\tomv003\tomh2222\tomh0000\tomv003\,U\,.
\enddisplay
And we can factor $V$ and $\Lambda$ as well, because such tilings
can begin in only two ways:
\begindisplay
V&=\tomv020\tomh1010\tomv020\,U+
 \tomv020\tomh2230\tomv001\tomh0002\tomv020\tomh0000\tomv001\,V\,,\cr
\Lambda&=\tomv002\tomh0101\tomv002\,U+
 \tomv002\tomh0322\tomv100\tomh2000\tomv002\tomh0000\tomv100\,\Lambda\,.\cr
\enddisplay
Now we have three equations in three unknowns ($U$, $V$, and $\Lambda$). We can
solve them by first solving for $V$ and~$\Lambda$ in terms of~$U$, then
plugging the results into the equation for $U$:
\begindisplay \openup4pt
V&=(\,\tomv003-
 \tomv020\tomh2230\tomv001\tomh0002\tomv020\tomh0000\tomv001\,)^{-1}
  \,\tomv020\tomh1010\tomv020\,U\,,\qquad
\Lambda=(\,\tomv003-
 \tomv002\tomh0322\tomv100\tomh2000\tomv002\tomh0000\tomv100\,)^{-1}
  \,\tomv002\tomh0101\tomv002\,U\,;\cr
U&=\tomv003\,+\,
 \tomv003\tomh1022\tomv020\tomh0000\tomv001\,
(\,\tomv003-
 \tomv020\tomh2230\tomv001\tomh0002\tomv020\tomh0000\tomv001\,)^{-1}
  \,\tomv020\tomh1010\tomv020\,U\,+\,
 \tomv003\tomh2201\tomv002\tomh0000\tomv100\,
(\,\tomv003-
 \tomv002\tomh0322\tomv100\tomh2000\tomv002\tomh0000\tomv100\,)^{-1}
  \,\tomv002\tomh0101\tomv002\,U\,+\,
\tomv003\tomh2222\tomh0000\tomv003\,U\,.
\enddisplay
And the final equation can be solved for $U$, giving the compact formula
\begindisplay
U={\hbox{\tomv003}\over\, \tomv003
\,-\,\tomv003\tomh1022\tomv020\tomh0000\tomv001\,
(\,\tomv003-
 \tomv020\tomh2230\tomv001\tomh0002\tomv020\tomh0000\tomv001\,)^{-1}
  \,\tomv020\tomh1010\tomv020\,
\,-\,\tomv003\tomh2201\tomv002\tomh0000\tomv100\,
(\,\tomv003-
 \tomv002\tomh0322\tomv100\tomh2000\tomv002\tomh0000\tomv100\,)^{-1}
  \,\tomv002\tomh0101\tomv002
\,-\,\tomv003\tomh2222\tomh0000\tomv003}\,.
\eqno
\enddisplay\endgroup % end of the special \medmuskip
This expression defines the infinite sum $U$,
\g I learned in another class about ``regular expressions.\qback'' If
I'm not mistaken, we can write\smallskip
\unitlength=2pt
$U=(
\tomv003\tomh1022\tomv020\tomh0000\tomv001\,\,
 \tomv020\tomh2230\tomv001\tomh0002\tomv020\tomh0000\tomv001^{\ast}
  \,\tomv020\tomh1010\tomv020$\par
\hfill${}+\tomv003\tomh2201\tomv002\tomh0000\tomv100\,\,
 \tomv002\tomh0322\tomv100\tomh2000\tomv002\tomh0000\tomv100\,^{\ast}
  \,\tomv002\tomh0101\tomv002
+\tomv003\tomh2222\tomh0000\tomv003\,)^{\ast}$\par\smallskip
in the language of regular expressions; so there must be some
connection between regular expressions and generating functions.\g
just as \eq(|t-fraction|) defines $T$.

The next step is to go commutative. Everything simplifies beautifully
when we detach all the dominoes and use only powers of $\Domv$ and~$\Domh$:
\begindisplay \openup5pt
U&={1\over
1-\Domv^2\Domh(1-\Domh^3)^{-1}
     -\Domv^2\Domh(1-\Domh^3)^{-1}-\Domh^3}\cr
&={1-\Domh^3\over
 (\,1-\Domh^3)^2-2\Domv^2\Domh}\cr
&={(1-\Domh^3)^{-1}\over
 1-2\Domv^2\Domh(1-\Domh^3)^{-2}}\cr
&={1\over1-\Domh^3}
 +{2\Domv^2\Domh\over(1-\Domh^3)^3}
 +{4\Domv^4\Domh^2\over(1-\Domh^3)^5}
 +{8\Domv^6\Domh^3\over(1-\Domh^3)^7}+\cdots\cr
&=\sum_{k\ge0}{2^k\Domv^{2k}\Domh^k\over(1-\Domh^3)^{2k+1}}\cr
&=\sum_{k,m\ge0}{m+2k\choose m}2^k\Domv^{2k}\Domh^{k+3m}\,.\cr
\enddisplay
(This derivation deserves careful scrutiny.
The last step uses the formula $(1-w)^{-2k-1}=\sum_m{m+2k\choose m}w^m$,
identity \equ(5.|neg-binomial1|).)
Let's take a good look at the bottom line to see what it tells us. First, it says
that every $3\times n$ tiling uses an even number of vertical dominoes. Moreover,
if there are $2k$ verticals, there must be at least $k$ horizontals, and the
total number of horizontals must be $k+3m$ for some $m\ge0$. Finally, the
number of possible tilings with $2k$ verticals and $k+3m$ horizontals is
exactly ${m+2k\choose m}2^k$.

We now are able to analyze the $3\times4$ tilings that left us doubtful
when we began looking at the $3\times n$ problem. When $n=4$ the total
area is~$12$, so we need six dominoes altogether. There are $2k$ verticals
and $k+3m$ horizontals,
 for some $k$ and~$m$; hence $2k+k+3m=6$. In other
words, $k+m=2$. If we use no verticals, then $k=0$ and $m=2$;
the number of possibilities is
${2+0\choose2}2^0=1$. (This accounts for the tiling
$\tomv003\tomh4444\tomh0000\tomv003\tomh0000\tomh0000\tomv003\,$.)
If we use two verticals, then $k=1$ and $m=1$; there are
${1+2@\choose1}2^1=6$ such tilings. And if we use four verticals, then
$k=2$ and $m=0$; there are ${0+4\choose0}2^2=4$ such tilings, making a
total of\/ $U_4=11$. In general if $n$ is even, this reasoning shows that
$k+m=\half n$, hence
${m+2k\choose m}={n/2+k\choose n/2-k}$ and the total
 number of $3\times n$ tilings is
\begindisplay
U_n=\sum_k{n/2+k\choose n/2-k}2^k=\sum_m{n-m\choose m}2^{n/2-m}\,.
\eqno\eqref|u-sum|
\enddisplay

As before, we can also substitute $z$ for both \Domv\ and~\Domh, getting
a generating function that doesn't discriminate between dominoes of
particular persuasions. The result is
\begindisplay
U={1\over 1-z^3(1-z^3)^{-1}-z^3(1-z^3)^{-1}-z^3}={1-z^3\over 1-4z^3+z^6}\,.
\eqno\eqref|u-with-cubes|
\enddisplay
If we expand this quotient into a power series, we get
\begindisplay
U=1+U_2\,z^3+U_4\,z^6+U_6\,z^9+U_8\,z^{12}+\cdots\,,
\enddisplay
a generating function for the numbers $U_n$. (There's a curious mismatch
between subscripts and exponents in this formula, but it is easily explained.
The coefficient of~$z^9$, for example, is $U_6$, which counts the tilings
of a $3\times6$ rectangle. This is what we want, because every
such tiling contains nine~dominoes.)

We could proceed to analyze \thiseq\ and get a closed form for the coefficients,
but it's better to save that for later in the chapter after we've gotten
more experience. So let's divest ourselves of dominoes for the moment
%\g Nuns do it out of habit.\g % too risqu\'e
and proceed to the next advertised problem, ``"change".\qback''

\newdimen\coinmm \coinmm=.33333pt
\unitlength=\coinmm
\let\tiny=\mathsubsubtext % 5pt
\newbox\bnullp \newbox\bnulln \newbox\bnulld \newbox\bnullq \newbox\bnullh
\setbox\bnullp=\hbox{\kern1pt\beginpicture(14,18)(-7,0)
	\put(0,9){\makebox(0,0){\lower.2ex\hbox{$\not\mskip4mu$}%
		\tiny 1}}\endpicture\kern1pt}
%\setbox\bnullt=\hbox{\kern1pt\beginpicture(14,18)(-7,0)
%	\put(0,9){\makebox(0,0){\lower.2ex\hbox{$\not\mskip4mu$}%
%		\tiny 2}}\endpicture\kern1pt}
\setbox\bnulln=\hbox{\kern1pt\beginpicture(14,18)(-7,0)
	\put(0,9){\makebox(0,0){\lower.2ex\hbox{$\not\mskip4mu$}%
		\tiny 5}}\endpicture\kern1pt}
\setbox\bnulld=\hbox{\kern1pt\beginpicture(16,18)(-8,0)
	\put(0,9){\makebox(0,0){\lower.2ex\hbox{$\not\mskip1.5mu$}%
		\tiny 1$\mskip-1mu$0}}\endpicture\kern1pt}
\setbox\bnullq=\hbox{\kern1pt\beginpicture(16,18)(-8,0)
	\put(0,9){\makebox(0,0){\lower.2ex\hbox{$\not\mskip1.5mu$}%
		\tiny 2$\mskip-1mu$5}}\endpicture\kern1pt}
\setbox\bnullh=\hbox{\kern1pt\beginpicture(16,18)(-8,0)
	\put(0,9){\makebox(0,0){\lower.2ex\hbox{$\not\mskip1.5mu$}%
		\tiny 5$\mskip-1mu$0}}\endpicture\kern1pt}
% Note, I've added 3mm to all Oren's measurements, for more clearance
\newbox\bpenny \newbox\bnickel \newbox\bdime \newbox\bquarter \newbox\bhalfdollar
\setbox\bpenny=\hbox{\kern1pt\beginpicture(22,18)(-11,0)
	\put(0,9){\circle{22}}
	\put(0,9){\makebox(0,0){\kern.5pt\tiny 1}}
	\endpicture\kern1pt}
\setbox\bnickel=\hbox{\kern1pt\beginpicture(24.2,18)(-12.1,0)
	\put(0,9){\circle{24.2}}
	\put(0,9){\makebox(0,0){\tiny 5}}
	\endpicture\kern1pt}
\setbox\bdime=\hbox{\kern1pt\beginpicture(20.9,18)(-10.45,0)
	\put(0,9){\circle{20.9}}
	\put(0,9){\makebox(0,0){\tiny 1$\mskip-1.5mu$0}}
	\endpicture\kern1pt}
\setbox\bquarter=\hbox{\kern1pt\beginpicture(27.3,18)(-13.65,0)
	\put(0,9){\circle{27.3}}
	\put(0,9){\makebox(0,0){\tiny 2$\mskip-1mu$5}}
	\endpicture\kern1pt}
\setbox\bhalfdollar=\hbox{\kern1pt\beginpicture(33.6,18)(-16.8,0)
	\put(0,9){\circle{33.6}}
	\put(0,9){\makebox(0,0){\tiny 50}}
	\endpicture\kern1pt}
\def\nullp{\copy\bnullp}
\def\nulln{\copy\bnulln}
\def\nulld{\copy\bnulld}
\def\nullq{\copy\bnullq}
\def\nullh{\copy\bnullh}
\def\penny{\copy\bpenny}
\def\nickel{\copy\bnickel}
\def\dime{\copy\bdime}
\def\quarter{\copy\bquarter}
\def\halfdollar{\copy\bhalfdollar}

\smallskip
How many ways are there to pay 50~cents?
We assume that the payment must be made with
pennies \penny, nickels \nickel, dimes \dime, quarters
\quarter, and half-dollars \halfdollar.
\g Ah yes, I remember when we had half-dollars.\g
George "P\'olya" [|polya-pictures|] popularized this
problem by showing that it can be solved with generating functions in
an instructive way.

Let's set up infinite sums that represent all possible ways to give change,
just as we tackled the domino problems by working with infinite sums
that represent all possible domino patterns. It's simplest to start
by working with fewer varieties of coins, so let's suppose first that
we have nothing but pennies. The sum of all ways to leave some number
of pennies (but just pennies) in change can be written
\begindisplay
P&=\nullp+\penny+\penny\penny+\penny\penny\penny+\penny\penny\penny\penny
   +\cdots\cr
 &=\nullp+\penny+\penny^2+\penny^3+\penny^4+\cdots\,.\cr
\enddisplay
The first term stands for the way to leave no pennies,
the second term stands for one penny,
then two pennies, three pennies, and so on.
Now if we're allowed to use both pennies and nickels, the sum of all possible
ways is
\begindisplay
N&=P+\nickel\,P+\nickel\nickel\,P+\nickel\nickel\nickel\,P
 +\nickel\nickel\nickel\nickel\,P+\cdots\cr
 &=(\nulln+\nickel+\nickel^2+\nickel^3+\nickel^4+\cdots\,)\,P\,,\cr
\enddisplay
since each payment has a certain number of nickels chosen from the first
factor and a certain number of pennies chosen from~$P$. (Notice that
$N$ is {\it not\/} the sum $\nullp+\penny+\nickel+(\penny+\nickel)^2+
(\penny+\nickel)^3+\cdots\,$, because such a sum includes many types of
payment more than once. For example, the term $(\penny+\nickel)^2=
\penny\penny+\penny\nickel+\nickel\penny+\nickel\nickel$ treats \penny\nickel\
and \nickel\penny\ as if they were different, but we want to list each
set of coins only once without respect to order.)

Similarly, if dimes are permitted as well, we get the infinite sum
\begindisplay
D&=(\nulld+\dime+\dime^2+\dime^3+\dime^4+\cdots\,)\,N\,,\cr
\enddisplay
which includes terms like $\dime^3\nickel^3\penny^5=
\dime\dime\dime\nickel\nickel\nickel\penny\penny\penny\penny\penny$
when it is expanded in full. Each of these terms is a different way to make change.
Adding quarters and then half-dollars to the realm of possibilities
\g Coins of the realm.\g
gives
\begindisplay
Q&=(\nullq+\quarter+\quarter^2+\quarter^3+\quarter^4+\cdots\,)\,D\,;\cr
C&=(\nullh+\halfdollar+\halfdollar^2+\halfdollar^3+\halfdollar^4+\cdots\,)\,Q\,.
\enddisplay
Our problem is to find the number of
terms in $C$ worth exactly 50\rlap/c.

A simple trick solves this problem nicely: We can replace \penny\ by~$z$,
\nickel\ by~$z^5$, \dime\ by~$z^{10}$, \quarter\ by~$z^{25}$, and
\halfdollar\ by~$z^{50}$. Then each term is replaced by~$z^n$, where $n$~is
the monetary value of the original term. For example, the term
\halfdollar\dime\nickel\nickel\penny\ becomes $z^{50+10+5+5+1}=z^{71}$.
The four ways of paying 13~cents, namely $\dime\penny^3$, $\nickel\penny^8$,
$\nickel^2\penny^3$, and $\penny^{13}$, each reduce to~$z^{13}$; hence the
coefficient of~$z^{13}$ will be~$4$ after the $z$-substitutions are made.

Let $P_n$, $N_n$, $D_n$, $Q_n$, and $C_n$ be
the numbers of ways to pay $n$ cents when we're allowed to use coins that
are worth at most $1$, $5$, $10$, $25$, and~$50$ cents, respectively.
Our analysis tells us that these
are the coefficients of~$z^n$ in the respective power series
\begindisplay
P&=1+z+z^2+z^3+z^4+\cdots\,,\cr
N&=(1+z^5+z^{10}+z^{15}+z^{20}+\cdots\,)@P\,,\cr
D&=(1+z^{10}+z^{20}+z^{30}+z^{40}+\cdots\,)@N\,,\cr
Q&=(1+z^{25}+z^{50}+z^{75}+z^{100}+\cdots\,)@D\,,\cr
C&=(1+z^{50}+z^{100}+z^{150}+z^{200}+\cdots\,)@Q\,.\cr
\enddisplay
Obviously $P_n=1$ for all $n\ge0$.
\g How many pennies are there, really?\par
If\/ $n$ is greater than, say, $10^{10}$,
I~bet that\/ $P_n=0$ in the ``real world.\qback''\g
And a little thought proves that we have $N_n=\lfloor n/5\rfloor+1$:
To make $n$~cents out of pennies and nickels, we must choose either
$0$~or $1$ or \dots\ or~$\lfloor n/5\rfloor$ nickels, after which
there's only one way to supply the requisite number of pennies.
Thus $P_n$ and $N_n$ are simple; but the values of $D_n$, $Q_n$, and
$C_n$ are increasingly more complicated.

One way to deal with these formulas is to realize that
$1+z^m+z^{2m}+\cdots$ is just $1/(1-z^m)$. Thus we can write
\begindisplay
P&=1/(1-z)\,,\cr
N&=P/(1-z^5)\,,\cr
D&=N/(1-z^{10})\,,\cr
Q&=D/(1-z^{25})\,,\cr
C&=Q/(1-z^{50})\,.\cr
\enddisplay
Multiplying by the denominators, we have
\begindisplay
(1-z)\,P&=1\,,\cr
(1-z^5)\,N&=P\,,\cr
(1-z^{10})\,D&=N\,,\cr
(1-z^{25})\,Q&=D\,,\cr
(1-z^{50})\,C&=Q\,.\cr
\enddisplay
Now we can equate coefficients of $z^n$ in these equations, getting
recurrence relations from which the desired coefficients
can quickly be computed:
\begindisplay
P_n&=P_{n-1}+\[n=0]\,,\cr
N_n&=N_{n-5}+P_n\,,\cr
D_n&=D_{n-10}+N_n\,,\cr
Q_n&=Q_{n-25}+D_n\,,\cr
C_n&=C_{n-50}+Q_n\,.\cr
\enddisplay
For example, the coefficient of $z^n$ in $D=(1-z^{25})Q$ is equal to
$Q_n-Q_{n-25}$;
so we must have $Q_n-Q_{n-25}=D_n$, as claimed.

We could unfold these recurrences and find, for example, that
$Q_n=D_n+D_{n-25}+D_{n-50}+D_{n-75}+\cdots\,$, stopping when the subscripts
get negative. But the non-iterated form is convenient because each
coefficient is computed with just one addition, as in Pascal's triangle.

Let's use the recurrences to find $C_{50}$. First, $C_{50}=C_0+Q_{50}$;
so we want to know $Q_{50}$. Then $Q_{50}=Q_{25}+D_{50}$, and
$Q_{25}=Q_0+D_{25}$; so we also want to know $D_{50}$ and~$D_{25}$.
These $D_n$ depend in turn on $D_{40}$, $D_{30}$, $D_{20}$, $D_{15}$,
$D_{10}$,~$D_5$, and on $N_{50}$,~$N_{45}$, \dots,~$N_5$. A simple calculation
therefore suffices to determine all the necessary coefficients:
\begindisplay \let\preamble=\tablepreamble \offinterlineskip
n&&0&5&10&15&20&25&30&35&40&45&50\cr
\omit&height2pt\cr
\noalign{\hrule}
\omit&height2pt\cr
P_n&&1&1&1&1&1&1&1&1&1&1&1\cr
N_n&&1&2&3&4&5&6&7&8&9&10&11\cr
D_n&&1&2&4&6&9&12&16&&25&&36\cr
Q_n&&1&&&&&13&&&&&49\cr
C_n&&1&&&&&&&&&&50\cr
\enddisplay
The final value in the table gives us our answer, $C_{50}$: There are
exactly 50~ways to leave a 50-cent tip.
\g(Not counting the option of charging the tip to a credit card.)\g

How about a closed form for $C_n$? Multiplying the equations together
gives us the compact expression
\begindisplay
C={1\over1-z}\,{1\over1-z^5}\,{1\over1-z^{10}}\,{1\over1-z^{25}}\,
 {1\over1-z^{50}}\,,
\eqno\eqref|change-gf|
\enddisplay
but it's not obvious how to get from here to the coefficient of $z^n$.
Fortunately there is a way; we'll return to this problem later in the
chapter.

More elegant formulas arise if we consider the problem of giving change
when we live in a land that mints
 coins of every positive integer denomination
(\penny,
\kern1pt\beginpicture(25,18)(-12.5,0)
	\put(0,9){\circle{25}}
	\put(0,9){\makebox(0,0){\tiny 2}}
	\endpicture\kern1pt,
\kern1pt\beginpicture(28,18)(-14,0)
	\put(0,9){\circle{28}}
	\put(0,9){\makebox(0,0){\tiny 3}}
	\endpicture\kern1pt, \dots~)
instead of just the five we allowed before.
The corresponding generating function is
an infinite product of fractions,
\begindisplay
{1\over (1 - z) (1 - z^2) (1 - z^3) \ldots}\,,
\enddisplay
and the coefficient of $z^n$ when these factors are fully
multiplied out is called $p(n)$, the number of {\it"partitions"\/} of~$n$.
A partition of~$n$ is a representation of~$n$ as a sum of positive integers,
disregarding order. For example, there are seven different partitions of~$5$,
namely
\begindisplay \tightplus \let\displaythick=\normalthick
5=4+1=3+2=3+1+1=2+2+1=2+1+1+1=1+1+1+1+1\,;
\enddisplay
hence $p(5)=7$. (Also
$p(2)=2$, $p(3)=3$, $p(4)=5$, and $p(6)=11$; it begins to look as if
$p(n)$ is always a prime number. But $p(7)=15$, spoiling the pattern.)
 There is no closed form
for $p(n)$, but the theory of partitions is a fascinating branch of
mathematics in which many remarkable discoveries have been made.
For example, "Ramanujan" proved that $p(5n+4)\=0$ \tmod5,
$p(7n+5)\=0$ \tmod7, and $p(11n+6)\=0$ \tmod{11},
by making ingenious transformations of generating functions
(see "Andrews"~[|andrews-partitions|, Chapter~10]).

\beginsection 7.2 Basic Maneuvers

Now let's look more closely
at some of the techniques that make
power series powerful.

First a few words about terminology and notation.
Our generic generating function has the form
\begindisplay
G(z)
	= g_0 + g_1 z + g_2 z^2 + \cdots
	= \sum_{n \geq 0} g_n z^n \,,
\eqno\eqref|gf-ord|
\enddisplay
and we say that $G(z)$, or $G$ for short,
is the generating function for the sequence
$\<g_0,g_1,g_2,\ldots\,\>$,
which we also call~$\<g_n\>$.
The coefficient $g_n$ of $z^n$ in~$G(z)$ is often
denoted $[z^n]\,G(z)$, as in Section~5.4.

The sum in \thiseq\ runs over all $n\geq0$, but we often
find it more convenient
to extend the sum over all integers~$n$.
We can do this by simply regarding
$g_{-1} = g_{-2} = \cdots = 0$.
In such cases we might still talk about
the sequence $\<g_0,g_1,g_2,\ldots\,\>$,
as if the $g_n$'s didn't exist for negative~$n$.

Two kinds of ``"closed forms"'' come up when we work with generating functions.
We might have a "closed form" for $G(z)$, expressed in terms of~$z$;
or we might have a closed form for $g_n$, expressed in terms of~$n$.
For example, the generating function for "Fibonacci numbers" has the
closed form $z/(1-z-z^2)$; the Fibonacci numbers themselves have the
closed form $(\phi^n-\phihat^n)/\!\sqrt5$. The context will explain what
kind of closed form is meant.

Now a few words about perspective.
"!philosophy"
The generating function~$G(z)$ appears to be two different entities,
depending on how we view it.
Sometimes it is a function of a complex variable~$z$,
satisfying all the standard properties proved in calculus books.
And sometimes it is simply a "formal power series",
\g If physicists can get away with viewing light
sometimes as a wave and sometimes as a particle,
mathematicians should be able to view generating functions in
two different ways.\g
with $z$~acting as a placeholder. In the previous section, for example,
we used the second interpretation;
we saw several examples in which $z$ was substituted for some
feature of a combinatorial object in a ``sum''
of such objects. The coefficient of~$z^n$ was then the
number of combinatorial objects having $n$ occurrences of that feature.

When we view $G(z)$ as a function of a complex variable,
its "convergence" becomes an issue.
We said in Chapter~2 that the infinite series $\sum_{n\ge0}g_nz^n$
converges (absolutely) if and only if there's a bounding constant $A$
such that the finite sums $\sum_{0\le n\le N}\vert g_nz^n\vert$ never
exceed~$A$, for any~$N$. Therefore it's easy to see that if $\sum_{n\ge0}
g_nz^n$ converges for some value $z=z_0$, it also converges for all $z$
with $\vert z\vert<\vert z_0\vert$. Furthermore, we must have
$\lim_{n\to\infty}\vert g_nz_0^n\vert=0@$; hence, in the
notation of Chapter~9, $g_n=O\bigl(\vert1/z_0\vert^n
\bigr)$ if there is convergence at~$z_0$. And conversely if $g_n=O(M^n)$,
the series $\sum_{n\ge0}g_nz^n$ converges for all $\vert z\vert<1/M$.
These are the basic facts about convergence of power series.

But for our purposes convergence is usually a red herring, unless we're
trying to study the asymptotic behavior of the coefficients.
Nearly every operation we perform on generating functions can be
justified rigorously as an operation on formal power series, and such
operations are legal even when the series don't converge.
(The relevant theory can be found, for example, in "Bell"~[|bell-gf|],
"Niven"~[|niven-gf|], and "Henrici"~[|henrici|, Chapter~1].)

Furthermore, even if we throw all caution to the winds
\g Even if we remove the tags from our mattresses.\g
 and derive formulas
without any rigorous justification, we generally can take the
results of our derivation and prove them by induction. For example,
the generating function for the Fibonacci numbers converges only when
$\vert z\vert<1/\phi\approx0.618$, but we didn't need to know that
when we proved the formula $F_n=(\phi^n-\phihat^n)/\!\sqrt5$. The latter
formula, once discovered, can be verified directly, if we don't
trust the theory of formal power series. Therefore we'll
ignore questions of convergence in this chapter; it's more a hindrance than
a help.

So much for perspective.
Next we look at our main tools
for reshaping generating functions\dash---%
adding, shifting, changing variables,
differentiating, integrating, and multiplying.
In what follows we assume that, unless stated otherwise,
$F(z)$ and $G(z)$ are the generating functions for
the sequences $\<f_n\>$ and $\<g_n\>$.
We also assume that the $f_n$'s and $g_n$'s
are zero for negative~$n$,
since this saves us some bickering with the limits of summation.

It's pretty obvious what happens when we add
constant multiples of $F$~and~$G$ together:
\begindisplay \advance\belowdisplayskip-3pt
\alpha F(z) + \beta G(z)
&= \alpha \sum_n f_n z^n \;+\; \beta \sum_n g_n z^n\cr
&= \sum_n \,(\alpha f_n + \beta g_n)\, z^n \,.
\eqno
\enddisplay
This gives us the generating function for the sequence
$\<\alpha f_n + \beta g_n\>$.

Shifting a generating function isn't much harder.
To shift~$G(z)$ right by $m$~places, that is,
to form the generating function for the sequence
$\<0,\ldots, 0,\allowbreak g_0,g_1,\ldots\,\>=\<g_{n-m}\>$
with $m$~leading $0$'s,
we simply multiply by~$z^m$:
\begindisplay \advance\belowdisplayskip-3pt
z^m G(z)
	= \sum_n g_n\, z^{n+m}
	= \sum_n g_{n-m}\, z^n \,,
\qquad\hbox{integer $m\ge0$}.
\eqno\eqref|gf-shift-up|
\enddisplay
This is the operation we used (twice), along with addition,
to deduce the equation $(1 - z - z^2) F(z) = z$
on our way to finding a closed form for the Fibonacci numbers
in Chapter~6.

And to shift~$G(z)$ left $m$~places\dash---that is,
to form the generating function for the sequence
$\<g_m,g_{m+1},g_{m+2},\ldots\,\>=\<g_{n+m}\>$ with the first
$m$ elements discarded\dash---
we subtract off the first~$m$ terms
and then divide by~$z^m$:
\begindisplay \postdisplaypenalty=10000 \advance\belowdisplayskip-1pt
{G(z){-}g_0{-}g_1 z{-}\cdots{-}g_{m-1} z^{m-1}\over z^m}
	= \sum_{n \geq m} g_n z^{n-m}
	= \sum_{n \geq 0} g_{n+m} z^n.
\eqno
\enddisplay
(We can't extend this last sum over all~$n$ unless $g_0=\cdots=g_{m-1}=0$.)

Replacing the $z$ by a constant multiple
is another of our tricks:
\begindisplay \advance\belowdisplayskip-3pt
G(cz)
	= \sum_n g_n (cz)^n
	= \sum_n c^n g_n z^n \,;
\eqno
\enddisplay
this yields the generating function for
the sequence $\<c^n g_n\>$.
The special case $c = -1$ is particularly useful.

Often we want to bring down a factor of~$n$ into
\g I fear $d$generating-function $dz$'s.\g "!disease"
 the coefficient.
Differentiation is what lets us do that:
\begindisplay \advance\belowdisplayskip-3pt
G'(z)
	= g_1 + 2 g_2 z + 3 g_3 z^2 + \cdots
	= \sum_n (n+1) g_{n+1}\,z^n \,.
\eqno\eqref|gf-deriv|
\enddisplay
Shifting this right one place gives us a form
that's sometimes more useful,
\begindisplay \advance\belowdisplayskip-3pt
z G'(z)
	= \sum_n n g_n\,z^n \,.
\eqno
\enddisplay
This is the generating function for
the sequence $\<n g_n\>$. Repeated differentiation would allow us to multiply
$g_n$ by any desired polynomial in~$n$.

Integration, the inverse operation,
lets us divide the terms by~$n$:
\begindisplay
\int_0^z \! G(t) \, dt
	= g_0 z + {1\over 2} g_1 z^2 + {1\over 3} g_2 z^3 + \cdots
	= \sum_{n \geq 1} {1\over n} g_{n-1}\, z^n\,.
\eqno
\enddisplay
(Notice that the constant term is zero.) If we want the generating
function for $\<g_n/n\>$ instead of $\<g_{n-1}/n\>$,
we should first shift left one place, replacing $G(t)$ by $\bigl(G(t)-g_0\bigr)/t$
in the integral.

Finally, here's how we multiply generating functions together:
\begindisplay
F(z)@G(z)
	&= (f_0 + f_1 z + f_2 z^2 + \cdots\,)
			(g_0 + g_1 z + g_2 z^2 + \cdots\,) \cr
\noalign{\smallskip}
	&= (f_0 g_0) \,+\, (f_0 g_1 + f_1 g_0) z
					\,+\, (f_0 g_2 + f_1 g_1 + f_2 g_0) z^2
						\,+\, \cdots \cr
\noalign{\smallskip}
	&= \sum_n \Bigl( \sum_k f_k@g_{n-k} \Bigr) z^n \,.
\eqno
\enddisplay
As we observed in Chapter~5,
this gives the generating function for the sequence
$\<h_n\>$, the {\it"convolution"\/} of $\<f_n\>$
and~$\<g_n\>$. The sum
$h_n=\sum_k f_k g_{n-k}$ can also be written $h_n=\sum_{k=0}^n f_k g_{n-k}$,
because $f_k=0$ when $k<0$ and $g_{n-k}=0$ when $k>n$.
Multiplication/convolution is a little more complicated than the other operations,
but it's very useful\dash---%
so useful that we will spend all of Section~7.5 below
looking at examples of it.

Multiplication has several special cases that are worth
considering as operations in themselves.
We've already seen one of these:
When $F(z) = z^m$ we get
the shifting operation~\eq(|gf-shift-up|).
In that case the sum~$h_n$ becomes the single term~$g_{n-m}$,
because all $f_k$'s are~$0$ except for $f_m = 1$.

Another useful special case arises when $F(z)$ is the
familiar function ${1/(1-z)} = 1 + z + z^2 + \cdots\,$; then
all $f_k$'s (for $k \geq 0$) are~$1$
and we have the important formula
\begindisplay
{1\over 1-z} G(z)
	= \sum_n \biggl( \sum_{k\ge0} g_{n-k} \biggr) z^n 
	= \sum_n \biggl( \sum_{k\le n} g_k \biggr) z^n \,.
\eqno\eqref|gf-cum-sum|
\enddisplay
Multiplying a generating function by $1/(1-z)$
gives us the generating function for the cumulative sums
of the original sequence.

\topinsert
\table Generating function manipulations.\tabref|gf-manip|
\begindisplay\abovedisplayskip=-2pt \belowdisplayskip=6pt \openup7pt%
 \advance\displayindent-\parindent\advance\displaywidth\parindent
\alpha F(z) + \beta G(z)
 &= \sum_n (\alpha f_n + \beta g_n) z^n\cr
z^m G(z)
 &= \sum_n g_{n-m} \,z^n\,,\qquad\hbox{integer $m\ge0$}\cr
{G(z) - g_0 - g_1 z - \cdots - g_{m-1} z^{m-1}\over z^m}
 &= \sum_{n \geq 0} g_{n+m} \,z^n\,,\qquad\hbox{integer $m\ge0$}\cr
G(cz)
 &= \sum_n c^n g_n \,z^n \cr
G'(z)
 &= \sum_n (n+1) g_{n+1} \,z^n\cr
z G'(z)
 &= \sum_n n g_n \,z^n \cr
\int_0^z G(t) \, dt
 &= \sum_{n \geq 1} {1\over n} g_{n-1} \,z^n\cr
F(z)@G(z)
 &= \sum_n \biggl( \sum_k f_k g_{n-k} \biggr) z^n \cr
{1\over 1-z} G(z)
 &= \sum_n \biggl( \sum_{k\le n} g_k \biggr) z^n \cr
\enddisplay
\hrule width\hsize height.5pt
\kern4pt
\endinsert

\topinsert
\table Simple sequences and their generating functions.\tabref|gf-simp|
\kern3pt\openup5pt
\halign to\hsize{$\left\<#,\ldots\,\right\>\hfil$%
 \tabskip=2em plus 100pt&
 $\displaystyle\sum\nolimits_{n\ge0}#\,z^n\hfil$&
 $\displaystyle#\hfil$\tabskip=0pt\cr
\omit\strut sequence\hfil&\omit generating function\hfil&\omit closed form\hfil\cr
\noalign{\hrule width\hsize\kern5pt}
1,0,0,0,0,0&\[n=0]&1\cr
0,\ldots,0,1,0,0&\[n=m]&z^m\cr
1,1,1,1,1,1&&{1\over1-z}\cr
1,-1,1,-1,1,-1&(-1)^n&{1\over1+z}\cr
1,0,1,0,1,0&\[2\divides n]&{1\over 1-z^2}\cr
1,0,\ldots,0,1,0,\ldots,0,1,0&\[m\divides n]&{1\over 1-z^m}\cr
1,2,3,4,5,6&(n+1)&{1\over(1-z)^2}\cr
1,2,4,8,16,32&2^n&{1\over1-2z}\cr
1,4,6,4,1,0,0&{4\choose n}&(1+z)^4\cr
1,c,{c\choose2},{c\choose3}&{c\choose n}&(1+z)^c\cr
1,c,{c+1\choose2},{c+\2\choose 3}&{c{+}n{-}1\choose n}&{1\over(1-z)^c}\cr
1,c,c^2,c^3&c^n&{1\over1-cz}\cr
1,{m+1\choose m},{m+\2\choose m},{m+3\choose m}&{m{+}n\choose m}&{1\over(1-z)^{m+1}}\cr
0,1,{1\over2},{1\over3},{1\over4}&
 \omit$\displaystyle\sum\nolimits_{n\ge1}{1\over n}\,z^n\hfil$&
 \ln{1\over1-z}\cr
0,1,-{1\over2},{1\over3},-{1\over4}&
 \omit$\displaystyle\sum\nolimits_{n\ge1}{(-1)^{n+1}\over n}\,z^n\hfil$&
 \ln(1+z)\cr
1,1,{1\over2},{1\over6},{1\over24},{1\over120}&{1\over n!}&e^z\cr}
\kern6pt
\hrule width\hsize height.5pt
\kern1pt
\endinsert

Table |gf-manip| summarizes the operations we've discussed so far.
To use all these manipulations effectively
it helps to have a healthy repertoire
of generating functions in stock.
Table~|gf-simp| lists the simplest ones; we can use those to get started
and to solve quite a few problems.

Each of the generating functions in Table |gf-simp| is important enough
to be memorized. Many of them are special cases of the others, and many
of them can be derived quickly from the others by using the basic operations
\g\vskip-27pt Hint: If the sequence consists of binomial coefficients,
its generating function usually involves a binomial, $1\pm z$.\g
of Table~|gf-manip|; therefore the memory work isn't very hard.

For example, let's consider the sequence $\<1,2,3,4,\ldots\,\>$,
whose generating function $1/(1-z)^2$ is often useful. This
generating function
appears near the middle of Table~|gf-simp|, and it's also the special
case $m=1$ of $\<1,{m+1\choose m},\allowbreak{m+\2\choose m},\allowbreak
{m+3\choose m},\ldots\,\>$, which appears further down;
it's also the special case $c=2$ of the closely related
sequence $\<1,c,{c+1\choose2},{c+\2\choose3},
\ldots\,\>$. We can derive it from the generating function for
$\<1,1,1,1,\ldots\,\>$ by taking cumulative sums
as in \eq(|gf-cum-sum|); that is, by
dividing $1/(1-z)$ by $(1-z)$. Or we can derive it from
\g OK, OK, I'm convinced already.\g
$\<1,1,1,1,\ldots\,\>$ by differentiation, using \eq(|gf-deriv|).

The sequence $\<1,0,1,0,\ldots\,\>$ is another one whose generating
function can be obtained in many ways. We can obviously derive the
formula $\sum_nz^{2n}=1/(1-z^2)$ by substituting $z^2$ for $z$ in the
identity $\sum_nz^n=1/(1-z)$; we can also apply cumulative summation
to the sequence $\<1,-1,1,-1,\ldots\,\>$, whose generating function
is $1/(1+z)$, getting $1/(1+z)(1-z)=1/(1-z^2)$. And there's also a third
way, which is based on a general method for extracting the even-numbered
terms $\<g_0,0,g_2,0,g_4,0,\ldots\,\>$ of {\it any\/} given sequence:
If we add $G(-z)$ to $G(+z)$ we get
\begindisplay
G(z)+G(-z)=\sum_ng_n\bigl(1+(-1)^n\bigr)z^n=2\sum_n g_n\[\hbox{$n$ even}]z^n\,;
\enddisplay
therefore
\begindisplay
{G(z)+G(-z)\over2}=\sum_n g_{2n}\,z^{2n}\,.
\eqno\eqref|gf-even|
\enddisplay
The odd-numbered terms can be extracted in a similar way,
\begindisplay
{G(z)-G(-z)\over2}=\sum_n g_{2n+1}\,z^{2n+1}\,.
\eqno\eqref|gf-odd|
\enddisplay
In the special case where $g_n=1$ and $G(z)=1/(1-z)$, the generating function
for $\<1,0,1,0,\ldots\,\>$ is $\half\bigl(G(z)+G(-z)\bigr)=
\half\bigl({1\over1-z}+{1\over1+z}\bigr)={1\over1-z^{2\mathstrut}}$.

Let's try this extraction trick on the generating function for Fibonacci
numbers. We know that $\sum_nF_nz^n=z/(1-z-z^2)$; hence
\begindisplay
\sum_nF_{2n}z^{2n}&=\half\biggl({z\over1-z-z^2}+{-z\over1+z-z^2}\biggr)\cr
&=\half\biggl({z+z^2-z^3-z+z^2+z^3\over(1-z^2)^2-z^2}\biggr)
={z^2\over1-3z^2+z^4}\,.
\enddisplay
This generates the sequence $\<F_0,0,F_2,0,F_4,\ldots\,\>$; hence
the sequence of alternate $F$'s, $\advance\thinmuskip0mu minus 1mu
\<F_0,F_2,F_4,F_6,\ldots\,\>=\<0,1,3,8,\ldots\,\>$,
has a simple generating function:
\begindisplay
\sum_n F_{2n}z^n={z\over1-3z+z^2}\,.
\eqno\eqref|fib-gf-even|
\enddisplay

\beginsection 7.3 Solving Recurrences

Now let's focus our attention on one of the most important uses of generating
functions: the "solution" of "recurrence" relations.

Given a sequence $\<g_n\>$ that satisfies a given recurrence,
we seek a closed form for $g_n$ in terms of~$n$. A solution to this problem
via generating functions proceeds in four steps that are almost mechanical
enough to be programmed on a computer:

\nobreak\smallskip
\item{1} Write down a single equation that expresses $g_n$ in terms of
other elements of the sequence. This equation should be valid for all
integers~$n$, assuming that $g_{-1}=g_{-2}=\cdots=0$.

\smallbreak
\item{2} Multiply both sides of the equation by $z^n$ and sum over all~$n$.
This gives, on the left, the sum $\sum_n g_nz^n$, which is the generating
function~$G(z)$. The right-hand side should be manipulated so that it
becomes some other expression involving~$G(z)$.

\smallbreak
\item{3} Solve the resulting equation, getting a closed form for~$G(z)$.

\smallbreak
\item{4} Expand $G(z)$ into a power series and read off the coefficient
of~$z^n$; this is a closed form for~$g_n$.

\nobreak\smallskip\noindent
This method works because the single function $G(z)$ represents the entire
sequence $\<g_n\>$ in such a way that many manipulations
are possible.

\subhead Example 1: Fibonacci numbers revisited.

For example, let's rerun the derivation of Fibonacci numbers from Chapter~6.
"!Fibonacci numbers, generating function"
In that chapter we were feeling our way, learning a new method; now we
can be more systematic. The given recurrence is
\begindisplay
g_0&=0\,;\qquad g_1=1\,;\cr
g_n&=g_{n-1}+g_{n-2}\,,\qquad\hbox{for $n\ge2$}.
\enddisplay
We will find a closed form for $g_n$ by using the four steps above.

Step 1 tells us to write the recurrence as a ``single equation'' for~$g_n$.
We could say
\begindisplay
g_n=\cases{0,&if $n\le0@$;\cr 1,&if $n=1$;\cr
 g_{n-1}+g_{n-2},&if $n>1$;\cr}
\enddisplay
but this is cheating. Step 1 really asks for a formula that doesn't
involve a case-by-case construction. The single equation
\begindisplay
g_n=g_{n-1}+g_{n-2}
\enddisplay
works for $n\ge2$, and it also holds when $n\le0$ (because we have $g_0=0$
and $g_{\rm negative}=0$). But when $n=1$ we get $1$~on the left
and $0$~on the right. Fortunately the problem is easy to fix, since
we can add $\[n=1]$ to the right; this adds~$1$ when $n=1$, and it makes
no change when $n\ne1$. So, we have
\begindisplay
g_n=g_{n-1}+g_{n-2}+\[n=1]\,;
\enddisplay
this is the equation called for in Step 1.

Step 2 now asks us to transform the equation for $\<g_n\>$
into an equation for $G(z)=\sum_ng_nz^n$. The task is not difficult:
\begindisplay \openup3pt
G(z)=\sum_ng_nz^n
&=\sum_ng_{n-1}\,z^n+\sum_ng_{n-2}\,z^n+\sum_n\[n=1]z^n\cr
&=\sum_ng_n\,z^{n+1}+\sum_ng_n\,z^{n+2}\;+\;z\cr
&=z@ G(z)+z^2G(z)+z\,.\cr
\enddisplay
Step 3 is also simple in this case; we have
\begindisplay
G(z)={z\over1-z-z^2}\,,
\enddisplay
which of course comes as no surprise.

Step 4 is the clincher. We carried it out in Chapter~6 by having a
sudden flash of inspiration; let's go more slowly now, so that we can
get through Step~4 safely later, when we meet problems that are
more difficult. What is
\begindisplay
[z^n]\,{z\over 1-z-z^2}\,,
\enddisplay
the coefficient of
$z^n$ when $z/(1-z-z^2)$ is expanded in a power series? More generally,
if we are given any "rational function"
\begindisplay
R(z)={P(z)\over Q(z)}\,,
\enddisplay
 where $P$ and~$Q$
are polynomials, what is the coefficient $[z^n]\,R(z)$?

There's one kind of rational function whose coefficients are particularly nice,
namely
\begindisplay
{a\over(1-\rho z)^{m+1}}=\sum_{n\ge0}{m+n\choose m}a\rho^nz^n\,.
\eqno
\enddisplay
(The case $\rho=1$ appears in Table~|gf-simp|, and we can get the general
formula shown here by substituting $\rho z$ for~$z$.) A finite sum of functions
like \thiseq,
\begindisplay
S(z)=
{a_1\over(1-\rho_1z)^{m_1+1}}+
{a_2\over(1-\rho_2z)^{m_2+1}}+\cdots+
{a_l\over(1-\rho_lz)^{m_l+1}}\,,
\eqno\eqref|s-parts|
\enddisplay
also has nice coefficients,
\begindisplay
[z^n]\,S(z)&=
a_1{m_1+n\choose m_1}\rho_1^n+
a_2{m_2+n\choose m_2}\rho_2^n\cr
&\hskip12em+\cdots+
a_l{m_l+n\choose m_l}\rho_l^n\,.
\eqno
\enddisplay
We will show that every rational function $R(z)$ such that $R(0)\ne\infty$
can be expressed in the form
\begindisplay
R(z)=S(z)+T(z)\,,
\eqno\eqref|r-s-t|
\enddisplay
where $S(z)$ has the form \eq(|s-parts|) and $T(z)$ is a polynomial.
Therefore there is a closed form for the coefficients $[z^n]\,R(z)$.
Finding $S(z)$ and~$T(z)$ is equivalent to finding the
``"partial fraction expansion"'' of~$R(z)$.

Notice that $S(z)=\infty$ when $z$ has the values $1/\rho_1$,
\dots,~$1/\rho_l$. Therefore the numbers $\rho_k$ that we need to find,
if we're going to succeed in expressing $R(z)$ in the desired form
$S(z)+T(z)$, must be the reciprocals of the numbers $\alpha_k$ where
$Q(\alpha_k)=0$. (Recall that $R(z)=P(z)/Q(z)$, where $P$ and~$Q$ are
polynomials; we have $R(z)=\infty$ only if $Q(z)=0$.)

Suppose $Q(z)$ has the form
\begindisplay
Q(z)=q_0+q_1z+\cdots+q_mz^m\,,\qquad\hbox{where $q_0\ne0$ and $q_m\ne0$.}
\enddisplay
The ``reflected'' polynomial "!reflected polynomial"
\begindisplay
Q^{\rm R}(z)=q_0z^m+q_1z^{m-1}+\cdots+q_m
\enddisplay
has an important relation to $Q(z)$:
\begindisplay
&Q^{\rm R}(z)=q_0(z-\rho_1)\ldots(z-\rho_m)\cr
&\qquad\iff
 Q(z)=q_0(1-\rho_1z)\ldots(1-\rho_mz)\,.
\enddisplay
Thus, the roots of $Q^{\rm R}$ are the reciprocals of the roots of~$Q$,
and vice versa. We can therefore
find the numbers $\rho_k$ we seek by factoring
the reflected polynomial $Q^{\rm R}(z)$.

For example, in the Fibonacci case we have
\begindisplay
Q(z)=1-z-z^2\,;\qquad Q^{\rm R}(z)=z^2-z-1\,.
\enddisplay
The roots of $Q^{\rm R}$ can be found by setting $(a,b,c)=(1,-1,-1)$
in the quad\-ratic formula $\bigl(-b\pm\mskip-1mu
 \sqrt{\,b^{\mathstrut2}-4ac}\,\bigr)
/2a$; we find that they are
\begindisplay \postdisplaypenalty=10000
\phi={1+\sqrt5\over2}\And \phihat={1-\sqrt5\over2}\,.
\enddisplay
Therefore $Q^{\rm R}(z)=(z-\phi)(z-\phihat)$ and $Q(z)=(1-\phi z)(1-\phihat z)$.

Once we've found the $\rho$'s, we can proceed to find the partial fraction
expansion. It's simplest if all the roots are distinct, so let's consider that
special case first. We might as well state and prove the general result
formally:

\subhead Rational Expansion Theorem for Distinct Roots.

{\sl If\/ $R(z)=P(z)/Q(z)$, where $Q(z)=q_0(1-\rho_1z)\ldots(1-\rho_l z)$
and the numbers $(\rho_1,\ldots,\rho_l)$ are distinct, and if\/ $P(z)$
is a polynomial of degree less than~$l$, then}
\begindisplay
[z^n]\,R(z)=a_1\rho_1^n+\cdots+a_l\rho_l^n\,,\qquad
\hbox{where \ $a_k=\displaystyle{-\rho_k P(1/\rho_k)\over Q'(1/\rho_k)}$}.
\eqno\eqref|rat-distinct|
\enddisplay
Proof: Let $a_1$, \dots, $a_l$ be the stated constants. Formula \thiseq\
holds if $R(z)=P(z)/Q(z)$ is equal to
\begindisplay
S(z)={a_1\over1-\rho_1z}+\cdots+{a_l\over1-\rho_lz}\,.
\enddisplay
And we can prove that $R(z)=S(z)$ by showing that the function
$T(z)=R(z)-S(z)$ is not
\g Impress your parents by leaving the book open at this page.\g
infinite as $z\to1/\rho_k$. For this will show that the rational function~$T(z)$
is never infinite; hence $T(z)$ must be a polynomial. We also can show that
$T(z)\to0$ as $z\to\infty$; hence $T(z)$~must
be zero.

Let $\alpha_k=1/\rho_k$. To prove that $\lim_{z\to\alpha_k}T(z)\ne\infty$,
it suffices to show that $\lim_{z\to\alpha_k}\,(z-\alpha_k)T(z)=0$,
because $T(z)$ is a rational function of $z$.
Thus we want to show that
\begindisplay
\lim_{z\to\alpha_k}\,(z-\alpha_k)R(z)
=\lim_{z\to\alpha_k}\,(z-\alpha_k)S(z)\,.
\enddisplay
The right-hand limit equals
$\lim_{z\to\alpha_k}a_k(z-\alpha_k)/(1-\rho_kz)=-a_k/\rho_k$, because
$(1-\rho_kz)=-\rho_k(z-\alpha_k)$ and $(z-\alpha_k)/(1-\rho_jz)\to0$
for $j\ne k$. The left-hand limit is
\begindisplay
\lim_{z\to\alpha_k}\,(z-\alpha_k){P(z)\over Q(z)}=P(\alpha_k)\lim_{z\to\alpha_k}
{z-\alpha_k\over Q(z)}={P(\alpha_k)\over Q'(\alpha_k)}\,,
\enddisplay
by L'Hospital's rule. Thus the theorem is proved.

Returning to the Fibonacci example,
we have $P(z)=z$ and $Q(z)=1-z-z^2=(1-\phi z)(1-\phihat z)$;
hence $Q'(z)=-1-2z$, and
\begindisplay
{-\rho P(1/\rho)\over Q'(1/\rho)}={-1\over-1-2/\rho}={\rho\over\rho+2}\,.
\enddisplay
According to \eq(|rat-distinct|),
the coefficient of $\phi^n$ in $[z^n]\,R(z)$ is therefore $\phi/(\phi+2)=
1/\!\sqrt5@$; the coefficient of $\phihat^n$ is
$\phihat/(\phihat+2)=-1/\!\sqrt5$. So the theorem
tells us that $F_n=(\phi^n-\phihat^n)/\mskip-1mu
\sqrt5$, as in \equ(6.|fib-sol|).

When $Q(z)$ has repeated roots, the calculations become more difficult,
but we can beef up the proof of the theorem and prove the following
more general result:

\subhead General Expansion Theorem for Rational Generating Functions.

{\sl If\/ $R(z)=P(z)/Q(z)$, where\/ $Q(z)=q_0(1-\rho_1z)^{d_1}\ldots
(1-\rho_lz)^{d_l}$ and the numbers\/ $(\rho_1,\ldots,\rho_l)$ are
distinct, and if\/ $P(z)$ is a polynomial of degree less than\/
$d_1+\cdots+d_l$, then
\begindisplay
[z^n]\,R(z)=f_1(n)\rho_1^n\,+\,\cdots\,+\,f_l(n)\rho_l^n
\qquad\hbox{\rm for all $n\ge0$},
\eqno\eqref|gen-exp-thm|
\enddisplay
where each\/ $f_k(n)$ is a polynomial of degree\/ $d_k-1$ with
leading coefficient}
\begindisplay \tightplus \openup3pt
a_k&={(-\rho_k)^{d_k}P(1/\rho_k)d_k\over Q^{(d_k)}(1/\rho_k)}\cr
&={P(1/\rho_k)\over (d_k-1)!\,q_0\prod_{j\ne k}(1-\rho_j/\rho_k)^{d_j}}\,.
\eqno\eqref|gen-exp-a|
\enddisplay
This can be proved by induction on $\max(d_1,\ldots,d_l)$, using the fact that
\begindisplay
R(z)-{a_1(d_1-1)!\over(1-\rho_1z)^{d_1}}
-\cdots-{a_l(d_l-1)!\over(1-\rho_lz)^{d_l}}
\enddisplay
is a rational function whose denominator polynomial
is not divisible by\break ${(1-\rho_kz)^{d_k}}$ for any~$k$.

\subhead Example 2: A more-or-less random recurrence.

Now that we've seen some general methods, we're ready to tackle new problems.
Let's try to find a closed form for the recurrence
\begindisplay
g_0&=g_1=1\,;\cr
g_n&=g_{n-1}+2g_{n-2}+(-1)^n\,,\qquad\hbox{for $n\ge2$}.
\eqno\eqref|more-less-random|
\enddisplay
It's always a good idea to make a table of small cases first, and the
recurrence lets us do that easily:
\begindisplay \let\preamble=\tablepreamble
n&&0&1&2&3&4&5&6&7\cr
\noalign{\hrule}
(-1)^n&&1&-1&1&-1&1&-1&1&-1\cr
\noalign{\hrule}
g_n&&1&1&4&5&14&23&52&97\cr
\enddisplay
No closed form is evident, and this sequence isn't even listed in
"Sloane"'s {\sl Handbook\/}~[|sloane|];
so we need to go through the four-step process if we want to discover the solution.

Step 1 is easy, since we merely need to insert fudge factors to fix
things when $n<2$: The equation
\begindisplay
g_n=g_{n-1}+2g_{n-2}+(-1)^n\[n\ge0]+\[n=1]
\enddisplay
holds for all integers $n$. Now we can carry out Step 2:
\g\vskip30pt N.B.: The upper index on $\sum_{n=1}z^n$ is not missing!\g
\begindisplay
G(z)=\sum_ng_nz^n&=\!\sum_ng_{n-1}z^n+2\sum_ng_{n-2}z^n+\!\sum_{n\ge0}(-1)^nz^n
+\!\sum_{n=1}z^n\cr
&=zG(z)+2z^2G(z)+{1\over1+z}+z\,.\cr
\enddisplay
(Incidentally, we could also have used $-1\choose n$ instead of $(-1)^n\[n\ge0]$,
thereby getting $\sum_n{-1\choose n}z^n=(1+z)^{-1}$ by the binomial theorem.)
Step~3 is elementary algebra, which yields
\begindisplay
G(z)={1+z(1+z)\over(1+z)(1-z-2z^2)}={1+z+z^2\over(1-2z)(1+z)^2}\,.
\enddisplay
And that leaves us with Step 4.

The squared factor in the denominator is a bit troublesome,
since we know that repeated roots are more complicated than distinct roots;
but there it is.
We have two roots, $\rho_1=2$ and $\rho_2=-1$; the general expansion theorem
\eq(|gen-exp-thm|) tells us that
\begindisplay
g_n=a_12^n+(a_2n+c)(-1)^n
\enddisplay
for some constant~$c$, where
\begindisplay
a_1={1+1/2+1/4\over(1+1/2)^2}={7\over9}\,;\qquad
a_2={1-1+1\over1-2/(-1)}={1\over3}\,.
\enddisplay
(The second formula for $a_k$ in \eq(|gen-exp-a|) is easier to use than the
first one when the denominator has nice factors. We simply substitute
$z=1/\rho_k$ everywhere in $R(z)$, except in the factor where this gives zero,
and divide by $(d_k-1)!$; this gives the coefficient of $n^{d_k-1}\rho_k^n$.)
Plugging in $n=0$ tells us that the value of the remaining constant~$c$ had better
be $2\over9$; hence our answer is
\begindisplay
g_n=\textstyle{7\over9}2^n+\Bigl({1\over3}n+{2\over9}\Bigr)(-1)^n\,.
\eqno
\enddisplay
It doesn't hurt to check the cases $n=1$ and $2$, just to be sure that we didn't
foul up. Maybe we should even try $n=3$, since this formula looks weird.
But it's correct, all right.

Could we have discovered \thiseq\ by guesswork? Perhaps after tabulating a
few more values we may have observed that $g_{n+1}\approx2g_n$ when $n$~is
large. And with chutzpah and luck we might even have been able to smoke out
the constant $7\over9$. But it sure is simpler and more reliable to have
generating functions as a tool.

\subhead Example 3: Mutually recursive sequences.

Sometimes we have two or more recurrences that depend on each other.
Then we can form generating functions for both of them, and solve both
by a simple extension of our four-step method.

For example, let's return to the problem of $3\times n$ domino tilings
that we explored earlier this chapter. If we want to know only the total
number of ways, $U_n$, to cover a $3\times n$ rectangle with dominoes,
without breaking this number down into vertical dominoes versus
horizontal dominoes, we needn't go into as much detail as we did before.
We can merely set up the recurrences
\begindisplay \def\preamble{&\hfill$##$&${}##$\hfill\qquad}
U_0&=1\,,\qquad U_1=0\,;&V_0&=0\,,\qquad V_1=1\,;\cr
U_n&=2V_{n-1}+U_{n-2}\,,&V_n&=U_{n-1}+V_{n-2}\,,
\qquad\hbox{for $n\ge2$}.\cr
\enddisplay
Here $V_n$ is the number of ways to cover a $3\times n$ rectangle-minus-corner,
using $(3n-1)/2$ dominoes. These recurrences are easy to discover, if we
consider the possible domino configurations at the rectangle's
left edge, as before. Here are
the values of $U_n$ and $V_n$ for small~$n$:
"!Seaver, George Thomas"
\begindisplay \let\preamble=\tablepreamble
n&&0&1&2&3&4&5&6&7\cr
\noalign{\hrule}
U_n&&1&0&3&0&11&0&41&0\eqno\eqref|u-v-table|\cr
\noalign{\hrule}
V_n&&0&1&0&4&0&15&0&56\cr
\enddisplay

Let's find closed forms, in four steps.
First (Step~1), we have
\begindisplay
U_n=2V_{n-1}+U_{n-2}+\[n=0]\,,\qquad
V_n=U_{n-1}+V_{n-2}\,,
\enddisplay
for all $n$. Hence (Step~2),
\begindisplay
U(z)=2zV(z)+z^2U(z)+1\,,\,\qquad V(z)=zU(z)+z^2V(z)\,.
\enddisplay
Now (Step 3) we must solve two equations in two unknowns; but these are
easy, since the second equation yields $V(z)=zU(z)/(1-z^2)$; we find
\begindisplay
U(z)={1-z^2\over1-4z^2+z^4}\,;\qquad V(z)={z\over1-4z^2+z^4}\,.
\eqno
\enddisplay
\looseness=-1
(We had this formula for $U(z)$ in \eq(|u-with-cubes|), but with $z^3$
instead of~$z^2$. In that derivation, $n$~was the number of dominoes;~now
it's the width of the rectangle.)

The denominator $1-4z^2+z^4$ is a function of $z^2$; this is what
makes $U_{2n+1}=0$ and $V_{2n}=0$, as they should be.
We can take advantage of this nice property of~$z^2$ by retaining~$z^2$
when we factor the denominator:
We need not take $1-4z^2+z^4$ all the way to a product
of four factors $(1-\rho_kz)$, since two factors of the form $(1-\rho_kz^2)$
will be enough to tell us the coefficients. In other words if we consider
the generating function
\begindisplay
W(z)={1\over1-4z+z^2}=W_0+W_1\,z+W_2\,z^2+\cdots\,,
\eqno\eqref|1-4+1|
\enddisplay
we will have $V(z)=zW(z^2)$ and $U(z)=(1-z^2)W(z^2)$; hence
$V_{2n+1}=W_n$ and $U_{2n}=W_n-W_{n-1}$. We save time and energy
by working with the simpler function~$W(z)$.

The factors of $1-4z+z^2$ are $(z-2-\sqrt3\,)$ and $(z-2+\sqrt3\,)$,
and they can also be written $\bigl(1-(2{+}\sqrt3\,)z\bigr)$ and
$\bigl(1-(2{-}\sqrt3\,)z\bigr)$ because this polynomial is its own reflection.
Thus it turns out that we have
\begindisplay \openup3pt
&V_{2n+1}=W_n={3{+}2\sqrt3\over6}(2+\sqrt3\,)^n+
 {3{-}2\sqrt3\over6}(2-\sqrt3\,)^n\,;\hidewidth\cr
&U_{2n}=W_n-W_{n-1}&={3{+}\sqrt3\over6}(2+\sqrt3\,)^n+
 {3{-}\sqrt3\over6}(2-\sqrt3\,)^n\cr
&&= {(2+\sqrt3\,)^n\over3-\sqrt3}
 +{(2-\sqrt3\,)^n\over3+\sqrt3}\,.
\eqno
\enddisplay
This is the desired closed form for the number of $3\times n$ domino tilings.

Incidentally, we can simplify the formula for $U_{2n}$ by realizing that
the second term always lies between $0$ and~$1$. The number $U_{2n}$ is an
integer, so we have
\begindisplay
U_{2n}=\biggl\lceil{(2+\sqrt3\,)^n\over3-\sqrt3}\biggr\rceil\,,
\qquad\hbox{for $n\ge0$}.
\eqno\eqref|u-closed|
\enddisplay
In fact, the other term $(2-\sqrt3\,)^n\!/(3+\sqrt3\,)$ is extremely small
when $n$~is large, because $2-\sqrt3\approx0.268$. This needs to be
taken into account if we try to use formula \thiseq\ in numerical
calculations. For example, a fairly expensive name-brand hand "calculator"
comes up with $413403.0005$ when asked to compute $(2+\sqrt3@)^{10}\!/
(3-\sqrt3@)$. This is correct to nine significant figures; but the
true value is slightly {\it less\/} than $413403$, not slightly greater.
Therefore it would be a mistake to take the ceiling of $413403.0005$;
the correct answer, $U_{20}=413403$, is obtained by {\it rounding\/} to
the nearest integer. Ceilings
\g I've known slippery floors too.\g
can be hazardous.

\subhead Example 4: A closed form for "change".

When we left the problem of making change, we had just calculated
the number of ways to pay 50\rlap/c. Let's try now to count the number
of ways there are to change a dollar, or a million dollars\dash---%
still using only pennies, nickels, dimes, quarters, and halves.

\goodbreak
The generating function derived earlier is
\begindisplay
C(z)={1\over1-z}\,{1\over1-z^5}\,{1\over1-z^{10}}\,{1\over1-z^{25}}\,
{1\over1-z^{50}}\,;
\enddisplay
this is a rational function of $z$ with a denominator of degree~91.
Therefore we can decompose the denominator into 91 factors and
come up with a 91-term ``closed form'' for $C_n$, the number of
ways to give $n$ cents in change. But that's too horrible to contemplate.
Can't we do better than the general method suggests, in this particular case?

One ray of hope suggests itself immediately, when we notice that
the denominator is almost a function of $z^5$. The trick we just used
to simplify the calculations by noting that $1-4z^2+z^4$ is a function
of $z^2$ can be applied to $C(z)$, if we replace $1/(1-z)$ by
$(1+z+z^2+z^3+z^4)/(1-z^5)$:
\begindisplay \openup3pt
C(z)&={1+z+z^2+z^3+z^4\over1-z^5}\,
 {1\over1-z^5}\,{1\over1-z^{10}}\,{1\over1-z^{25}}\,
 {1\over1-z^{50}}\cr
 &= (1+z+z^2+z^3+z^4)\check C(z^5)\,,\cr
\noalign{\vskip1pt}
\check C(z)&={1\over1-z}\,
 {1\over1-z}\,{1\over1-z^2}\,{1\over1-z^5}\,{1\over1-z^{10}}\,.
\enddisplay
The compressed function $\check C(z)$ has a denominator whose degree is only~19,
so it's much more tractable than the original. This new expression for
$C(z)$ shows us, incidentally, that
$C_{5n}=C_{5n+1}=C_{5n+2}=C_{5n+3}=C_{5n+4}$; and indeed, this set of
equations
is obvious in retrospect: The number of ways to
leave a 53\rlap/c tip is the same as the number of ways to leave a 50\rlap/c
tip, because the number of pennies is predetermined modulo~$5$.

 But $\check C(z)$ still doesn't have a
\g Now we're also getting compressed reasoning.\g
really simple closed form based on the roots of the denominator.
The easiest way to compute the coefficients of $\check C(z)$ is probably to
recognize that each of the denominator factors is a divisor of $1-z^{10}$.
Hence we can write
\begindisplay
\check C(z)={A(z)\over(1-z^{10})^5}\,,\quad
\hbox{where $A(z)=A_0+A_1z+\cdots+A_{31}z^{31}$}.
\eqno\eqref|change-compressed|
\enddisplay
The actual value of $A(z)$, for the curious, is
\begindisplay \openup2pt
&(1+z+\cdots+z^9)^2(1+z^2+\cdots+z^8)(1+z^5)\cr
&\quad=1+2z+4z^2+6z^3+9z^4+13z^5+18z^6+24z^7\cr
&\quad\qquad+31z^8+39z^9+45z^{10}+52z^{11}+57z^{12}+63z^{13}+67z^{14}+69z^{15}\cr
&\quad\qquad+69z^{16}+67z^{17}+63z^{18}+57z^{19}+52z^{20}
 +45z^{21}+39z^{22}+31z^{23}\cr
&\quad\qquad+24z^{24}+18z^{25}+13z^{26}+9z^{27}
 +6z^{28}+4z^{29}+2z^{30}+z^{31}\,.
\enddisplay
Finally, since $1/(1-z^{10})^5=\sum_{k\ge0}{k+4\choose4}z^{10k}$, we can determine
the coefficient ${\check C}_n=[z^n]\,\check C(z)$ as follows, when $n=10q+r$
and $0\le r<10$:
\begindisplay
{\check C}_{10q+r}&=\sum_{j,k}\textstyle A_j{k+4\choose4}\[10q+r=10k+j]\cr
&=\textstyle A_r{q+4\choose4}+A_{r+10}{q+3\choose4}
 +A_{r+20}{q+\2\choose4}+A_{r+30}{q+1\choose4}\,.
\eqno\eqref|change-closed|
\enddisplay
This gives ten cases, one for each value of $r$; but it's a pretty good closed
form, compared with alternatives that involve powers of complex numbers.

For example, we can use this expression to deduce the value of $C_{50q}=
"!change, large amounts"
{\check C}_{10q}$. Then $r=0$ and we have
\begindisplay
C_{50q}={q{+}4\choose4}+45{q{+}3\choose4}+52{q{+}2\choose4}+2{q{+}1\choose4}\,.
\enddisplay
The number of ways to change 50\rlap/c is ${5\choose4}+45{4\choose4}=50$;
the number of ways to change \$1 is ${6\choose4}+45{5\choose4}+52{4\choose4}
=292$; and the number of ways to change \$1,000,000 is
\begindisplay
&{2000004\choose 4}+45{2000003\choose4}+52{2000002\choose4}+2{2000001\choose4}\cr
\noalign{\medskip}
&\hskip12em=66666793333412666685000001\,.
\enddisplay

\subhead Example 5: A divergent series.

Now let's try to get a closed form for the numbers $g_n$ defined by
\begindisplay
g_0&=1\,;\cr
g_n&=ng_{n-1}\,,\qquad\hbox{for $n>0$}.
\enddisplay
After staring at this for a few nanoseconds we realize that $g_n$ is
\g Nowadays people are talking \undertext{femto}seconds.\g
just $n!$; in fact, the method of summation factors described in
Chapter~2 suggests this answer immediately. But let's try to solve
the recurrence with generating functions, just to see what happens.
(A powerful technique should be able to handle easy recurrences like this,
as well as others that have answers we can't guess so easily.)

The equation
\begindisplay
g_n=ng_{n-1}+\[n=0]
\enddisplay
holds for all $n$, and it leads to
\begindisplay
G(z)=\sum_ng_nz^n=\sum_nng_{n-1}\,z^n+\sum_{n=0}z^n\,.
\enddisplay
{\displaywidowpenalty=-70 % force desirable page break (DEK, May 88)
To complete Step 2, we want to express $\sum_nng_{n-1}\,z^n$ in terms of~$G(z)$,
and the basic maneuvers in Table~|gf-manip| suggest that the derivative
$G'(z)=\sum_nng_nz^{n-1}$ is somehow involved. So we steer toward that
kind of sum:
\begindisplay
G(z)&=1+\sum_n(n+1)g_n\,z^{n+1}\cr
&=1+\sum_nng_n\,z^{n+1}+\sum_ng_n\,z^{n+1}\cr
&=1+z^2G'(z)+zG(z)\,.
\enddisplay
} % end displaywidowpenalty kludge

Let's check this equation, using the values of $g_n$ for small~$n$. Since
\begindisplay
G&=1&+ z&+2z^2&+\phantom06z^3&+24z^4&+\cdots\,,\cr
G'&=&\mathbin{\phantom+}1&+4z&+18z^2&+96z^3&+\cdots\,,\cr
\enddisplay
we have
\begindisplay
z^2G'&=&&\mathbin{\phantom+}z^2&+4z^3&+18z^4&+96z^5&+\cdots\,,\cr
zG&=&z&+z^2&+2z^3&+\phantom06z^4&+24z^5&+\cdots\,,\cr
1&=1\,.\cr
\enddisplay
These three lines add up to~$G$, so we're fine so far. Incidentally, we
often find it convenient to write `$G$' instead of `$G(z)$'; the extra
`$(z)$' just clutters up the formula when we aren't changing $z$.

Step 3 is next, and it's different from what we've done before because we
have a differential equation to solve. But this is a differential equation
that we can handle with the hypergeometric series techniques of
Section~5.6; those techniques aren't too bad. (Readers who are unfamiliar with
hypergeometrics needn't worry\dash---this will be quick.)%
\g\noindent
\llap{``}This will be quick.'' That's what the doctor said just before he
stuck me with that needle. Come to think of it, ``hypergeometric''
sounds a lot like ``hypodermic.\qback''\g

First we must get rid of the constant `$1$', so we take the derivative
of both sides:
\begindisplay
G'=(z^2G'+zG+1)'&=(2zG'+z^2G'')+(G+zG')\cr
&=z^2G''+3zG'+G\,.
\enddisplay
The theory in Chapter 5 tells us
to rewrite this using the $\vartheta$ operator, and we know
from exercise~6.|small-vartheta| that
\begindisplay
\vartheta G=zG'\,,\qquad \vartheta^2G=z^2G''+zG'\,.
\enddisplay
Therefore the desired form of the differential equation is
\begindisplay
\vartheta G=z\vartheta^2G+2z\vartheta G+zG=z(\vartheta+1)^2G\,.
\enddisplay
According to \equ(5.|hyp-alt-diff-eq|), the solution with $g_0=1$ is the
hypergeometric series $F(1,1;;z)$.

Step 3 was more than we bargained for; but now that we know what the
function $G$~is,
Step~4 is easy\dash---the hypergeometric definition \equ(5.|hyp-def|) gives us the
power series expansion:
\begindisplay
G(z)=\hyp{1,1}{}z=\sum_{n\ge0}{1\_^n\,1\_^n\,z^n\over n!}=\sum_{n\ge0}n!\,z^n\,.
\enddisplay
We've confirmed the closed form we knew all along, $g_n=n!$.

Notice that the technique gave the right answer even though $G(z)$ diverges
for all nonzero~$z$. The sequence $n!$ grows so fast, the terms $\vert n!\,z^n
\vert$ approach $\infty$ as $n\to\infty$, unless $z=0$. This shows
that formal power series can be manipulated algebraically
without worrying about convergence.

\subhead Example 6: A recurrence that goes all the way back.

Let's close this section by applying generating functions to a problem in graph
theory. A {\it"fan"\/} of order~$n$ is a graph on the vertices
$\{0,1,\ldots,n\}$ with $2n-1$ edges defined as follows: Vertex~$0$
is connected by an edge to each of the other $n$ vertices,
and vertex~$k$ is connected by an edge
to vertex $k+1$, for $1 \leq k <n$.
Here, for example, is the fan of order~4,
which has five vertices and seven edges.
\begindisplay
\unitlength=13pt
\beginpicture(3,3)(-.5,0)
\put(0,0){\disk{.3}}
\put(2,0){\disk{.3}}
\put(2,1){\disk{.3}}
\put(2,2){\disk{.3}}
\put(2,3){\disk{.3}}
\put(-.5,0){\makebox(0,0){$0$}}
\put(2.5,0){\makebox(0,0){$1$}}
\put(2.5,1){\makebox(0,0){$2$}}
\put(2.5,2){\makebox(0,0){$3$}}
\put(2.5,3){\makebox(0,0){$4$}}
\thicklines
\put(0,0){\line(1,0){2}}
\put(0,0){\line(2,1){2}}
\put(0,0){\line(1,1){2}}
\put(0,0){\line(2,3){2}}
\put(2,0){\line(0,1){3}}
\endpicture
\enddisplay
The problem of interest:
How many spanning "tree"s~$f_n$ are in such a graph?
A~{\it"spanning tree"\/}
is a subgraph containing
all the vertices, and containing enough edges to make the
subgraph connected yet not so many that it has a cycle.
It turns out that every spanning tree
of a "graph" on $n+1$ vertices has exactly $n$~edges.
With fewer than $n$~edges the subgraph wouldn't be connected,
and with more than~$n$ it would have a cycle;
graph theory books prove this.

There are $2n-1 \choose n$ ways to choose
$n$ edges from among the $2n-1$ present in a fan of order~$n$,
but these choices don't always yield a spanning tree.
For instance the subgraph
\begindisplay
\unitlength=13pt
\beginpicture(3,3)(-.5,0)
\put(0,0){\disk{.3}}
\put(2,0){\disk{.3}}
\put(2,1){\disk{.3}}
\put(2,2){\disk{.3}}
\put(2,3){\disk{.3}}
\put(-.5,0){\makebox(0,0){$0$}}
\put(2.5,0){\makebox(0,0){$1$}}
\put(2.5,1){\makebox(0,0){$2$}}
\put(2.5,2){\makebox(0,0){$3$}}
\put(2.5,3){\makebox(0,0){$4$}}
\thicklines
\put(0,0){\line(1,1){2}}
\put(0,0){\line(2,3){2}}
\put(2,0){\line(0,1){1}}
\put(2,2){\line(0,1){1}}
\endpicture
\enddisplay
has four edges but is not a spanning tree; it has a cycle
from $0$ to~$4$ to $3$ to~$0$,
and it has no connection between
$\{1,2\}$ and the other vertices. We want to count
how many of the $2n-1\choose n$ choices actually do yield spanning trees.

Let's look at some small cases.
It's pretty easy to enumerate
the spanning trees for $n = 1$,~$2$, and~$3$:
\begindisplay \openup6pt \advance\abovedisplayskip-13pt
\unitlength=10pt
\def\fan#1#2#3{\hbox{\beginpicture(2,#1)(0,0)
 \put(0,0){\disk{.3}}
 \multiput(2,0)(0,1){#1}{\disk{.3}}
 \let\\=\diags #2 \let\\=\verts #3 \endpicture}}
\def\diags#1{\put(0,0){\ifcase#1 \or\line(1,0)2\or\line(2,1)2\or\line(1,1)2\fi}}
\def\verts#1{\put(2,#1){\line(0,-1)1}}
\vbox{\halign{\hfil#\hfil\cr
\fan1{\\1}{}\cr
$f_1=1$\cr}}
\hskip4em
\vbox{\halign{\hfil#\hfil\cr
\fan2{\\2\\1}{}\cr
\fan2{\\2}{\\1}\qquad
\fan2{\\1}{\\1}\cr
$f_2=3$\cr}}
\hskip4em
\vbox{\halign{\hfil#\hfil\cr
\fan3{\\3\\2\\1}{}\qquad
\fan3{\\3\\2}{\\1}\qquad
\fan3{\\3\\1}{\\1}\qquad
\fan3{\\3\\1}{\\2}\cr
\fan3{\\3}{\\1\\2}\qquad
\fan3{\\2\\1}{\\2}\qquad
\fan3{\\2}{\\1\\2}\qquad
\fan3{\\1}{\\1\\2}\cr
$f_3=8$\cr}}
\enddisplay
(We need not show the labels on the vertices, if we always draw
vertex~$0$ at the left.)
What about the case $n=0$? At first it seems reasonable to set $f_0 = 1$;
but we'll take $f_0 = 0$,
because the existence of a fan of order~$0$
"!null case" "!empty case"
(which should have $2n-1 = -1$ edges) is dubious.

Our four-step procedure tells
us to find a recurrence for~$f_n$ that holds for all~$n$.
We can get a recurrence by observing how the topmost vertex
(vertex~$n$)
is connected to the rest of the spanning tree.
If it's not connected to vertex~$0$, it must be
connected to vertex~$n-1$,
since it must be connected to the rest of the graph.
In this case,
any of the $f_{n-1}$ spanning trees for the remaining fan
(on the vertices 0~through $n-1$) will complete a spanning tree
for the whole graph.
Otherwise vertex~$n$ {\it is\/} connected to~$0$,
and there's some number~$k\le n$
such that vertices $n$, $n-1$, \dots,~$k$ are connected directly
but the edge between $k$ and~$k-1$ is not present. Then
there can't be any edges between $0$ and $\{n-1,\ldots,k\}$, or there
would be a cycle.
If $k=1$, the spanning tree is therefore determined completely.
And if $k>1$, any of the $f_{k-1}$ ways to produce
a spanning tree on $\{0,1,\ldots,k-1\}$ will yield a spanning tree
on the whole graph.
For example, here's what this analysis produces when $n=4$:
\begindisplay
\unitlength=10pt
\vcenter{\halign{\hfil#\hfil\cr
\phantom{$k=4$}\cr
\hbox{\beginpicture(4,4)(-1,0)
 \put(0,0){\disk{.3}}
 \multiput(2,0)(0,1)4{\disk{.3}}
 \put(0,0){\line(2,3)2}
 \put(0,0){\line(1,0)2}
 \put(2,0){\line(0,1)3}
 \multiput(.25,.15)(0,.28284)1{\disk{.1}}
 \multiput(.65,.15)(0,.28284)3{\disk{.1}}
 \multiput(1.05,.15)(0,.28284)5{\disk{.1}}
 \multiput(1.45,.15)(0,.28284)7{\disk{.1}}
 \multiput(1.85,.15)(0,.28284)9{\disk{.1}}
 \multiput(.45,.29242)(0,.28284)1{\disk{.1}}
 \multiput(.85,.29242)(0,.28284)3{\disk{.1}}
 \multiput(1.25,.29242)(0,.28284)5{\disk{.1}}
 \multiput(1.65,.29242)(0,.28284)8{\disk{.1}}
 \endpicture}\cr
\hbox{$f_4$}\cr}}
=
\vcenter{\halign{\hfil#\hfil\cr
\phantom{$k=4$}\cr
\hbox{\beginpicture(4,4)(-1,0)
 \put(0,0){\disk{.3}}
 \multiput(2,0)(0,1)4{\disk{.3}}
 \put(0,0){\line(1,1)2}
 \put(0,0){\line(1,0)2}
 \put(2,0){\line(0,1)2}
 \multiput(.25,.15)(0,.28284)1{\disk{.1}}
 \multiput(.65,.15)(0,.28284)2{\disk{.1}}
 \multiput(1.05,.15)(0,.28284)3{\disk{.1}}
 \multiput(1.45,.15)(0,.28284)5{\disk{.1}}
 \multiput(1.85,.15)(0,.28284)7{\disk{.1}}
 \multiput(.45,.29242)(0,.28284)1{\disk{.1}}
 \multiput(.85,.29242)(0,.28284)2{\disk{.1}}
 \multiput(1.25,.29242)(0,.28284)4{\disk{.1}}
 \multiput(1.65,.29242)(0,.28284)5{\disk{.1}}
 \thicklines \put(2,2){\line(0,1)1}
 \endpicture}\cr
\hbox{$f_3$}\cr}}
+
\vcenter{\halign{\hfil#\hfil\cr
\hbox{$k=4$}\cr
\hbox{\beginpicture(4,4)(-1,0)
 \put(0,0){\disk{.3}}
 \multiput(2,0)(0,1)4{\disk{.3}}
 \put(0,0){\line(1,1)2}
 \put(0,0){\line(1,0)2}
 \put(2,0){\line(0,1)2}
 \multiput(.25,.15)(0,.28284)1{\disk{.1}}
 \multiput(.65,.15)(0,.28284)2{\disk{.1}}
 \multiput(1.05,.15)(0,.28284)3{\disk{.1}}
 \multiput(1.45,.15)(0,.28284)5{\disk{.1}}
 \multiput(1.85,.15)(0,.28284)7{\disk{.1}}
 \multiput(.45,.29242)(0,.28284)1{\disk{.1}}
 \multiput(.85,.29242)(0,.28284)2{\disk{.1}}
 \multiput(1.25,.29242)(0,.28284)4{\disk{.1}}
 \multiput(1.65,.29242)(0,.28284)5{\disk{.1}}
 \thicklines \put(0,0){\line(2,3)2}
 \endpicture}\cr
\hbox{$f_3$}\cr}}
+
\vcenter{\halign{\hfil#\hfil\cr
\hbox{$k=3$}\cr
\hbox{\beginpicture(4,4)(-1,0)
 \put(0,0){\disk{.3}}
 \multiput(2,0)(0,1)4{\disk{.3}}
 \put(0,0){\line(2,1)2}
 \put(0,0){\line(1,0)2}
 \put(2,0){\line(0,1)1}
 \multiput(.6,.1)(0,.28284)1{\disk{.1}}
 \multiput(1,.1)(0,.28284)2{\disk{.1}}
 \multiput(1.4,.1)(0,.28284)3{\disk{.1}}
 \multiput(1.8,.1)(0,.28284)4{\disk{.1}}
 \multiput(.8,.24242)(0,.28284)1{\disk{.1}}
 \multiput(1.2,.24242)(0,.28284)2{\disk{.1}}
 \multiput(1.6,.24242)(0,.28284)2{\disk{.1}}
 \thicklines \put(0,0){\line(2,3)2} \put(2,2){\line(0,1)1}
 \endpicture}\cr
\hbox{$f_2$}\cr}}
+
\vcenter{\halign{\hfil#\hfil\cr
\hbox{$k=2$}\cr
\hbox{\beginpicture(4,4)(-1,0)
 \put(0,0){\disk{.3}}
 \multiput(2,0)(0,1)4{\disk{.3}}
 \put(0,0){\line(1,0)2}
 \thicklines \put(0,0){\line(2,3)2} \put(2,1){\line(0,1)2} \put(0,0){\line(1,0)2}
 \endpicture}\cr
\hbox{$f_1$}\cr}}
+
\vcenter{\halign{\hfil#\hfil\cr
\hbox{$k=1$}\cr
\hbox{\beginpicture(4,4)(-1,0)
 \put(0,0){\disk{.3}}
 \multiput(2,0)(0,1)4{\disk{.3}}
 \thicklines \put(0,0){\line(2,3)2} \put(2,0){\line(0,1)3}
 \endpicture}\cr
\hbox{$1$}\cr}}
\enddisplay
The general equation, valid for $n \geq 1$, is
\begindisplay
 f_n
	= f_{n-1} + f_{n-1}+ f_{n-2} + f_{n-3} + \cdots + f_1 + 1\,.
\enddisplay
(It almost seems as though the `$1$' on the end is $f_0$
and we should have chosen $f_0 = 1$;
but we will doggedly stick with our choice.) A few changes suffice
to make the equation valid for all integers~$n$:
\begindisplay
f_n=f_{n-1}\;+\;\sum_{k<n}f_k\;+\;\[n>0]\,.
\eqno\eqref|fan-rec|
\enddisplay
This is a recurrence that ``goes all the way back'' from $f_{n-1}$ through
all previous values, so it's different from the other recurrences
we've seen so far in this chapter. We used a special method to get
rid of a similar right-side sum in Chapter~2, when we solved the
quicksort recurrence \equ(2.|quicksort-rec|);
namely, we subtracted one instance of the recurrence from another
($f_{n+1}-f_n$). This trick would get rid of the $\sum$ now, as it
did then; but we'll see that
generating functions allow us to work directly with such sums.
(And it's a good thing that they do, because we will be seeing
much more complicated recurrences before long.)

Step 1 is finished; Step 2 is where we need to do a new thing:
\begindisplay \openup3pt
F(z)=\sum_n{f_nz^n}&=\sum_nf_{n-1}z^n+\sum_{k,n}f_kz^n\[k<n]+\sum_n\[n>0]z^n\cr
&=zF(z)\;+\;\sum_kf_kz^k\sum_n\[n>k]z^{n-k}\;+\;{z\over1-z}\cr
&=zF(z)\;+\;F(z)\sum_{m>0}z^m\;+\;{z\over1-z}\cr
&=zF(z)\;+\;F(z)\,{z\over1-z}\;+\;{z\over1-z}\,.\cr
\enddisplay
The key trick here was to change $z^n$
to $z^k\,z^{n-k}$; this made it possible to express the value of the double
sum in terms of $F(z)$, as required in Step~2.

Now Step~3 is simple algebra, and we find
\begindisplay
 F(z)
	= {z\over 1 - 3z + z^2} \,.
\enddisplay
Those of us with a zest for memorization will
recognize this as the generating function~\eq(|fib-gf-even|)
for the even-numbered Fibonacci numbers. So, we needn't go through Step~4;
we have found a somewhat surprising answer to the spans-of-fans problem:
\begindisplay
 f_n
	= F_{2n} \,, \qquad\hbox{for $n \geq 0$.}
\eqno
\enddisplay

\beginsection 7.4 Special Generating Functions

Step 4 of the four-step procedure becomes much easier if we know the
coefficients of lots of different power series. The expansions in
Table~|gf-simp| are quite useful, as far as they go, but many
other types of closed forms are possible. Therefore we ought to supplement
that table with another one, which lists power series that correspond
to the ``special numbers'' considered in Chapter~6.

\pageinsert
\table Generating functions for special numbers.\tabref|gf-special|
"!Stirling numbers, generating functions for"
"!Eulerian numbers, generating function for"
\begindisplay\abovedisplayskip=-2pt \belowdisplayskip=6pt \openup5.6pt%
 \advance\displayindent-\parindent\advance\displaywidth\parindent
%{1\over(1-z)}\ln{1\over1-z}
% &=\sum_{n\ge0} H_n\,z^n
% \eqno\eqref|harmonic-gf|\cr
{1\over(1-z)^{m+1}}\ln{1\over1-z}
 &=\sum_{n\ge0}(H_{m+n}-H_m){m{+}n\choose n}z^n
 \eqno\eqref|harmonic-general|\cr
{z\over e^z-1}
 &=\sum_{n\ge0} B_n{z^n\over n!}
 \eqno\eqref|bernoulli-gf|\cr
{F_mz\over1-(F_{m-1}{+}F_{m+1})z+(-1)^mz^2}
 &=\sum_{n\ge0} F_{mn}\,z^n
 \eqno\eqref|fib-general|\cr
\sum_k{m\brace k}{k!\,z^k\over(1-z)^{k+1}}
 &=\sum_{n\ge0} n^mz^n
 \eqno\eqref|mth-power-gf|\cr
\bigl(z^{-1}\bigr)\_^{-m}={z^m\over(1-z)(1-2z)\ldots(1-mz)}
\hskip-3em&\hskip3em=\sum_{n\ge0}{n\brace m}z^n
 \eqno\eqref|stirl2-gf|\cr
z\_^m=z(z+1)\ldots(z+m-1)
\hskip-3em&\hskip3em=\sum_{n\ge0}{m\brack n}z^n
 \eqno\eqref|stirl1-gf|\cr
\bigl(e^z-1\bigr)^m
 &=m!\sum_{n\ge0}{n\brace m}{z^n\over n!}
 \eqno\eqref|exp-power-gf|\cr
\Bigl(\ln{1\over1-z}\Bigr)^{\!m}
 &=m!\sum_{n\ge0}{n\brack m}{z^n\over n!}
 \eqno\eqref|log-power-gf|\cr
\biggl({z\over\ln(1+z)}\biggr)^{\!m}
 &=\sum_{n\ge0}{z^n\over n!}{m\brace m{-}n}\bigg/{m{-}1\choose n}
 \eqno\eqref|dual-log-power-gf|\cr
\biggl({z\over1-e^{-z}}\biggr)^{\!m}
 &=\sum_{n\ge0}{z^n\over n!}{m\brack m{-}n}\bigg/{m{-}1\choose n}
 \eqno\eqref|dual-exp-power-gf|\cr
\noalign{\smallskip}
e^{z+wz}
 &=\sum_{m,n\ge0}{n\choose m}w^m{z^n\over n!}
 \eqno\eqref|bc-double-gf|\cr
e^{w(e^z-1)}
 &=\sum_{m,n\ge0}{n\brace m}w^m{z^n\over n!}
 \eqno\eqref|stirl2-double-gf|\cr
{1\over(1-z)^w}
 &=\sum_{m,n\ge0}\,{n\brack m}\,w^m{z^n\over n!}
 \eqno\eqref|stirl1-double-gf|\cr
{1-w\over e^{(w-1)z}-w}
 &=\sum_{m,n\ge0}{n\euler m}w^m{z^n\over n!}
 \eqno\eqref|euler-double-gf|\cr
\enddisplay
\hrule width\hsize height.5pt
\endinsert

\begingroup % squeezing to preserve the old page layout!
\advance\abovedisplayskip 0pt minus 3pt
\advance\belowdisplayskip 0pt minus 3pt
Table |gf-special| is the database we need. The identities in this
table are not difficult to prove, so we needn't dwell on them; this table
is primarily for reference when we meet a new problem. But there's a
nice proof of the first formula, \eq(|harmonic-general|), that deserves
mention: We start with the identity
\begindisplay
{1\over(1-z)^{x+1}}=\sum_n{x+n\choose n}z^n
\enddisplay
and differentiate it with respect to $x$. On the left, $(1-z)^{-x-1}$ is
equal to $e^{(x+1)\ln(1/(1-z))}$, so $d/dx$ contributes a factor of
$\ln\bigl(1/(1-z)\bigr)$. On the right, the numerator of $x+n\choose n$
is $(x+n)\ldots(x+1)$, and $d/dx$ splits this into $n$ terms whose sum
is equivalent to multiplying $x+n\choose n$ by
\begindisplay
{1\over x+n}+\cdots+{1\over x+1}=H_{x+n}-H_x\,.
\enddisplay
Replacing $x$ by $m$ gives \eq(|harmonic-general|).
Notice that $H_{x+n}-H_x$ is meaningful even when $x$ is not an integer.

By the way,
this method of differentiating a complicated product\dash---%
leaving it as a product\dash---%
is usually better than expressing the derivative as a sum.
For example the right side of
\begindisplay\tightplus
{d\over dx} \bigl( (x+n)^n \ldots (x+1)^1 \bigr)
= (x+n)^n \ldots (x+1)^1
			\left( {n\over x+n} + \cdots + {1\over x+1} \right)
\enddisplay
would be a lot messier written out as a sum.

The general identities in Table |gf-special| include many important
special cases. For example,
\eq(|harmonic-general|) simplifies to
the generating function for~$H_n$ when $m=0$:
\begindisplay
{1\over1-z}\ln{1\over1-z}=\sum_n H_n@z^n\,.
\eqno\eqref|harmonic-gf|
\enddisplay
"!harmonic numbers, generating function"
This equation can also be derived in other ways; for example, we can take the
power series for $\ln\bigl(1/(1-z)\bigr)$ and divide it by~$1-z$
to get cumulative sums.

Identities \eq(|dual-log-power-gf|) and \eq(|dual-exp-power-gf|) involve
the respective ratios ${m\brace m-n}\big/{m-1\choose n}$ and
${m\brack m-n}\big/{m-1\choose n}$, which have the undefined form $0/0$
when $n\ge m$. However, there is a way to give them a proper meaning
using the "Stirling polynomials" of \equ(6.|stirl-poly-def|), because we have
\begindisplay \openup3pt
{m\brace m-n}\bigg/{m-1\choose n}&=(-1)^{n+1}n!\,m@\sigma_n(n-m)\,;\eqno\cr
{m\brack m-n}\bigg/{m-1\choose n}&=n!\,m@\sigma_n(m)\,.\eqno\cr
\enddisplay
Thus, for example, the case $m=1$ of \eq(|dual-log-power-gf|) should not
be regarded as the power series $\sum_{n\ge0}(z^n\!/n!){1\brace1-n}
\big/{0\choose n}$, but rather as
\begindisplay\postdisplaypenalty=-5000
{z\over\ln(1+z)}=-\sum_{n\ge0}(-z)^n\sigma_n(n-1)
=\textstyle 1+\half z-{1\over12}z^2+\cdots\,.
\enddisplay
\endgroup

Identities \eq(|bc-double-gf|),
\eq(|stirl1-double-gf|),
\eq(|stirl2-double-gf|), and \eq(|euler-double-gf|) are ``double generating
functions'' or ``"super generating functions"'' because they have the form
"!generating functions of generating functions"
$G(w,z)=\sum_{m,n}g_{m,n}w^mz^n$. The coefficient of $w^m$ is a generating
function in the variable~$z$; the coefficient of $z^n$ is a generating
function in the variable~$w$. Equation~\eq(|euler-double-gf|) can be
put into the more symmetrical form
\begindisplay
{e^w-e^z\over we^z-ze^w}=\sum_{m,n}{m+n+1\euler m}{w^mz^n\over(m+n+1)!}\,.
\eqno
\enddisplay
\vskip-10pt\null

\beginsection 7.5 Convolutions

The {\it"convolution"\/} of two given sequences
\g I always thought convolution was what happens to my~brain when I
try~to do a proof.\g
$\<f_0,f_1,\ldots\,\>=\<f_n\>$ and\break
$\<g_0,g_1,\ldots\,\>=\<g_n\>$ is the sequence
$\<f_0g_0,\,{f_0g_1+f_1g_0},\,\ldots\,\>=\<\sum_kf_kg_{n-k}\>$.
We have observed in Sections 5.4 and 7.2 that convolution of sequences
corresponds to multiplication of their generating functions. This fact
makes it easy to evaluate many sums that would otherwise be
difficult to handle.

\subhead Example 1: A Fibonacci convolution.

For example, let's try to evaluate $\sum_{k=0}^nF_kF_{n-k}$ in closed form.
This is the convolution of $\<F_n\>$ with itself, so the sum must
be the coefficient of~$z^n$ in $F(z)^2$, where $F(z)$ is the
generating function for $\<F_n\>$.
All we have to do is figure out the value of this coefficient.

The generating function $F(z)$ is $z/(1-z-z^2)$, a quotient of polynomials; so the
general expansion theorem for rational functions tells us that the
answer can be obtained from a partial fraction representation. We can
use the general expansion theorem \eq(|gen-exp-thm|)
and grind away; or we can use the fact that
\begindisplay \openup5pt
F(z)^2&=\biggl({1\over\sqrt5\strut}\biggl({1\over1-\phi z}-{1\over1-\phihat z}
\biggr)\biggr)^{\!2}\cr
&={1\over5}\biggl({1\over(1-\phi z)^2}-{2\over(1-\phi z)(1-\phihat z)}
 +{1\over(1-\phihat z)^2}\biggr)\cr
\noalign{\smallskip}
&={1\over5}\sum_{n\ge0}(n+1)\phi^nz^n-{2\over5}\sum_{n\ge0}F_{n+1}z^n
 +{1\over5}\sum_{n\ge0}(n+1)\phihat^nz^n\,.
\enddisplay
Instead of expressing the answer in terms of $\phi$ and $\phihat$, let's try
for a closed form in terms of Fibonacci numbers.
Recalling that $\phi+\phihat=1$, we have
\begindisplay\openup 5pt
\phi^n+\phihat^n
&=[z^n]\,\biggl({1\over1-\phi z}+{1\over1-\phihat z}\biggr)\cr
&=[z^n]\,{2-(\phi+\phihat)z\over(1-\phi z)(1-\phihat z)}
 =[z^n]\,{2-z\over1-z-z^2}=2F_{n+1}-F_n\,.
\enddisplay
Hence
\begindisplay
F(z)^2={1\over5}\sum_{n\ge0}\,(n+1)(2F_{n+1}-F_n)z^n
 -{2\over5}\sum_{n\ge0}F_{n+1}\,z^n\,,
\enddisplay
and we have the answer we seek:
\begindisplay
\sum_{k=0}^n F_k F_{n-k}
={2nF_{n+1}-(n+1)F_n\over5}\,.
\eqno\eqref|fib-conv|
\enddisplay
For example, when $n=3$ this formula
gives $F_0F_3+F_1F_2+F_2F_1+F_3F_0=0+1+1+0=2$ on the left and
$(6F_4-4F_3)/5=(18-8)/5=2$ on the right.

\subhead Example 2: Harmonic convolutions.

The efficiency of a certain
computer method called ``"samplesort"'' depends on the value "!sorting"
of the sum
\begindisplay
T_{m,n}=\sum_{0\le k<n}{k\choose m}{1\over n-k}\,,
\qquad\hbox{integers $m,n\ge0$}.
\enddisplay
Exercise 5.|samplesort-recurrence| obtains the value of this sum by a
somewhat intricate double induction, using summation factors. It's much
easier to realize that $T_{m,n}$ is just the $n$th term in the
convolution of $\<{0\choose m},{1\choose m},{\2\choose m},\ldots\,\>$
with $\<0,{1\over1},\half,\ldots\,\>$. Both sequences
have simple generating functions in Table~|gf-simp|:
\begindisplay
\sum_{n\ge0}{n\choose m}z^n={z^m\over(1-z)^{m+1}}\,;
\qquad \sum_{n>0}{z^n\over n}=\ln{1\over1-z}\,.
\enddisplay
Therefore, by \eq(|harmonic-general|),
\begindisplay\openup3pt
T_{m,n}=[z^n]\,{z^m\over(1-z)^{m+1}}\ln{1\over1-z}
&=[z^{n-m}]\,{1\over(1-z)^{m+1}}\ln{1\over1-z}\cr
&=(H_n-H_m){n\choose n-m}\,.
\enddisplay

In fact, there are many more sums that boil down to
this same sort of convolution, because we have
\begindisplay
{1\over(1-z)^{r+1}}\ln{1\over1-z}\,\cdot\,{1\over(1-z)^{s+1}}=
{1\over(1-z)^{r+s+2}}\ln{1\over1-z}
\enddisplay
for all $r$ and $s$. Equating coefficients of $z^n$ % on the left and right
gives the general identity
\begindisplay \postdisplaypenalty=-100
&\sum_k{r+k\choose k}{s+n-k\choose n-k}(H_{r+k}-H_r)\cr
&\qquad\qquad={r+s+n+1\choose n}(H_{r+s+n+1}-H_{r+s+1})\,.
\eqno\eqref|rshh-conv|
\enddisplay
This seems almost too good to be true.
\g Because it's so harmonic.\g
 But it checks, at least when $n=2$:
\begindisplay \openup3pt
&{r+1\choose1}{s+1\choose1}{1\over r+1}+
{r+2\choose2}{s+0\choose0}\biggl({1\over r+2}+{1\over r+1}\biggr)\cr
&\hskip9em={r+s+3\choose2}\biggl({1\over r+s+3}+{1\over r+s+2}\biggr)\,.
\enddisplay
Special cases like $s=0$ are as remarkable as the general case.

And there's more.
We can use the convolution identity
\begindisplay
\sum_k{r+k\choose k}{s+n-k\choose n-k}={r+s+n+1\choose n}
\enddisplay
to transpose $H_r$ to the other side, since
 $H_r$ is independent of $k$:
\begindisplay
&\sum_k{r+k\choose k}{s+n-k\choose n-k}H_{r+k}\cr
&\qquad\qquad={r+s+n+1\choose n}(H_{r+s+n+1}-H_{r+s+1}+H_r)\,.
\eqno\eqref|rsh-conv|
\enddisplay
There's still more: If $r$ and $s$ are nonnegative integers $l$ and~$m$,
we can replace $r+k\choose k$ by $l+k\choose l$ and $s+n-k\choose n-k$
by $m+n-k\choose m$; then we can change $k$ to~$k-l$ and $n$~to $n-m-l$,
getting
\begindisplay
\sum_{k=0}^n{k\choose l}{n-k\choose m}H_k&=
{n+1\choose l+m+1}(H_{n+1}-H_{l+m+1}+H_l)\,,\cr
&\hskip7em\hbox{integers $l,m,n\ge0$}.
\eqno\eqref|lmnh-conv|\cr
\enddisplay
Even the special case $l=m=0$ of this identity was difficult for us to handle
in Chapter~2! $\bigl($See \equ(2.|harm-sum|).$\bigr)$ We've
come a long way.

\subhead Example 3: Convolutions of convolutions.

If we form the convolution of $\<f_n\>$ and $\<g_n\>$,
then convolve this with a third sequence $\<h_n\>$, we get a
sequence whose $n$th term is
\begindisplay
\sum_{j+k+l=n}f_j\,g_k\,h_l\,.
\enddisplay
The generating function of this three-fold convolution
is, of course, the three-fold product $F(z)@G(z)@H(z)$. In a similar
way, the $m$-fold
convolution of a sequence $\<g_n\>$ with itself has $n$th term equal to
\begindisplay \postdisplaypenalty=10000
\sum_{k_1+k_2+\cdots+k_m=n}g_{k_1}\,g_{k_2}\,\ldots \,g_{k_m}
\enddisplay
and its generating function is $G(z)^m$.

We can apply these observations to the spans-of-fans problem considered
"!spanning trees"
earlier (Example~6 in Section~7.3). It turns
out that there's another way to compute $f_n$,
the number of spanning trees of an $n$-fan, based on the configurations of
tree edges between the vertices $\{1,2,\ldots,n\}$: The edge between vertex~$k$
and vertex~$k+1$ may or may not be selected for the tree; and each of the
ways to select these edges connects up certain blocks
\g Concrete blocks.\g
 of adjacent vertices.
For example, when $n=10$ we might connect vertices $\{1,2\}$, $\{3\}$,
$\{4,5,6,7\}$, and $\{8,9,10\}$:
\begindisplay
\unitlength=12pt
\beginpicture(4,9)(-1,0)
\put(0,0){\disk{.3}}
\multiput(2,0)(0,1){10}{\disk{.3}}
\put(2,0){\line(0,1)1}
\put(2,3){\line(0,1)3}
\put(2,7){\line(0,1)2}
\put(-.5,0){\makebox(0,0){$0$}}
\put(2.5,0){\makebox(0,0){$1$}}
\put(2.5,1){\makebox(0,0){$2$}}
\put(2.5,2){\makebox(0,0){$3$}}
\put(2.5,3){\makebox(0,0){$4$}}
\put(2.5,4){\makebox(0,0){$5$}}
\put(2.5,5){\makebox(0,0){$6$}}
\put(2.5,6){\makebox(0,0){$7$}}
\put(2.5,7){\makebox(0,0){$8$}}
\put(2.5,8){\makebox(0,0){$9$}}
\put(2.6,9){\makebox(0,0){$10$}}
\endpicture
\enddisplay
How many spanning trees can we make, by adding additional edges to vertex~$0$?
We need to connect $0$ to each of the four blocks; and there are two
ways to join $0$ with $\{1,2\}$, one way to join it with $\{3\}$, four
ways with $\{4,5,6,7\}$, and three ways with $\{8,9,10\}$, or
$2\cdt1\cdt4\cdt3=24$ ways altogether. Summing over all possible ways to
make blocks gives us the following expression for the total number of
spanning trees:
\begindisplay
f_n=\sum_{m>0}\sum\twoconditions{k_1+k_2+\cdots+k_m=n}{k_1,k_2,\ldots,k_m>0}
 k_1k_2\ldots k_m\,.
\eqno
\enddisplay
For example, $f_4=4+3\cdt1+2\cdt2+1\cdt3+2\cdt1\cdt1+1\cdt2\cdt1+1\cdt1\cdt2
+1\cdt1\cdt1\cdt1=21$.

This is the sum of $m$-fold convolutions of the sequence $\<0,1,2,3,\ldots\,
\>$, for $m=1$,~$2$, $3$,~\dots; hence the generating function for
$\<f_n\>$ is
\begindisplay
F(z)=G(z)+G(z)^2+G(z)^3+\cdots\,={G(z)\over1-G(z)}
\enddisplay
where $G(z)$ is the generating function for $\<0,1,2,3,\ldots\,\>$,
namely $z/(1-z)^2$. Consequently we have
\begindisplay
F(z)={z\over(1-z)^2-z}={z\over1-3z+z^2}\,,
\enddisplay
as before. This approach to $\<f_n\>$ is more symmetrical and
appealing than the complicated recurrence we had earlier.

\subhead Example 4: A convoluted recurrence.

Our next example is especially important. In fact, it's the ``classic example''
of why generating functions are useful in the solution of recurrences.

Suppose we have $n+1$ variables $x_0$, $x_1$, \dots, $x_n$ whose product
is to be computed by doing $n$~multiplications. How many ways $C_n$ are there
to insert parentheses into the product $x_0\cdot x_1\cdot\ldots\cdot x_n$
so that the order of multiplication is completely specified? For example,
when $n=2$ there are two ways, $x_0\cdot(x_1\cdt x_2)$ and
$(x_0\cdt x_1)\cdot x_2$. And when $n=3$ there are five ways,
\begindisplay \openup6pt
&x_0\cdt(x_1\cdt(x_2\cdt x_3))\,,\;
x_0\cdt((x_1\cdt x_2)\cdt x_3)\,,\;
(x_0\cdt x_1)\cdt(x_2\cdt x_3)\,,\cr
&\hskip12em(x_0\cdt(x_1\cdt x_2))\cdt x_3\,,\;
((x_0\cdt x_1)\cdt x_2)\cdt x_3\,.
\enddisplay
Thus $C_2=2$, $C_3=5$; we also have $C_1=1$ and $C_0=1$.

Let's use the four-step procedure of Section 7.3. What is a recurrence for
the $C$'s? The key observation is that there's exactly one `$\,\cdot\,$' operation
outside all of the parentheses, when $n>0$; this is the final multiplication
that ties everything together. If this `$\,\cdot\,$' occurs between $x_k$ and~$x_{k+1}$,
there are $C_k$ ways to fully parenthesize $x_0\cdt\ldots\cdt x_k$, and there
are $C_{n-k-1}$ ways to fully parenthesize $x_{k+1}\cdt\ldots\cdt x_n$; hence
\begindisplay
C_n=C_0C_{n-1}+C_1C_{n-2}+\cdots+C_{n-1}C_0\,,\qquad\hbox{if $n>0$}.
\enddisplay
By now we recognize this expression as a convolution, and we know how to patch the
formula so that it holds for all integers $n$:
\begindisplay
C_n=\sum_k C_kC_{n-1-k}\,+\,\[n=0]\,.
\eqno\eqref|cat-rec|
\enddisplay
Step 1 is now complete. Step 2 tells us to multiply by $z^n$ and sum:
\begindisplay \openup4pt
C(z)&=\sum_n C_nz^n\cr
&=\sum_{k,n}C_kC_{n-1-k}z^n+\sum_{n=0}z^n\cr
\noalign{\smallskip}
&=\sum_k C_kz^k\sum_nC_{n-1-k}z^{n-k}\,+\,1\cr
&=C(z)\cdot zC(z)+1\,.
\enddisplay
Lo and behold, the convolution has become a product, in the generating-function
\g The authors jest.\g
world. Life is full of surprises.

Step 3 is also easy. We solve for $C(z)$ by the quadratic formula:
\begindisplay
C(z)={1\pm\sqrt{\,\mathstrut1-4z}\over2z}\,.
\enddisplay
But should we choose the $+$ sign or the $-$ sign? Both choices yield a
function that satisfies $C(z)=zC(z)^2+1$, but only one of the choices is
suitable for our problem. We might choose the $+$~sign on the grounds that
positive thinking is best; but we soon discover that this
choice gives $C(0)=\infty$,
contrary to the facts. (The correct function $C(z)$ is supposed to
have $C(0)=C_0=1$.)
 Therefore we conclude that
\begindisplay
C(z)={1-\sqrt{\,\mathstrut1-4z}\over2z}\,.
\enddisplay

Finally, Step 4. What is $[z^n]\,C(z)$? The binomial theorem tells us that
\begindisplay
\sqrt{\,1-4z}=\sum_{k\ge0}{1/2\choose k}(-4z)^k
=1+\sum_{k\ge1}{1\over2k}{-1/2\choose k-1}(-4z)^k\,;
\enddisplay
hence, using \equ(5.|minus-half-bc|),
\begindisplay
{1-\sqrt{\,\mathstrut1-4z}\over2z}
&=\sum_{k\ge1}{1\over k}{-1/2\choose k-1}(-4z)^{k-1}\cr
&=\sum_{n\ge0}{-1/2\choose n}{(-4z)^n\over n+1}
 =\sum_{n\ge0}{2n\choose n}{z^n\over n+1}\,.
\enddisplay
The number of ways to parenthesize, $C_n$, is ${2n\choose n}{1\over n+1}$.

We anticipated this result in Chapter 5, when we introduced the
\g So the convoluted recurrence has led us to an oft-recurring convolution.\g
sequence of {\it "Catalan numbers"\/} $\<1,1,2,5,14,\ldots\,\>=
\<C_n\>$. This sequence arises in dozens of problems that
seem at first to be unrelated to each other~[|catalan-history|], because
many situations have a recursive structure that corresponds to the
convolution recurrence~\eq(|cat-rec|).

For example, let's consider the following problem: How many sequences
$\<a_1,a_2\ldots,a_{2n}\>$ of $+1$'s and $-1$'s have the property that
\begindisplay
a_1+a_2+\cdots+a_{2n}=0
\enddisplay
and have all their partial sums
\begindisplay
a_1,\quad a_1+a_2,\quad \ldots,\quad a_1+a_2+\cdots+a_{2n}
\enddisplay
nonnegative? There must be $n$ occurrences of $+1$ and $n$ occurrences of~$-1$.
We can represent this problem graphically by plotting the sequence of partial sums
$s_n=\sum_{k=1}^na_k$ as a function of~$n$: The five solutions for $n=3$ are
\countdef\m=4 \countdef\n=6
\def\init{\thinlines\put(0,0){\line(1,0)6}
 \thicklines \m=0 \n=0 \put(0,0){\disk{.2}}}
\def\+{\put(\number\m,\number\n){\line(1,1)1}\advance\m1 \advance\n1
 \put(\number\m,\number\n){\disk{.2}}}
{\def\-{\put(\number\m,\number\n){\line(1,-1)1}\advance\m1 \advance\n-1
 \put(\number\m,\number\n){\disk{.2}}}
\begindisplay \unitlength=10pt
&\beginpicture(6,3)(0,0)\init\+\+\+\-\-\-\endpicture\;,\quad
\beginpicture(6,3)(0,0)\init\+\+\-\+\-\-\endpicture\;,\quad
\beginpicture(6,3)(0,0)\init\+\-\+\+\-\-\endpicture\;,\cr
&\hskip12em\beginpicture(6,3)(0,0)\init\+\+\-\-\+\-\endpicture\;,\quad
\beginpicture(6,3)(0,0)\init\+\-\+\-\+\-\endpicture\;.\cr
\enddisplay
These are ``"mountain ranges"'' of width $2n$ that can be drawn with line segments
of the forms
\begingroup
\unitlength=10pt
\def\init{\thicklines \m=0 \n=0 \put(0,0){\disk{.2}}}%
\beginpicture(1,1)(0,.2)\init\+\endpicture\ and
\beginpicture(1,1)(0,-.8)\init\-\endpicture\thinspace.\endgroup}
It turns out that there are exactly $C_n$ ways to do this, and the sequences
can be related to the parenthesis problem in the following way: Put an
extra pair of "parentheses" around the entire formula, so that there are
$n$~pairs of parentheses corresponding to the $n$~multiplications. Now
replace each~`$\,\cdot\,$'
by~$+1$ and each `$@)@$' by~$-1$
 and erase everything else. For example, the formula
$x_0\cdt\bigl((x_1\cdt x_2)\cdt(x_3\cdt x_4)\bigr)$ corresponds to the
sequence $\<+1,+1,-1,+1,+1,-1,-1,-1\>$ by this rule. The five ways to
parenthesize $x_0\cdt x_1\cdt x_2\cdt x_3$ correspond to the five
mountain ranges for $n=3$ shown above.

Moreover, a slight reformulation of our sequence-counting
problem leads to a
surprisingly simple combinatorial solution that avoids the use of
generating functions: How many sequences $\<a_0,a_1,a_2,\ldots,a_{2n}\>$ of
$+1$'s and $-1$'s have the property that
\begindisplay
a_0+a_1+a_2+\cdots+a_{2n}=1\,,
\enddisplay
when all the partial sums
\begindisplay
a_0,\quad a_0+a_1,\quad a_0+a_1+a_2,\quad\ldots,\quad a_0+a_1+\cdots+a_{2n}
\enddisplay
are required to be {\it positive\/}? Clearly these are just the sequences of the
previous problem, with the additional element $a_0=+1$ placed in front.
But the sequences in the new problem can be enumerated by a simple
counting argument, using a remarkable fact discovered by
George "Raney"~[|raney|] in 1959:
{\sl If $\<x_1,x_2,\ldots,x_m\>$ is any sequence of integers whose sum is $+1$,
exactly one of the cyclic shifts
\begindisplay
\<x_1,x_2,\ldots,x_m\>,\;
\<x_2,\ldots,x_m,x_1\>,\; \ldots,\;
\<x_m,x_1,\ldots,x_{m-1}\>
\enddisplay
has all of its partial sums positive}. \ For example, consider the sequence
$\<3,-5,2,-2,3,0\>$. Its cyclic shifts are
\begindisplay \def\preamble{&$##$\hfill\qquad\qquad} \postdisplaypenalty=10000
\<3,-5,2,-2,3,0\>&\<-2,3,0,3,-5,2\>\cr
\<-5,2,-2,3,0,3\>&\<3,0,3,-5,2,-2\>\,\,{\scriptstyle\sqrt{\vphantom2}}\cr
\<2,-2,3,0,3,-5\>&\<0,3,-5,2,-2,3\>\cr
%&\<3,-5,2,-2,3,0\>\cr
%&\<-5,2,-2,3,0,3\>\cr
%&\<2,-2,3,0,3,-5\>\cr
%&\<-2,3,0,3,-5,2\>\cr
%&\<3,0,3,-5,2,-2\>\,\,{\scriptstyle\sqrt{\vphantom2}}\cr
%&\<0,3,-5,2,-2,3\>\cr
\enddisplay
and only the one that's checked has entirely positive partial sums.

Raney's lemma can be proved by a simple geometric argument. Let's
extend the sequence periodically to get an infinite sequence
\begindisplay
\<x_1,x_2,\ldots,x_m,x_1,x_2,\ldots,x_m,x_1,x_2,\ldots\,\>\,;
\enddisplay
thus we let $x_{m+k}=x_k$ for all $k\ge0$.
If we now plot the partial sums $s_n=x_1+\cdots+x_n$ as a function of~$n$,
the graph of~$s_n$ has an ``average slope'' of $1/m$, because $s_{m+n}=
s_n+1$. For example, the graph corresponding to our example sequence
$\<3,-5,2,-2,3,0,3,-5,2,\ldots\,\>$ begins as follows:
\begindisplay \advance\belowdisplayskip-13pt
\unitlength=10pt
\countdef\m=4 \countdef\n=6
\def\init{\thicklines \m=0 \n=0 \put(0,0){\disk{.2}}}
\def\+#1,{\put(\number\m,\number\n){\line(1,#1)1}\advance\m1 \advance\n#1
 \put(\number\m,\number\n){\disk{.2}}}
\beginpicture(20,12)(0,-4)
\init\+3,\+-5,\+2,\+-2,\+3,\+0,
\+3,\+-5,\+2,\+-2,\+3,\+0,
\+3,\+-5,\+2,\+-2,\+3,\+0,
\+3,\+-5,\+2,\+-2,\+3,\+0,
\advance\m 2
\setbox0=\hbox{$.\,$}%
\put(\number\m,\number\n){\copy0\raise.16667\wd0\copy0\raise.33333\wd0\copy0}
\thinlines
\put(0,0){\line(1,0){30}}
\put(4,-2){\line(6,1){26}}
\put(4,-2){\line(-6,-1){4}}
\put(1,3){\line(6,1){29}}
\put(1,3){\line(-6,-1){1}}
\endpicture
\enddisplay
\g Ah, if stock prices would only continue to rise like this.\g
The entire graph can be contained between two lines of slope $1/m$, as shown;
we have $m=6$ in the illustration.
In general these bounding lines touch the graph just
once in each cycle of $m$~points, since lines of
slope~$1/m$ hit points with integer coordinates only once per $m$~units.
The unique lower point of intersection is the only place in the cycle from
which all partial sums will be positive, because every other point on the
curve has an intersection point within $m$ units to its right.

With Raney's lemma
\g (Attention, computer scientists: The partial sums in this problem represent the
"stack size" as a function of time, when a product of $n+1$ factors is
evaluated, because each ``push'' operation changes the size by $+1$
and each multiplication changes it by $-1$.)\g
we can easily enumerate the sequences $\<a_0,\ldots,a_{2n}\>$
of $+1$'s and $-1$'s whose partial sums are entirely positive and whose
total sum is~$+1$. There are $2n+1\choose n$ sequences with $n$ occurrences
of~$-1$ and $n+1$ occurrences of~$+1$, and Raney's lemma tells us that
exactly $1/(2n+1)$ of these sequences have all partial sums positive.
(List all $N={2n+1\choose n}$ of these sequences and all $2n+1$ of their
cyclic shifts, in an $N\times(2n+1)$ array. Each row contains exactly one
solution. Each solution appears exactly once in each column. So there are
$N/(2n+1)$ distinct solutions in the array, each appearing $(2n+1)$ times.)
The total number of sequences with positive partial sums is
\begindisplay
{2n+1\choose n}{1\over2n+1}={2n\choose n}{1\over n+1}=C_n\,.
\enddisplay

\subhead Example 5: A recurrence with m-fold convolution.

We can generalize the problem just considered by looking at sequences
$\<a_0,\ldots,a_{mn}\>$ of $+1$'s and $(1-m)$'s whose partial sums
are all positive and whose total sum is~$+1$. Such sequences can be
called {\it $m$-"Raney sequences"}.
If there are $k$ occurrences of $(1-m)$ and $mn+1-k$ occurrences of~$+1$,
we have
\begindisplay
k(1-m)+(mn+1-k)=1\,,
\enddisplay
\g (Attention, computer scientists:
 The stack interpretation now applies with respect to
an $m$-ary operation, instead of the binary multiplication considered earlier.)\g
hence $k=n$. There are $mn+1\choose n$ sequences with $n$ occurrences of
$(1-m)$ and $mn+1-n$ occurrences of~$+1$, and Raney's lemma tells us that
the number of such sequences with all partial sums positive is exactly
\begindisplay
{mn+1\choose n}{1\over mn+1}={mn\choose n}{1\over(m-1)n+1}\,.
\eqno
\enddisplay
So this is the number of $m$-Raney sequences. Let's call this a
"Fuss-Catalan number" $C_n^{(m)}$, because the sequence $\<C_n^{(m)}\>$
was first investigated
by N.\thinspace I. "Fuss"~[|fuss|]
in 1791 (many years before "Catalan" himself got into
the act). The ordinary Catalan numbers are $C_n=C_n^{(2)}$.

Now that we know the answer, \thiseq, let's play ``"Jeopardy"'' and figure
out a question that leads to it. In the case $m=2$ the question was:
``What numbers $C_n$ satisfy the recurrence $C_n=\sum_k C_kC_{n-1-k}+
\[n=0]$?'' We will try to find a similar question (a similar recurrence) in the
general case.

The trivial sequence $\<+1\>$ of length~1 is clearly an $m$-Raney
sequence. If we put the number $(1-m)$ at the right of any $m$ sequences
that are $m$-Raney, we get an $m$-Raney sequence; the partial sums
stay positive as they increase to $+2$, then $+3$, \dots,~$+m$, and~$+1$.
Conversely, we can show that all $m$-Raney sequences $\<a_0,\ldots,a_{mn}\>$
arise in this way, if $n>0$: The last term $a_{mn}$ must be $(1-m)$.
The partial sums $s_j=a_0+\cdots+a_{j-1}$ are positive for $1\le j\le mn$, and
$s_{mn}=m$ because $s_{mn}+a_{mn}=1$. Let $k_1$ be the largest index $\le mn$
such that $s_{k_1}=1$; let $k_2$ be largest such that $s_{k_2}=2$; and so on.
Thus $s_{k_j}=j$ and $s_k>j$, for $k_j<k\le mn$ and $1\le j\le m$. It follows
that $k_m=mn$, and we can verify without difficulty that
each of the subsequences $\<a_0,\ldots,a_{k_1-1}\>$, $\<a_{k_1},\ldots,
a_{k_2-1}\>$, \dots, $\<a_{k_{m-1}},\ldots,a_{k_m-1}\>$ is an $m$-Raney
sequence. We must have $k_1=mn_1+1$, \ $k_2-k_1=mn_2+1$, \ \dots, \
$k_m-k_{m-1}= mn_m+1$, for some nonnegative integers $n_1$, $n_2$, \dots,~$n_m$.

Therefore ${mn+1\choose n}{1\over mn+1}$ is the answer to the following
two interesting questions: ``What are the numbers $C_n^{(m)}$ defined by
the recurrence
\begindisplay
C_n^{(m)}=\Biggl(@\sum_{n_1+n_2+\cdots+n_m=n-1}\!\!
 C_{n_1}^{(m)}C_{n_2}^{(m)}\ldots C_{n_m}^{(m)}\Biggr)\;+\;\[n=0]
\eqno
\enddisplay
for all integers $n$?'' \ ``If $G(z)$ is a power series that satisfies
\begindisplay
G(z)=z\,G(z)^m\,+\,1\,,
\eqno\eqref|m-conv-def|
\enddisplay
what is $[z^n]\,G(z)$?''

Notice that these are not easy questions.
In the ordinary Catalan case ($m=2$), we solved \thiseq\ for
$G(z)$ and its coefficients by using the quadratic formula and the binomial
theorem; but when $m=3$, none of the standard techniques gives any clue
about how to solve the cubic equation $G=zG^3+1$. So it has turned out to
be easier to answer this question before asking it.

Now, however, we know enough to ask even harder questions and deduce their
answers. How about this one: ``What is $[z^n]\,G(z)^l$, if $l$ is a
positive integer and if $G(z)$ is the power series defined by~\thiseq?''
The argument we just gave can be used to show that $[z^n]\,G(z)^l$
 is the number of
sequences of length~$mn+l$ with the following three properties:
\smallskip
\item{$\bullet$}Each element is either $+1$ or $(1-m)$.
\smallskip
\item{$\bullet$}The partial sums are all positive.
\smallskip
\item{$\bullet$}The total sum is $l@$.
\smallskip
\noindent For we get all such sequences in a unique way by putting together
$l$~sequences that have the $m$-Raney property. The number of ways to
do this is
\begindisplay
\sum_{n_1+n_2+\cdots+n_l=n}C_{n_1}^{(m)}C_{n_2}^{(m)}\ldots C_{n_l}^{(m)}
=[z^n]\,G(z)^l\,.
\enddisplay

"Raney" proved a generalization of his lemma that tells us how to count such
sequences: {\sl If\/ $\<x_1,x_2,\ldots,x_m\>$ is any sequence of integers
with $x_j\le1$ for all\/~$j$, and with $x_1+x_2+\cdots+x_m=l>0$,
then exactly $l$ of the cyclic shifts
\begindisplay
\<x_1,x_2,\ldots,x_m\>,\;
\<x_2,\ldots,x_m,x_1\>,\; \ldots,\;
\<x_m,x_1,\ldots,x_{m-1}\>
\enddisplay
have all positive partial sums}.

 For example, we can
check this statement on the sequence
$\<-2,1,-1,\allowbreak
0,\allowbreak1,\allowbreak1,-1,1,1,1\>$. The cyclic shifts are
\begindisplay \def\preamble{&$##$\hfill\qquad\qquad}\openup2pt
\<-2,1,-1,0,1,1,-1,1,1,1\>&\<1,-1,1,1,1,-2,1,-1,0,1\>\cr
\<1,-1,0,1,1,-1,1,1,1,-2\>&\<-1,1,1,1,-2,1,-1,0,1,1\>\cr
\<-1,0,1,1,-1,1,1,1,-2,1\>&\<1,1,1,-2,1,-1,0,1,1,-1\>\,\,{\scriptstyle\sqrt{\vphantom2}}\cr
\<0,1,1,-1,1,1,1,-2,1,-1\>&\<1,1,-2,1,-1,0,1,1,-1,1\>\cr
\<1,1,-1,1,1,1,-2,1,-1,0\>\,\,{\scriptstyle\sqrt{\vphantom2}}&\<1,-2,1,-1,0,1,1,-1,1,1\>\cr
\enddisplay
and only the two examples marked `${\scriptstyle\sqrt{\vphantom2}\,}$'
 have all partial sums positive.
This generalized lemma is proved in exercise~|prove-raney|.

A sequence of $+1$'s and $(1-m)$'s that has length $mn+l$ and total
sum~$l$ must have exactly $n$ occurrences of $(1-m)$. The generalized
lemma tells us that $l/(mn+l)$ of these $mn+l\choose n$ sequences
have all partial sums positive; hence our tough question has
a surprisingly simple answer:
\begindisplay
[z^n]\,G(z)^l={mn+l\choose n}{l\over mn+l}\,,
\eqno\eqref|m-cat-l|
\enddisplay
for all integers $l>0$.

Readers who haven't forgotten Chapter~5 might well be experiencing
{\it d\'ej\`a vu\/}: ``That formula looks familiar; haven't we
seen it before?'' Yes, indeed; "Lambert"'s
equation \equ(5.|t-series-power|) says that
\begindisplay
[z^n]\,\Bscr_t(z)^r={tn+r\choose n}{r\over tn+r}\,.
\enddisplay
Therefore the generating function $G(z)$ in \eq(|m-conv-def|) must
actually be the generalized binomial series $\Bscr_m(z)$. Sure enough,
equation \equ(5.|t-series-rec|) says
\begindisplay
\Bscr_m(z)^{1-m}-\Bscr_m(z)^{-m}=z\,,
\enddisplay
which is the same as
\begindisplay
\Bscr_m(z)-1=z\Bscr_m(z)^m\,.
\enddisplay

Let's switch to the notation of Chapter 5, now that we know we're
dealing with generalized binomials. Chapter~5 stated a bunch of identities
without proof. We have now closed part of
the gap by proving that the power series
$\Bscr_t(z)$ defined by
\begindisplay
\Bscr_t(z)=\sum_n{tn+1\choose n}{z^n\over tn+1}
\enddisplay
has the remarkable property that
\begindisplay
\Bscr_t(z)^r=\sum_n{tn+r\choose n}{r\,z^n\over tn+r}\,,
\enddisplay
whenever $t$ and $r$ are positive integers.

Can we extend these results to {\it arbitrary\/} values of $t$ and~$r$?
Yes; because the coefficients ${tn+r\choose n}{r\over tn+r}$ are polynomials
in $t$ and~$r$. The general $r$th power defined by
\begindisplay
\Bscr_t(z)^r\!=\!e^{r\ln \Bscr_t(z)}\!=\!\!
\sum_{n\ge0}\!{\bigl(r\ln\Bscr_t(z)\bigr)^{\!n}
\over n!}\!=\!\!\sum_{n\ge0}{r^n\over n!}\biggl(-\!\sum_{m\ge1}
\!{\bigl(1{-}\Bscr_t(z)\bigr)^{\!m}\over m}\biggr)^{\!n}
\enddisplay
has coefficients that are polynomials in $t$ and $r@$;
and those polynomials
are equal to ${tn+r\choose n}{r\over tn+r}$ for infinitely many values
of $t$ and~$r$. So the two sequences of polynomials must be identically equal.

Chapter 5 also mentions the generalized exponential series
\begindisplay
\Escr_t(z)=\sum_{n\ge0}{(tn+1)^{n-1}\over n!}z^n\,,
\enddisplay
which is said in \equ(5.|t-series-power|) to have an equally remarkable property:
\begindisplay
[z^n]\,\Escr_t(z)^r={r(tn+r)^{n-1}\over n!}\,.
\eqno\eqref|escr-power|
\enddisplay
We can prove this as a limiting case of the formulas for $\Bscr_t(z)$,
because it is not difficult to show that
\begindisplay
\Escr_t(z)^r=\lim_{x\to\infty}\Bscr_{xt}(z/x)^{xr}\,.
\enddisplay

\begingroup \def\uppercase#1{\kern-.05em EXPONENTIAL GF'S} % KLUDGE!
\beginsection 7.6 Exponential Generating Functions

\endgroup % END KLUDGE!
Sometimes a sequence $\<g_n\>$ has a generating function whose properties
are quite complicated, while the related sequence $\<g_n/n!@\>$ has
a generating function that's quite simple. In such cases we naturally
prefer to work with $\<g_n/n!@\>$ and then multiply by $n!$ at the end.
This trick works sufficiently often that we have a special name for it:
We call the power series
\begindisplay
\widehat G(z)=\sum_{n\ge0}g_n{z^n\over n!}
\eqno\eqref|egf-def|
\enddisplay
the {\it"exponential generating function"\/} or ``"egf"'' of the
sequence \kern-.26pt$\<g_0,g_1,g_2,\ldots\,\>$.
This name arises because the exponential function~$e^z$ is
the egf of $\<1,1,1,\ldots\,\>$.

Many of the generating functions in Table~|gf-special|
are actually egf's. For example, equation \eq(|log-power-gf|) says that
$\bigl(\ln{1\over1-z}\bigr){}^m\!/m!$ is the egf for the sequence
$\<{0\brack m},{1\brack m},{2\brack m},\ldots\,\>$. The ordinary
generating function for this sequence is much more complicated
(and also divergent).

Exponential generating functions have their own basic maneuvers, analogous
to the operations we learned in Section~7.2. For example, if we multiply
the egf of~$\<g_n\>$ by~$z$, we get
\begindisplay
\sum_{n\ge0}g_n\,{z^{n+1}\over n!}=\sum_{n\ge1}g_{n-1}\,{z^n\over(n-1)!}
=\sum_{n\ge0}ng_{n-1}\,{z^n\over n!}\,;
\enddisplay
this is the egf of $\<0,g_0,2g_1,\ldots\,\>=\<ng_{n-1}\>$.

Differentiating
the egf of $\<g_0,g_1,g_2,\ldots\,\>$ with respect to~$z$ gives
\g Are we having fun~yet?\g
\begindisplay
\sum_{n\ge0}ng_n\,{z^{n-1}\over n!}=\sum_{n\ge1}g_n\,{z^{n-1}\over(n-1)!}
=\sum_{n\ge0}g_{n+1}\,{z^n\over n!}\,;
\eqno
\enddisplay
this is the egf of $\<g_1,g_2,\ldots\,\>$. Thus differentiation
on egf's corresponds to the left-shift operation $\bigl(G(z)-g_0\bigr)/z$
on ordinary gf's. (We used this left-shift property of egf's when
we studied hypergeometric series, \equ(5.|hyp/dz|).) Integration of an egf
gives
\begindisplay
\int_0^z@\sum_{n\ge0}g_n\,{t^n\over n!}\,dt=\sum_{n\ge0}g_n\,{z^{n+1}\over
 (n+1)!}=\sum_{n\ge1}g_{n-1}\,{z^n\over n!}\,;
\eqno
\enddisplay
this is a right shift, the egf of $\<0,g_0,g_1,\ldots\,\>$.

The most interesting operation on egf's, as on ordinary gf's, is
multiplication. If $\widehat F(z)$ and $\widehat G(z)$ are egf's for
$\<f_n\>$ and $\<g_n\>$, then $\widehat F(z)@\widehat G(z)=\widehat H(z)$ is
the egf for a sequence $\<h_n\>$ called the {\it"binomial
convolution"\/} of $\<f_n\>$ and $\<g_n\>$:
"!convolution, binomial"
\begindisplay
h_n=\sum_k{n\choose k}f_k\,g_{n-k}\,.
\eqno
\enddisplay
Binomial coefficients appear here because
${n\choose k}=n!/k!\,(n-k)!$, hence
\begindisplay
{h_n\over n!}=\sum_{k=0}^n {f_k\over k!}\,{g_{n-k}\over(n-k)!}\,;
\enddisplay
in other words, $\<h_n/n!@\>$ is the ordinary convolution of
$\<f_n/n!@\>$ and $\<g_n/n!@\>$.

Binomial convolutions occur frequently in applications. For example,
we defined the Bernoulli numbers in \equ(6.|bern-def|) by the
implicit recurrence
\begindisplay
\sum_{j=0}^m{m+1\choose j}B_j=\[m=0]\,,\qquad\hbox{for all $m\ge0$};
\enddisplay
this can be rewritten as a binomial convolution, if we substitute $n$ for
$m+1$ and add the term $B_n$ to both sides:
\begindisplay
\sum_k{n\choose k}B_k=B_n+\[n=1]\,,\qquad\hbox{for all $n\ge0$}.
\eqno\eqref|bern-def-new|
\enddisplay
We can now relate this recurrence to power series (as promised in Chapter~6) by
introducing the egf for Bernoulli numbers,
$\widehat B(z)=\sum_{n\ge0}B_nz^n\!/n!$. The left-hand side of \thiseq\
is the binomial convolution of $\<B_n\>$ with the constant
sequence $\<1,1,1,\ldots\,\>$; hence the egf of the left-hand side
is $\widehat B(z)e^z$. The egf of the right-hand side is
$\sum_{n\ge0}\bigl(B_n+\[n=1]\bigr)z^n\!/n!=\widehat B(z)+z$. Therefore
we must have $\widehat B(z)=z/(e^z-1)$; we have proved equation
\equ(6.|bern-gf|), which appears also in Table~|gf-special| as
equation \eq(|bernoulli-gf|).

\smallbreak
Now let's look again at a sum that has been popping up
frequently in this book,
\begindisplay
S_m(n)=0^m+1^m+2^m+\cdots+(n-1)^m=\sum_{0\le k<n} k^m\,.
\enddisplay
This time we will try to analyze the problem with generating functions,
in hopes that it will suddenly become simpler.
We will consider $n$ to be
fixed and $m$~variable; thus our goal is to understand the
coefficients of the power series
\begindisplay
S(z)=S_0(n)+S_1(n)\,z+S_2(n)\,z^2+\cdots=\sum_{m\ge0}S_m(n)\,z^m\,.
\enddisplay
We know that the generating function for $\<1,k,k^2,\ldots\,\>$ is
\begindisplay
{1\over1-kz}=\sum_{m\ge0}k^m\,z^m\,,
\enddisplay
hence
\begindisplay
S(z)=\sum_{m\ge0}\,\sum_{0\le k<n}k^m\,z^m=\sum_{0\le k<n}{1\over1-kz}
\enddisplay
by interchanging the order of summation. We can put this sum in
closed form,
\begindisplay \postdisplaypenalty=10000
S(z)&={1\over z}\biggl({1\over z^{-1}-0}+{1\over z^{-1}-1}+\cdots+
 {1\over z^{-1}-n+1}\biggr)\cr
&={1\over z}\bigl(H_{z^{-1}}-H_{z^{-1}-n}\bigr)\,;
\eqno
\enddisplay
but we know nothing about expanding such a closed form in powers of~$z$.

Exponential generating functions come to the rescue. The egf of
our sequence
$\<S_0(n),S_1(n),S_2(n),\ldots\,\>$ is
\begindisplay
\widehat S(z,n)=S_0(n)+S_1(n)\,{z\over1!}+S_2(n)\,{z^2\over2!}+\cdots
=\sum_{m\ge0}S_m(n)\,{z^m\over m!}\,.
\enddisplay
To get these coefficients $S_m(n)$
we can use the egf for $\<1,k,k^2,\ldots\,\>$, namely
\begindisplay
e^{kz}=\sum_{m\ge0}k^m\,{z^m\over m!}\,,
\enddisplay
and we have
\begindisplay
\widehat S(z,n)=\sum_{m\ge0}\,\sum_{0\le k<n}k^m\,{z^m\over m!}=\sum_{0\le k<n}e^{kz}\,.
\enddisplay
And the latter sum is a geometric progression, so there's a closed form
\begindisplay
\widehat S(z,n)={e^{nz}-1\over e^z-1}\,.
\eqno\eqref|szn-closed|
\enddisplay
Eureka!
All we need to do is figure out the coefficients of this relatively simple
function, and we'll know $S_m(n)$, because $S_m(n)=m!\,[z^m]\widehat S(z,n)$.

Here's where "Bernoulli numbers" come into the picture. We observed a
moment ago that the egf for Bernoulli numbers is
\begindisplay
\widehat B(z)=\sum_{k\ge0}B_k\,{z^k\over k!}={z\over e^z-1}\,;
\enddisplay
hence we can write
\begindisplay
\widehat S(z,n)&=\widehat B(z)\,{e^{nz}-1\over z}\cr
&=\Bigl(B_0@{z^0\over0!}+B_1{z^1\over1!}+B_2{z^2\over2!}+\cdots\,\Bigr)
 \Bigl(n@{z^0\over1!}+n^2{z^1\over2!}+n^3{z^2\over3!}+\cdots\,\Bigr)\,.
\enddisplay
The sum $S_m(n)$ is $m!$ times the coefficient of $z^m$ in this product.
For example,
\begindisplay\openup6pt
S_0(n)&=0!\,\Bigl(B_0@{n\over1!\,0!}\Bigr)&=n\,;\cr
S_1(n)&=1!\,\Bigl(B_0@{n^2\over2!\,0!}+B_1@{n\over1!\,1!}\Bigr)
 &=\textstyle\half n^2-\half n\,;\cr
S_2(n)&=2!\,\Bigl(B_0@{n^3\over3!\,0!}+B_1@{n^2\over2!\,1!}
 +B_2@{n\over1!\,2!}\Bigr)
 &=\textstyle{1\over3}n^3-\half n^2+{1\over6}n\,.\cr
\enddisplay
We have therefore derived the formula $\Sq_n=S_2(n)={1\over3}n(n-\half)(n-1)$
"!sum of consecutive squares"
"!sum of consecutive cubes"
for the umpteenth time, and this was the simplest derivation of all:
In a few lines we have found the general behavior of $S_m(n)$ for all~$m$.

The general formula can be written
\begindisplay
S_{m-1}(n)={1\over m}\bigl(B_m(n)-B_m(0)\bigr)\,,
\eqno\eqref|m-pwr-gen|
\enddisplay
where $B_m(x)$ is the {\it "Bernoulli polynomial"\/} defined by
\begindisplay
B_m(x)=\sum_k{m\choose k}B_k\,x^{m-k}\,.
\eqno
\enddisplay
Here's why: The Bernoulli polynomial
is the binomial convolution of the sequence $\<B_0,B_1,B_2,\ldots\,\>$
with $\<1,x,x^2,\ldots\,\>$; hence the exponential generating function for
$\<B_0(x),B_1(x),B_2(x),\ldots\,\>$ is the product of their egf's,
\begindisplay
\widehat B(z,x)=\sum_{m\ge0}B_m(x){z^m\over m!}
={z\over e^z-1}\sum_{m\ge0}x^m{z^m\over m!}={ze^{xz}\over e^z-1}\,.
\eqno\eqref|bern-poly-gf|
\enddisplay
Equation \eq(|m-pwr-gen|) follows because
the egf for $\<0,S_0(n),2S_1(n),\ldots\,\>$ is, by \eq(|szn-closed|),
\begindisplay
z\,{e^{nz}-1\over e^z-1}=\widehat B(z,n)-\widehat B(z,0)\,.
\enddisplay

Let's turn now to another problem for which egf's are just
the thing: How many "spanning trees" are possible in the {\it"complete graph"\/}
on $n$ vertices $\{1,2,\ldots,n\}$? Let's call this number~$t_n$.
The complete graph has $\half n(n-1)$
edges, one edge joining each pair of distinct vertices; so we're
essentially looking for the total number of ways to connect up $n$ given
things by drawing $n-1$ lines between them.

\def\ft#1#2#3#4#5#6{\beginpicture(1,1)(0,0)
\put(0,0){\disk{.2}}\put(0,1){\disk{.2}}\put(1,0){\disk{.2}}\put(1,1){\disk{.2}}
\if#1T\put(0,0){\line(0,1)1}\fi
\if#2T\put(0,0){\line(1,1)1}\fi
\if#3T\put(0,0){\line(1,0)1}\fi
\if#4T\put(1,1){\line(0,-1)1}\fi
\if#5T\put(1,0){\line(-1,1)1}\fi
\if#6T\put(1,1){\line(-1,0)1}\fi \endpicture}

We have $t_1=t_2=1$. Also $t_3=3$, because a complete graph on three vertices
is a fan of order~$2$; we know that $f_2=3$. And there are sixteen spanning trees
when $n=4$:
\begindisplay \openup14pt \unitlength=15pt \advance\abovedisplayskip-8pt
&\ft TFFFTT\quad
\ft FTFFTT\quad
\ft FFTFTT\quad\qquad
\ft TFFTFT\quad
\ft FTFTFT\quad
\ft FFTTFT\quad\qquad
\ft TFFTTF\quad
\ft FTFTTF\quad
\ft FFTTTF\cr
&\ft TTFTFF\quad
\ft TFTTFF\quad\qquad
\ft TTFFTF\quad
\ft FTTFTF\quad\qquad
\ft TFTFFT\quad
\ft FTTFFT\quad\qquad\qquad
\ft TTTFFF
\eqno
\enddisplay
Hence $t_4=16$.

Our experience with the analogous problem for fans suggests that the
best way to tackle this problem is to single out one vertex, and to look at
the blocks or components that the spanning tree joins together when we
ignore all edges that touch the special vertex. If the non-special vertices
form $m$~components of sizes $k_1$, $k_2$, \dots,~$k_m$, then we can
connect them to the special vertex in $k_1k_2\ldots k_m$ ways. For
example, in the case $n=4$, we can consider the lower left vertex to
be special. The top row of \thiseq\ shows $3t_3$ cases where the other
three vertices are joined among themselves in $t_3$~ways and then connected
to the lower left in $3$~ways. The bottom row shows $2\cdt1\times t_2t_1\times
{3\choose2}$ solutions where the other three vertices are divided into
components of sizes $2$ and~$1$ in $3\choose2$ ways; there's also the
case $\,\vcenter{\hbox{\unitlength=10pt \ft TTTFFF}}\,$
where the other three vertices are completely unconnected among themselves.

This line of reasoning leads to the recurrence
\begindisplay
t_n=\!\!\sum_{m>0}{1\over m!}\sum_{k_1+\cdots+k_m=n-1}
 {n-1\choose k_1,k_2,\ldots,k_m}\,
 k_1k_2\ldots k_m \,t_{k_1}t_{k_2}\ldots t_{k_m}
\enddisplay
for all $n>1$. Here's why: There are $n-1\choose k_1,k_2,\ldots,k_m$
ways to assign $n-1$ elements to a sequence of $m$ components of
respective sizes $k_1$, $k_2$, \dots,~$k_m$; there are
$t_{k_1}t_{k_2}\ldots t_{k_m}$ ways to connect up those individual components
with spanning trees; there are $k_1k_2\ldots k_m$ ways to connect vertex~$n$
to those components; and we divide by~$m!$ because we want to disregard the
order of the components. For example, when $n=4$ the recurrence says that
\begindisplay
\textstyle t_4=3t_3+\half\bigl({3\choose1,2}2t_1t_2+{3\choose2,1}2t_2t_1\bigr)
+{1\over6}\bigl({3\choose1,1,1}t_1^3\bigr)=3t_3+6t_2t_1+t_1^3\,.
\enddisplay

The recurrence for $t_n$ looks formidable at first, possibly even frightening;
but it really isn't bad, only convoluted. We can define
\begindisplay
u_n=n\,t_n
\enddisplay
and then everything simplifies considerably:
\begindisplay
{u_n\over n!}=\sum_{m>0}{1\over m!}\sum_{k_1+k_2+\cdots+k_m=n-1}
{u_{k_1}\over k_1!}
{u_{k_2}\over k_2!}\ldots
{u_{k_m}\over k_m!}\,,\qquad\hbox{if $n>1$}.
\eqno
\enddisplay
The inner sum is the coefficient of $z^{n-1}$ in the egf $\widehat U(z)$,
raised to the $m$th power; and we obtain the correct formula also
when $n=1$, if we add in the term $\widehat U(z)^0$ that corresponds
to the case $m=0$. So
\begindisplay
{u_n\over n!}=[z^{n-1}]\,\sum_{m\ge0}{1\over m!}\,\widehat U(z)^m
=[z^{n-1}]\,e^{\widehat U(z)}=[z^n]\,ze^{\widehat U(z)}
\enddisplay
for all $n>0$, and we have the equation
\begindisplay
\widehat U(z)=z\,e^{\widehat U(z)}\,.
\eqno
\enddisplay
Progress! Equation \thiseq\ is almost like
\begindisplay
\Escr(z)=e^{z\,\Escr(z)}\,,
\enddisplay
which defines the generalized exponential series $\Escr(z)=\Escr_1(z)$
in \equ(5.|t-series-rec|) and \eq(|escr-power|); indeed, we have
\begindisplay
\widehat U(z)=z\,\Escr(z)\,.
\enddisplay
So we can read off the answer to our problem:
\begindisplay
t_n={u_n\over n}={n!\over n}\,[z^n]\,\widehat U(z)=(n-1)!\,[z^{n-1}]\,\Escr(z)=n^{n-2}\,.
\eqno
\enddisplay
The complete graph on $\{1,2,\ldots,n\}$ has exactly $n^{n-2}$ spanning
trees, for all $n>0$.

\beginsection 7.7 Dirichlet Generating Functions

There are many other possible ways to generate a sequence from a series;
any system of ``"kernel"'' functions $K_n(z)$ such that
\begindisplay
\sum_n g_n\,K_n(z)=0\;\implies\;\hbox{$g_n=0$ for all $n$}
\enddisplay
can be used, at least in principle. Ordinary generating functions use
$K_n(z)=z^n$, and exponential generating functions use $K_n(z)=z^n\!/n!$;
we could also try falling factorial powers $z\_n$, or
binomial coefficients $z\_n/n!={z\choose n}$.

The most important alternative to gf's and egf's uses the
kernel functions $1/n^z$; it is intended for sequences $\<g_1,g_2,\ldots\,\>$
that begin with $n=1$ instead of $n=0$:
\begindisplay
\widetilde G(z)=\sum_{n\ge1}{g_n\over n^z}\,.
\eqno\eqref|dgf-def|
\enddisplay
This is called a {\it "Dirichlet generating function"\/} ("dgf"), because
the German mathematician
Gustav Lejeune "Dirichlet" (1805--1859) made much of it.

For example, the dgf of the constant sequence $\<1,1,1,\ldots\,\>$ is
\begindisplay
\sum_{n\ge1}{1\over n^z}=\zeta(z)\,.
\eqno\eqref|zeta-def|
\enddisplay
This is "Riemann"'s {\it "zeta function"}, which we have also called the
"generalized harmonic number" $H_\infty^{(z)}$ when $z>1$.

The product of Dirichlet generating functions corresponds to a
special kind of convolution:
\begindisplay
\widetilde F(z)@\widetilde G(z)
&=\sum_{l,m\ge1}{f_l\over l^z}\,{g_m\over m^z}
 =\sum_{n\ge1}{1\over n^z}\sum_{l,m\ge1}f_l\,g_m\,\[l\cdt m=n]\,.
\enddisplay
Thus $\widetilde F(z)@\widetilde G(z)=\widetilde H(z)$ is the dgf
of the sequence "!sums over divisors"
\begindisplay
h_n=\sum_{d\divides n}f_d\,g_{n/d}\,.
\eqno\eqref|dirichlet-conv|
\enddisplay

For example, we know from \equ(4.|mobius-def|) that
$\sum_{d\divides n}\mu(d)=\[n=1]$; this is the Dirichlet convolution
of the M\"obius sequence $\<\mu(1),\mu(2),\mu(3),\ldots\,\>$ with
$\<1,1,1,\ldots\,\>$, hence
\begindisplay
\widetilde M(z)@\zeta(z)=\sum_{n\ge1}{\[n=1]\over n^z}=1\,.
\eqno\eqref|inverse-zeta|
\enddisplay
In other words, the dgf of
$\<\mu(1),\mu(2),\mu(3),\ldots\,\>$ is $\zeta(z)^{-1}$.

\displaywidowpenalty=1000
Dirichlet generating functions are particularly valuable when the
sequence $\<g_1,g_2,\ldots\,\>$ is a {\it"multiplicative function"},
namely when
\begindisplay
g_{mn}=g_m\,g_n\qquad\hbox{for $m\rp n@$}.
\enddisplay
\looseness=-1
In such cases the values of $g_n$ for all $n$ are determined by
the values of $g_n$ when $n$ is a power of a prime, and we can
factor the dgf into a product over primes:
\begindisplay
\widetilde G(z)=\prod_{p\ \rm prime}\biggl(1+
{g_p\over p^z}+
{g_{p^2}\over p^{2z}}+
{g_{p^3}\over p^{3z}}+\cdots\,\biggr)\,.
\eqno
\enddisplay
If, for instance, we set $g_n=1$ for all $n$, we obtain a product
representation of Riemann's zeta function:
\begindisplay
\zeta(z)=\prod_{p\ \rm prime}\biggl({1\over 1-p^{-z}}\biggr)\,.
\eqno\eqref|zeta-product|
\enddisplay
The "M\"obius function" has $\mu(p)=-1$ and $\mu(p^k)=0$ for $k>1$, hence
its dgf is
\begindisplay
\widetilde M(z)=\prod_{p\ \rm prime}(1-p^{-z})\,;
\eqno
\enddisplay
this agrees, of course,
with \eq(|inverse-zeta|) and \eq(|zeta-product|).
"!totient"
Euler's $\varphi$ function has $\varphi(p^k)=p^k-p^{k-1}$, hence its dgf
has the factored form
\begindisplay
\widetilde\Phi(z)=\prod_{p\ \rm prime}\biggl(1+{p-1\over p^z-p}\biggr)
=\prod_{p\ \rm prime}{1-p^{-z}\over1-p^{1-z}}\,.
\eqno
\enddisplay
We conclude that $\widetilde\Phi(z)=\zeta(z-1)/\zeta(z)$.

\beginexercises

\subhead \kern-.05em Warmups

\ex:
\unitlength=3pt
An eccentric collector of $2\times n$ domino tilings pays \$4 for each vertical
domino and \$1 for each horizontal domino. How many tilings are worth
exactly \$$m$ by this criterion? For example, when $m=6$ there are
three solutions: \domi\domv\domhh\kern1pt, \domi\domhh\domv\kern1pt,
and \domi\domhh\domhh\domhh\kern1pt.
\answer\unitlength=3pt
\def\Domh{\beginpicture(3,1)(-.5,0) % horizontal, stand-alone
\put(2,0){\line(0,1){1}}
\put(0,0){\line(0,1){1}}
\put(0,0){\line(1,0){2}}
\put(0,1){\line(1,0){2}}
\endpicture}%
\def\Domv{\beginpicture(2,2)(-.5,0) % vertical, stand-alone
\put(0,0){\line(0,1){2}}
\put(1,0){\line(0,1){2}}
\put(0,0){\line(1,0){1}}
\put(0,2){\line(1,0){1}}
\endpicture}%
Substitute $z^4$ for \Domv\ and $z$ for \Domh\ in the generating
function, getting $1/(1-z^4-z^2)$. This is like the generating function
for~$T$, but with $z$ replaced by~$z^2$. Therefore the answer is
zero if $m$ is odd, otherwise $F_{m/2+1}$.

\ex:
Give the generating function and the exponential generating function
for the sequence $\<2,5,13,35,\ldots\,\>=\<2^n+3^n\>$ in closed form.
\answer $G(z)=1/(1-2z)+1/(1-3z)$; \ $\widehat G(z)=e^{2z}+e^{3z}$.
\source{[|knuth1|, exercise 1.2.9--1].}

\ex:
What is $\sum_{n\ge0}H_n/10^n$?
\answer Set $z=1/10$ in the generating function, getting
${10\over9}\ln{10\over9}$.

\ex:
The general expansion theorem for rational functions $P(z)/Q(z)$
is not completely general, because it restricts the degree of~$P$ to be
less than the degree of~$Q$. What happens if $P$ has a larger degree
than this?
\answer Divide $P(z)$ by $Q(z)$, getting a quotient $T(z)$ and a
remainder $P_0(z)$ whose degree is less than the degree of~$Q$. The
coefficients of $T(z)$ must be added to the coefficients $[z^n]\,P_0(z)/Q(z)$
for small~$n$. (This is the polynomial $T(z)$ in \equ(7.|r-s-t|).)

\ex:
Find a generating function $S(z)$ such that
\begindisplay
[z^n]\,S(z)=\sum_k{r\choose k}{r\choose n-2k}\,.
\enddisplay
\answer This is the convolution of $(1+z^2)^r$ with $(1+z)^r$, so
\begindisplay
S(z)=(1+z+z^2+z^3)^r\,.
\enddisplay
Incidentally, no simple form is known
for the coefficients of this generating function; hence the stated
sum probably has no simple closed form. (We can use generating
functions to obtain negative results as well as positive ones.)

\subhead Basics

\ex:
Show that the recurrence \eq(|more-less-random|)
can be solved by the "repertoire method", without using generating functions.
\answer Let the solution to $g_0=\alpha$, $g_1=\beta$, $g_n=g_{n-1}+
2g_{n-2}+(-1)^n\gamma$ be $g_n=A(n)\alpha+B(n)\beta+C(n)\gamma$.
The function $2^n$ works when $\alpha=1$, $\beta=2$, $\gamma=0$;
the function $(-1)^n$ works when $\alpha=1$, $\beta=-1$, $\gamma=0$;
the function $(-1)^nn$ works when $\alpha=0$, $\beta=-1$, $\gamma=3$.
Hence $A(n)+2B(n)=2^n$, $A(n)-B(n)=(-1)^n$, and $-B(n)+3C(n)=(-1)^nn$.

\ex:
Solve the recurrence
\begindisplay
g_0&=1\,;\cr
g_n&=g_{n-1}+2g_{n-2}+\cdots+ng_0\,,\qquad\hbox{for $n>0$}.
\enddisplay
\answer $G(z)=\bigl(z/(1-z)^2\bigr)G(z)+1$, hence
\begindisplay
G(z)={1-2z+z^2\over1-3z+z^2}=1+{z\over1-3z+z^2}\,;
\enddisplay
we have $g_n=F_{2n}+\[n=0]$.
\g\vskip-30pt I bet that the controversial ``fan of order zero''
"!null case" "!empty case"
does have one spanning tree.\g

\ex:
What is $[z^n]\,\bigl(\ln(1-z)\bigr)^2\!/(1-z)^{m+1}$?
\answer Differentiate $(1-z)^{-x-1}$ twice with respect to~$x$, obtaining
\begindisplay
{x+n\choose n}\bigl((H_{x+n}-H_x)^2-(H_{x+n}^{(2)}-H_x^{(2)})\bigr)\,.
\enddisplay
Now set $x=m$.
\source{"Zave" [|zave|].}

\ex:
Use the result of the previous exercise to evaluate $\sum_{k=0}^n
H_kH_{n-k}$.
\answer $(n+1)(H_n^2-H_n^{(2)})-2n(H_n-1)$.
\source{[|knuth1|, exercise 1.2.7--22].}

\ex:
Set $r=s=-1/2$ in identity \eq(|rshh-conv|) and then remove all occurrences
of $1/2$ by using tricks like \equ(5.|n-1/2-bc|). What amazing
identity do you deduce?
\g I deduce that Clark "Kent" is really Superman.\g
\answer The identity $H_{k-1/2}-H_{-1/2}={2\over2k-1}+\cdots+{2\over1}=2H_{2k}-H_k$
implies that $\sum_k{2k\choose k}{2n-2k\choose n-k}(2H_{2k}-H_k)=4^nH_n$.

\ex:
This problem, whose three parts are independent, gives practice in the
manipulation of generating functions. We assume that $A(z)=\sum_n a_nz^n$,
$B(z)=\sum_n b_nz^n$, $C(z)=\sum_n c_nz^n$, and that the coefficients
are zero for negative~$n$.
\vskip2pt
\itemitem{a}If $c_n=\sum_{j+2k\le n}a_jb_k$, express $C$ in terms of $A$ and~$B$.
\vskip2pt
\itemitem{b}If $nb_n=\sum_{k=0}^n2^ka_k/(n-k)!$, express $A$ in terms of~$B$.
\vskip2pt
\itemitem{c}If $r$ is a real number and if $a_n=\sum_{k=0}^n{r+k\choose k}b_{n-k}$,
express $A$ in terms of~$B$; then use your formula to find coefficients
$f_k(r)$ such that $b_n=\sum_{k=0}^n f_k(r)@a_{n-k}$.
\answer (a) $C(z)=A(z)B(z^2)/(1-z)$. (b) $zB'(z)=A(2z)e^z$, hence
$A(z)={z\over2}e^{-z/2}B'({z\over2})$.
(c)~$A(z)=B(z)/(1-z)^{r+1}$, hence $B(z)=(1-z)^{r+1}A(z)$ and we have
$f_k(r)={r+1\choose k}(-1)^k$.
\source{1971 final exam.}

\ex:
How many ways are there to put the numbers $\{1,2,\ldots,2n\}$ into a
$2\times n$ array so that rows and columns are in increasing order
from left to right and from top to bottom? For example, one solution when
$n=5$ is
\begindisplay
\pmatrix{1&2&4&5&8\cr 3&6&7&9&10\cr}\,.
\enddisplay
\answer $C_n$. The numbers in the upper row correspond to the positions
of $+1$'s in a sequence of $+1$'s and~$-1$'s that defines a
``mountain range''; the numbers in the lower row correspond to the
positions of $-1$'s. For example, the given array corresponds to
\begindisplay
\unitlength=10pt
\countdef\m=4 \countdef\n=6
\def\init{\thinlines\put(0,0){\line(1,0){10}}
 \thicklines \m=0 \n=0 \put(0,0){\disk{.2}}}
\def\+{\put(\number\m,\number\n){\line(1,1)1}\advance\m1 \advance\n1
 \put(\number\m,\number\n){\disk{.2}}}
\def\-{\put(\number\m,\number\n){\line(1,-1)1}\advance\m1 \advance\n-1
 \put(\number\m,\number\n){\disk{.2}}}
\beginpicture(10,3)(0,0)\init\+\+\-\+\+\-\-\+\-\-\endpicture\;.
\enddisplay
\source{[|knuth3|, pp.~63--64].}

\ex:\exref|prove-raney|%
Prove "Raney"'s generalized lemma, which is stated just before \eq(|m-cat-l|).
\answer Extend the sequence periodically (let $x_{m+k}=x_k$) and define
$s_n=x_1+\cdots+x_n$. We have $s_m=l$, $s_{2m}=2l$, etc. There must be
a largest index $k_j$ such that $s_{k_j}=j$, $s_{k_j+m}=l+j$, etc.
These indices $k_1$, \dots,~$k_l$ (modulo~$m$) specify the cyclic
shifts in question.\par
For example, in the sequence
$\<-2,1,-1,0,1,1,-1,1,1,1\>$ with $m=10$ and $l=2$ we have $k_1=17$, $k_2=24$.
\source{"Raney" [|raney|].}

\ex:
Solve the recurrence
\begindisplay \openup3pt
g_0&=0\,,\qquad g_1=1\,,\cr
g_n&=-2ng_{n-1}+\sum_k{n\choose k}g_k g_{n-k}\,,\qquad\hbox{for $n>1$},
\enddisplay
by using an exponential generating function.
\answer $\widehat G(z)=-2z\widehat G(z)+\widehat G(z)^2+z$ (be careful about the final
term!) leads via the quadratic formula to
\begindisplay
\widehat G(z)={1+2z-\sqrt{1+4z^{2\mathstrut}}\over2}\,.
\enddisplay
Hence $g_{2n+1}=0$ and $g_{2n}=(-1)^n(2n)!\,C_{n-1}$, for all $n>0$.

\ex:\exref|bell-number-gf|%
The {\it "Bell number"\/} $\varpi_n$ is the number of ways to partition
$n$~things
into subsets. For example, $\varpi_3=5$ because we can partition $\{1,2,3\}$ in
the following ways:
\begindisplay
\{1,2,3\}\,;\quad
\{1,2\}\cup\{3\}\,;\quad
\{1,3\}\cup\{2\}\,;\quad
\{1\}\cup\{2,3\}\,;\quad
\{1\}\cup\{2\}\cup\{3\}\,.
\enddisplay
Prove that $\varpi_{n+1}=\sum_k{n\choose k}\varpi_{n-k}$,
 and use this recurrence to find
a closed form for the exponential generating function
 $P(z)=\sum_n\varpi_nz^n\!/n!$.
\answer There are ${n\choose k}\varpi_{n-k}$ partitions with $k$ other objects
in the subset containing~$n+1$. Hence $\widehat P'(z)=e^z\widehat P(z)$. The solution
to this differential equation is $\widehat P(z)=e^{e^z+c}$, and $c=-1$ since
$\widehat P(0)=1$. (We can also get this result by summing
\equ(7.|exp-power-gf|) on~$m$, since $\varpi_n=\sum_m{n\brace m}$.)
\source{"Bell" [|bell-numbers|].}

\ex:
Two sequences $\<a_n\>$ and $\<b_n\>$ are related by the
convolution formula
\begindisplay
b_n=\!\sum_{k_1+2k_2+\cdots nk_n=n}{a_1{+}k_1{-}1\choose k_1}
{a_2{+}k_2{-}1\choose k_2}\ldots{a_n{+}k_n{-}1\choose k_n}\,;
\enddisplay
also $a_0=0$ and $b_0=1$. Prove that the corresponding generating functions
satisfy
%\begindisplay
$\ln B(z)=A(z)+\half A(z^2)+{1\over3}A(z^3)+\cdots\,$.
%\enddisplay
\answer One way is to take the logarithm of
\begindisplay
B(z)=1\big/\bigl((1-z)^{a_1}(1-z^2)^{a_2}(1-z^3)^{a_3}(1-z^4)^{a_4}\ldots\,\bigr),
\enddisplay
then use the formula for $\ln{1\over1-z}$ and interchange the order of
summation.
\source{"P\'olya" [|polya-counting|, p.~149]; [|knuth1|, exercise 2.3.4.4--1].}

\ex:
Show that the exponential generating function $\widehat G(z)$ of a sequence
is related to the ordinary generating function $G(z)$ by the formula
\begindisplay
\int_0^\infty \widehat G(zt)e^{-t}\,dt=G(z)\,,
\enddisplay
if the integral exists.
\answer This follows since $\int_0^\infty t^ne^{-t}\,dt=n!@$.
There's also a formula that goes in the other direction:
\begindisplay
\widehat G(z)={1\over2\pi}\int_{-\pi}^{+\pi}G(ze^{-i\theta})\,e^{e^{i\theta}}
d\theta\,.
\enddisplay

\ex:
Find the "Dirichlet generating functions" for the sequences
\par\nobreak\vskip2pt
\itemitem{a}$g_n=\sqrt n\mkern1mu$;
\itemitem{b}$g_n=\ln n@$;
\itemitem{c}$g_n=\[\hbox{$n$ is squarefree}]$.
\par\nobreak\vskip2pt\item{}%
Express your answers in terms of the zeta function. (Squarefreeness
is defined in exercise 4.|squarefree-def|.)
\answer (a) $\zeta(z-\half)$; \ (b) $-\zeta'(z)$; \ (c) $\zeta(z)/\zeta(2z)$.
Every positive integer is uniquely representable as $m^2q$, where $q$
is squarefree.

\ex:\exref|general-conv-principles|%
Every power series $F(z)=\sum_{n\ge0}f_nz^n$ with $f_0=1$ defines a
sequence of polynomials $f_n(x)$ by the rule
\begindisplay
F(z)^x=\sum_{n\ge0}f_n(x)@z^n\,,
\enddisplay
where $f_n(1)=f_n$ and $f_n(0)=\[n=0]$. In general, $f_n(x)$ has degree~$n$.
\g What do you mean, ``in general''? If $f_1=f_2=\cdots=f_{m-1}=0$, the
degree of $f_n(x)$ is at most $\lfloor n/m\rfloor$.\g
Show that such polynomials always satisfy the convolution formulas
\begindisplay\openup4pt
\sum_{k=0}^n f_k(x)@f_{n-k}(y)&=f_n(x+y)\,;\cr
(x+y)\sum_{k=0}^n kf_k(x)@f_{n-k}(y)&=xn@f_n(x+y)\,.\cr
\enddisplay
(The identities in Tables |conv-table| and |stirling-convolutions| are special
cases of this trick.)
\answer If $n>0$, the coefficient $[z^n]\,\exp\bigl(x\ln F(z)\bigr)$ is
a polynomial of degree~$n$ in~$x$ that's a multiple of~$x$. The first
convolution formula comes from equating coefficients of~$z^n$ in
$F(z)^xF(z)^y=F(z)^{x+y}$. The second comes from equating coefficients of
$z^{n-1}$ in $F'(z)@F(z)^{x-1}F(z)^y=F'(z)@F(z)^{x+y-1}$, because we have
\begindisplay
F'(z)@F(z)^{x-1}=x^{-1}{\partial\over\partial z}\bigl(F(z)^x\bigr)
=x^{-1}\sum_{n\ge0}nf_n(x)z^{n-1}\,.
\enddisplay
(Further convolutions follow by taking $\partial/\partial x$, as
in \equ(7.|harmonic-general|).)\par
Still more is true, as shown in [|knuth-cp|]: We have
\begindisplay
\sum_{k=0}^n{x@f_k(x+tk)\over x+tk}{y@f_{n-k}(y+t(n-k))\over y+t(n-k)}=
{(x+y)f_n(x+y+tn)\over x+y+tn}\,,
\enddisplay
for arbitrary $x$, $y$, and $t$. In fact, $x@f_n(x+tn)/(x+tn)$ is the
sequence of polynomials for the coefficients of $\Fscr_t(z)^x$, where
\begindisplay
\Fscr_t(z)=F\bigl(z@\Fscr_t(z)^t\bigr)\,.
\enddisplay
(We saw special cases in \equ(5.|t-series-rec|) and \equ(6.|t-series-stirl|).)
\source{[|knuth-cp|].}

\ex:\exref|d-finite-defined|%
A power series $G(z)$ is called {\it"differentiably finite"\/} if
there exist finitely many polynomials $P_0(z)$, \dots, $P_m(z)$, not
all zero, such that
\begindisplay
P_0(z)@G(z)+P_1(z)@G'(z)+\cdots+P_m(z)@G^{(m)}(z)=0\,.
\enddisplay
A sequence of numbers $\<g_0,g_1,g_2,\ldots\,\>$ is called {\it"polynomially
recursive"\/} if there exist finitely many polynomials $p_0(z)$, \dots,
$p_m(z)$, not all zero, such that
\begindisplay
p_0(n)@g_n+p_1(n)@g_{n+1}+\cdots+p_m(n)@g_{n+m}=0
\enddisplay
for all integers $n\ge0$. Prove that a generating function is differentiably
finite if and only if its sequence of coefficients is polynomially
recursive.
\answer Let $G(z)=\sum_{n\ge0}g_nz^n$. Then
\begindisplay
z^lG^{(k)}(z)=\sum_{n\ge0}
n\_kg_nz^{n-k+l}=\sum_{n\ge0}(n+k-l)\_kg_{n+k-l}z^n
\enddisplay
for all $k,l\ge0$,
if we regard $g_n=0$ for $n<0$. Hence if $P_0(z)$, \dots,~$P_m(z)$
are polynomials, not all zero, having maximum degree~$d$, then there are
polynomials $p_0(n)$, \dots,~$p_{m+d}(n)$ such that
\begindisplay
P_0(z)@G(z)+\cdots+P_m(z)@G^{(m)}(z)=\sum_{n\ge0}\sum_{j=0}^{m+d}p_j(n)@g_{n+j-d}z^n\,.
\enddisplay
Therefore a differentiably finite $G(z)$ implies that
\begindisplay
\sum_{j=0}^{m+d}p_j(n+d)@g_{n+j}=0\,,\qquad\hbox{for all $n\ge0$}.
\enddisplay
The converse is similar.
(One consequence is that $G(z)$ is differentiably finite if and only
if the corresponding egf, $\widehat G(z)$, is differentiably finite.)
\source{"Jungen" [|jungen|, p.~299] credits A. "Hurwitz".}

\subhead Homework exercises

\ex:
A robber holds up a bank and demands \$500 in tens and twenties.
He also demands to know the number of ways in which the cashier can
give him the money. Find a generating function $G(z)$
\g Will he settle for $2\times n$ domino tilings?\g
 for which this number
is $[z^{500}]\,G(z)$, and a more compact generating function $\check G(z)$ for
which this number is $[z^{50}]\,\check G(z)$. Determine the required
number of ways by (a)~using partial fractions; (b)~using a method
like \eq(|change-compressed|).
\answer This is the problem of giving change with denominations $10$
and $20$, so $G(z)=1/(1-z^{10})(1-z^{20})=\check G(z^{10})$, where
$\check G(z)=1/(1-z)(1-z^2)$. (a)~The partial fraction decomposition
\g This slow method of finding the answer is just the cashier's way
of stalling until the police come.\g
of $\check G(z)$ is $\half(1-z)^{-2}+{1\over4}(1-z)^{-1}+{1\over4}(1+z)
^{-1}$, so $[z^n]\,\check G(z)={1\over4}\bigl(2n+3+(-1)^n\bigr)$.
Setting $n=50$ yields $26$ ways to make the payment.
(b)~$\check G(z)=(1+z)/(1-z^2)^2=(1+z)(1+2z^2+3z^4+\cdots\,)$, so
$[z^n]\,\check G(z)=\lfloor n/2\rfloor+1$. (Compare this with the
value $N_n=\lfloor n/5\rfloor+1$ in the text's coin-changing problem.
The bank robber's problem is equivalent to the problem of making change
\g The USA has two\hbox{-}cent pieces,
 but they haven't been minted since 1873.\g
with pennies and tuppences.)

\def\tri#1{\cpic(4,4)(0,0) \put(0,0){\line(1,0){4}}
 \put(0,0){\line(1,2){2}} \put(4,0){\line(-1,2){2}}#1\endcpic}
\def\sqr#1{\cpic(4.5,4.5)(0,0) \put(0,0){\line(0,1){4.5}}
 \put(0,0){\line(1,0){4.5}} \put(0,4.5){\line(1,0){4.5}}\put(4.5,0){\line(0,1){4.5}}
 #1\endcpic}
\def\pnt#1{\cpic(6,5)(-1,0) \put(0,0){\line(1,0){4}}
 \put(0,0){\line(-1,3){1}} \put(-1,3){\line(3,2){3}} \put(2,5){\line(3,-2){3}}
 \put(4,0){\line(1,3){1}} #1\endcpic}
\def\dgn{\cpic(3,2)(0,0)\put(0,0){\line(1,0){3}}\endcpic}

\ex:\exref|triangulations|%
Let $P$ be the sum of all ways to ``triangulate'' polygons:
\begindisplay \unitlength=3.33333pt \openup3pt \advance\medmuskip5mu
P=\dgn +\tri{}&+\sqr{\put(4.5,0){\line(-1,1){4.5}}}
	+\sqr{\put(0,0){\line(1,1){4.5}}}\cr
&+\pnt{\put(4,0){\line(-5,3){5}}\put(4,0){\line(-2,5){2}}}
	+\pnt{\put(4,0){\line(-5,3){5}}\put(5,3){\line(-1,0){6}}}
	+\pnt{\put(0,0){\line(2,5){2}}\put(4,0){\line(-2,5){2}}}
	+\pnt{\put(0,0){\line(2,5){2}}\put(0,0){\line(5,3){5}}}
	+\pnt{\put(-1,3){\line(1,0){6}}\put(0,0){\line(5,3){5}}}
	+\cdots\,.
\enddisplay
(The first term represents a degenerate polygon with only two vertices;
every other term shows a polygon that has been divided into triangles.
For example, a pentagon can be triangulated in five ways.)
Define a ``multiplication'' operation $A\triangle B$ on triangulated
polygons $A$ and~$B$ so that the equation
\begindisplay \unitlength=3.33333pt
P=\,\,\dgn\;+\; P\triangle P
\enddisplay
is valid. Then replace each triangle by `$z$'; what does this tell you
about the number of ways to decompose an $n$-gon into triangles?
\answer Each polygon has a ``base'' (the line segment at the bottom).
If $A$ and~$B$ are triangulated polygons, let $A\triangle B$ be the
result of pasting the base of~$A$ to the upper left diagonal of $\triangle$,
and pasting the base of~$B$ to the upper right diagonal. Thus, for
example,
\def\tri#1{\cpic(4,4)(0,0) \put(0,0){\line(1,0){4}}
 \put(0,0){\line(1,2){2}} \put(4,0){\line(-1,2){2}}#1\endcpic}%
\def\sqr#1{\cpic(3,3)(0,0) \put(0,0){\line(0,1){3}}
 \put(0,0){\line(1,0){3}} \put(0,3){\line(1,0){3}}\put(3,0){\line(0,1){3}}
 #1\endcpic}%
\def\pnt#1{\cpic(6,5)(-1,0) \put(0,0){\line(1,0){4}}
 \put(0,0){\line(-1,3){1}} \put(-1,3){\line(3,2){3}} \put(2,5){\line(3,-2){3}}
 \put(4,0){\line(1,3){1}} #1\endcpic}%
\def\dgn{\cpic(3,2)(0,0)\put(0,0){\line(1,0){3}}\endcpic}%
\begindisplay
\unitlength=3.33333pt
\sqr{\put(0,0){\line(1,1){3}}}\;\triangle\;\dgn=
\pnt{\put(0,0){\line(2,5){2}}\put(0,0){\line(5,3){5}}}\,\,.
\enddisplay
(The polygons might need to be warped a bit and/or banged into shape.)
Every triangulation arises in this way, because the base line is part of
a unique triangle and there are triangulated polygons $A$ and~$B$ at its
left and right.\par
 Replacing each triangle by~$z$ gives a power series in which
the coefficient of~$z^n$ is the number of triangulations with $n$~triangles,
namely the number of ways to decompose an $(n+2)$-gon into triangles.
Since $P=1+zP^2$, this is the generating function for Catalan numbers
$C_0+C_1z+C_2z^2+\cdots\,$; the number of ways to triangulate an $n$-gon
is $C_{n-2}={2n-4\choose n-2}/(n-1)$.
\source{"P\'olya" [|polya-pictures|].}

\ex:
In how many ways can a $2\times2\times n$ pillar be built out of
\g At union rates, as many as you can afford, plus a few.\g
$2\times1\times1$ bricks?
\answer Let $a_n$ be the stated number, and $b_n$ the number of ways
with a $2\times1\times1$ notch missing at the top. By considering
the possible patterns visible on the top surface, we have
\begindisplay
a_n&=2a_{n-1}+4b_{n-1}+a_{n-2}+\[n=0]\,;\cr
b_n&=a_{n-1}+b_{n-1}\,.
\enddisplay
Hence the generating functions satisfy $A=2zA+4zB+z^2A+1$,
$B=zA+zB$, and we have
\begindisplay
A(z)={1-z\over(1+z)(1-4z+z^2)}\,.
\enddisplay
This formula relates to the problem of $3\times n$ domino tilings; we have
$a_n={1\over3}\bigl(U_{2n}+V_{2n+1}+(-1)^n\bigr)=
{1\over6}(2+\sqrt3\,)^{n+1}+{1\over6}(2-\sqrt3\,)^{n+1}+{1\over3}(-1)^n$,
which is $(2+\sqrt3\,)^{n+1}\!/6$ rounded to the nearest integer.
\source{1983 homework.}

\ex:
How many "spanning trees" are in an $n$-"wheel" (a "graph" with $n$ ``outer''
vertices in a cycle, each connected to an $(n+1)$st ``hub'' vertex),
when $n\ge3$?
\answer $n\sum_{k_1+\cdots+k_m=n}k_1\cdot\ldots\cdot k_m/m=F_{2n+1}+
F_{2n-1}-2$. (Consider the coefficient
$[z^{n-1}]\,{d\over dz}\ln\bigl(1/(1-G(z))\bigr)$,
where $G(z)=z/(1-z)^2$.)
\source{"Myers" [|myers|]; "Sedl\'a\v cek" [|sedlacek|].}

\ex:\exref|bmod-omega|%
Let $m\ge2$ be an integer. What is a closed form for the generating
function of the sequence $\<n\bmod m\>$, as a function of $z$ and~$m$?
Use this generating function to express `$n\bmod m$' in terms of the
complex number $\omega=e^{2\pi i/m}$. (For example, when $m=2$ we have
$\omega=-1$ and $n\bmod2=\half-\half(-1)^n$.)
\answer The generating function is $P(z)/(1-z^m)$, where
$P(z)=z+2z^2+\cdots+(m-1)z^{m-1}=\bigl((m-1)z^{m+1}-mz^m+z)/(1-z)^2$.
The denominator is $Q(z)=1-z^m=(1-\omega^0z)(1-\omega^1z)\ldots(1-\omega^{m-1}z)$.
By the rational expansion theorem for distinct roots, we obtain
\begindisplay
n\bmod m={m-1\over2}+\sum_{k=1}^{m-1}{\omega^{-kn}\over\omega^k-1}\,.
\enddisplay
\source{[|knuth2|, "Carlitz"'s proof of lemma 3.3.3B].}

\ex:
\def\FF{{\frak F}}%
The "second-order Fibonacci numbers" $\<\FF_n\>$ are defined by the recurrence
"!Fibonacci numbers, second-order"
\begindisplay
\FF_0&=0\,;\qquad \FF_1=1\,;\cr
\FF_n&=\FF_{n-1}+\FF_{n-2}+F_n\,,\qquad\hbox{for $n>1$}.
\enddisplay
Express $\FF_n$ in terms of the usual Fibonacci numbers $F_n$ and $F_{n+1}$.
\answer $(1-z-z^2){\frak F}(z)=F(z)$ leads to ${\frak F}_n=\bigl(
2(n+1)F_n+nF_{n+1}\bigr)/5$ as in equation~\equ(7.|fib-conv|).
\source{[|knuth1|, exercise 1.2.8--12].}

\ex:\exref|fib-squared|%
A $2\times n$ domino tiling can also be regarded as a way to draw $n$
disjoint lines in a $2\times n$ array of points:
\begindisplay
\unitlength=8pt
\beginpicture(10,1)(0,0)
\multiput(0,0)(1,0){10}{\disk{.25}}
\multiput(0,1)(1,0){10}{\disk{.25}}
\put(0,0){\line(0,1)1}
\put(5,0){\line(0,1)1}
\put(8,0){\line(0,1)1}
\put(9,0){\line(0,1)1}
\put(1,0){\line(1,0)1}
\put(1,1){\line(1,0)1}
\put(3,0){\line(1,0)1}
\put(3,1){\line(1,0)1}
\put(6,0){\line(1,0)1}
\put(6,1){\line(1,0)1}
\endpicture
\enddisplay
If we superimpose two such patterns, we get a set of cycles, since
every point is touched by two lines. For example, if the lines above
are combined with the lines
\begindisplay
\unitlength=8pt
\beginpicture(10,1)(0,0)
\multiput(0,0)(1,0){10}{\disk{.25}}
\multiput(0,1)(1,0){10}{\disk{.25}}
\put(0,0){\line(0,1)1}
\put(1,0){\line(0,1)1}
\put(2,0){\line(1,0)1}
\put(2,1){\line(1,0)1}
\put(4,0){\line(1,0)1}
\put(4,1){\line(1,0)1}
\put(6,0){\line(1,0)1}
\put(6,1){\line(1,0)1}
\put(8,0){\line(1,0)1}
\put(8,1){\line(1,0)1}
\endpicture\,,
\enddisplay
the result is
\begindisplay
\unitlength=8pt
\beginpicture(10,1)(0,0)
\multiput(0,0)(1,0){10}{\disk{.25}}
\multiput(0,1)(1,0){10}{\disk{.25}}
\put(-.1,0){\line(0,1)1}
\put(5,0){\line(0,1)1}
\put(8,0){\line(0,1)1}
\put(9,0){\line(0,1)1}
\put(1,0){\line(1,0)1}
\put(1,1){\line(1,0)1}
\put(3,0){\line(1,0)1}
\put(3,1){\line(1,0)1}
\put(6,-.1){\line(1,0)1}
\put(6,.9){\line(1,0)1}
\put(.1,0){\line(0,1)1}
\put(1,0){\line(0,1)1}
\put(2,0){\line(1,0)1}
\put(2,1){\line(1,0)1}
\put(4,0){\line(1,0)1}
\put(4,1){\line(1,0)1}
\put(6,.1){\line(1,0)1}
\put(6,1.1){\line(1,0)1}
\put(8,0){\line(1,0)1}
\put(8,1){\line(1,0)1}
\endpicture\,.
\enddisplay
The same set of cycles is also obtained by combining
\begindisplay
\unitlength=8pt
\beginpicture(10,1)(0,.1)
\multiput(0,0)(1,0){10}{\disk{.25}}
\multiput(0,1)(1,0){10}{\disk{.25}}
\put(0,0){\line(0,1)1}
\put(1,0){\line(0,1)1}
\put(2,0){\line(1,0)1}
\put(2,1){\line(1,0)1}
\put(4,0){\line(1,0)1}
\put(4,1){\line(1,0)1}
\put(8,0){\line(0,1)1}
\put(9,0){\line(0,1)1}
\put(6,0){\line(1,0)1}
\put(6,1){\line(1,0)1}
\endpicture
\qquad\hbox{with}\qquad
\beginpicture(10,1)(0,.1)
\multiput(0,0)(1,0){10}{\disk{.25}}
\multiput(0,1)(1,0){10}{\disk{.25}}
\put(0,0){\line(0,1)1}
\put(1,0){\line(1,0)1}
\put(1,1){\line(1,0)1}
\put(3,0){\line(1,0)1}
\put(3,1){\line(1,0)1}
\put(5,0){\line(0,1)1}
\put(6,0){\line(1,0)1}
\put(6,1){\line(1,0)1}
\put(8,0){\line(1,0)1}
\put(8,1){\line(1,0)1}
\endpicture
\,.
\enddisplay
But we get a unique way to reconstruct the original patterns from the
\def\\{{/\penalty0}}%
superimposed ones if we assign orientations to the vertical lines by using
arrows that go alternately up\\down\\up\\down\\$\cdots\,$\
 in the first pattern and alternately down\\up\\down\\up\\$\cdots\,$\
in the second. For example,
\begindisplay \unitlength=8pt
\beginpicture(9,1)(0,.1)
\multiput(0,0)(1,0){10}{\disk{.25}}
\multiput(0,1)(1,0){10}{\disk{.25}}
\put(0,0){\line(0,1)1} \put(0,0){\vector(0,1){1.0}}
\put(5,0){\line(0,1)1} \put(5,1){\vector(0,-1){1.0}}
\put(8,0){\line(0,1)1} \put(8,0){\vector(0,1){1.0}}
\put(9,0){\line(0,1)1} \put(9,1){\vector(0,-1){1.0}}
\put(1,0){\line(1,0)1}
\put(1,1){\line(1,0)1}
\put(3,0){\line(1,0)1}
\put(3,1){\line(1,0)1}
\put(6,0){\line(1,0)1}
\put(6,1){\line(1,0)1}
\endpicture
\;\;+\;\;
\beginpicture(9,1)(0,.1)
\multiput(0,0)(1,0){10}{\disk{.25}}
\multiput(0,1)(1,0){10}{\disk{.25}}
\put(0,0){\line(0,1)1} \put(0,1){\vector(0,-1){1.0}}
\put(1,0){\line(0,1)1} \put(1,0){\vector(0,1){1.0}}
\put(2,0){\line(1,0)1}
\put(2,1){\line(1,0)1}
\put(4,0){\line(1,0)1}
\put(4,1){\line(1,0)1}
\put(6,0){\line(1,0)1}
\put(6,1){\line(1,0)1}
\put(8,0){\line(1,0)1}
\put(8,1){\line(1,0)1}
\endpicture
\;=\;
\beginpicture(10,1)(0,.1)
\multiput(0,0)(1,0){10}{\disk{.25}}
\multiput(0,1)(1,0){10}{\disk{.25}}
\put(-.1,0){\line(0,1)1} \put(-.1,0){\vector(0,1){1.0}}
\put(5,0){\line(0,1)1} \put(5,1){\vector(0,-1){1.0}}
\put(8,0){\line(0,1)1} \put(8,0){\vector(0,1){1.0}}
\put(9,0){\line(0,1)1} \put(9,1){\vector(0,-1){1.0}}
\put(1,0){\line(1,0)1}
\put(1,1){\line(1,0)1}
\put(3,0){\line(1,0)1}
\put(3,1){\line(1,0)1}
\put(6,-.1){\line(1,0)1}
\put(6,.9){\line(1,0)1}
\put(.1,0){\line(0,1)1} \put(.1,1){\vector(0,-1){1.0}}
\put(1,0){\line(0,1)1} \put(1,0){\vector(0,1){1.0}}
\put(2,0){\line(1,0)1}
\put(2,1){\line(1,0)1}
\put(4,0){\line(1,0)1}
\put(4,1){\line(1,0)1}
\put(6,.1){\line(1,0)1}
\put(6,1.1){\line(1,0)1}
\put(8,0){\line(1,0)1}
\put(8,1){\line(1,0)1}
\endpicture
\,.
\enddisplay
The number of such oriented cycle patterns must therefore be $T_n^2=F_{n+1}^2$,
and we should be able to prove this via algebra.
Let $Q_n$ be the number of oriented $2\times n$ cycle patterns. Find a
recurrence for $Q_n$, solve it with generating functions, and deduce algebraically
that $Q_n=F_{n+1}^2$.
\answer Each oriented cycle pattern begins with
{\unitlength=8pt
\beginpicture(.5,1)(-.25,.1)
\multiput(0,0)(0,1){2}{\disk{.25}}
\put(-.1,0){\line(0,1)1} \put(-.1,0){\vector(0,1){1.0}}
\put(.1,0){\line(0,1)1} \put(.1,1){\vector(0,-1){1.0}}
\endpicture}
or
{\unitlength=8pt
\beginpicture(1.2,1)(-.1,.1)
\multiput(0,0)(1,0){2}{\disk{.25}}
\multiput(0,1)(1,0){2}{\disk{.25}}
\put(0,-.1){\line(1,0)1}
\put(0,.9){\line(1,0)1}
\put(0,.1){\line(1,0)1}
\put(0,1.1){\line(1,0)1}
\endpicture}
or a $2\times k$ cycle (for some $k\ge2$) oriented in one of two ways. Hence
\begindisplay
Q_n=Q_{n-1}+Q_{n-2}+2Q_{n-2}+2Q_{n-3}+\cdots+2Q_0
\enddisplay
for $n\ge2$; \ $Q_0=Q_1=1$. The generating function is therefore
\begindisplay \openup3pt
Q(z)&=zQ(z)+z^2Q(z)+2z^2Q(z)/(1-z)+1\cr
&=1/\bigl(1-z-z^2-2z^2\!/(1-z)\bigr)\cr
&={(1-z)\over(1-2z-2z^2+z^3)}\cr
&={\phi^2\!/5\over 1-\phi^2z}+{\phi^{-2}\!/5\over1-\phi^{-2}z}+{2/5\over 1+z}\,,\cr
\enddisplay
and $Q_n=\bigl(\phi^{2n+2}+\phi^{-2n-2}+2(-1)^n\bigr)/5
=\bigl((\phi^{n+1}-\hat\phi^{n+1})/\!\sqrt5\,\bigr)^2=F_{n+1}^2$.

\ex:
The coefficients of $A(z)$ in \eq(|change-compressed|) satisfy
$A_r+A_{r+10}+A_{r+20}+A_{r+30}=100$ for $0\le r<10$. Find a ``simple''
explanation for this.
\answer In general if $A(z)=(1+z+\cdots+z^{m-1})B(z)$, we have
$A_r+A_{r+m}+A_{r+2m}+\cdots=B(1)$ for $0\le r<m$. In this case $m=10$
and $B(z)=(1+z+\cdots+z^9)(1+z^2+z^4+z^6+z^8)(1+z^5)$.

\ex:
What is the sum of Fibonacci products
\begindisplay
\sum_{m>0}\,\,\sum\twoconditions{k_1+k_2+\cdots+k_m=n}{k_1,k_2,\ldots,k_m>0}
F_{k_1}F_{k_2}\ldots F_{k_m}\,?
\enddisplay
\answer $F(z)+F(z)^2+F(z)^3+\cdots=z/(1-z-z^2-z)=\bigl(1/(1-(1+\sqrt2\,)z)
-(1/(1-(1-\sqrt2\,)z)\bigr)/\sqrt8$, so the answer is
$\bigl((1+\sqrt2\,)^n-(1-\sqrt2\,)^n\bigr)/\sqrt8$.

\ex:
If the generating function $G(z)=1/(1-\alpha z)(1-\beta z)$ has
the partial fraction decomposition $a/(1-\alpha z)+b/(1-\beta z)$, what
is the partial fraction decomposition of $G(z)^n$?
\answer $\sum_{k=1}^n{2n-1-k\choose n-1}\bigl(a^nb^{n-k}\!/(1-\alpha z)^k
+a^{n-k}b^n\!/(1-\beta z)^k\bigr)$, by exercise 5.|ax+by-expansion|.

\ex: What function $g(n)$ of the positive integer~$n$ satisfies the recurrence
\begindisplay
\sum_{d\divides n}g(d)\,\varphi(n/d)=1\,,
\enddisplay
where $\varphi$ is Euler's totient function?
\answer The dgf is $\zeta(z)^2\!/\zeta(z-1)$; hence we find $g(n)$ is
the product of $(k+1-kp)$ over all prime powers $p^k$ that exactly
divide~$n$.

\ex:
An {\it"arithmetic progression"\/} is an infinite set of integers
\begindisplay
\{an+b\}=\{b,a+b,2a+b,3a+b,\ldots\,\}\,.
\enddisplay
A set of arithmetic "progression"s $\{a_1n+b_1\}$, \dots,~%
$\{a_mn+b_m\}$ is called an {\it"exact cover"\/} if every nonnegative
integer occurs in one and only one of the progressions.
For example, the three progressions $\{2n\}$, $\{4n+1\}$,
$\{4n+3\}$ constitute an exact cover.
Show that if $\{a_1n+b_1\}$, \dots,~%
$\{a_mn+b_m\}$ is an exact cover such that $2\le a_1\le\cdots\le a_m$,
then $a_{m-1}=a_m$. \Hint: Use generating functions.
\answer We may assume that each $b_k\ge0$. A set of arithmetic progressions
forms an exact cover if and only if
\begindisplay
{1 \over 1-z}={z^{b_1}\over 1-z^{a_1}}+\cdots+{z^{b_m}\over 1-z^{a_m}}\,.
\enddisplay
Subtract $z^{b_m}/(1-z^{a_m})$ from both sides and set $z=e^{2\pi i/a_m}$.
The left side is infinite, and the right side will be finite unless
$a_{m-1}=a_m$.
\source{[|erdos-graham|, pp. 25--26] credits L. "Mirsky" and M. "Newman".}

\subhead Exam problems

\ex:
What is $[w^mz^n]\,\bigl(\ln(1+z)\bigr)/(1-wz)$?
\answer $(-1)^{n-m+1}\[n>m]/(n-m)$.
\source{1971 final exam.}

\ex:
Find a closed form for the generating function $\sum_{n\ge0}G_n(z)w^n$,
if
\begindisplay
G_n(z)=\sum_{k\le n/m}{n-mk\choose k}z^{mk}\,.
\enddisplay
(Here $m$ is a fixed positive integer.)
\answer We can also write $G_n(z)=\sum_{k_1+(m+1)k_{m+1}=n}{k_1+k_{m+1}
"!multinomial coefficient"
\choose k_{m+1}}(z^m)^{k_{m+1}}$. In general, if
\begindisplay
G_n=\sum_{k_1+2k_2+\cdots+rk_r=n}{k_1+k_2+\cdots+k_r\choose k_1,k_2,
\ldots,k_r}z_1^{k_1}z_2^{k_2}\ldots z_r^{k_r}\,,
\enddisplay
we have $G_n=z_1G_{n-1}+z_2G_{n-2}+\cdots+z_rG_{n-r}+\[n=0]$, and the
generating function is $1/(1-z_1w-z_2w^2-\cdots-z_rw^r)$. In the stated
special case
the answer is $1/(1-w-z^mw^{m+1})$. (See \equ(5.|bc-gen-fib|) for
the case $m=1$.)
\source{Tom\'as "Feder".*}

\ex:
Evaluate the sum $\sum_{0<k<n}1/k(n-k)$ in two ways:
\itemitem{a}Expand the summand in partial fractions.
\itemitem{b}Treat the sum as a convolution and use generating functions.
\answer (a) ${1\over n}\sum_{0<k<n}\bigl(1/k+1/(n-k)\bigr)={2\over n}H_{n-1}$.
(b)~$[z^n]\,\bigl(\ln{1\over1-z}\bigr)^2={2!\over n!}{n\brack2}={2\over n}
H_{n-1}$ by \equ(7.|log-power-gf|) and \equ(6.|stirl-harmonic|).
Another way to do part~(b) is to use the rule $[z^n]\,F(z)={1\over n}[z^{n-1}]\,
F'(z)$ with $F(z)=\bigl(\ln{1\over1-z}\bigr)^2$.

\ex:
Let $A(z)$ be the generating function for $\<a_0,a_1,a_2,a_3,\ldots\,\>$.
Express $\sum_n a_{\lfloor n/m\rfloor}z^n$ in terms of $A$, $z$, and $m$.
\answer ${1-z^m\over1-z}A(z^m)$.
\source {1974 final exam.}

\ex:
Let $a_n$ be the number of ways to write the positive integer~$n$ as
a sum of powers of~$2$, disregarding order. For example, "!binary partitions"
$a_4=4$, since $4=2+2=2+1+1=1+1+1+1$. By convention we let $a_0=1$.
Let $b_n=\sum_{k=0}^na_k$ be the cumulative sum of the first $a$'s.
\itemitem{a}Make a table of the $a$'s and $b$'s up through $n=10$.
What amazing relation do you observe in your table? (Don't prove it yet.)
\itemitem{b}Express the generating function $A(z)$ as an infinite product.
\itemitem{c}Use the expression from part (b) to prove the result of part (a).
\answer (a) The amazing identity $a_{2n}=a_{2n+1}=b_n$ holds in the table
\begindisplay \let\preamble=\tablepreamble
n&&0&1&2&3&4&5&6&7&8&9&10\cr
\noalign{\hrule}
a_n&&1&1&2&2&4&4&6&6&10&10&14\cr
\noalign{\hrule}
b_n&&1&2&4&6&10&14&20&26&36&46&60\cr
\enddisplay
(b) $A(z)=1/\bigl((1-z)(1-z^2)(1-z^4)(1-z^8)\ldots\,\,\bigr)$.
(c) $B(z)=A(z)/(1-z)$, and we want to show that $A(z)=(1+z)B(z^2)$.
This follows from $A(z)=A(z^2)/(1-z)$.
\source{"Euler" [|euler-partitions|, \S50]; 1971 final exam.}

\ex:
Find a closed form for the double generating function
\begindisplay
M(w,z)=\sum_{m,n\ge0}\min(m,n)\,w^mz^n\,.
\enddisplay
Generalize your answer to obtain, for fixed $m\ge2$, a closed form for
\begindisplay
M(z_1,\ldots,z_m)=\sum_{n_1,\ldots,n_m\ge0}\min(n_1,\ldots,n_m)\,z_1^{n_1}\!
\ldots z_m^{n_m}\,.
\enddisplay
\answer $(1-wz)M(w,z)=\sum_{m,n\ge1}\bigl(\min(m,n)-\min(m{-}1,n{-}1)\bigr)
w^mz^n=\sum_{m,n\ge1}w^mz^n=wz/(1-w)(1-z)$.
In general,
\begindisplay
M(z_1,\ldots,z_m)={z_1\ldots z_m\over(1-z_1)\ldots(1-z_m)(1-z_1\ldots z_m)}\,.
\enddisplay
\source{"Carlitz" [|carlitz-max|].}

\ex:
Given positive integers $m$ and $n$, find closed forms for
\begindisplay
\sum_{1\le k_1<k_2<\cdots<k_m\le n}k_1k_2\ldots k_m \And
\hskip-1em\sum_{1\le k_1\le k_2\le\cdots\le k_m\le n}k_1k_2\ldots k_m\,.
\enddisplay
(For example, when $m=2$ and $n=3$ the sums are $1\cdt2+1\cdt3+2\cdt3$ and
$1\cdt1+1\cdt2+1\cdt3+2\cdt2+2\cdt3+3\cdt3$.) \Hint:
What are the coefficients of $z^m$ in the generating functions
$(1+a_1z)\ldots(1+a_nz)$ and
$1/(1-a_1z)\ldots(1-a_nz)$?
\answer The answers to the hint are
"!Stirling numbers as sums of products"
\begindisplay
\sum_{1\le k_1<k_2<\cdots<k_m\le n}\hskip-1.5em
 a_{k_1}a_{k_2}\ldots a_{k_m} \And\!\!\!
\sum_{1\le k_1\le k_2\le\cdots\le k_m\le n}\hskip-1.5em
a_{k_1}a_{k_2}\ldots a_{k_m}\,,
\enddisplay
respectively. Therefore: (a) We want the coefficient of $z^m$ in the product
$(1+z)(1+2z)\ldots(1+nz)$. This is the reflection of $(z+1)\_^n$, so it is
${n+1\brack n+1}+{n+1\brack n}z+\cdots+{n+1\brack 1}z^n$ and the answer is
$n+1\brack n+1-m$.
(b)~The coefficient of $z^m$ in $1/\bigl((1-z)(1-2z)\ldots(1-nz)\bigr)$ is
$m+n\brace n$ by \equ(7.|stirl2-gf|).
\source{[|knuth1|, exercise 1.2.9--18].}

\ex:
Express $\sum_k{n\choose k}(kF_{k-1}-F_k)(n-k)\?$ in closed form.
\answer The egf for $\<nF_{n-1}-F_n\>$ is $(z-1)\widehat F(z)$ where $\widehat F(z)=
\sum_{n\ge0}F_nz^n\!/n!=(e^{\phi z}-e^{\phihat z})/\mkern-1mu\sqrt5$. The
egf for $\<n\?@\>$ is $e^{-z}\!/(1-z)$. The product is
\begindisplay
5^{-1/2}\bigl(e^{(\phihat-1)z}-e^{(\phi-1)z}\bigr)
=5^{-1/2}(e^{-\phi z}-e^{-\phihat z})\,.
\enddisplay
We have $\widehat F(z)e^{-z}=-\widehat F(-z)$. So the answer is $(-1)^nF_n$.

\ex:
An {\it"up-down permutation"\/} of order $n$ is an arrangement
$a_1a_2\ldots a_n$ of the integers $\{1,2,\ldots,n\}$ that
goes alternately up and down:
\begindisplay
a_1<a_2>a_3<a_4>\cdots\,.
\enddisplay
For example, $35142$ is an up-down permutation of order~$5$. If $A_n$
denotes the number of up-down permutations of order~$n$, show that
the exponential generating function of $\<A_n\>$ is $(1+\sin z)/\!\cos z$.
\answer The number of up-down permutations with the largest
element~$n$ in position~$2k$ is
${n-1\choose2k-1}A_{2k-1}A_{n-2k}$. Similarly, the number of up-down permutations
with the smallest element~$1$ in position~$2k+1$ is $\smash{n-1\choose2k}A_{2k}
A_{n-2k-1}$, because down-up permutations and up-down permutations are equally
numerous. Summing over all possibilities gives
\begindisplay
2A_n=\sum_k{n-1\choose k}\,A_k\,A_{n-1-k}+2\[n=0]+\[n=1]\,.
\enddisplay
The egf $\widehat A$ therefore satisfies $2\widehat A'(z)=\widehat A(z)^2+1$ and
 $\widehat A(0)=1$;
the given function
solves this differential equation.
(Consequently $A_n=\vert E_n\vert+T_n$ is a "secant number" when $n$ is even,
"!Euler number" a "tangent number" when $n$ is odd.)
\source{"Andr\'e" [|andre|]; [|knuth3|, exercise 5.1.4--22].}

\ex:
A space probe has discovered that organic material on Mars
has DNA composed of five symbols, denoted by $(a,b,c,d,e)$, instead of
the four components in earthling DNA. The four pairs $cd$, $ce$, $ed$,
and $ee$ never occur consecutively in a string of "Martian DNA", but
"!DNA, Martian"
any string without forbidden pairs is possible.
(Thus $bbcda$ is forbidden but $bbdca$ is OK.) How many Martian DNA
strings of length~$n$ are possible? (When $n=2$ the answer is $21$,
because the left and right ends of a string are distinguishable.)
\answer Let $a_n$ be the number of Martian DNA strings that don't end
with $c$ or~$e$; let $b_n$ be the number that do. Then
\begindisplay \openup3pt
&a_n=3a_{n-1}+2b_{n-1}+\[n=0]\,,\qquad&b_n=2a_{n-1}+b_{n-1}\,;\cr
&A(z)=3zA(z)+2zB(z)+1\,,\qquad&B(z)=2zA(z)+zB(z)\,;\cr
&A(z)={1-z\over1-4z-z^2}\,,\qquad&B(z)={2z\over1-4z-z^2}\,;\cr
\enddisplay
and the total number is $[z^n](1+z)/(1-4z-z^2)=F_{3n+2}$.
\source{1974 final exam.}

\ex:
The {\it"Newtonian generating function"} of a sequence $\<g_n\>$ is defined to be
\begindisplay
\dot G(z)=\sum_n g_n{z\choose n}\,.
\enddisplay
Find a convolution formula that defines the relation between
sequences $\<f_n\>$, $\<g_n\>$, and $\<h_n\>$ whose Newtonian
generating functions are related by the equation $\dot F(z)\dot G(z)=\dot H(z)$.
Try to make your formula as simple and symmetric as possible.
\answer By \equ(5.|newton-series|), $g_n=\Delta^n \dot G(0)$.
The $n$th difference of a product can be written
\begindisplay
\Delta^n A(z)B(z)=\sum_k{n\choose k}\bigl(\Delta^k E^{n-k}A(z)\bigr)
 \bigl(\Delta^{n-k}B(z)\bigr)\,,
\enddisplay
and $E^{n-k}=(1+\Delta)^{n-k}=\sum_j{n-k\choose j}\Delta^j$.
Therefore we find
\begindisplay
h_n=\sum_{j,k}{n\choose k}{n-k\choose j}\,f_{j+k}\,g_{n-k}\,.
\enddisplay
This is a sum over all trinomial coefficients; it can be put into
the more symmetric form "!trinomial"
\begindisplay
h_n=\sum_{j+k+l=n}{n\choose j,k,l}\,f_{j+k}\,g_{k+l}\,.
\enddisplay

\ex:
Let $q_n$ be the number of possible outcomes when $n$ numbers
$\{x_1,\ldots,x_n\}$ are compared with each other. For example,
$q_3=13$ because the possibilities are "!sorting"
\begindisplay \displaythick=\thinmuskip
&x_1<x_2<x_3\,;\quad
x_1<x_2=x_3\,;\quad
x_1<x_3<x_2\,;\quad
x_1=x_2<x_3\,;\cr
&x_1=x_2=x_3\,;\quad
x_1=x_3<x_2\,;\quad
x_2<x_1<x_3\,;\cr
&x_2<x_1=x_3\,;\quad
x_2<x_3<x_1\,;\quad
x_2=x_3<x_1\,;\cr
&x_3<x_1<x_2\,;\quad
x_3<x_1=x_2\,;\quad
x_3<x_2<x_1\,.
\enddisplay
Find a closed form for the egf
$\widehat Q(z)=\sum_n q_nz^n\!/n!$. Also find sequences $\<a_n\>$, $\<b_n\>$,
$\<c_n\>$ such that
\begindisplay
q_n=\sum_{k\ge0}k^na_k=\sum_k{n\brace k}b_k=\sum_k{n\euler k}c_k\,,
\quad\hbox{for all $n>0$}.
\enddisplay
\answer Each partition into $k$ nonempty subsets can be ordered
\g The empty set\par is pointless.\g
in $k!$ ways, so $b_k=k!$. Thus $\widehat Q(z)=\sum_{n,k\ge0}{n\brace k}k!\,z^n\!/n!
=\sum_{k\ge0}(e^z-1)^k=1/(2-e^z)$. And this is the geometric
series $\sum_{k\ge0}e^{kz}\!/2^{k+1}$,
hence $a_k=1/2^{k+1}$. Finally, $c_k=2^k$; consider all permutations
when the $x$'s are distinct, change each `$>$' between subscripts to `$<$'
and allow each `$<$' between subscripts to become either `$<$' or `$=$'.
(For example, the permutation $x_1x_3x_2$ produces $x_1<x_3<x_2$ and
$x_1=x_3<x_2$, because $1<3>2$.)
\source{"Gross" [|gross|]; [|knuth3|, exercise 5.3.1--3].}

\ex:
Evaluate $\sum_{m,n>0}\[m\rp n]/m^2n^2$.
\answer This sum is $\sum_{n\ge1}r(n)/n^2$, where $r(n)$ is the number of
ways to write $n$ as a product of two relatively prime factors. If $n$~is
divisible by~$t$ distinct primes, $r(n)=2^t$. Hence $r(n)/n^2$ is
multiplicative and the sum is
\begindisplay
\prod_p\biggl(1+{2\over p^2}+{2\over p^4}\cdots\,\biggr)
&=\prod_p\biggl(1+{2\over p^2-1}\biggr)\cr
&=\prod_p\biggl({p^2+1\over p^2-1}\biggr)=\zeta(2)^2\!/\zeta(4)={5\over2}\,.
\enddisplay
\source{"de Bruijn" [|de-br-prob9|].}

\ex:
Evaluate
\begindisplay
\sum_{0\le k\le n/2}{n-2k\choose k}\left(-4\over27\right)^{\!k}
\enddisplay
in closed form. \Hint: $z^3-z^2+{4\over27}=(z+{1\over3})(z-{2\over3})^2$.
\answer Let $S_n=\sum_{0\le k\le n/2}{n-2k\choose k}\alpha^k$. Then
$S_n=S_{n-1}+\alpha S_{n-3}+\[n=0]$, and the generating function is
$1/(1-z-\alpha z^3)$. When $\alpha=-{4\over27}$, the hint tells us that
this has a nice factorization $1/(1+{1\over3}z)(1-{2\over3}z)^2$.
The general expansion theorem now tells us that
$S_n=({2\over3}n+c)({2\over3})^n
+{1\over9}(-{1\over3})^n$, and the remaining constant~$c$
turns out to be $8\over9$.

\ex:
Show that the numbers $U_n$ and $V_n$ of $3\times n$ domino
tilings, as given in \eq(|u-v-table|),
are closely related to the fractions
"!square root of 3"
in the "Stern--Brocot tree"
 that converge to $\sqrt3$.
\answer The Stern--Brocot representation of $\sqrt3$ is $R(LR^2)^\infty$,
because
\begindisplay
\sqrt3+1=2+{1\over\displaystyle1+{\strut1\over\sqrt3+1}}\,.
\enddisplay
The fractions are $1\over1$, $2\over1$, $3\over2$, $5\over3$, $7\over4$,
$12\over7$, $19\over11$, $26\over15$, \dots; they eventually
have the cyclic pattern
\begindisplay
{V_{2n-1}{+}V_{2n+1}\over U_{2n}},\,
{U_{2n}{+}V_{2n+1}\over V_{2n+1}},\,
{U_{2n+2}{+}V_{2n-1}\over U_{2n}{+}V_{2n+1}},\,
{V_{2n+1}{+}V_{2n+3}\over U_{2n+2}},\
\ldots\,.
\enddisplay
\source{"Waugh" and "Maxfield" [|rt-3|].}

\ex:
A certain sequence $\<g_n\>$ satisfies the recurrence
\begindisplay
ag_n+bg_{n+1}+cg_{n+2}+d=0\,,\qquad \hbox{integer $n\ge0$},
\enddisplay
for some integers $(a,b,c,d)$ with $\gcd(a,b,c,d)=1$. It also has the
closed form
\begindisplay
g_n=\bigl\lfloor\alpha(1+\sqrt2\,)^n\bigr\rfloor\,,
\qquad \hbox{integer $n\ge0$},
\enddisplay
for some real number $\alpha$ between $0$ and $1$. Find $a$, $b$, $c$, $d$,
and $\alpha$.
\answer We have $g_0=0$, and if $g_1=m$ the generating function satisfies
\begindisplay
aG(z)+bz^{-1}G(z)+cz^{-2}\bigl(G(z)-mz\bigr)+{d\over1-z}=0\,.
\enddisplay
Hence $G(z)=P(z)/(az^2+bz+c)(1-z)$ for some polynomial $P(z)$. Let
$\rho_1$ and $\rho_2$ be the roots of $cz^2+bz+a$, with $\vert\rho_1\vert
\ge\vert\rho_2\vert$. If $b^2-4ac\le0$ then $\vert\rho_1\vert^2=
\rho_1\rho_2=a/c$ is rational, contradicting the fact that $\root n\of{g_n}$
approaches $1+\sqrt2$. Hence $\rho_1=(-b+\sqrt{\,b^{\mathstrut2}-4ca})/2c=
1+\sqrt2@$; and this implies that $a=-c$, $b=-2c$, $\rho_2=1-\sqrt2$. The
generating function now takes the form
\begindisplay \openup3pt
G(z)&={z\bigl(m-(r+m)z\bigr)\over(1-2z-z^2)(1-z)}\cr
&={-r+(2m+r)z\over2(1-2z-z^2)}+{r\over2(1-z)}=mz+(2m-r)z^2+\cdots\,,
\enddisplay
where $r=d/c$. Since $g_2$ is an integer, $r$ is an integer. We also have
\begindisplay
g_n=\alpha(1+\sqrt2\,)^n+\hat\alpha(1-\sqrt2\,)^n+{\textstyle\half}r
=\bigl\lfloor\alpha(1+\sqrt2\,)^n\bigr\rfloor\,,
\enddisplay
and this can hold only if $r=-1$, because $(1-\sqrt2\,)^n$ alternates in
sign as it approaches zero.
Hence $(a,b,c,d)=\pm(1,2,-1,1)$.
Now we find $\alpha={1\over4}(1+\sqrt2\,m)$,
which is between $0$ and~$1$ only if $0\le m\le 2$. Each of these
values actually gives a solution; the sequences $\<g_n\>$ are
$\<0,0,1,3,8,\ldots\,\>$,
$\<0,1,3,8,20,\ldots\,\>$, and
$\<0,2,5,13,32,\ldots\,\>$.
\source{1984 final exam.}

\ex:
This is a problem about powers and parity.
\g\vskip-5pt "Kissinger", take note.\g
\itemitem{a}Consider the sequence $\<a_0,a_1,a_2,\ldots\,\>=
\<2,2,6,\ldots\,\>$ defined by the formula
\begindisplay
a_n=(1+\sqrt2@)^n+
(1-\sqrt2@)^n\,.
\enddisplay
Find a simple recurrence relation that is satisfied by this sequence.
\itemitem{b} Prove that $\bigl\lceil(1+\sqrt2@)^n\bigr\rceil\=
n$ \tmod2 for all integers $n>0$.
\itemitem{c} Find a number $\alpha$ of the form $(p+\sqrt q@)/2$,
where $p$ and~$q$ are positive integers, such that $\lfloor\alpha^n\rfloor
\=n$ \tmod2 for all integers $n>0$.
\answer (a) The denominator of $\bigl(1/\bigl(1-(1+\sqrt2@)z\bigr)
+1/\bigl(1-(1-\sqrt2@)z\bigr)\bigr)$ is $1-2z-z^2$; hence
$a_n=2a_{n-1}+a_{n-2}$ for $n\ge2$. (b)~True because $a_n$ is even
and $-1<1-\sqrt2<0$. (c)~Let
\begindisplay
b_n=\biggl({p+\sqrt q\over2}\biggr)^n+
    \biggl({p-\sqrt q\over2}\biggr)^n\,.
\enddisplay
We would like $b_n$ to be odd for all $n>0$, and $-1<(p-\sqrt q@)/2<0$.
Working as in part~(a), we find $b_0=2$, $b_1=p$, and
$b_n=pb_{n-1}+{1\over4}(q-p^2)b_{n-2}$ for $n\ge2$.
One satisfactory solution has $p=3$ and $q=17$.
\source{"Waterhouse" [|waterhouse|].}

\subhead Bonus problems

\ex: Continuing exercise |triangulations|, consider the sum of all ways
to decompose "polygons" into polygons:
\begindisplay \unitlength=3.33333pt \openup7pt \advance\medmuskip1mu%
 \advance\abovedisplayskip-7pt
Q&=\dgn \,+\,\tri{}\,+\,\sqr{}\,+\,\sqr{\put(4.5,0){\line(-1,1){4.5}}}
	\,+\,\sqr{\put(0,0){\line(1,1){4.5}}}\cr
&\qquad	+\pnt{}+\pnt{\put(4,0){\line(-2,5){2}}}
	+\pnt{\put(4,0){\line(-5,3){5}}}
	+\pnt{\put(5,3){\line(-1,0){6}}}
	+\pnt{\put(0,0){\line(2,5){2}}}
	+\pnt{\put(0,0){\line(5,3){5}}}
	+\pnt{\put(4,0){\line(-5,3){5}}\put(4,0){\line(-2,5){2}}}
	+\cdots\,.
\enddisplay
Find a symbolic equation for $Q$ and use it to find a generating function
for the number of ways to draw nonintersecting diagonals inside a
convex $n$-gon.
(Give a closed form for the generating function as a function of~$z$; you
need not find a closed form for the coefficients.)
\answer Extending the multiplication idea of exercise |triangulations|, we
have
\begindisplay \advance\abovedisplayskip10pt
\unitlength=3.33333pt
\def\\(#1,#2){\put(#1,#2){\makebox(0,0){$Q$}}}
\def\sqr#1{\cpic(4,4)(0,0) \put(0,0){\line(0,1){4}}
 \put(0,0){\line(1,0){4}} \put(0,4){\line(1,0){4}}\put(4,0){\line(0,1){4}}
 #1\endcpic}%
Q=\dgn +\quad\tri{\\(-1.1,3)\\(5,3)}\quad
+\;\quad\sqr{\\(-2,2)\\(2,6)\\(6,2)}\;\quad
+\,\quad\pnt{\\(-2.6,1)\\(-1.3,5.4)\\(4.9,5.2)\\(6.7,1)}\quad\,
+\cdots\,.
\enddisplay
Replace each $n$-gon by $z^{n-2}$.
This substitution
behaves properly under multiplication,
because the pasting operation takes
an $m$-gon and an $n$-gon into an $(m+n-2)$-gon.
 Thus the generating function is
\begindisplay
Q=1+zQ^2+z^2Q^3+z^3Q^4+\cdots=1+{zQ^2\over1-zQ}
\enddisplay
and the quadratic formula gives $Q=\bigl(1+z-\sqrt{@1-6z+z^{\mathstrut2}}
\,\bigr)/4z$.
The coefficient of $z^{n-2}$ in this power series is the number of
ways to put nonoverlapping diagonals into a convex $n$-gon. These
coefficients apparently
have no closed form in terms of other quantities that we have discussed in
\g\vskip-22pt Give me "Legendre" polynomials and I'll~give you a "closed~form".\g
this book,
but their asymptotic behavior is known [|knuth1|, exercise
2.2.1--12].\par
Incidentally, if each $n$-gon in $Q$ is replaced by $wz^{n-2}$ we get
\begindisplay
Q={1+z-\sqrt{\,1-\smash{(4w+2)}z+z^{2\mathstrut}}\over 2(1+w)z}\,,
\enddisplay
a formula in which the coefficient of $w^mz^{n-2}$ is the number of
ways to divide an $n$-gon into $m$ polygons by nonintersecting diagonals.
\source{"Schr\"oder" [|schroeder|]; [|knuth1|, exercise 2.3.4.4--31].}

\ex:
Prove that the product
\begindisplay
2^{mn/2}\prod\twoconditions{1\le j\le m}{1\le k\le n}
\biggl(\Bigl(\cos^2{j\pi\over m+1}\Bigr)\Domv^2
   +\Bigl(\cos^2{k\pi\over n+1}\Bigr)\Domh^2\biggr)^{1/4}
\enddisplay
is the generating function for tilings of an $m\times n$ rectangle
with dominoes. (There are $mn$ factors, which we can imagine are
written in the $mn$ cells of the rectangle. If $mn$ is odd, the
middle factor is zero. The coefficient of $\Domv^j\Domh^k$ is the
number of ways to do the tiling with $j$ vertical and $k$ horizontal
dominoes.) \Hint: This
\g Is this a hint or a warning?\g
 is a difficult problem, really beyond the
scope of this book. You may wish to simply verify the formula
in the case $m=3$, $n=4$.
\answer The key first step is to observe that the square of the number
of ways is the number of cycle patterns of a certain kind, generalizing
exercise |fib-squared|. These can be enumerated by evaluating the
determinant of a matrix whose eigenvalues are not difficult to determine.
When $m=3$ and $n=4$, the fact that $\cos 36^\circ=\phi/2$ is helpful
(exercise 6.|cos36|).
\source{"Fisher" [|fisher|]; "Percus" [|percus|, pp.~89--123];
 "Stanley"~[|stanley-dimers|].}

\ex:
Prove that the polynomials defined by the recurrence
\def\\{\atopwithdelims\vert\vert}
\begindisplay
p_n(y)&=\Bigl(y-{1\over4}\Bigr)^{\!n}-\,\sum_{k=0}^{n-1}{2n\choose2k}\!
 \left(-1\over4\right)^{\!n-k}\!p_k(y)\,,\quad\hbox{integer $n\ge0$},
\enddisplay
have the form $p_n(y)=\sum_{m=0}^n{n\\m}y^n$, where $n\\m$ is a positive integer
for $1\le m\le n$. \Hint: This exercise is very instructive but not very easy.
\answer The first few cases are $p_0(y)=1$, $p_1(y)=y$, $p_2(y)=y^2+y$,
$p_3(y)=y^3+3y^2+3y$. Let $p_n(y)=q_{2n}(x)$ where $y=x(1-x)$; we seek
a generating function that defines $q_{2n+1}(x)$ in a convenient way.
One such function
 is $\sum_n q_n(x)z^n\!/n!=2e^{ixz}\!/(e^{iz}+1)$, from which it follows
that $q_n(x)=i^nE_n(x)$, where $E_n(x)$ is called an "Euler polynomial".
We have $\sum(-1)^xx^n\,\delta x=\half(-1)^{x+1}E_n(x)$, so Euler polynomials
are analogous to Bernoulli polynomials, and they have factors analogous to
those in \equ(6.|bern-factors|). By exercise 6.|z/ez+1| we have $nE_{n-1}(x)
=\sum_{k=0}^n{n\choose k}B_kx^{n-k}(2-2^{k+1})$; this polynomial
has integer coefficients
by exercise~6.|prove-staudt-clausen|. Hence $q_{2n}(x)$, whose coefficients
have denominators that are powers of~$2$, must have integer coefficients.
Hence $p_n(y)$ has integer coefficients. Finally, the relation
$(4y-1)p_n''(y)+2p_n'(y)=2n(2n-1)p_{n-1}(y)$ shows that
\def\\{\atopwithdelims\vert\vert}
\begindisplay
2m(2m-1){n\\m}=m(m+1){n\\m+1}+2n(2n-1){n-1\\m-1}\,,
\enddisplay
and it follows that the $n\\m$'s are positive. (A similar proof shows that
the related quantity $(-1)^n(2n+2)
E_{2n+1}(x)/(2x-1)$ has positive integer coefficients, when expressed as
an $n$th degree polynomial in~$y$.) It can be shown that $n\\1$ is
the "Genocchi number" $(-1)^{n-1}(2^{2n+1}-2)B_{2n}$
(see exercise 6.|genocchi-answer|), and that
${n\\n-1}={n\choose2}$, ${n\\n-2}=2{n+1\choose4}+3{n\choose4}$, etc.
\source{"Hammersley" [|hammersley-undergrad|].}

\ex: The sequence of {\it "pentagonal numbers"\/} $\<1,5,12,22,\ldots\,\>$
generalizes the triangular and square numbers in an obvious way:
\begindisplay
\unitlength=3.33333pt
\beginpicture(24,18)(-12,-18)
\put(0,0){\line(4,-3){12}}
\put(0,0){\line(-4,-3){12}}
\put(4,-3){\line(-1,-3)1}
\put(-4,-3){\line(1,-3)1}
\put(8,-6){\line(-1,-3)2}
\put(-8,-6){\line(1,-3)2}
\put(12,-9){\line(-1,-3)3}
\put(-12,-9){\line(1,-3)3}
\put(-3,-6){\line(1,0)6}
\put(-6,-12){\line(1,0){12}}
\put(-9,-18){\line(1,0){18}}
\multiput(0,0)(4,-3)3{\disk{1.3}}
\multiput(12,-9)(-1,-3)3{\disk{1.3}}
\multiput(9,-18)(-6,0)3{\disk{1.3}}
\multiput(-9,-18)(-1,3)3{\disk{1.3}}
\multiput(-12,-9)(4,3)3{\disk{1.3}}
\multiput(-3,-6)(6,0)2{\disk{1.3}}
\multiput(-6,-12)(6,0)3{\disk{1.3}}
\multiput(-7,-9)(14,0)2{\disk{1.3}}
\endpicture
\enddisplay
Let the $n$th triangular number be $T_n=n(n+1)/2$; let the $n$th pentagonal
number be $P_n=n(3n-1)/2$; and let $U_n$ be the $3\times n$ domino-tiling
number defined in \eq(|u-closed|). Prove that the triangular number
$T_{(U_{4n+2}-1)/2}$ is also a pentagonal number.
\Hint: $3U_{2n}^2=(V_{2n-1}+V_{2n+1})^2+2$.
\answer It is $P_{(1+V_{4n+1}+V_{4n+3})/6}$.
Thus, for example, $T_{20}=P_{12}=210$; $T_{285}=P_{165}=40755$.
\source{"Euler" [|euler-algebra|, part 2, section 2, chapter 6, \S91].}

\ex:
Consider the following curious construction:
\begindisplay \def\preamble{&\hfil$##$\hfil\enspace} \def\\{\phantom{00}}
\\&\\&\\&\\&\\&\\&\\&\\\cr
\noalign{\vskip-\baselineskip}
1&2&3&4&5&6&7&8&9&10&11&12&13&14&15&16&\ldots\cr
1&2&3&4& &6&7&8&9&  &11&12&13&14&  &16&\ldots\cr
1&3&6&10& &16&23&31&40&  &51&63&76&90&  &106&\ldots\cr
1&3&6&  & &16&23&31&  &  &51&63&76&  &  &106&\ldots\cr
1&4&10&  & &26&49&80&  &  &131&194&270&  &  &376&\ldots\cr
1&4&  &  & &26&49&  &  &  &131&194&   &  &  &376&\ldots\cr
1&5&  &  & &31&80&  &  &  &211&405&   &  &  &781&\ldots\cr
1& &  &  & &31&  &  &  &  &211&   &   &  &  &781&\ldots\cr
1& &  &  & &32&  &  &  &  &243&   &   &  &  &\kern-.5em1024&\ldots\cr
\enddisplay
(Start with a row containing all the positive integers. Then delete
every $m$th column; here $m=5$. Then replace the remaining entries by
partial sums. Then delete every $(m-1)$st column. Then replace with
partial sums again, and so on.)
Use generating
functions to show that the final result is the
sequence of $m$th powers. For example, when $m=5$ we get
$\<1^5,2^5,3^5,4^5,\ldots\,\>$ as shown.
\answer Let $E_k$ be the operation on power series that sets all coefficients
to zero except those of $z^n$
where $n\bmod m=k$. The stated
construction is equivalent to the operation
\begindisplay
E_0\,S\,E_0\,S\,(E_0+E_1)\,S\,\ldots\,S\,(E_0+E_1+\cdots+E_{m-1})
\enddisplay
applied to $1/(1-z)$, where $S$ means ``multiply by $1/(1-z)$.''
There are $m!$ terms
\begindisplay
E_0\,S\,E_{k_1}\,S\,E_{k_2}\,S\,\ldots\,S\,E_{k_m}
\enddisplay
where $0\le k_j<j$, and every such term evaluates to $z^{rm}\!/(1-z^m)^{m+1}$ if
$r$ is the number of places where $k_j<k_{j+1}$. Exactly $m\euler r$ terms
have a given value of~$r$, so the coefficient of $z^{mn}$ is
$\sum_{r=0}^{m-1}{m\euler r}{n+m-r\choose m}=(n+1)^m$ "!Eulerian numbers"
by \equ(6.|expand-ord-to-consec|).
(The fact that operation $E_k$ can be expressed with complex roots
of unity seems to be of no help in this problem.)
\source{"Moessner" [|moessner|].}

\ex:
Prove that if the power series $F(z)$ and $G(z)$ are "differentiably
finite" (as defined in exercise~|d-finite-defined|), then so are $F(z)+G(z)$
and $F(z)@G(z)$.
\answer Suppose that $P_0(z)@F(z)+\cdots+P_m(z)@F^{(m)}(z)=Q_0(z)@G(z)+\cdots+
Q_n(z)@G^{(n)}(z)=0$, where $P_m(z)$ and $Q_n(z)$ are nonzero. (a)~Let
$H(z)=F(z)+G(z)$. Then there are rational functions $R_{k,l}(z)$ for $0\le l<m+n$
such that $H^{(k)}(z)=R_{k,0}(z)@F^{(0)}(z)+\cdots+R_{k,m-1}(z)@F^{(m-1)}(z)
+R_{k,m}(z)@G^{(0)}(z)+\cdots+R_{k,m+n-1}(z)@G^{(n-1)}(z)$. The $m+n+1$ vectors
$\bigl(R_{k,0}(z),\ldots,R_{k,m+n-1}(z)\bigr)$ are linearly dependent in the
$(m+n)$-dimensional vector space whose components are rational functions;
hence there are rational functions $S_l(z)$, not all zero, such that
$S_0(z)@H^{(0)}(z)+\cdots+S_{m+n}(z)@H^{(m+n)}(z)=0$. (b)~Similarly, let
$H(z)=F(z)@G(z)$. There are rational $R_{k,l}(z)$ for $0\le l<mn$ with
$H^{(k)}(z)=\sum_{i=0}^{m-1}\sum_{j=0}^{n-1}R_{k,ni+j}(z)@F^{(i)}(z)@G^{(j)}(z)$,
hence $S_0(z)@H^{(0)}(z)+\cdots+S_{mn}(z)@H^{(mn)}(z)=0$ for some rational
$S_l(z)$, not all zero. (A similar proof shows that if $\<f_n\>$ and $\<g_n\>$
are polynomially recursive, so are $\<f_n+g_n\>$ and $\<f_ng_n\>$.
Incidentally, there is no similar result for quotients; for example,
$\cos z$ is differentiably finite, but $1/\!\cos z$ is not.)
\source{"Stanley" [|stanley-d-finite|].}

\subhead Research problems

\ex:
Prove that there is no ``simple "closed form"'' for the coefficient of $z^n$
in $(1+z+z^2)^n$, as a function of~$n$, in some large class of
``simple closed forms.\qback''
\answer "Euler" [|euler-middle|] showed that this number is also
$[z^n]\,1\mskip-1mu/\mskip-2mu\sqrt{1{-}2z{-}3z^{2\mathstrut}}$, and
he gave the formula $t_n=\sum_{k\ge0}n\_{2k}/k!^2=\sum_k{n\choose k}{n-k
\choose k}$. He also
discovered a ``memorable failure of induction'' "!induction, failure"
while examining these numbers: Although $3t_n-t_{n+1}$ is equal to
$F_{n-1}(F_{n-1}+1)$ for $0\le n\le8$, this empirical law mysteriously
breaks down when $n$ is $9$ or more! George "Andrews"~[|andrews-euler|]
has explained the mystery by showing that the sum $\sum_k[z^{n+10k}]\,
(1+z+z^2)^n$ can be expressed as a closed form in terms of "Fibonacci
numbers".
\par
H.\thinspace S. "Wilf" observes that $[z^n]\,(a+bz+cz^2)^n$=
$[z^n]\,1/f(z)$, where $f(z)=\sqrt{1-2bz+(b^{2\mathstrut}-4ac)z^2}$ (see
[|wilfology|, page~159]), and it follows that the coefficients satisfy
\begindisplay
(n+1)A_{n+1}-(2n+1)b@A_n+n(b^2-4ac)A_{n-1}=0\,.
\enddisplay
The algorithm of "Petkov\v{s}ek" [|petk|] can be used to prove that this
recurrence has a closed form solution as a finite sum of "hypergeometric
terms" if and only if $abc(b^2-4ac)=0$. Therefore in particular, the middle
trinomial coefficients have no such closed form.
\g\vskip-22pt Give me "Legendre polynomials" and I'll~give you a closed~form.\g
The next step is presumably to extend this result to a larger class of
closed forms (including harmonic numbers and/or Stirling numbers, for example).
\source{"Euler" [|euler-middle|].}

\ex:
Prove or disprove:
If all the coefficients of $G(z)$ are either $0$ or $1$,
and if all the coefficients of $G(z)^2$ are less than some constant~$M$, then
infinitely many of the coefficients of $G(z)^2$ are zero.
\answer (Paul "Erd\H os" "!reward"
currently offers \$500 for a solution.)
\source{[|erdos-graham|, p.~48] credits P. "Erd\H os" and P.~"Tur\'an".}

