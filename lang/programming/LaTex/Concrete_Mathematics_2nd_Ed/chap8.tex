% Chapter 8 of Concrete Mathematics
% (c) Addison-Wesley, all rights reserved.
\input gkpmac
\refin bib

\refin chap2
\refin chap5
\refin chap6
\refin chap7
\refin chap9

\pageno=381
\beginchapter 8 Discrete Probability

THE ELEMENT OF CHANCE enters into many of our attempts to understand the world
we live in. A mathematical {\it"theory of probability"\/} allows us to
"!probability, theory of"
calculate the likelihood of complex events if we assume that the events
are governed by appropriate axioms. This theory has significant applications
in all branches of science, and it has strong connections with the
techniques we have studied in previous chapters.

Probabilities are called ``"discrete"'' if we can compute the probabilities
of all events by summation instead of by integration. We are getting pretty good
at sums, so it should come as no great surprise that we are ready to
apply our knowledge to some interesting calculations of probabilities
and averages.

\def\dieone{\die{\diedot22}}
\def\dietwo{\die{\diedot11\diedot33}}
\def\diethree{\die{\diedot11\diedot22\diedot33}}
\def\diefour{\die{\diedot11\diedot33\diedot13\diedot31}}
\def\diefive{\die{\diedot11\diedot22\diedot33\diedot13\diedot31}}
\def\diesix{\die{\diedot11\diedot12\diedot13\diedot31\diedot32\diedot33}}
\def\dieseven{\die{\diedot11\diedot12\diedot13\diedot31\diedot32\diedot33\diedot22}}
\def\dieeight{\die{\diedot11\diedot12\diedot13\diedot31\diedot32\diedot33%
 \diedot21\diedot23}}
\def\die#1{{\unitlength=2.5pt\beginpicture(5,4)(-.5,1)
 \put(0,0){\line(0,1)4}
 \put(0,0){\line(1,0)4}
 \put(0,4){\line(1,0)4}
 \put(4,0){\line(0,1)4}#1\endpicture}}
\def\diedot#1#2{\put(#1,#2){\disk1}}

\vskip-1.1pt % have to make this page slightly tight! (DEK May 88)
\beginsection 8.1 Definitions

Probability theory starts with the idea of a {\it"probability space"},
\g(Readers unfamiliar with probability theory will, with high
probability, benefit from a perusal of "Feller"'s classic
introduction to the subject [|feller|].)\g
which is a set~$\Omega$ of all things that can happen in a given
problem together with a rule that assigns a probability $\Pr(\omega)$
to each elementary event $\omega\in\Omega$. The probability $\Pr(\omega)$
must be a nonnegative real number, and the condition
\begindisplay \advance\belowdisplayskip-1.2pt
\sum_{\omega\in\Omega}\Pr(\omega)=1
\eqno\eqref|all-probs|
\enddisplay
must hold in every discrete probability space. Thus, each value $\Pr(\omega)$
must lie in the interval $[@0\dts1]$.
We speak of Pr as a {\it"probability distribution"}, because it distributes
a total probability of~$1$ among the events~$\omega$.

Here's an example: If we're
rolling a pair of "dice", the set~$\Omega$
of elementary events is $D^2=\{\,\dieone\dieone,\,\dieone\dietwo,\,\ldots,\,
\diesix\diesix\,\}$, where
\begindisplay
D=\{\,\dieone,\,\dietwo,\,\diethree,\,\diefour,\,\diefive,\,\diesix\,\}
\enddisplay
is the set of all six ways that a given die can land. Two rolls such as
\g Never say die.\g
\smash{\dieone\dietwo} and \smash{\dietwo\dieone}
are considered to be distinct;
hence this probability space has a total of $6^2=36$ elements.

We usually assume that dice are ``fair''\dash---that each of the
six possibilities for a particular die has probability~$1\over6$, and that
"!fair dice" "!loaded dice"
each of the 36 possible rolls in~$\Omega$ has probability~$1\over36$.
\g Careful: They might go off.\g
But we can also consider ``loaded'' dice in which there is a different
distribution of probabilities. For example, let
\begindisplay
\Pr_1(\dieone)&=\Pr_1(\diesix)=\textstyle{1\over4}\,;\cr
\Pr_1(\dietwo)&=\Pr_1(\diethree)=\Pr_1(\diefour)=\Pr_1(\diefive)=
\textstyle{1\over8}\,.\cr
\enddisplay
Then $\sum_{d\in D}\Pr_1(d)=1$, so $\Pr_1$ is a probability distribution
on the set~$D$,
 and we can assign probabilities to the
elements of $\Omega=D^2$ by the rule
\begindisplay
\Pr_{11}(d\,d')=\Pr_1(d)\,\Pr_1(d')\,.
\eqno\eqref|pr1-def|
\enddisplay
For example, $\Pr_{11}(\diesix\diethree)={1\over4}\cdt{1\over8}={1\over32}$.
This is a valid distribution because
\begindisplay \openup3pt
\sum_{\omega\in\Omega}\Pr_{11}(\omega)
&=\sum_{dd'\in D^2}\Pr_{11}(d\,d')
 =\sum_{d,d'\in D}\Pr_1(d)\,\Pr_1(d')\cr
&=\sum_{d\in D}\Pr_1(d)\,\sum_{d'\in D}\Pr_1(d')=1\cdot1=1\,.\cr
\enddisplay
We can also consider the case of one fair die and one loaded die,
\begindisplay
\Pr_{01}(d\,d')=\Pr_0(d)\,\Pr_1(d')\,,\qquad\hbox{where $\Pr_0(d)={1\over6}$},
\eqno\eqref|pr2-def|
\enddisplay
in which case $\Pr_{01}(\diesix\diethree)={1\over6}\cdt{1\over8}={1\over48}$.
Dice in the ``real world'' can't really be expected to turn up equally often
\g If all sides of a cube were identical, how could we tell which
side is face up?\g
on each side, because they aren't perfectly symmetrical; but $1\over6$ is
usually pretty close to the truth.

An {\it"event"\/} is a subset of $\Omega$. In dice games, for example,
the set
\begindisplay
\{\,\dieone\dieone,\,\dietwo\dietwo,\,\diethree\diethree,\,
\diefour\diefour,\,\diefive\diefive,\,\diesix\diesix\,\}
\enddisplay
is the event that ``doubles are thrown.\qback'' 
 The individual elements~$\omega$ of~$\Omega$
are called {\it"elementary events"\/} because they cannot be decomposed
into smaller subsets; we can think of $\omega$ as a one-element
event~$\{\omega\}$.

The probability of an
event~$A$ is defined by the formula
\begindisplay
\Pr\prp(\omega\in A)=\sum_{\omega\in A}\Pr(\omega)\,;
\eqno\eqref|prob-def|
\enddisplay
and in general if $R(\omega)$ is any statement about $\omega$, we write
`$\Pr\bigl(R(\omega)\bigr)$' for the sum of all $\Pr(\omega)$ such that
$R(\omega)$ is true. Thus, for example, the
probability of doubles with fair dice is ${1\over36}+{1\over36}+{1\over36}+
{1\over36}+{1\over36}+{1\over36}={1\over6}$; but
when both dice are loaded with probability distribution~$\Pr_1$
it is ${1\over16}+{1\over64}+{1\over64}+
{1\over64}+{1\over64}+{1\over16}={3\over16}>{1\over6}$. Loading the dice makes
the event ``doubles are thrown'' more probable.

(We have been using $\sum$-notation in a more general sense here than
defined in Chapter~2: The sums in
\eq(|all-probs|) and \eq(|prob-def|) occur over all elements~$\omega$ of an
arbitrary set, not over integers only. However, this new development
is not really alarming; we can agree to
use special notation under a $\sum$
whenever nonintegers are intended, so there will be no confusion with
our ordinary conventions. The other definitions in Chapter~2 are still valid;
in particular, the definition of infinite sums in that chapter
gives the appropriate
interpretation to our sums when
the set~$\Omega$ is infinite. Each probability is nonnegative, and the
sum of all probabilities is bounded, so the probability of event~$A$ in
\eq(|prob-def|) is well defined for all subsets $A\subseteq\Omega$.)

A {\it"random variable"\/} is a function defined on the elementary
events~$\omega$ of a probability space. For example, if $\Omega=D^2$
we can define  $S(\omega)$ to be the sum of the spots on the dice roll~$\omega$,
so that $S(\diesix\diethree)=6+3=9$. The probability that the spots total seven
is the probability of the event $S(\omega)=7$, namely
\begindisplay \openup3pt
&\Pr(\dieone\diesix)+\Pr(\dietwo\diefive)+\Pr(\diethree\diefour)\cr
&\qquad+\Pr(\diefour\diethree)+\Pr(\diefive\dietwo)+\Pr(\diesix\dieone)\,.
\enddisplay
With fair dice ($\Pr=\Pr_{00}$), this happens with probability~$1\over6$;
but with loaded dice ($\Pr=\Pr_{11}$), it happens with probability
${1\over16}+{1\over64}+{1\over64}+
{1\over64}+{1\over64}+{1\over16}={3\over16}$, the same as we observed
for doubles.

It's customary to drop the `$(\omega)$' when we talk about random variables,
because there's usually only one probability space involved when we're
working on any particular problem. Thus we say simply `$S=7$' for the
event that a $7$ was rolled, and `$S=4$' for the event
$\{\,\dieone\diethree,\,\dietwo\dietwo,\,\diethree\dieone\,\}$.

A random variable can be characterized by the probability distribution of
its values. Thus, for example, $S$~takes on eleven possible values
$\{2,3,\ldots,12\}$, and we can tabulate the probability that $S=s$
for each $s$ in this set:
\begindisplay \let\preamble=\tablepreamble \offinterlineskip
s&&2&3&4&5&6&7&8&9&10&11&12&\kern-1em\kern1pt\cr
\noalign{\hrule}
\omit&height 1pt\cr
\Pr_{00}\prp(S=s)&&{1\over36}&{2\over36}&{3\over36}&{4\over36}&
 {5\over36}&{6\over36}&{5\over36}&{4\over36}&
 {3\over36}&{2\over36}&{1\over36}\cr
\omit&height 2pt\cr
\noalign{\hrule}
\Pr_{11}\prp(S=s)&&{4\over64}&{4\over64}&{5\over64}&{6\over64}&
 {7\over64}&{12\over64}&{7\over64}&{6\over64}&
 {5\over64}&{4\over64}&{4\over64}\cr
\enddisplay
If we're working on a problem that involves only the random variable~$S$
and no other properties of dice, we can compute the answer from these
probabilities alone, without regard to the details of the set $\Omega=D^2$.
In fact, we could define the probability space to be the smaller
set $\Omega=\{2,3,\ldots,12\}$, with whatever probability distribution
$\Pr(s)$ is desired. Then `$S=4$' would be an elementary event.
Thus we can often ignore the underlying probability space~$\Omega$
and work directly with random variables and their distributions.

If two random variables $X$ and $Y$ are defined over the same probability
space~$\Omega$, we can characterize their behavior without knowing
everything about~$\Omega$ if we know the ``"joint distribution"''
\g Just Say No.\g
\begindisplay
\Pr\prp(X=x\ {\rm and}\ Y=y)
\enddisplay
for each $x$ in the range of\/~$X$ and each $y$ in the range of\/~$Y$.
"!independent random variables"
We say that $X$ and $Y$ are {\it independent\/} random variables if
\begindisplay
\Pr\prp(X=x\ {\rm and}\ Y=y)=\Pr\prp(X=x)\cdot\Pr\prp(Y=y)
\eqno\eqref|indep-def|
\enddisplay
for all $x$ and $y$.  Intuitively, this means that the value of\/~$X$
has no effect on the value of\/~$Y$.

For example, if $\Omega$ is the set of dice rolls $D^2$, we can let
$S_1$ be the number of spots on the first die and $S_2$ the number
of spots on the second. Then the random variables $S_1$ and~$S_2$
are independent with respect to each of the probability distributions
$\Pr_{00}$, $\Pr_{11}$, and $\Pr_{01}$ discussed earlier,
 because we defined the dice probability for each elementary
event~$dd'$ as a product of a probability for~$S_1=d$ multiplied by
a probability for~$S_2=d'$. We could have defined probabilities differently
so that, say,
\g\vskip19pt A dicey inequality.\g
\begindisplay
\Pr(\dieone\diefive)\,\big/\,\Pr(\dieone\diesix) \ne
\Pr(\dietwo\diefive)\,\big/\,\Pr(\dietwo\diesix)\,;
\enddisplay
but we didn't do that, because different dice aren't supposed to
influence each other. With our definitions, both of these ratios are
$\Pr\prp(S_2=5)/\Pr\prp(S_2=6)$.

We have defined $S$ to be the sum of the two spot values, $S_1+S_2$.
Let's consider another random variable $P$, the product $S_1S_2$.
Are $S$ and~$P$ independent? Informally, no; if we are told that
$S=2$, we know that $P$ must be~$1$. Formally, no again, because
the independence condition \eq(|indep-def|) fails spectacularly (at least
in the case of fair dice):
For all legal values of $s$ and~$p$, we have $0<\Pr_{00}\prp(S=s)\cdt
\Pr_{00}\prp(P=p)\le{1\over6}\cdt{1\over9}$; this can't equal
$\Pr_{00}\prp(S=s\,{\rm and}\,P=p)$,
which is a multiple of~$1\over36$.

If we want to understand the typical behavior of a given random variable,
we often ask about its ``average'' value. But the notion of ``average''
is ambiguous; people generally speak about three different kinds of
averages when a sequence of numbers is given:
\smallskip
\item{$\bullet$}the {\it"mean"\/} (which
is the sum of all values, divided by the number of values); 
\item{$\bullet$}the {\it"median"\/} (which
is the middle value, numerically); 
\item{$\bullet$}the {\it"mode"\/} (which
is the value that occurs most often).
\smallskip\noindent
For example, the mean of $(3,1,4,1,5)$ is ${3+1+4+1+5\over5}=2.8$; the
median is~$3$; the mode is~$1$.

But probability theorists usually work with random variables instead of with
sequences of numbers, so we want to define the notion of an ``average''
for random variables too.
Suppose we repeat an experiment over and over again, making independent
trials in such a way that each value of\/~$X$ occurs with a frequency
approximately proportional to its probability. (For example, we might
roll a pair of dice many times, observing the values of $S$ and/or $P$.)
We'd like to define the average value of a random variable so that such
experiments will usually produce a sequence of numbers whose mean, median,
or mode is approximately the same as the mean, median, or mode of\/~$X$,
according to our definitions.

 Here's how it can be done: The {\it"mean"\/}
of a random real-valued variable\/~$X$ on a probability space~$\Omega$
is defined to be
\begindisplay
\sum_{x\in X(\Omega)}x\cdt\Pr\prp(X=x)
\eqno\eqref|mean1|
\enddisplay
if this potentially infinite sum exists.
(Here $X(\Omega)$ stands for the set of all values that $X$ can assume.)
The {\it"median"\/} of\/~$X$ is defined to be the set of all~$x$ such that
\begindisplay
\Pr\prp(X\le x)\ge\half\And\Pr\prp(X\ge x)\ge\half\,.
\eqno\eqref|median1|
\enddisplay
And the {\it"mode"\/} of\/~$X$ is defined to be the set of all~$x$ such that
\begindisplay
\Pr\prp(X=x)\ge\Pr\prp(X=x')\qquad\hbox{for all $x'\in X(\Omega)$}.
\eqno
\enddisplay
In our dice-throwing example, the mean of $S$ turns out to be
$2\cdt{1\over36}+3\cdt{2\over36}+\cdots+12\cdt{1\over36}=7$ in distribution
$\Pr_{00}$, and it also turns out to be~$7$ in distribution~$\Pr_{11}$.
The median and mode both turn out to be $\{7\}$ as well, in both
distributions. So $S$ has the same average under all three definitions.
On the other hand the
$P$ in distribution~$\Pr_{00}$ turns out to have a mean value
 of ${49\over4}=12.25$; its median
is~$\{10\}$, and its mode is $\{6,12\}$. The mean of\/~$P$ is unchanged if
we load the dice with distribution $\Pr_{11}$, but the median drops to~$\{8\}$
and the mode becomes $\{6\}$ alone.

Probability theorists have a special name and notation for the mean
of a random variable: They call it the {\it"expected value"}, and write
\begindisplay
EX=\sum_{\omega\in\Omega}X(\omega)\Pr(\omega)\,.
\eqno\eqref|mean2|
\enddisplay
In our dice-throwing example, this sum has 36 terms (one for each element
of~$\Omega$), while \eq(|mean1|) is a sum of only eleven terms. But
both sums have the same value, because they're both equal to
\begindisplay
\sum\twoconditions{\omega\in\Omega}{x\in X(\Omega)}
 x\Pr(\omega)\bigi[x=X(\omega)\bigr]\,.
\enddisplay

The mean of
a random variable turns out to be
\g I get it:\par On average, ``average'' means ``mean.\qback''\g
more meaningful in applications than the other kinds of averages, so
we shall largely forget about medians and modes from now on.
We will use the terms ``expected value,\qback'' ``mean,\qback'' and ``average''
almost interchangeably in the rest of this chapter.

If $X$ and $Y$ are any two random variables defined on the same probability
space, then $X+Y$ is also a random variable on that space. By formula
\eq(|mean2|), the average of their sum is the sum of their averages:
\begindisplay
E(X+Y)=\sum_{\omega\in\Omega}\bigl(X(\omega)+Y(\omega)\bigr)\Pr(\omega)
=EX+EY\,.
\eqno
\enddisplay
Similarly, if $\alpha$ is any constant we have the simple rule
\begindisplay
E(\alpha X)=\alpha EX\,.
\eqno
\enddisplay
But the corresponding rule for multiplication of random variables is more
complicated in general; %this happens because
the expected value is defined as a sum over elementary events,
and sums of products don't often have a simple form.
In spite of this difficulty, there is a very nice
formula for the mean of a product in the special case that the
random variables are independent:
\begindisplay
E(XY)=(EX)(EY),\qquad\hbox{if $X$ and $Y$ are independent}.
\eqno
\enddisplay
We can prove this by the distributive law for products,
\begindisplay \openup3pt
E(XY)&=\sum_{\omega\in\Omega}X(\omega)Y(\omega)\cdt\Pr(\omega)\cr
&=\sum\twoconditions{x\in X(\Omega)}{y\in Y(\Omega)}
 xy\cdt\Pr\prp(X=x\ {\rm and}\ Y=y)\cr
&=\sum\twoconditions{x\in X(\Omega)}{y\in Y(\Omega)}
 xy\cdt\Pr\prp(X=x)\Pr\prp(Y=y)\cr
&=\sum_{x\in X(\Omega)}x\Pr\prp(X=x)\;\cdot\!
  \sum_{y\in Y(\Omega)}y\Pr\prp(Y=y)
=(EX)(EY)\,.
\enddisplay

For example, we know that $S=S_1+S_2$ and $P=S_1S_2$, when $S_1$ and~$S_2$
are the numbers of spots on the first and second of a pair of random dice.
We have $ES_1=ES_2={7\over2}$, hence $ES=7$; furthermore $S_1$ and~$S_2$
are independent, so $EP={7\over2}\cdt{7\over2}={49\over4}$, as claimed earlier.
We also have $E(S+P)=ES+EP=7+{49\over4}$. But $S$ and~$P$ are not
independent, so we cannot assert that $E(SP)=7\cdt{49\over4}={343\over4}$.
In fact, the expected value of $SP$ turns out to equal $637\over6$
in distribution $\Pr_{00}$, while
it equals $112$ (exactly) in distribution $\Pr_{11}$.

\beginsection 8.2 Mean and Variance

The next most important property of a random variable, after we know
its expected value, is its {\it"variance"}, defined as the mean
square deviation from the mean:
\begindisplay
VX=E\bigl((X-EX)^2\bigr)\,.
\eqno\eqref|var1|
\enddisplay
If we denote $EX$ by $\mu$, the variance $VX$ is the expected value of
$(X-\mu)^2$. This measures the ``spread'' of $X$'s distribution.

As a simple example of variance computation, let's suppose we have
just been made an offer we can't refuse: Someone has given us
"!greed"
two gift certificates for a certain "lottery". The lottery organizers
sell 100 tickets for each weekly drawing. One of these tickets is
selected by a uniformly random process\dash---that is,
each ticket is equally likely to be chosen\dash---and the lucky ticket holder
wins a hundred million dollars. The other 99 ticket holders win nothing.

We can use our gift in two ways: Either we buy two tickets in the
\g(Slightly subtle point:\par There are two probability spaces,
depending on what strategy we use; but\/ $EX_1$ and\/ $EX_2$
are the same in both.)\g
same lottery, or we buy one ticket in each of two lotteries.
Which is a better strategy? Let's try to analyze this by letting
$X_1$ and $X_2$ be random variables that represent the amount we
win on our first and second ticket. The expected value of $X_1$,
in millions, is
\begindisplay
EX_1=\textstyle{99\over100}\cdot0+{1\over100}\cdot100=1\,,
\enddisplay
and the same holds for $X_2$. Expected values are additive, so our
average total winnings will be
\begindisplay
E(X_1+X_2)=EX_1+EX_2=2\hbox{ million dollars,}
\enddisplay
regardless of which strategy we adopt.

Still, the two strategies seem different. Let's look beyond expected
values and study the exact probability distribution of $X_1+X_2$:
\begindisplay \def\preamble{\bigstrut\hfil##\quad&\vrule##&&\quad\hfil$##$\hfil}%
 \offinterlineskip
&&\multispan3\hfil\quad winnings (millions)\hfil\cr
&&0&100&200\cr
\noalign{\hrule}
\omit&height 2pt\cr
same drawing&&.9800&.0200\cr
different drawings&&.9801&.0198&.0001\cr
\enddisplay
If we buy two tickets in the same lottery we have a 98\% chance of
winning nothing and a 2\% chance of winning \$100 million.
If we buy them in different lotteries we have a 98.01\% chance of
winning nothing, so this is slightly more likely than before;
and we have a 0.01\% chance of winning \$200 million, also slightly
more likely than before; and our chances of winning \$100 million
are now 1.98\%. So the distribution of $X_1+X_2$ in this second
situation is slightly more spread out; the middle value, \$100 million,
is slightly less likely, but the extreme values are slightly more likely.

It's this notion of the spread of a random variable that the variance is
intended to capture. We measure the spread in terms of the squared
deviation of the random variable from its mean. In case~1, the variance
is therefore \setmathsize{.98(0M-2M)^2\,+\,.02(100M-2M)^2}
\begindisplay
\mathsize{.98(0M-2M)^2\,+\,.02(100M-2M)^2}=196M^2\,;
\enddisplay
in case 2 it is
\begindisplay
&.9801(0M-2M)^2\,+\,.0198(100M-2M)^2+.0001(200M-2M)^2\cr
&\mathsize{}=198M^2\,.
\enddisplay
As we expected, the latter variance is slightly larger, because the
distribution of case~2 is slightly more spread out.

When we work with variances, everything is squared, so the numbers
can get pretty big. (The factor $M^2$ is one trillion,
\g Interesting:
The variance of a dollar amount is expressed in units of square dollars.\g
which is somewhat imposing even for high-stakes gamblers.) To convert the
numbers back to the more meaningful original scale, we often take the
square root of the variance. The resulting number is called the
{\it"standard deviation"}, and it is usually denoted by the Greek letter~"$\sigma$":
\begindisplay
\sigma=\sqrt{VX}\,.
\eqno
\enddisplay
The standard deviations of the random variables $X_1+X_2$ in our two
lottery strategies are $\sqrt{196M^{\mathstrut2}}=14.00M$ and 
$\sqrt{198M^{\mathstrut2}}\approx14.071247M$. In some sense the second
alternative is about \$71,247 riskier.

How does the variance help us choose a strategy? It's not
clear. The strategy with higher variance is a little riskier; but do we
get the most for our money by taking more risks or by playing it safe?
\g Another way to reduce risk might be to bribe the lottery officials.
I~guess that's where probability becomes indiscreet.\bigskip
(N.B.: Opinions expressed in these margins do not necessarily represent
the opinions of the management.)\g
Suppose we had the chance to buy 100 tickets instead of only~two. Then we
could have a guaranteed victory in a single lottery (and the variance
would be zero); or we could gamble on a hundred different lotteries,
with a $.99^{100}\approx.366$ chance of winning nothing but also with
a nonzero probability of winning up to \$10,000,000,000. To decide
between these alternatives is beyond the scope of this book; all we can
do here is explain how to do the calculations.

In fact, there is a simpler way to calculate the variance, instead
of using the definition \eq(|var1|). (We suspect that there must be
something going on in the mathematics behind the scenes, because
the variances in the lottery example magically came out to be
integer multiples of\/~$M^2$.) We have
\begindisplay
E\bigl((X-EX)^2\bigr)&=E\bigl(X^2-2X(EX)+(EX)^2\bigr)\cr
&=E(X^2)-2(EX)(EX)+(EX)^2\,,\cr
\enddisplay
since $(EX)$ is a constant; hence
\begindisplay \postdisplaypenalty=10000
VX=E(X^2)-(EX)^2\,.
\eqno\eqref|var2|
\enddisplay
``The variance is the mean of the square minus the square of the mean.''

For example, the mean of $(X_1+X_2)^2$ comes to $.98(0M)^2+.02(100M)^2=
200M^2$ or to $.9801(0M)^2+.0198(100M)^2+.0001(200M)^2=202M^2$ in the
lottery problem. Subtracting $4M^2$ (the square of the mean) gives the
results we obtained the hard way.

There's an even easier formula yet, if we want to calculate $V(X+Y)$
when $X$ and~$Y$ are independent: We have
\begindisplay \openup3pt
E\bigl((X+Y)^2\bigr)&=E(X^2+2XY+Y^2)\cr
&=E(X^2)+2(EX)(EY)+E(Y^2)\,,
\enddisplay
since we know that $E(XY)=(EX)(EY)$ in the independent case. Therefore
\begindisplay \openup3pt
V(X+Y)&=E\bigl((X+Y)^2\bigr)-(EX+EY)^2\cr
&=E(X^2)+2(EX)(EY)+E(Y^2)\cr
&\qquad-(EX)^2-2(EX)(EY)-(EY)^2\cr
&=E(X^2)-(EX)^2+E(Y^2)-(EY)^2\cr
&=VX+VY\,.
\eqno
\enddisplay
``The variance of a sum of independent random variables is the sum of their
variances.'' For example, the variance of the amount we can win with a
single lottery ticket is
\begindisplay
E(X_1^2)-(EX_1)^2=.99(0M)^2+.01(100M)^2-(1M)^2=99M^2\,.
\enddisplay
Therefore the variance of the total winnings of two lottery tickets
in two separate (independent) lotteries is $2\times99M^2=198M^2$.
And the corresponding variance for $n$ independent lottery tickets
is $n\times99M^2$.

The variance of the dice-roll sum $S$ drops out of this same formula,
since $S=S_1+S_2$ is the sum of two independent random variables.
We have
\begindisplay
VS_1={1\over6}(1^2+2^2+3^2+4^2+5^2+6^2)-\left(7\over2\right)^{\!2}={35\over12}
\enddisplay
when the dice are fair;
hence $VS={35\over12}+{35\over12}={35\over6}$. The loaded die has
\begindisplay
VS_1={1\over8}(2\cdt1^2+2^2+3^2+4^2+5^2+2\cdt6^2)-\left(7\over2\right)^{\!2}
={45\over12}\,;
\enddisplay
hence $VS={45\over6}=7.5$ when both dice are loaded.
 Notice that the loaded dice give
$S$ a larger variance, although $S$ actually assumes its average value~$7$
more often than it would with fair dice. If our goal is to shoot lots of
lucky~$7$'s, the variance is not our best indicator of success.

OK, we have learned how to compute variances. But we haven't really
seen a good reason why the variance is a natural thing to compute.
Everybody does it, but why? The main reason is
\g If he proved it in 1867, it's a classic '67 Chebyshev.\g
{\it "Chebyshev's inequality"\/} ([|bienayme|] and [|chebyshev-ineq|]),
which states that the variance has a significant property:
\begindisplay
\Pr\pbigi((X-EX)^2\ge\alpha\bigr)\le VX/\alpha\,,\qquad
\hbox{for all $\alpha>0$}.
\eqno\eqref|cheb1|
\enddisplay
(This is different from the monotonic inequalities of "Chebyshev" that we
encountered in Chapter~2.)
Very roughly, \thiseq\ tells us that a random variable~$X$ will rarely
be far from its mean~$EX$ if its variance~$VX$ is small.
The proof is amazingly simple. We have
\begindisplay \openup3pt
VX&=\sum_{\omega\in\Omega}\bigl(X(\omega)-EX\bigr)^{\!2}\,\Pr(\omega)\cr
&\ge\sum\twoconditions{\omega\in\Omega}{(X(\omega)-EX)^2\ge\alpha}
 \!\!\!\bigl(X(\omega)-EX\bigr)^{\!2}\,\Pr(\omega)\cr
&\ge\sum\twoconditions{\omega\in\Omega}{(X(\omega)-EX)^2\ge\alpha}
 \!\!\!\alpha\Pr(\omega)
\;=\;\alpha\cdt\Pr\pbigi((X-EX)^2\ge\alpha\bigr)\,;
\enddisplay
dividing by $\alpha$ finishes the proof.

If we write $\mu$ for the mean and $\sigma$ for the standard deviation,
and if we replace $\alpha$ by $c^2VX$ in \eq(|cheb1|), the condition
$(X-EX)^2\ge c^2VX$ is the same as $(X-\mu)^2\ge(c\sigma)^2$; hence
\eq(|cheb1|) says that
\begindisplay
\Pr\pbigi(\vert X-\mu\vert\ge c\sigma\bigr)\le 1/c^2\,.
\eqno\eqref|cheb2|
\enddisplay
Thus, $X$ will lie within $c$ standard deviations of its mean value
except with probability at most $1/c^2$. A random variable will lie
within $2\sigma$ of~$\mu$ at least 75\% of the time; it will lie
between $\mu-10\sigma$ and $\mu+10\sigma$ at least 99\% of the time.
These are the cases $\alpha=4VX$ and $\alpha=100VX$ of Chebyshev's inequality.

If we roll a pair of fair dice $n$ times, the total value of the $n$~rolls
will almost always be near $7n$, for large~$n$.
Here's why: The variance of $n$ independent
rolls is ${35\over6}n$. A variance of ${35\over6}n$ means a standard
deviation of only
\begindisplay
\textstyle\sqrt{{35\over6}n}\,.
\enddisplay
So Chebyshev's inequality
tells us that the final sum will lie between
\begindisplay \postdisplaypenalty=10000
\textstyle 7n-10\sqrt{{35\over6}n}\And
7n+10\sqrt{{35\over6}n}
\enddisplay
in at least 99\% of all experiments when $n$ fair dice are rolled.
For example, the odds are better
than 99 to~1 that the total value of a million rolls
will be between 6.976~million and 7.024~million.

In general, let $X$ be {\it any\/}
 random variable over a probability space~$\Omega$,
having finite mean~$\mu$ and finite standard deviation~$\sigma$.
Then we can consider the probability space $\Omega^n$ whose elementary
events are $n$-tuples $(\omega_1,\omega_2,\ldots,\omega_n)$ with
each $\omega_k\in\Omega$, and whose probabilities are
\begindisplay
\Pr(\omega_1,\omega_2,\ldots,\omega_n)=\Pr(\omega_1)\Pr(\omega_2)\ldots
\Pr(\omega_n)\,.
\enddisplay
If we now define random variables $X_k$ by the formula
\begindisplay
X_k(\omega_1,\omega_2,\ldots,\omega_n)=X(\omega_k)\,,
\enddisplay
the quantity
\begindisplay
X_1+X_2+\cdots+X_n
\enddisplay
is a sum of $n$ independent random variables, which corresponds to taking
$n$ independent ``samples'' of\/~$X$ on $\Omega$ and adding them together.
The mean of $X_1+X_2+\cdots+X_n$ is $n\mu$, and the standard deviation
is $\sqrt n\,\sigma$; hence the average of the $n$ samples,
\begindisplay
{1\over n}(X_1+X_2+\cdots+X_n)\,,
\enddisplay
will lie between $\mu-10\sigma/\mskip-1mu\sqrt n$ and
$\mu+10\sigma/\mskip-1mu\sqrt n$ at least 99\% of the time.
\g(That is, the average will fall between the stated limits
in at least 99\% of all cases when we look at a set of $n$ independent
samples, for any fixed value of $n$. Don't misunderstand
this as a statement about the
averages of an infinite sequence $X_1$, $X_2$, $X_3$, \dots\ as~$n$ varies.)\g
 In other words, if we
choose a large enough value of~$n$, the average of $n$ independent samples
will almost always be very near the expected value $EX$.
(An even stronger theorem called the Strong
"Law of Large Numbers" is proved in textbooks of probability theory;
but the simple consequence of Chebyshev's inequality that we have
just derived is enough for our purposes.)

Sometimes we don't know the characteristics of a probability space, and
we want to estimate the mean of a random variable~$X$ by sampling its
value repeatedly. (For example, we might want
to know the average temperature at noon on a January day in San Francisco;
or we may wish to know the mean life expectancy of insurance agents.)
If we have obtained independent empirical observations $X_1$, $X_2$,
\dots,~$X_n$, we can guess that the true mean is approximately
\begindisplay
\widehat EX={X_1+X_2+\cdots+X_n\over n}\,.
\eqno\eqref|hat-e|
\enddisplay
And we can also make an estimate of the variance, using the formula
\begindisplay
\widehat VX={X_1^2+X_2^2+\cdots+X_n^2\over n-1}
 -{(X_1+X_2+\cdots+X_n)^2\over n(n-1)}\,.
\eqno\eqref|hat-v|
\enddisplay
The $(n-1)$'s in this formula look like typographic errors; it seems
they should be $n$'s, as in \eq(|hat-e|), because
the true variance $VX$ is defined by expected values in \eq(|var2|).
Yet we get a better estimate with $n-1$ instead of~$n$
here, because definition \thiseq\ implies that
\begindisplay
E(\widehat VX)=VX\,.
\eqno
\enddisplay
Here's why:
\begindisplay \openup5pt
E(\widehat VX)&={1\over n-1}E\Bigl(\,\sum_{k=1}^n X_k^2\,-\,{1\over n}
 \sum_{j=1}^n\,\sum_{k=1}^n X_jX_k\Bigr)\cr
&={1\over n-1}\Bigl(\,\sum_{k=1}^n E(X_k^2)\,-\,{1\over n}
 \sum_{j=1}^n\,\sum_{k=1}^n E(X_jX_k)\Bigr)\cr
&={1\over n-1}\Bigl(\,\sum_{k=1}^n E(X^2)\,-\,{1\over n}
 \sum_{j=1}^n\,\sum_{k=1}^n \bigl(E(X)^2\[j\ne k]+E(X^2)\[j=k]\bigr)\Bigr)\cr
&={1\over n-1}\Bigl(nE(X^2)-{1\over n}\bigl(nE(X^2)+n(n-1)E(X)^2\bigr)\Bigr)\cr
&=E(X^2)-E(X)^2=VX\,.
\enddisplay
(This derivation uses the independence of the observations when it
replaces $E(X_jX_k)$ by $(EX)^2\[j\ne k]+E(X^2)\[j=k]$.)

In practice, experimental results about a random variable~$X$ are usually
obtained by calculating a sample mean $\hat\mu=\widehat EX$ and a sample
standard deviation $\hat\sigma=\sqrt{\widehat VX}$, and presenting the answer
in the form `$\,\hat\mu\pm\hat\sigma/\mskip-1mu\sqrt n\,$'.
 For example, here are ten
rolls of two supposedly fair dice:
\begindisplay \openup4pt \advance\abovedisplayskip-2pt
\diefour\diethree\qquad	\diefive\diethree\qquad	\diethree\dieone\qquad
	\diesix\diefour\qquad	\dietwo\diesix\cr
\diefive\diesix\qquad	\diefour\dieone\qquad	\diefive\dieone\qquad
	\dietwo\diesix\qquad	\diefour\diethree\cr
\enddisplay
The sample mean of the spot sum $S$ is
\begindisplay
\hat\mu=(7+11+8+5+4+6+10+8+8+7)/10=7.4\,;
\enddisplay
the sample variance is
\begindisplay
(7^2+11^2+8^2+5^2+4^2+6^2+10^2+8^2+8^2+7^2-10{\hat\mu}^2)/9\approx2.1^2\,.
\enddisplay
We estimate the average spot sum of these dice to be $7.4\pm2.1/\mskip-1mu
\sqrt{10}=7.4\pm0.7$,
on the basis of these experiments.

\smallskip
Let's work one more example of means and variances, in order to show how
they can be calculated theoretically instead of empirically. One of the
questions we considered in Chapter~5 was
the ``"football victory problem",\qback'' where $n$~"hats" are thrown
into the air and the result is a random permutation of hats. We showed
in equation \equ(5.|subfactorial-sol|) that there's a probability
of $n\?/n!\approx1/e$
that nobody gets the right hat back. We also derived
the formula
\begindisplay
P(n,k)={1\over n!}{n\choose k}(n-k)\?={1\over k!}{(n-k)\?\over(n-k)!}
\eqno
\enddisplay
for the probability that exactly $k$ people end up with their own hats.

Restating these results in the formalism just learned, we can consider
the probability space $\Pi_n$ of all $n!$ permutations~$\pi$ of $\{1,2,\ldots
,n\}$, where $\Pr(\pi)=1/n!$ for all $\pi\in\Pi_n$. The random variable
\begindisplay
F_n(\pi)=\hbox{number of ``"fixed points"'' of $\pi$}\,,\qquad
\hbox{for $\pi\in\Pi_n$},
\enddisplay
\g\vskip-31pt Not to be confused with a Fibonacci number.\g
measures the number of correct hat-falls in the football victory problem.
Equation \thiseq\ gives $\Pr\prp(F_n=k)$, but let's pretend that we don't
know any such formula; we merely want to study the average value of~$F_n$,
and its standard deviation.

The average value is, in fact, extremely easy to calculate, avoiding
all the complexities of Chapter~5. We simply observe that
\begindisplay
F_n(\pi)&=F_{n,1}(\pi)+F_{n,2}(\pi)+\cdots+F_{n,n}(\pi)\,,\cr
F_{n,k}(\pi)&=[@\hbox{position $k$ of $\pi$ is a fixed point}]\,,
\qquad\hbox{for $\pi\in\Pi_n$}.
\enddisplay
Hence
\begindisplay
EF_n=EF_{n,1}+EF_{n,2}+\cdots+EF_{n,n}\,.
\enddisplay
And the expected value of $F_{n,k}$ is simply the probability that $F_{n,k}=1$,
which is $1/n$ because exactly $(n-1)!$ of the $n!$ permutations
$\pi=\pi_1\pi_2\ldots\pi_n\in\Pi_n$ have $\pi_k=k$. Therefore
\begindisplay
EF_n=n/n=1\,,\qquad\hbox{for $n>0$}.
\eqno\eqref|e-fixedpts|
\enddisplay
On the average,
\g One the average.\g
 one hat will be in its correct place. ``A random permutation
has one fixed point, on the average.''

Now what's the standard deviation? This question is more difficult,
because the $F_{n,k}$'s are not independent of each other. But we can
calculate the variance by analyzing the mutual dependencies among them:
\begindisplay \openup3pt
E(F_n^2)&=E\biggl(\Bigl(\,\sum_{k=1}^nF_{n,k}\Bigr)^{\!2\,}\biggr)
=E\Bigl(\,\sum_{j=1}^n\,\sum_{k=1}^nF_{n,j\,}F_{n,k}\Bigr)\cr
&=\sum_{j=1}^n\sum_{k=1}^n E(F_{n,j\,}F_{n,k})
=\sum_{1\le k\le n}E(F_{n,k}^2)\!+\!2\!\!
 \sum_{1\le j<k\le n}E(F_{n,j\,}F_{n,k})\,.\cr
\enddisplay
(We used a similar trick when we derived \equ(2.|upper-triangle|)
in Chapter~2.) Now $F_{n,k}^2=F_{n,k}$, since $F_{n,k}$ is either $0$ or~$1$;
hence $E(F_{n,k}^2)=EF_{n,k}=1/n$ as before. And if $j<k$ we have
$E(F_{n,j\,}F_{n,k})=\Pr(\pi$ has both $j$ and $k$ as fixed points$)=(n-2)!/n!
=1/n(n-1)$. Therefore
\begindisplay
E(F_n^2)={n\over n}+{n\choose 2}{2\over n(n-1)}=2\,,
\qquad\hbox{for $n\ge2$}.
\eqno\eqref|mu2-fixedpts|
\enddisplay
(As a check when $n=3$, we have ${2\over6}0^2+{3\over6}1^2+{0\over6}2^2+
{1\over6}3^2=2$.) The variance is $E(F_n^2)-(EF_n)^2=1$, so the standard
deviation (like the mean) is~$1$. ``A random permutation of $n\ge2$ elements
has $1\pm1$ fixed points.''

\begingroup \advance\parindent-7.5pt % KLUDGE!
\beginsection 8.3 Probability Generating Functions

If $X$ is a random variable that takes only nonnegative integer values, we
\endgroup % end kludge
can capture its probability distribution nicely by using the techniques
of Chapter~7. The {\it"probability generating function"\/} or "pgf" of\/~$X$ is
\begindisplay
G_X(z)=\sum_{k\ge0}\Pr\prp(X=k)\,z^k\,.
\eqno
\enddisplay
This power series in $z$ contains all the information about the random
variable~$X$. We can also express it in two other ways:
\begindisplay
G_X(z)=\sum_{\omega\in\Omega}\Pr(\omega)\,z^{X(\omega)}=E(z^X)\,.
\eqno
\enddisplay

The coefficients of $G_X(z)$ are nonnegative, and they sum to~$1$; the
latter condition can be written
\begindisplay
G_X(1)=1\,.
\eqno\eqref|pgf0|
\enddisplay
Conversely, any power series $G(z)$ with nonnegative coefficients and
with $G(1)=1$ is the pgf of some random variable.

The nicest thing about pgf's is that they usually simplify the computation
of means and variances. For example, the mean is easily expressed:
\begindisplay
EX&=\sum_{k\ge0}k\cdt\Pr\prp(X=k)\cr
&=\sum_{k\ge0}\Pr\prp(X=k)\cdt kz^{k-1}\big\vert_{z=1}\cr
&=G'_X(1)\,.
\eqno\eqref|pgf1|
\enddisplay
We simply differentiate the pgf with respect to $z$ and set $z=1$.

The variance is only slightly more complicated:
\begindisplay
E(X^2)&=\sum_{k\ge0}k^2\cdt\Pr\prp(X=k)\cr
&=\sum_{k\ge0}\Pr\prp(X=k)\cdt\bigl( k(k-1)z^{k-2}+kz^{k-1}\bigr)\,\big\vert_{z=1}
=G''_X(1)+G'_X(1)\,.
\enddisplay
Therefore
\begindisplay
VX=G''_X(1)+G'_X(1)-G'_X(1)^2\,.
\eqno\eqref|pgf2|
\enddisplay
Equations \eq(|pgf1|) and \eq(|pgf2|) tell us that we can compute the
mean and variance if we can compute the values of two derivatives,
$G'_X(1)$ and $G''_X(1)$. We don't have to know a closed form for the
probabilities; we don't even have to know a closed form for $G_X(z)$ itself.

It is convenient to write
\begindisplay
\Mean(G)&=G'(1)\,,\eqno\eqref|pgf-mean|\cr
\Var(G)&=G''(1)+G'(1)-G'(1)^2\,,\eqno\eqref|pgf-var|\cr
\enddisplay
when $G$ is any function, since we frequently want to compute these
combinations of derivatives.

The second-nicest thing about pgf's is that they are comparatively
simple functions of~$z$, in many important cases. For example, let's
look at the
{\it"uniform distribution"\/} of order~$n$, in which the random variable
takes on each of the values $\{0,1,\ldots,n-1\}$ with probability $1/n$.
The pgf in this case is
\begindisplay
U_n(z)={1\over n}(1+z+\cdots+z^{n-1})={1\over n}{1-z^n\over 1-z}\,,
\qquad\hbox{for $n\ge1$}.
\eqno\eqref|u-pgf|
\enddisplay
We have a closed form for $U_n(z)$ because this is a geometric series.

\looseness=-1
But this closed form proves to be somewhat embarrassing: When
we plug in $z=1$ (the value of~$z$ that's most critical for the pgf\/),
we get the undefined ratio $0/0$, even though $U_n(z)$ is a polynomial
that is perfectly well defined at any value of~$z$. The value
$U_n(1)=1$ is obvious from the non-closed form $(1+z+\cdots+z^{n-1})/n$,
yet it seems that we must resort to "L'Hospital's rule" to find
$\lim_{z\to1}U_n(z)$ if we want to determine $U_n(1)$ from the closed form.
The determination of $U'_n(1)$ by L'Hospital's rule will be even harder,
because there will be a factor of $(z-1)^2$ in the denominator; $U_n''(1)$
will be harder still.

Luckily
there's a nice way out of this dilemma. If $G(z)=\sum_{n\ge0}g_nz^n$
is any power series that converges for at least one value of~$z$ with
$\vert z\vert>1$, the power series $G'(z)=\sum_{n\ge0}ng_nz^{n-1}$
will also have this property, and so will $G''(z)$, $G'''(z)$, etc.
Therefore by "Taylor's theorem" we can write
\begindisplay
G(1+t)=G(1)+{G'(1)\over1!}t+{G''(1)\over2!}t^2+
{G'''(1)\over3!}t^3+\cdots\,;
\eqno\eqref|derivs-at-1|
\enddisplay
all derivatives of $G(z)$ at $z=1$ will appear as coefficients, when
$G(1+t)$ is expanded in powers of~$t$.

For example, the derivatives of the uniform pgf\/ $U_n(z)$ are easily
found in this way:
\begindisplay \openup3pt
U_n(1+t)&={1\over n}{(1+t)^n-1\over t}\cr
&={1\over n}{n\choose1}
 +{1\over n}{n\choose2}t
 +{1\over n}{n\choose3}t^2+\cdots
 +{1\over n}{n\choose n}t^{n-1}\,.
\enddisplay
Comparing this to \eq(|derivs-at-1|) gives
\begindisplay
U_n(1)=1\,;\quad U_n'(1)={n-1\over2}\,;\quad
U_n''(1)={(n-1)(n-2)\over3}\,;
\eqno
\enddisplay
and in general $U_n^{(m)}(1)=(n-1)\_m/(m+1)$, although we need only the
cases $m=1$ and $m=2$ to compute the mean and the variance. The
mean of the uniform distribution is
\begindisplay
U_n'(1)={n-1\over2}\,,
\eqno\eqref|unif-mean|
\enddisplay
and the variance is
\begindisplay \openup3pt
U_n''(1)+U_n'(1)-U_n'(1)^2
&=4\,{(n-1)(n-2)\over12}+6\,{(n-1)\over12}-3\,{(n-1)^2\over12}\cr
&={n^2-1\over12}\,.
\eqno\eqref|unif-var|
\enddisplay

The third-nicest thing about pgf's is that the product of pgf's corresponds
to the sum of independent random variables. We learned in Chapters
5 and~7 that the product of generating functions corresponds to the
convolution of sequences; but it's even more important in
applications to know that the convolution of probabilities corresponds
to the sum of independent random variables. Indeed, if $X$ and~$Y$
are random variables that take on nothing but integer values, the
probability that $X+Y=n$ is
\begindisplay
\Pr\prp(X+Y=n)=\sum_k\Pr\prp(X=k\ {\rm and}\ Y=n-k)\,.
\enddisplay
If $X$ and~$Y$ are independent, we now have
\begindisplay
\Pr\prp(X+Y=n)=\sum_k\Pr\prp(X=k)\,\Pr\prp(Y=n-k)\,,
\enddisplay
a convolution. Therefore\dash---and this is the punch line\dash---
\begindisplay
G_{X+Y}(z)=G_X(z)\,G_Y(z)\,,\qquad\hbox{if $X$ and $Y$ are independent}.
\eqno
\enddisplay
Earlier this chapter we observed that $V(X+Y)=VX+VY$ when $X$ and~$Y$
are independent. Let $F(z)$ and~$G(z)$ be the pgf's for $X$ and~$Y$,
and let $H(z)$ be the pgf for $X+Y$. Then
\begindisplay
H(z)=F(z)@G(z)\,,
\enddisplay
and our formulas \eq(|pgf1|) through \eq(|pgf-var|) for mean and variance
tell us that we must have
\begindisplay
\Mean(H)&=\Mean(F)+\Mean(G)\,;\eqno\eqref|ind-sum1|\cr
\Var(H)&=\Var(F)+\Var(G)\,.\eqno\eqref|ind-sum2|\cr
\enddisplay
These formulas, which are properties of the derivatives
$\Mean(H)=H'(1)$ and
$\Var(H)=H''(1)+H'(1)-H'(1)^2$,
aren't valid for arbitrary function products $H(z)=F(z)@G(z)$;
we have
\begindisplay
H'(z)&=F'(z)@G(z)+F(z)@G'(z)\,,\cr
H''(z)&=F''(z)@G(z)+2F'(z)@G'(z)+F(z)@G''(z)\,.
\enddisplay
But if we set $z=1$,
we can see that \eq(|ind-sum1|) and \eq(|ind-sum2|)
will be valid in general provided only that
\begindisplay
F(1)=G(1)=1
\eqno
\enddisplay
and that the derivatives exist.
The ``probabilities'' don't have to be in $[@0\dts1]$ for these formulas to hold.
 We can normalize the functions $F(z)$
and $G(z)$ by dividing through by $F(1)$ and $G(1)$ in order to make
this condition valid, whenever $F(1)$
and $G(1)$ are nonzero.

Mean and variance aren't the whole story. They are merely two
of an infinite series of so-called {\it"cumulant"\/} statistics introduced
\g I'll graduate magna cum ulant.\g
by the Danish astronomer Thorvald Nicolai "Thiele"~[|thiele|] in 1903.
The first two cumulants $\kappa_1$ and $\kappa_2$ of a random variable
are what we have called the mean and the variance; there also are
higher-order cumulants that express more subtle properties of a distribution.
The general formula
\begindisplay
\ln G(e^t)={\kappa_1\over1!}t+
 {\kappa_2\over2!}t^2+
 {\kappa_3\over3!}t^3+
 {\kappa_4\over4!}t^4+\cdots
\eqno\eqref|cum-def|
\enddisplay
defines the cumulants of all orders, when $G(z)$ is the pgf of a random variable.

Let's look at cumulants more closely.
If $G(z)$ is the pgf for $X$, we have
\begindisplay\openup3pt
G(e^t)&=\sum_{k\ge0}\Pr\prp(X=k)e^{kt}&=\sum_{k,m\ge0}\Pr\prp(X=k){k^mt^m\over m!}\cr
&&=1+{\mu_1\over1!}t+{\mu_2\over2!}t^2+{\mu_3\over3!}t^3+\cdots\,,\eqno\cr
\enddisplay
where
\begindisplay
\mu_m&=\sum_{k\ge0}k^m\Pr\prp(X=k)=E(X^m)\,.}\hidewidth{\eqno\eqref|moment-def|\cr
\enddisplay
This quantity $\mu_m$ is called the ``$m$th "moment"'' of\/~$X$. We can take
exponentials on both sides of \eq(|cum-def|), obtaining another formula
for $G(e^t)$:
\begindisplay \openup3pt
G(e^t)&=1+{(\kappa_1t+\half\kappa_2t^2+\cdots\,)\over1!}+
 {(\kappa_1t+\half\kappa_2t^2+\cdots\,)^2\over2!}+\cdots\cr
&=1+\kappa_1t+\textstyle\half(\kappa_2+\kappa_1^2)t^2+\cdots\,.
\enddisplay
Equating coefficients of powers of $t$ leads to a series of formulas
%\g You forgot the coefficients of~$t^0$.\g
\begindisplay
\kappa_1&=\mu_1\,,\eqno\cr
\kappa_2&=\mu_2-\mu_1^2\,,\eqno\cr
\kappa_3&=\mu_3-3\mu_1\mu_2+2\mu_1^3\,,\eqno\cr
\kappa_4&=\mu_4-4\mu_1\mu_3+12\mu_1^2\mu_2-3\mu_2^2-6\mu_1^4\,,\eqno\cr
\kappa_5&=\mu_5-5\mu_1\mu_4+20\mu_1^2\mu_3-10\mu_2\mu_3\cr
&\hskip7em+30\mu_1\mu_2^2-60\mu_1^3\mu_2+24\mu_1^5\,,\eqno\cr
\noalign{\vskip-3pt}
&\quad\vdots
\enddisplay
defining the cumulants in terms of the moments. Notice that $\kappa_2$
is indeed the variance, $E(X^2)-(EX)^2$, as claimed.

Equation \eq(|cum-def|) makes it clear that the cumulants defined by
\g \noindent\llap{``}For these higher half-invariants we shall propose no special names.''
\par\hfill\kern-4pt\dash---T.\thinspace N.\thinspace"Thiele"\thinspace[|thiele|]\g
the product $F(z)@G(z)$ of two pgf's will be the sums of the corresponding
cumulants of $F(z)$ and $G(z)$, because logarithms of products are sums.
 Therefore
all cumulants of the sum of independent random variables are additive, just
as the mean and variance are. This property makes cumulants more
important than moments.

If we take a slightly different tack, writing
\begindisplay
G(1+t)=1+{\alpha_1\over1!}t+{\alpha_2\over2!}t^2+{\alpha_3\over3!}t^3+\cdots\,,
\enddisplay
equation \eq(|derivs-at-1|) tells us that the $\alpha$'s are the ``factorial
moments''
\begindisplay \openup3pt
\alpha_m&=G^{(m)}(1)\cr
&=\sum_{k\ge0}\Pr\prp(X=k)k\_m\,z^{k-m}\,\big\vert_{z=1}\cr
&=\sum_{k\ge0}k\_m\Pr\prp(X=k)\cr
&=E(X\_m)\,.
\eqno
\enddisplay
It follows that
\begindisplay \openup3pt
G(e^t)&=1+{\alpha_1\over1!}(e^t-1)+{\alpha_2\over2!}(e^t-1)^2+\cdots\cr
&=1+{\alpha_1\over1!}(t+{\textstyle\half} t^2+\cdots\,)+
   {\alpha_2\over2!}(t^2+t^3+\cdots\,)+\cdots\cr
&=\textstyle1+\alpha_1t+\half(\alpha_2+\alpha_1)t^2+\cdots\,,
\enddisplay
and we can express the cumulants in terms of the derivatives $G^{(m)}(1)$:
\begindisplay
\kappa_1&=\alpha_1\,,\eqno\cr
\kappa_2&=\alpha_2+\alpha_1-\alpha_1^2\,,\eqno\cr
\kappa_3&=\alpha_3+3\alpha_2+\alpha_1-3\alpha_2\alpha_1-3\alpha_1^2+2\alpha_1^3\,,
\eqno\cr
&\quad\vdots
\enddisplay
This sequence of formulas yields ``additive'' identities that extend
\eq(|ind-sum1|) and \eq(|ind-sum2|) to all the cumulants.

Let's get back down to earth and apply these ideas to simple examples. The
simplest case of a random variable is a ``random constant,\qback'' where
$X$~has a certain fixed value~$x$ with probability~$1$. In this case
$G_X(z)=z^x$, and $\ln G_X(e^t)=xt$; hence the mean is~$x$ and all other
cumulants are zero. It follows that the operation of multiplying any pgf
by $z^x$ increases the mean by~$x$ but leaves the variance and all other
cumulants unchanged.

How do probability generating functions apply to dice? The
distribution of spots on one fair die has the pgf
\begindisplay
G(z)={z+z^2+z^3+z^4+z^5+z^6\over6}=zU_6(z)\,,
\enddisplay
where $U_6$ is the pgf for the uniform distribution of order~$6$. The
factor~`$z$' adds $1$ to the mean, so the mean is $3.5$ instead of~${n-1\over2}
=2.5$ as given in~\eq(|unif-mean|);
but an extra `$z$'
 does not affect the variance \eq(|unif-var|), which equals $35\over12$.

The pgf for total spots on two independent dice is the square of the pgf for
spots on one die,
\begindisplay \advance\medmuskip-.6mu
G_S(z)&={z^2+2z^3+3z^4+4z^5+5z^6+6z^7+5z^8+4z^9+3z^{10}+2z^{11}+z^{12}\over36}\cr
&=z^2U_6(z)^2\,.
\enddisplay
If we roll a pair of fair dice $n$ times,
 the probability that we get a total of $k$~spots
overall is, similarly,
\begindisplay \openup2pt
[z^k]\,G_S(z)^n&=[z^k]\,z^{2n}U_6(z)^{2n}\cr
&=[z^{k-2n}]\,U_6(z)^{2n}\,.
\enddisplay

In the hats-off-to-football-victory problem considered earlier,
\g Hat distribution is a different kind of uniform distribution.\g
otherwise known as the problem of
enumerating the fixed points of a random permutation, we know from
\equ(5.|subfactorial-rec|)
that the pgf is
\begindisplay
F_n(z)=\sum_{0\le k\le n}{(n-k)\?\over(n-k)!}{z^k\over k!}\,,\qquad
\hbox{for $n\ge0$}.
\eqno\eqref|football-pgf|
\enddisplay
Therefore
\begindisplay
F_n'(z)&=\sum_{1\le k\le n}{(n-k)\?\over(n-k)!}{z^{k-1}\over(k-1)!}\cr
&=\sum_{0\le k\le n-1}{(n-1-k)\?\over(n-1-k)!}{z^k\over k!}\cr
\noalign{\smallskip}
&=F_{n-1}(z)\,.\cr
\enddisplay
Without knowing the details of the coefficients, we can conclude from
this recurrence $F'_n(z)=F_{n-1}(z)$ that $F_n^{(m)}(z)=F_{n-m}(z)$; hence
\begindisplay
F_n^{(m)}(1)=F_{n-m}(1)=\[n\ge m]\,.
\eqno
\enddisplay
This formula makes it easy to calculate the mean and variance;
we find as before (but more quickly) that they are both equal
to~$1$ when $n\ge2$.

In fact, we can now show that the $m$th cumulant $\kappa_m$ of this
random variable is equal to~$1$ whenever $n\ge m$. For the $m$th
cumulant depends only on $F'_n(1)$, $F''_n(1)$, \dots,~$F^{(m)}_n(1)$,
and these are all equal to~$1$; hence we obtain the same answer for
the $m$th cumulant as we do when we replace $F_n(z)$ by the limiting pgf
\begindisplay
F_\infty(z)=e^{z-1}\,,
\eqno\eqref|football-infty|
\enddisplay
which has $F_\infty^{(m)}(1)=1$ for derivatives of all orders. The cumulants
of $F_\infty$ are identically equal to~$1$, because
\begindisplay
\ln F_\infty(e^t)=\ln e^{e^t-1}=e^t-1
={t\over1!}+{t^2\over2!}+{t^3\over3!}+\cdots\,.
\enddisplay

\beginsection 8.4 Flipping coins

Now let's turn to processes that have just two outcomes.
If we flip a coin, there's probability $p$ that it comes up
\g Con artists know that\/ $p\approx0.1$ when you spin a
newly minted U.S. penny on a~smooth table. "!cheating"
(The weight distribution makes "Lincoln"'s head fall downward.)\g
heads and probability $q$ that it comes up tails, where
\begindisplay
p+q=1\,.
\enddisplay
(We assume that the coin doesn't come to rest on its edge, or
fall into a hole, etc.) Throughout this section, the numbers $p$
and~$q$ will always sum to~$1$. If the coin is {\it fair},
"!fair coin" "!biased coin"
we have $p=q=\half$; otherwise the coin is said to be {\it biased}.

The probability generating function for the number of heads after
one toss of a coin is
\begindisplay
H(z)=q+pz\,.
\eqno
\enddisplay
If we toss the coin $n$ times, always assuming that different coin
tosses are independent, the number of heads
is generated by
\begindisplay
H(z)^n=(q+pz)^n=\sum_{k\ge0}{n\choose k}p^kq^{n-k}z^k\,,
\eqno\eqref|binom-pgf|
\enddisplay
according to
the binomial theorem. Thus, the chance that we obtain exactly~$k$
heads in $n$ tosses is ${n\choose k}p^kq^{n-k}$.
This sequence of probabilities is
called the {\it"binomial distribution"}.

Suppose we toss a coin repeatedly until heads first turns up. What is
the probability that exactly $k$~tosses will be required? We have $k=1$
with probability~$p$ (since this is the probability of heads on
the first flip); we have $k=2$ with probability $qp$ (since this
is the probability of tails first, then heads); and for general~$k$
the probability is $q^{k-1}p$. So the generating function is
\begindisplay
pz+qpz^2+q^2pz^3+\cdots\,={pz\over1-qz}\,.
\eqno\eqref|geom-pgf|
\enddisplay
Repeating the process until $n$ heads are obtained gives the pgf
"!Bernoulli trials, see coin flipping"
\begindisplay
\left( pz\over1-qz\right)^{\!n}&=p^nz^n\sum_k{n+k-1\choose k}(qz)^k\cr
&=\sum_k{k-1\choose k-n}p^nq^{k-n}z^k\,.
\eqno\eqref|bernoulli-trials|
\enddisplay
This, incidentally, is $z^n$ times
\begindisplay
\left(p\over1-qz\right)^{\!n}=\sum_k{n+k-1\choose k}p^nq^kz^k\,,
\eqno\eqref|negbinom-pgf|
\enddisplay
the generating function for the {\it"negative binomial distribution"}.
%(The special case $n=1$ of a negative binomial distribution is called
%a {\it"geometric distribution"}, because the corresponding generating
%function is a geometric series.)

The probability space in example \eq(|bernoulli-trials|),
where we flip a coin until $n$~heads have appeared,
 is different from the
probability spaces we've seen earlier in this chapter, because it
contains infinitely many elements. Each element is a finite sequence
of heads and/or tails, containing precisely $n$ heads in all, and
\g Heads I win,\par tails you lose.\smallskip
No? OK; tails you lose, heads I~win.\smallskip
No? Well, then,\par heads you lose,\par tails I win.\g
ending with heads; the probability of such a sequence is $p^nq^{k-n}$,
where $k-n$ is the number of tails. Thus, for example, if $n=3$ and
if we write $\tt H$ for heads and $\tt T$ for tails, the sequence
$\tt THTTTHH$ is an element of the probability space, and its
probability is $qpqqqpp=p^3q^4$.

Let $X$ be a random variable with the binomial distribution \eq(|binom-pgf|),
and let $Y$ be a random variable with the negative binomial
distribution \eq(|negbinom-pgf|). These distributions depend on $n$ and~$p$.
The mean of\/~$X$ is $nH'(1)=np$, since its pgf is $H(z)^n$; the variance is
\begindisplay
n\bigl(H''(1)+H'(1)-H'(1)^2\bigr)=n(0+p-p^2)=npq\,.
\eqno\eqref|v-binom|
\enddisplay
Thus the standard deviation is $\sqrt{@npq\vphantom1}\,$: If we
toss a coin $n$~times, we expect to get heads about
$np\pm\sqrt{@npq\vphantom1}$ times.  The mean and variance of $Y$
can be found in a similar way: If we let
\begindisplay
G(z)={p\over1-qz}\,,
\enddisplay
we have
\begindisplay \openup3pt
G'(z)&={pq\over(1-qz)^2}\,,\cr
G''(z)&={2pq^2\over(1-qz)^3}\,;
\enddisplay
hence $G'(1)=pq/p^2=q/p$ and $G''(1)=2pq^2\!/p^3=2q^2\!/p^2$. It follows that
the mean of\/~$Y$ is $nq/p$ and the variance is $nq/p^2$.

A simpler way to derive the mean and variance of\/~$Y$ is to use the
reciprocal generating function
\begindisplay
F(z)={1-qz\over p}={1\over p}-{q\over p}z\,,
\eqno
\enddisplay
and to write
\begindisplay
G(z)^n=F(z)^{-n}\,.
\eqno
\enddisplay
This polynomial $F(z)$ is not a probability generating function, because
it has a negative coefficient. But it does satisfy the crucial condition
$F(1)=1$.
Thus $F(z)$ is formally a binomial that corresponds to a coin for which
we get heads with ``probability'' equal to $-q/p$; and $G(z)$ is formally
\g The probability is negative that I'm getting younger.
\medskip Oh? Then it's ${>}\,1$ that you're getting older, or
staying the~same.\g
equivalent to flipping such a coin $-1$ times\kern1pt(!).
The negative binomial distribution
with parameters $(n,p)$ can therefore be regarded as the ordinary
binomial distribution with parameters $(n',p')=(-n,-q/p)$. Proceeding
formally, the mean must be $n'p'=(-n)(-q/p)=nq/p$, and the variance must
be $n'p'q'=(-n)(-q/p)(1+q/p)=nq/p^2$. This formal derivation involving
negative probabilities is valid, because our derivation for ordinary
binomials was based on identities between formal power series in which
the assumption $0\le p\le1$ was never used.

\unitlength=3pt
\def\domi{\beginpicture(0,2)(0,0) % identity
	\put(0,0){\line(0,1){2}}
	\endpicture}
\def\domv{\beginpicture(1,2)(0,0) % vertical without left edge
	\put(1,0){\line(0,1){2}}
	\put(0,0){\line(1,0){1}}
	\put(0,2){\line(1,0){1}}
	\endpicture}
\def\domhh{\beginpicture(2,2)(0,0) % two horizontals without left edge
	\put(2,0){\line(0,1){2}}
	\put(0,0){\line(1,0){2}}
	\put(0,1){\line(1,0){2}}
	\put(0,2){\line(1,0){2}}
	\endpicture}

\smallbreak
Let's move on to another example: How many times do we have to flip
a coin until we get heads twice in a row? The probability space now
consists of all sequences of $\tt H$'s and $\tt T$'s that end
with $\tt HH$ but have no consecutive $\tt H$'s until the final position:
\begindisplay
\Omega=\tt\{@ HH,THH,TTHH,HTHH,TTTHH,THTHH,HTTHH,\ldots\,\}\,.
\enddisplay
The probability of any given sequence is obtained by replacing $\tt H$
by $p$ and $\tt T$ by~$q$; for example, the sequence $\tt THTHH$
will occur with probability
\begindisplay
\Pr(@{\tt THTHH}@)=qpqpp=p^3q^2\,.
\enddisplay

We can now play with generating functions as we did at the beginning
of Chapter~7, letting $S$ be the infinite sum
\begindisplay
S=\tt  HH+THH+TTHH+HTHH+TTTHH+THTHH+HTTHH+\cdots
\enddisplay
of all the elements of $\Omega$. If we replace each $\tt H$ by $pz$
and each $\tt T$ by $qz$, we get the probability generating function
for the number of flips needed until two consecutive heads turn up.

There's a curious relation between $S$ and the sum of domino tilings
\begindisplay \postdisplaypenalty=1000
T	= \domi+
 \domi\domv+\domi\domv\domv + \domi\domhh
 + \domi\domv\domv\domv + \domi\domv\domhh + \domi\domhh\domv
 + \cdots
\enddisplay
in equation \equ(7.|t-sum|). Indeed, we obtain $S$ from $T$ if we
replace each $\domi\domv$ by $\tt T$ and each $\domi\domhh$ by $\tt HT$,
then tack on an $\tt HH$ at the end. This correspondence is easy
to prove because each element of~$\Omega$ has the form $({\tt T+HT})^n
\tt HH$ for some $n\ge0$, and each term of~$T$ has the
form $(@\domi\domv+\domi\domhh@)^n$.
Therefore by \equ(7.|t-fraction|) we have
\begindisplay
S=(1-{\tt T-HT})^{-1}{\tt HH}\,,
\enddisplay
and the probability generating function for our problem is
\begindisplay \openup6pt
G(z)&=\bigl(1-qz-(pz)(qz)\bigr)^{\!-1}(pz)^2\cr
&={p^2z^2\over1-qz-pqz^2}\,.
\eqno
\enddisplay

Our experience with the negative binomial distribution gives us a clue
that we can most easily calculate the mean and
variance of \thiseq\ by writing
\begindisplay
G(z)={z^2\over F(z)}\,,
\enddisplay
where
\begindisplay
F(z)={1-qz-pqz^2\over p^2}\,,
\enddisplay
and by calculating the ``mean'' and ``variance'' of this pseudo-pgf\/ $F(z)$.
(Once again we've introduced a function with $F(1)=1$.)
We have
\begindisplay
F'(1)&=(-q-2pq)/p^2=2-p^{-1}-p^{-2}\,;\cr
F''(1)&=-2pq/p^2=2-2p^{-1}\,.\cr
\enddisplay
Therefore, since $z^2=F(z)@G(z)$, \ $\Mean(z^2)=2$, and
$\Var(z^2)=0$, the mean and variance of distribution $G(z)$ are
\begindisplay
\Mean(G)&=2-\Mean(F)&=p^{-2}+p^{-1}\,;\eqno\cr
\Var(G)&=-\Var(F)&=p^{-4}+2p^{-3}-2p^{-2}-p^{-1}\,.\eqno\cr
\enddisplay
When $p=\half$ the mean and variance are $6$ and~$22$, respectively.
(Exercise~|divide-gfs| discusses the calculation of means and variances
by subtraction.)

%\smallbreak
\goodbreak
Now let's try a more intricate experiment: We will flip coins until the
pattern {\tt THTTH} is first obtained. The sum of winning positions
is now
\begindisplay
S&=\tt THTTH+HTHTTH+TTHTTH\cr
&\qquad\tt+HHTHTTH+HTTHTTH+THTHTTH+TTTHTTH+\cdots\,;
\enddisplay
this sum is more difficult to describe than the previous one. If we go
back to the method by which we solved the domino problems in Chapter~7,
we can obtain a formula for~$S$ by considering it as a ``"finite state
language"'' defined by the following ``"automaton"'':
\g\noindent\llap{``}\thinspace`You really are an automaton\dash---a calculating
machine,' I cried. `There is something positively inhuman in you at
times.'\thinspace''\par\hfill\dash---J.\thinspace H.
 "Watson"~[|holmes-four|]\g % chap2
\begindisplay
\unitlength=4pt
\beginpicture(60,10.5)(-8,-8.5)
\multiput(0,0)(10,0)6{\circle4}
\multiput(-8,0)(10,0)6{\vector(1,0)6}
\put(0,.1){\makebox(0,0){$0$}}
\put(10,.1){\makebox(0,0){$1$}}
\put(20,.1){\makebox(0,0){$2$}}
\put(30,.1){\makebox(0,0){$3$}}
\put(40,.1){\makebox(0,0){$4$}}
\put(50,.1){\makebox(0,0){$5$}}
\put(3.25,1.75){\makebox(0,0){\tt T}}
\put(13.25,1.75){\makebox(0,0){\tt H}}
\put(23.25,1.75){\makebox(0,0){\tt T}}
\put(33.25,1.75){\makebox(0,0){\tt T}}
\put(43.25,1.75){\makebox(0,0){\tt H}}
\put(1.5,-3.5){\makebox(0,0){\tt H}}
\put(11.5,-3.5){\makebox(0,0){\tt T}}
\put(21.5,-3.5){\makebox(0,0){\tt H}}
\put(31.5,-3.5){\makebox(0,0){\tt H}}
\put(41.5,-3.5){\makebox(0,0){\tt T}}
\ovaltlfalse
\ovaltrfalse
\put(-2,-2){\oval(4,8)}
\put(8,-2){\oval(4,8)}
\put(7.5,-5){\oval(25,4)}
\put(23,-3.5){\oval(14,4)}
\put(22.5,-6.5){\oval(35,4)}
\put(-5,-5){\vector(0,1)5}
\put(-4,-2){\vector(0,1)2}
\put(5,-6.5){\vector(0,1){6.5}}
\put(6,-2){\vector(0,1)2}
\put(16,-3.5){\vector(0,1){3.5}}
\put(20,-5){\line(0,1)3}
\put(30,-3.5){\line(0,1){1.5}}
\put(40,-6.5){\line(0,1){4.5}}
\endpicture
\enddisplay
The elementary events in the probability space are the sequences of
$\tt H$'s and $\tt T$'s that lead from state~$0$ to state~$5$. Suppose,
for example, that we have just seen {\tt THT}; then we are in state~$3$.
Flipping tails now takes us to state~$4$; flipping heads in state~$3$
would take us to
state~$2$ (not all the way back to state~$0$, since the {\tt TH} we've
just seen may be followed by {\tt TTH}).

In this formulation, we can let $S_k$ be the sum of all sequences
of {\tt H}'s and {\tt T}'s that lead to state~$k$; it follows that
\begindisplay   \openup3pt
S_0&=1+S_0\,{\tt H}+S_2\,{\tt H}\,,\cr
S_1&=S_0\,{\tt T}+S_1\,{\tt T}+S_4\,{\tt T}\,,\cr
S_2&=S_1\,{\tt H}+S_3\,{\tt H}\,,\cr
S_3&=S_2\,{\tt T}\,,\cr
S_4&=S_3\,{\tt T}\,,\cr
S_5&=S_4\,{\tt H}\,.\cr
\enddisplay
Now the sum $S$ in our problem is $S_5$; we can obtain it by solving
these six equations in the six unknowns $S_0$, $S_1$, \dots,~$S_5$.
Replacing $\tt H$ by~$pz$ and $\tt T$ by~$qz$ gives generating
functions where the coefficient of~$z^n$ in~$S_k$ is the probability
that we are in state~$k$ after $n$ flips.

In the same way, any diagram of transitions between states, where the
transition from state~$j$ to state~$k$ occurs with given probability~$p_{j,k}$,
leads to a set of simultaneous linear equations whose solutions are
generating functions for the state probabilities after $n$~transitions
have occurred. Systems of this kind are called {\it "Markov processes"},
and the theory of their behavior is intimately related to the theory
of linear equations.

But the coin-flipping problem can be solved in a much simpler way,
without the complexities of the general finite-state approach. Instead
of six equations in six unknowns $S_0$, $S_1$, \dots,~$S_5$, we can
characterize~$S$ with only two equations in two unknowns. The trick is
to consider the auxiliary sum $N=S_0+S_1+S_2+S_3+S_4$ of all flip
sequences that don't contain any occurrences of the given pattern
{\tt THTTH}:
\begindisplay
N=1+\tt H+T+HH+\cdots+THTHT+THTTT+\cdots\,.
\enddisplay
We have
\begindisplay
1+N(@{\tt H+T}@)=N+S\,,
\eqno\eqref|thtth-1|
\enddisplay
\looseness=-1
because every term on the left either ends with $\tt THTTH$ (and
belongs to~$S$) or doesn't (and belongs to~$N$); conversely, every
term on the right is either empty or belongs to $N\,\tt H$
or~$N\,\tt T$. And we also have the important additional equation
\begindisplay
N\,{\tt THTTH}=S+S\,{\tt TTH}\,,
\eqno\eqref|thtth-2|
\enddisplay
because every term on the left completes a term of~$S$ after either
the first~$\tt H$ or the second~$\tt H$, and because every 
term on the right belongs to the left.

The solution to these two simultaneous equations is easily obtained:
We have $N=(1-S)(1-{\tt H}-{\tt T})^{-1}$ from~\eq(|thtth-1|), hence
\begindisplay
(1-S)(1-{\tt T}-{\tt H})^{-1}\,{\tt THTTH}=S(1+{\tt TTH})\,.
\enddisplay
As before, we get the probability generating function $G(z)$ for
the number of flips if we replace $\tt H$ by~$pz$ and $\tt T$ by~$qz$.
A bit of simplification occurs since $p+q=1$, and we find
\begindisplay
{\bigl(1-G(z)\bigr)\,p^2q^3z^5\over1-z}=G(z)(1+pq^2z^3)\,;
\enddisplay
hence the solution is
\begindisplay
G(z)={p^2q^3z^5\over p^2q^3z^5+(1+pq^2z^3)(1-z)}\,.
\eqno
\enddisplay
Notice that $G(1)=1$, if $pq\ne0$; we do eventually encounter the
pattern $\tt THTTH$, with probability~$1$, unless the coin is rigged
so that it always comes up heads or always tails.

To get the mean and variance of the distribution \thiseq, we invert
$G(z)$ as we did in the previous problem, writing $G(z)=z^5\!/F(z)$
where $F$~is a polynomial:
\begindisplay
F(z)={p^2q^3z^5+(1+pq^2z^3)(1-z)\over p^2q^3}\,.
\eqno
\enddisplay
The relevant derivatives are
\begindisplay
F'(1)&=5-(1+pq^2)/p^2q^3\,,\cr
F''(1)&=20-6pq^2\!/p^2q^3\,;
\enddisplay
and if $X$ is the number of flips we get
\begindisplay
EX&=\Mean(G)=5-\Mean(F)=p^{-2}q^{-3}+p^{-1}q^{-1}\,;}\hidewidth{
 \eqno\eqref|thtth-e|\cr
VX&=\Var(G)&=-\Var(F)\cr
&&=-25+p^{-2}q^{-3}+7p^{-1}q^{-1}+\Mean(F)^2\cr
&&=(EX)^2-9p^{-2}q^{-3}-3p^{-1}q^{-1}\,.\eqno\eqref|thtth-v|
\enddisplay
When $p=\half$, the mean and variance are $36$ and $996$.

\smallbreak
Let's get general: The problem we have just solved was ``random'' enough
to show us how to analyze the case that we are waiting for the first
appearance of an
{\it arbitrary\/} pattern $A$ of heads and tails. Again we let $S$
be the sum of all winning sequences of $\tt H$'s and $\tt T$'s, and
we let~$N$ be the sum of all sequences that haven't encountered 
the pattern $A$~yet.
Equation \eq(|thtth-1|) will remain the same; equation \eq(|thtth-2|)
will become
\begindisplay \openup4pt
NA&=S\bigl(1+A^{(1)}\,\[A^{(m-1)}=A_{(m-1)}]+
A^{(2)}\,\[A^{(m-2)}=A_{(m-2)}]\cr
&\hskip7em+\cdots+
A^{(m-1)}\,\[A^{(1)}=A_{(1)}]\bigr)\,,
\eqno\eqref|general-solitaire|
\enddisplay
where $m$ is the length of~$A$, and where $A^{(k)}$ and $A_{(k)}$ denote
respectively
the last~$k$ characters and the first~$k$ characters of~$A$.
For example, if $A$ is the pattern $\tt THTTH$ we just studied, we have
\begindisplay
&A^{(1)}=\tt H\,,\quad&A^{(2)}=\tt TH\,,\quad&A^{(3)}=\tt TTH\,,\quad
 &A^{(4)}=\tt HTTH\,;\cr
&A_{(1)}=\tt T\,,\quad&A_{(2)}=\tt TH\,,\quad&A_{(3)}=\tt THT\,,\quad
 &A_{(4)}=\tt THTT\,.\cr
\enddisplay
Since the only perfect match is $A^{(2)}=A_{(2)}$, equation \thiseq\
reduces to~\eq(|thtth-2|).

Let $\widetilde A$ be the result of substituting $p^{-1}$ for $\tt H$
and $q^{-1}$ for~$\tt T$ in the pattern~$A$. Then it is not difficult to
generalize our derivation of \eq(|thtth-e|) and \eq(|thtth-v|) to
conclude (exercise~|prove-aflips|) that the general mean and variance are
\begindisplay \openup3pt
EX&=\sum_{k=1}^m{\widetilde A}_{(k)}\,\[A^{(k)}=A_{(k)}]\,;\eqno\eqref|aflip-e|\cr
VX&=(EX)^2-\sum_{k=1}^m\,(2k-1){\widetilde A}_{(k)}\,\[A^{(k)}=A_{(k)}]\,.
\eqno\eqref|aflip-v|\cr
\enddisplay

In the special case $p=\half$ we can interpret these formulas in a particularly
simple way. Given a pattern~$A$ of $m$~heads and tails, let
\begindisplay
A{:}A=\sum_{k=1}^m2^{k-1}\,\[A^{(k)}=A_{(k)}]\,.
\eqno\eqref|a:a|
\enddisplay
We can easily find the binary representation of this number by placing
a `$1$' under each position such that the string matches itself perfectly
when it is superimposed on a copy of itself that has been shifted to
start in this position:
\def\chek{\rlap{$\qquad\scriptstyle\sqrt{\vphantom2}$}}
\begindisplay \def\preamble{\hfill$##$&${}##$\hfil&&\hfil\tt##\hfil}
A&=&H&T&H&T&H&H&T&H&T&H\cr
A{:}A&=(&$1$&$0$&$0$&$0$&$0$&$1$&$0$&$1$&$0$&$1$&\rlap{$)_2=512+16+4+1=533$}%
\phantom{$0$}&
\phantom{$0$}&
\phantom{$0$}&
\phantom{$0$}&
\phantom{$0$}&
\phantom{$0$}&
\phantom{$0$}&
\phantom{$0$}\cr
&&H&T&H&T&H&H&T&H&T&H\chek\cr
&&&H&T&H&T&H&H&T&H&T&H\cr
&&&&H&T&H&T&H&H&T&H&T&H\cr
&&&&&H&T&H&T&H&H&T&H&T&H\cr
&&&&&&H&T&H&T&H&H&T&H&T&H\cr
&&&&&&&H&T&H&T&H&H&T&H&T&H\chek\cr
&&&&&&&&H&T&H&T&H&H&T&H&T&H\cr
&&&&&&&&&H&T&H&T&H&H&T&H&T&H\chek\cr
&&&&&&&&&&H&T&H&T&H&H&T&H&T&H\cr
&&&&&&&&&&&H&T&H&T&H&H&T&H&T&H\chek\cr
\enddisplay
Equation \eq(|aflip-e|) now tells us that the expected number of flips
until pattern~$A$ appears is exactly $2(A{:}A)$, if we use a fair coin,
because ${\widetilde A}_{(k)}=2^k$ when $p=q=\half$. This result,
\g``Chem bol'she periodov u nashego slova,
tem pozzhe ono po\t\i avl\t\i aets\t\i a.''\hfill\dash---A.\thinspace D. Solov'ev\g
first discovered by the Soviet mathematician A.\thinspace D. "Solov'ev" in 1966~[|soloviev|],
seems paradoxical at first glance: Patterns with no self-overlaps
occur sooner than overlapping patterns do! It takes almost twice as long
to encounter $\tt HHHHH$ as it does to encounter $\tt HHHHT$ or $\tt THHHH$.

\smallbreak
Now let's consider an amusing game that was invented by (of all people)
Walter "Penney"~[|penney|] in 1969. Alice and Bill flip a coin until
either $\tt HHT$ or $\tt HTT$ occurs; Alice wins if the pattern
$\tt HHT$ comes first, Bill wins if $\tt HTT$ comes first. 
This game\dash---now called ``"Penney ante"''\dash---certainly seems to be
fair, if played with a fair coin, because both patterns $\tt HHT$
and $\tt HTT$ have the same characteristics if we look at them in isolation:
The probability generating function for the waiting time until
$\tt HHT$ first occurs is
\begindisplay
G(z)={z^3\over z^3-8(z-1)}\,,
\enddisplay
\g\vskip-15pt Of \undertext{course} not! Who could they have an advantage over?\g
and the same is true for $\tt HTT$. Therefore neither Alice nor Bill
has an advantage, if they play solitaire.

But there's an interesting interplay between the patterns when both are
considered simultaneously. Let $S_A$ be the sum of Alice's winning
configurations, and let $S_B$ be the sum of Bill's:
\begindisplay
S_A&=\tt HHT+HHHT+THHT+HHHHT+HTHHT+THHHT+\cdots\,;\cr
S_B&=\tt HTT+THTT+HTHTT+TTHTT+THTHTT+TTTHTT+\cdots\,.\cr
\enddisplay
Also\dash---taking our cue from the trick that worked when only one pattern was
involved\dash---let us denote by~$N$ the sum of all sequences in which neither
player has won so far:
\begindisplay
N=1+\tt H+T+HH+HT+TH+TT+HHH+HTH+THH+\cdots\,.
\eqno\eqref|n-hht:htt|
\enddisplay
Then we can easily verify the following set of equations:
\begindisplay
1+N(@{\tt H+T}@)&=N+S_A+S_B\,;\cr
N\,{\tt HHT}&=S_A\,;\eqno\eqref|hht:htt|\cr
N\,{\tt HTT}&=S_A\,{\tt T}+S_B\,.
\enddisplay
If we now set ${\tt H}={\tt T}=\half$, the resulting value of $S_A$
becomes the probability that Alice wins, and $S_B$ becomes the probability
that Bill wins. The three equations reduce to
\begindisplay
\textstyle 1+N=N+S_A+S_B\,;\qquad {1\over8}N=S_A\,;\qquad
{1\over8}N=\half S_A+S_B\,;
\enddisplay
and we find $S_A={2\over3}$, $S_B={1\over3}$. Alice will win about twice
as often as Bill!

In a generalization of this game, Alice and Bill choose patterns $A$ and~$B$ of
heads and tails, and they flip coins until either $A$ or~$B$ appears.
The two patterns need not have the same length, but we assume that
$A$~doesn't occur within~$B$, nor does $B$ occur within~$A$. (Otherwise
the game would be degenerate. For example, if $A=\tt HT$ and $B=\tt THTH$,
poor Bill could never win; and if $A=\tt HTH$ and $B=\tt TH$, both players might
claim victory simultaneously.) Then we can write three equations
analogous to \eq(|general-solitaire|) and \eq(|hht:htt|):
\begindisplay \openup5pt
&1+N(@{\tt H+T}@)=N+S_A+S_B\,;}\hidewidth{\cr
&NA&=S_A\sum_{k=1}^l A^{(l-k)}\,\[A^{(k)}=A_{(k)}]
 \;+\;S_B\!\!\sum_{k=1}^{\min(l,m)} A^{(l-k)}\,\[B^{(k)}=A_{(k)}]\,;\cr
&NB&=S_A\!\!\sum_{k=1}^{\min(l,m)}B^{(m-k)}\,\[A^{(k)}=B_{(k)}]
 \;+\;S_B\sum_{k=1}^m B^{(m-k)}\,\[B^{(k)}=B_{(k)}]\,.\cr
\eqno\eqref|general-duel|\cr
\enddisplay
Here $l$ is the length of~$A$ and $m$ is the length of~$B$. For example, if
we have $A=\tt HTTHTHTH$ and $B=\tt THTHTTH$,
 the two pattern-dependent equations are
\begindisplay
N\,{\tt HTTHTHTH}&=S_A\,{\tt TTHTHTH}+S_A+S_B\,{\tt TTHTHTH}+S_B\,{\tt THTH}\,;\cr
N\,{\tt THTHTTH}&=S_A\,{\tt THTTH}+S_A\,{\tt TTH}+S_B\,{\tt THTTH}+S_B\,.\cr
\enddisplay
We obtain the victory probabilities by setting ${\tt H}={\tt T}=\half$, if
we assume that a fair coin is being used; this reduces the two crucial equations
to
\begindisplay \openup2pt
\eqalign{
N&=S_A\sum_{k=1}^l 2^k\,\[A^{(k)}=A_{(k)}]
 +S_B\sum_{k=1}^{\min(l,m)} 2^k\,\[B^{(k)}=A_{(k)}]\,;\cr
N&=S_A\sum_{k=1}^{\min(l,m)} 2^k\,\[A^{(k)}=B_{(k)}]
 +S_B\sum_{k=1}^m 2^k\,\[B^{(k)}=B_{(k)}]\,.\cr}
\eqno\eqref|general-duel-half|
\enddisplay
We can see what's going on if we
generalize the $A{:}A$ operation of \eq(|a:a|) to a function
of two independent strings $A$ and~$B$:
\begindisplay
A{:}B=\sum_{k=1}^{\min(l,m)}2^{k-1}\,\[A^{(k)}=B_{(k)}]\,.
\eqno\eqref|a:b|
\enddisplay
Equations \eq(|general-duel-half|) now become simply
\begindisplay
S_A(A{:}A)+S_B(B{:}A)=S_A(A{:}B)+S_B(B{:}B)\,;
\enddisplay
the "odds" in Alice's favor are
\begindisplay
{S_A\over S_B}={B{:}B-B{:}A\over A{:}A-A{:}B}\,.
\eqno\eqref|alice-odds|
\enddisplay
(This beautiful formula was discovered by John Horton "Conway" [|gardner-coins|].)

For example, if $A=\tt HTTHTHTH$ and $B=\tt THTHTTH$ as above, we have
$A{:}A=(10000001)_2=129$, $A{:}B=(0001010)_2=10$, $B{:}A=(0001001)_2=9$,
and $B{:}B=(1000010)_2=66$; so the ratio $S_A/S_B$ is $(66-9)/(129-10)=57/119$.
Alice will win this one only $57$ times out of every $176$, on the average.

Strange things can happen in Penney's game. For example,
the pattern $\tt HHTH$ wins over the pattern $\tt HTHH$ with
$3/2$~odds, and $\tt HTHH$ wins over $\tt THHH$ with $7/5$~odds. So
$\tt HHTH$ ought to be much better than $\tt THHH$. Yet $\tt THHH$
actually wins over $\tt HHTH$, with $7/5$~odds! The relation
\g Odd, odd.\g
"!nontransitive paradox" "!transitive law, failure of" "!paradox"
between patterns is not transitive. In fact, 
exercise~|second-player-win| proves that if Alice chooses any pattern
$\tau_1\tau_2\ldots \tau_l$ of length $l\ge3$, Bill can always ensure
better than even chances of winning if he chooses
the pattern ${\overline\tau}_2\tau_1\tau_2\ldots
\tau_{l-1}$, where ${\overline\tau}_2$ is the heads/tails opposite
of~$\tau_2$.

\beginsection 8.5 Hashing

Let's conclude this chapter by applying probability theory to computer
programming. Several important algorithms for storing and "retrieving
information" inside a computer are based on a technique called "!searching a table"
``"hashing".\qback'' The general problem is to maintain a set of
\g \noindent\llap{``}Somehow the verb `to~hash' magically became standard terminology
for key transformation during the mid-1960s, yet nobody was rash enough
to use such an undignified word publicly until 1967.''\par
\hfill\dash---D.\thinspace E. Knuth~[|knuth3|]\g
records that each contain a ``key'' value, $K$, and some data $D(K)$
about that key; we want to be able to find $D(K)$ quickly when $K$
is given. For example, each key might be the name of a student,
and the associated data might be that student's homework grades.

In practice, computers don't have enough capacity to set aside
one memory cell for every possible key; billions of keys are
possible, but comparatively few keys
are actually present in any
one application. One solution to the problem is to maintain two
tables \array KEY[j] and \array DATA[j] for $1\le j\le N$, where $N$~is
the total number of records that can be accommodated; another variable~$n$
tells how many records are actually present. Then we can search
for a given key~$K$ by going through the table sequentially in
an obvious way:

\smallskip
\item{S1} Set $j:=1$. (We've searched through all positions $<j$.)
\item{S2} If $j>n$, stop. (The search was unsuccessful.)
\item{S3} If $\array KEY[j]=K$, stop. (The search was successful.)
\item{S4} Increase $j$ by~$1$ and return to step S2. (We'll try again.)

\smallskip\noindent
After a successful search, the desired data entry $D(K)$ appears in
\array DATA[j]. After an unsuccessful search, we can insert $K$ and
$D(K)$ into the table by setting
\begindisplay
n:=j,\;\;\array KEY[n]:=K,\;\;\array DATA[n]:=D(K),
\enddisplay
assuming that the table was not already filled to capacity.

This method works, but it can be dreadfully slow; we need to repeat
step~S2 a total of $n+1$ times whenever an unsuccessful search is made, and
$n$~can be quite large.

Hashing was invented to speed things up.
The basic idea, in one of its popular forms,
is to use $m$ separate lists instead of one giant list. A ``hash function''
transforms every possible key~$K$ into a list number $h(K)$ between
$1$ and~$m$. An auxiliary table \array FIRST[i] for $1\le i\le m$
points to the first record in list~$i$; another auxiliary table
\array NEXT[j] for $1\le j\le N$ points to the
record following record~$j$ in its list. We assume that
\begindisplay
\array FIRST[i]&=-1\,,\qquad&\hbox{if list $i$ is empty};\cr
\array NEXT[j]&=0\,,\qquad&\hbox{if record $j$ is the last in its list}.\cr
\enddisplay
As before, there's
a variable~$n$ that tells how many records have been stored altogether.

For example, suppose the keys are names, and suppose that there are
$m=4$ lists based on the first letter of a name:
\begindisplay
h({\rm name})=\cases{1,&for A--F;\cr
 2,&for G--L;\cr
 3,&for M--R;\cr
 4,&for S--Z.\cr}
\enddisplay
We start with four empty lists and with $n=0$. If, say, the first
record has Nora as its key, we have $h({\rm Nora})=3$, so Nora becomes
the key of the first item in list~$3$. If the next two names are 
Glenn and Jim, they both go into list~$2$. Now the tables
in memory look like this:
\begindisplay
&\array FIRST[1]\!=\!-1,\quad
 \array FIRST[2]\!=\!2,\quad
 \array FIRST[3]\!=\!1,\quad
 \array FIRST[4]\!=\!-1.}\hidewidth{\cr
&\array KEY[1]=\rm Nora,\qquad&\array NEXT[1]=0\,;\cr
&\array KEY[2]=\rm Glenn,\qquad&\array NEXT[2]=3\,;\cr
&\array KEY[3]=\rm Jim,\qquad&\array NEXT[3]=0\,;\qquad n=3.\cr
\enddisplay
(The values of \array DATA[1], \array DATA[2], and \array DATA[3] are
confidential and will not be shown.) After 18 records have been inserted,
\g Let's hear it for the Concrete Math students who sat in the
front rows and lent their names to this experiment.\g
the lists might contain the names
\begindisplay \def\preamble{&##\hfil\qquad} \openup-4pt
list $1$&list $2$&list $3$&list $4$\cr
\noalign{\kern7pt\hrule\kern7pt}
Dianne&Glenn&Nora&Scott\cr
Ari&Jim&Mike&Tina\cr
Brian&Jennifer&Michael\cr
Fran&Joan&Ray\cr
Doug&Jerry&Paula\cr
&Jean\cr
\enddisplay
and these names would appear intermixed in the {\tt KEY} array with
{\tt NEXT} entries to keep the lists effectively separate.
If we now want to search for John, we have to scan through the six
names in list~$2$ (which happens to be
the longest list); but that's not nearly
as bad as looking at all 18 names.

Here's a precise specification of the algorithm that searches for
key~$K$ in accordance with this scheme:
\smallskip
\item{H1}Set $i:=h(K)$ and $j:=\array FIRST[i]$.
\item{H2}If $j\le0$, stop. (The search was unsuccessful.)
\item{H3}If $\array KEY[j]=K$, stop. (The search was successful.)
\item{H4}Set $i:=j$, then set $j:=\array NEXT[i]$ and return to step H2.
(We'll try again.)
\smallskip\noindent
For example, to search for Jennifer in the example given,
step~H1 would set $i:=2$ and $j:=2$; step~H3 would find that $\rm Glenn
\ne Jennifer$;
\g I bet their parents are glad about that.\g
 step~H4 would set $j:=3$; and step~H3
would find $\rm Jim\ne Jennifer$. One more iteration of steps H4 and~H3
would locate Jennifer in the table.

After a successful search, the desired data $D(K)$ appears in \array DATA[j],
as in the previous algorithm. After an unsuccessful search, we can
enter $K$ and~$D(K)$ in the table by doing the following operations:
\begindisplay
&n:=n+1;\cr
&\hbox{{\bf if\/} \ $j<0$ \ {\bf then} \ $\array FIRST[i]:=n$ \ {\bf else}
 \ $\array NEXT[i]:=n$;}\cr
&\array KEY[n]:=K;\;\;\array DATA[n]:=D(K);\;\;\array NEXT[n]:=0.
\eqno\eqref|insert-into-hash|
\enddisplay
Now the table will once again be up to date.

We hope to get lists of roughly equal length, because this will make
the task of searching about $m$~times faster. The value of~$m$ is
usually much greater than~$4$, so a factor of $1/m$
will be a significant improvement.

We don't know in advance what keys will be present, but it is
generally possible to choose the hash function~$h$ so that we can
consider $h(K)$ to be a random variable that is uniformly distributed
between $1$ and~$m$, independent of the hash values of other
keys that are present. In such cases computing the hash function
is like rolling a die that has $m$~faces. There's a chance that
all the records will fall into the same list, just as there's a chance
that a die will always turn up \diesix\kern1pt; but probability theory
tells us that the lists will {\it almost always\/} be pretty evenly balanced.

\subhead Analysis of Hashing: Introduction.

``"Algorithmic analysis"'' is a branch of computer science that derives
"!analysis of algorithms"
quantitative information about
the efficiency of computer methods. ``"Probabilistic analysis of
an algorithm"'' is the study of an algorithm's running time, considered
as a random variable that depends on assumed characteristics of the input data.
Hashing is an especially good candidate for probabilistic analysis,
because it is an extremely efficient method on the average, even though
its worst case is too horrible to contemplate. (The worst case occurs
when all keys have the same hash value.) Indeed, a computer programmer
who uses hashing had better be a believer in probability theory.

Let $P$ be the number of times step H3 is performed when the algorithm
above is used to carry out a search. (Each execution of~H3 is called
a ``probe'' in the table.) If we know~$P$, we know how often each
step is performed, depending on whether the search is successful
or unsuccessful:
\begindisplay \def\preamble{&\hfil##\hfil\qquad}
Step&Unsuccessful search&Successful search\cr
\noalign{\kern2pt}
H1&$1$ \ time&$1$ \ time\cr
H2&$P+1$ \ times&$P$ \ times\cr
H3&$P$ \ times&$P$ \ times\cr
H4&$P$ \ times&$P-1$ \ times\cr
\enddisplay
Thus the main quantity that governs the running time of the search procedure
is the number of probes,~$P$.

We can get a good mental picture of the algorithm by imagining that
we are keeping an address book that is organized in a special way, with room for
only one entry per page. On the cover of the book we note down the page
number for the first entry in each of $m$ lists; each name~$K$ determines the
list $h(K)$ that it belongs to. Every page inside the book
refers to the successor page in its list. The number of probes needed
to find an address in such a book is the number of pages we must consult.

If $n$ items have been inserted, their positions in the table depend only
on their respective hash values, $\<h_1,h_2,\ldots,h_n\>$. Each of
the $m^n$ possible sequences $\<h_1,h_2,\ldots,h_n\>$ is considered
to be equally likely, and $P$ is a random variable depending on
such a sequence.

\subhead Case 1: The key is not present.\g Check under the doormat.\g

Let's consider first the behavior of\/~$P$ in an unsuccessful search,
assuming that $n$~records have previously been inserted into the hash table.
In this case the relevant probability space consists of $m^{n+1}$
elementary events
\begindisplay
\omega=(h_1,h_2,\ldots,h_n,h_{n+1})
\enddisplay
where $h_j$ is the hash value of the $j$th key inserted, and where
$h_{n+1}$ is the hash value of the key for which the search is
unsuccessful. We assume that the hash function~$h$ has been chosen
properly so that $\Pr(\omega)=1/m^{n+1}$ for every such~$\omega$.

For example, if $m=n=2$, there are eight equally likely
possibilities:
\begindisplay \def\preamble{$\hfil##\hfil$\quad&$\hfil##\hfil$\quad&%
 $\hfil##\hfil$:\quad&$\hfil##\hfil$} \openup-3.5pt
h_1&h_2&h_3&P\cr
\noalign{\kern5pt\hrule\kern5pt}
1&1&1&2\cr
1&1&2&0\cr
1&2&1&1\cr
1&2&2&1\cr
2&1&1&1\cr
2&1&2&1\cr
2&2&1&0\cr
2&2&2&2\cr
\enddisplay
If $h_1=h_2=h_3$ we make two unsuccessful probes before concluding that
the new key $K$ is not present; if $h_1=h_2\ne h_3$ we make none; and so
on. This list
of all possibilities shows that $P$ has a probability
distribution given by the pgf $({2\over8}+{4\over8}z+{2\over8}z^2)
=(\half+\half z)^2$, when $m=n=2$.

An unsuccessful search makes one probe for
every item in list number $h_{n+1}$, so
we have the general formula
\begindisplay
P=\[h_1=h_{n+1}]\,+\,
  \[h_2=h_{n+1}]\,+\,\cdots\,+\,
  \[h_n=h_{n+1}]\,.
\eqno
\enddisplay
The probability that $h_j=h_{n+1}$ is $1/m$, for $1\le j\le n$; so it
follows that
\begindisplay
EP=E\[h_1=h_{n+1}]+
   E\[h_2=h_{n+1}]+\cdots+
   E\[h_n=h_{n+1}]={n\over m}\,.
\enddisplay
Maybe we should do that more slowly: Let $X_j$ be the random variable
\begindisplay
X_j=X_j(\omega)=\[h_j=h_{n+1}]\,.
\enddisplay
Then $P=X_1+\cdots+X_n$, and $EX_j=1/m$ for all $j\le n$; hence
\begindisplay
EP=EX_1+\cdots+EX_n=n/m\,.
\enddisplay
Good: As we had hoped, the average number of probes is $1/m$~times what it was
without hashing.
Furthermore the random variables $X_j$ are independent, and they each
have the same probability generating function
\begindisplay
X_j(z)={m-1+z\over m}\,;
\enddisplay
therefore the pgf for the total number of probes in an unsuccessful
search is
\begindisplay
P(z)=X_1(z)\ldots X_n(z)=\Bigl({m-1+z\over m}\Bigr)^{\!n}\,.
\eqno
\enddisplay
This is a binomial distribution, with $p=1/m$ and $q=(m-1)/m$;
in other words, the number of probes in an unsuccessful search
behaves just like the number of heads when we toss a biased
coin whose probability of heads is $1/m$ on each toss.
Equation \eq(|v-binom|) tells us that the variance of\/~$P$ is therefore
\begindisplay
npq={n(m-1)\over m^2}\,.
\enddisplay
When $m$ is large, the variance of $P$ is approximately $n/m$, so the
standard deviation is approximately $\sqrt{n/m}$.

\subhead Case 2: The key is present.

Now let's look at successful searches. In this case the appropriate
probability space is a bit more complicated, depending on our
application: We will let $\Omega$ be the set of all elementary events
\begindisplay
\omega=(h_1,\ldots,h_n;k)\,,
\eqno\eqref|succ-omega|
\enddisplay
where $h_j$ is the hash value for the $j$th key as before, and where
$k$~is the index of the key being sought (the key whose hash value
is $h_k$). Thus we have $1\le h_j\le m$ for $1\le j\le n$, and $1\le k\le n$;
there are $m^n\cdt n$ elementary events~$\omega$ in all.

Let $s_j$ be the probability that we are searching for the $j$th
key that was inserted into the table. Then
\begindisplay
\Pr(\omega)=s_k/m^n
\eqno
\enddisplay
if $\omega$ is the event \eq(|succ-omega|). (Some applications search
most often for the items that were inserted first, or for the items
that were inserted last, so we will not assume that each $s_j=1/n$.)
Notice that $\sum_{\omega\in\Omega}\Pr(\omega)=\sum_{k=1}^n s_k=1$,
hence \thiseq\ defines a legal probability distribution.

The number of probes $P$ in a successful search is $p$ if key~$K$ was
the $p$th key to be inserted into its list. Therefore
\begindisplay
P(h_1,\ldots,h_n;k)=\[h_1=h_k]\,+\,
  \[h_2=h_k]\,+\,\cdots\,+\,
  \[h_k=h_k]\,;
\eqno\eqref|probe-function|
\enddisplay
or, if we let $X_j$ be the random variable $\[h_j=h_k]$, we have
\begindisplay
P=X_1+X_2+\cdots+X_k\,.
\eqno\eqref|succ-p|
\enddisplay
Suppose, for example, that we have $m=10$ and $n=16$, and that the
hash values have the following ``random'' pattern:
\g Where have I seen that pattern before?\g
\begindisplay \let\preamble=\tablepreamble \let\quad=\enspace
&\omit&(h_1,\ldots,h_{16})&=&3&1&4&1&5&9&2&6&5&3&5&8&9&7&9&3&;\cr
&\omit&(P_1,\ldots,P_{16})&=&1&1&1&2&1&1&1&1&2&2&3&1&2&1&3&3&.\cr
\enddisplay
The number of probes $P_j$ needed to find the $j$th key is shown below $h_j$.

Equation \thiseq\ represents $P$ as a sum of random variables, but we
can't simply calculate $EP$ as $EX_1+\cdots+EX_k$ because
the quantity $k$ itself is a random variable. What is the
probability generating function for~$P$? To answer this question
we should digress a moment to talk about {\it"conditional probability"}.
\g Equation \eq(|moment-def|) was also a momentary digression.\g

If $A$ and $B$ are events in a probability space, we say that
the conditional probability of~$A$, given~$B$, is
\begindisplay
\Pr\prp(\omega\in A\mid\omega\in B)
 ={\Pr\prp(\omega\in A\cap B)\over\Pr\prp(\omega\in B)}\,.
\eqno
\enddisplay
For example, if $X$ and $Y$ are random variables, the conditional
probability of the event $X=x$, given that $Y=y$, is
\begindisplay
\Pr\prp(X=x\mid Y=y)={\Pr\prp(X=x\ {\rm and}\ Y=y)\over\Pr\prp(Y=y)}\,.
\eqno\eqref|pr-x-given-y|
\enddisplay
For any fixed $y$ in the range of\/~$Y$, the sum of these conditional
prob\-abil\-ities over all $x$ in the range of\/~$X$ is $\Pr\prp(Y=y)/{\Pr\prp(Y=y)}=1$;
therefore \thiseq\ defines a probability distribution, and we can define
a new random variable `$@ X\given y@$' such that
$\Pr\pbigi((X\given y)=x\bigr)=\Pr\prp(X=x\mid Y=y)$.

 If $X$ and~$Y$ are independent,
the random variable $X\given y$ will be essentially the same as~$X$,
regardless of the value of~$y$, because $\Pr\prp(X=x\mid Y=y)$ is equal to
$\Pr\prp(X=x)$ by \eq(|indep-def|); that's what independence means.
But if $X$ and~$Y$ are dependent, the random variables
$X\given y$ and $X\given y'$ need not resemble each other in any way
when $y\ne y'$.

If $X$ takes only nonnegative integer values, we can decompose its pgf into
a sum of conditional pgf's with respect to any other random variable~$Y$:
\begindisplay
G_X(z)=\sum_{y\in Y(\Omega)}\Pr\prp(Y=y)@G_{X\given y}(z)\,.
\eqno
\enddisplay
This holds because the coefficient of $z^x$ on the left side is $\Pr\prp(X=x)$,
for all $x\in X(\Omega)$, and on the right it is
\begindisplay \openup5pt
\sum_{y\in Y(\Omega)}\Pr\prp(Y=y)\Pr\prp(X=x\mid Y=y)
&=\sum_{y\in Y(\Omega)}\Pr\prp(X=x\ {\rm and}\ Y=y)\cr
&=\Pr\prp(X=x)\,.
\enddisplay
For example, if $X$ is the product of the spots on two fair dice and if
$Y$ is the sum of the spots, the pgf for $X\given6$ is
\begindisplay
\textstyle G_{X\given6}(z)={2\over5}z^5+{2\over5}z^8+{1\over5}z^9
\enddisplay
because the conditional probabilities for $Y=6$ consist of five equally
probable events $\{\,\dieone\diefive,$ $\dietwo\diefour,$
$\diethree\diethree,$ $\diefour\dietwo,$ $\diefive\dieone\,\}$.
Equation \thiseq\ in this case reduces to
\begindisplay \openup6pt
G_X(z)&=\textstyle{1\over36}G_{X\given2}(z)+{2\over36}G_{X\given3}(z)+
 {3\over36}G_{X\given4}(z)+{4\over36}G_{X\given5}(z)\cr
&\qquad\qquad\textstyle{5\over36}G_{X\given6}(z)+{6\over36}G_{X\given7}(z)+
 {5\over36}G_{X\given8}(z)+{4\over36}G_{X\given9}(z)\cr
&\qquad\qquad\textstyle{3\over36}G_{X\given10}(z)+{2\over36}G_{X\given11}(z)+
 {1\over36}G_{X\given12}(z)\,,
\enddisplay
a formula that is obvious once you understand it. (End of digression.)
\g Oh, now I understand what mathematicians mean when they say something
is ``"obvious",\qback'' ``"clear",\qback'' or ``"trivial".\qback''\g

In the case of hashing, \thiseq\ tells us how to write down the
pgf for probes in a successful search, if we let $X=P$ and $Y=K$.
For any fixed $k$ between $1$ and~$n$, the random variable $P\given k$
is defined as a sum of independent random variables $X_1+\cdots+X_k$;
this is \eq(|succ-p|). So it has the pgf
\begindisplay \postdisplaypenalty=-500
G_{P\given k}(z)=\Bigl({m-1+z\over m}\Bigr)^{\!k-1}z\,.
\enddisplay
Therefore the pgf for $P$ itself is clearly
\g\noindent\llap{``}By clearly, I mean a good freshman should be able to do it, although
it's not completely trivial.''\par
\hfill\dash---Paul Erd\H os~[|erdos-scottish|].\g
\begindisplay \openup6pt
G_P(z)&=\sum_{k=1}^n s_kG_{P\given k}(z)\cr
&=\sum_{k=1}^n s_k\Bigl({m-1+z\over m}\Bigr)^{\!k-1}z\cr
&=z\,S\Bigl({m-1+z\over m}\Bigr)\,,
\eqno\eqref|p:s-of-binomial|
\enddisplay
where
\begindisplay
S(z)=s_1+s_2z+s_3z^2+\cdots+s_nz^{n-1}
\eqno
\enddisplay
is the pgf for the search probabilities $s_k$ (divided by~$z$ for convenience).

Good. We have a probability generating function for~$P$; we can now
find the mean and variance by differentiation. It's somewhat easier
to remove the $z$~factor first, as we've done before, thus
finding the mean and variance of $P-1$ instead:
\begindisplay \openup5pt
F(z)&=G_P(z)/z=S\Bigl({m-1+z\over m}\Bigr)\,;\cr
F'(z)&={1\over m}S'\Bigl({m-1+z\over m}\Bigr)\,;\cr
F''(z)&={1\over m^2}S''\Bigl({m-1+z\over m}\Bigr)\,.\cr
\enddisplay
Therefore
\begindisplay \openup2pt
EP&=1+\Mean(F)=1+F'(1)=1+m^{-1}\Mean(S)\,;}\hidewidth{\eqno\eqref|e-success|\cr
\noalign{\smallskip}
VP&=\Var(F)&=F''(1)+F'(1)-F'(1)^2\cr
&&=m^{-2}S''(1)+m^{-1}S'(1)-m^{-2}S'(1)^2\cr
&&=m^{-2}\Var(S)+(m^{-1}-m^{-2})\Mean(S)\,.\eqno\eqref|v-success|\cr
\enddisplay
These are general formulas expressing the mean and variance of the
number of probes~$P$ in terms of the mean and variance of the
assumed search distribution~$S$.

For example, suppose we have $s_k=1/n$ for $1\le k\le n$. This means we
are doing a purely ``random'' successful search, with all keys in the
table equally likely. Then $S(z)$ is the uniform probability distribution
$U_n(z)$ in \eq(|u-pgf|), and we have $\Mean(S)=(n-1)/2$, $\Var(S)=
(n^2-1)/12$. Hence
\begindisplay \openup3pt
EP&={n-1\over2m}+1\,;\eqno\eqref|e-unif-success|\cr
VP&={n^2-1\over12m^2}+{(m-1)(n-1)\over2m^2}={(n-1)(6m+n-5)\over12m^2}\,.
\eqno\eqref|v-unit-success|
\enddisplay
Once again we have gained the desired speedup factor of $1/m$.
If $m=n/\!@\ln n$ and $n\to\infty$, the average number of probes per
successful search in this case is about $\half \ln n$, and the standard
deviation is asymptotically $(\ln n)/\mskip-1mu\sqrt{12}$.

On the other hand, we might suppose that $s_k=(kH_n)^{-1}$ for $1\le k\le n$;
this distribution is called ``"Zipf's law".\qback'' Then $\Mean(G)=n/H_n$
and $\Var(G)=\half n(n+1)/H_n-n^2\!/H_n^2$. The average number of probes
for $m=n/\!@\ln n$ as $n\to\infty$ is approximately~$2$, with standard
deviation asymptotic to $\sqrt{\mskip2mu\ln n}/\sqrt2$.

In both cases the analysis allows the cautious souls among us, who fear
the worst case, to rest easily: Chebyshev's inequality tells us
that the lists will be nice and short, except in extremely rare cases.

\subhead Case 2, continued: Variants of the variance.

We have just computed the variance of the number of probes in a successful
search, by considering $P$~to be a random variable over a probability space
with $m^n\cdt n$ elements $(h_1,\ldots,h_n;k)$. But we could have
adopted another point of view: Each pattern $(h_1,\ldots,h_n)$ of hash
\g OK, gang, time to put on your skim~suits again.\par
\hfill\dash---Friendly TA\g
values defines a random variable $P\given(h_1,\ldots,h_n)$, representing
the probes we make in a successful search of a particular hash table
on $n$~given keys. The average value of $P\given(h_1,\ldots,h_n)$, %namely
\begindisplay
A(h_1,\ldots,h_n)=
 \sum_{p=1}^n p\cdot\Pr\pbigi({\bigl(P\given(h_1,\ldots,h_n)\bigr)}=p\bigr)\,,
\eqno
\enddisplay
can be said to represent the running time of a successful search. This
quantity $A(h_1,\ldots,h_n)$ is a random variable that depends only
on $(h_1,\ldots,h_n)$, not on the final component~$k$.
We can write it in the form
\begindisplay
A(h_1,\ldots,h_n)=\sum_{k=1}^n s_k\,P(h_1,\ldots,h_n;k)\,,
\enddisplay
where $P(h_1,\ldots,h_n;k)$ is defined in \eq(|probe-function|),
since $P\given(h_1,\ldots,h_n)=p$ with probability
\begindisplay \openup5pt
{\sum_{k=1\lower1pt\hbox{}}^n\Pr\pbigi(P(h_1,\ldots,h_n;k)=p\bigr)\over
\sum_{k=1}^{n\raise1ex\hbox{}}\Pr(h_1,\ldots,h_n;k)}
&={\sum_{k=1\lower1pt\hbox{}}^n m^{-n}s_k\bigi[P(h_1,\ldots,h_n;k)=p\bigr]\over
\sum_{k=1}^{n\raise1ex\hbox{}} m^{-n}s_k}\cr
&=\sum_{k=1}^n s_k\bigi[P(h_1,\ldots,h_n;k)=p\bigr]\,.
\enddisplay

The mean value of $A(h_1,\ldots,h_n)$, obtained by summing over all
$m^n$ possibilities $(h_1,\ldots,h_n)$ and dividing by~$m^n$, will
be the same as the mean value we obtained before in \eq(|e-success|).
But the {\it "variance"\/} of $A(h_1,\ldots,h_n)$ is something different;
this is a variance of $m^n$ averages, not a variance of ${m^n\cdt n}$
probe counts. For example, if $m=1$ (so that there is only one list),
the ``average'' value $A(h_1,\ldots,h_n)=A(1,\ldots,1)$ is actually
constant, so its variance $VA$ is zero; but the number of probes
in a successful search is not constant, so the variance $VP$ is nonzero.
\g But the VP is nonzero only in an election year.\g

We can illustrate this difference between variances by carrying out the
calculations for general $m$ and~$n$ in the simplest case, when
$s_k=1/n$ for $1\le k\le n$. In other words, we will assume temporarily
that there is a uniform distribution of search keys. Any given
sequence of hash values $(h_1,\ldots,h_n)$ defines $m$~lists that
contain respectively $(n_1,n_2,\ldots,n_m)$ entries for some numbers~$n_j$,
where
\begindisplay
n_1+n_2+\cdots+n_m=n\,.
\enddisplay
A successful search in which each of the $n$ keys in the table is equally
likely will have an average running time of
\begindisplay \openup4pt
A(h_1,\ldots,h_n)&={(1{+}\cdots{+}n_1)+
    (1{+}\cdots{+}n_2)+\cdots+
    (1{+}\cdots{+}n_m)\over n}\cr
&={n_1(n_1{+}1)+n_2(n_2{+}1)+\cdots+n_m(n_m{+}1)\over2n}\cr
&={n_1^2+n_2^2+\cdots+n_m^2+n\over2n}
\enddisplay
probes. Our goal is to calculate the variance of this quantity $A(h_1,\ldots,
h_n)$, over the probability space consisting of
all $m^n$ sequences $(h_1,\ldots,h_n)$.

The calculations will be simpler, it turns out, if we compute the
variance of a slightly different quantity,
\begindisplay
B(h_1,\ldots,h_n)={n_1\choose2}+{n_2\choose2}+\cdots+{n_m\choose2}\,.
\enddisplay
We have
\begindisplay
A(h_1,\ldots,h_n)=1+B(h_1,\ldots,h_n)/n\,,
\enddisplay
hence the mean and variance of~$A$ satisfy
\begindisplay
EA=1+{EB\over n}\,;\qquad VA={VB\over n^2}\,.
\eqno\eqref|a-reduced-to-b|
\enddisplay
The probability that the list sizes will be $n_1$, $n_2$, \dots, $n_m$
is the multinomial coefficient
\begindisplay
{n\choose n_1,n_2,\ldots,n_m}={n!\over n_1!\,n_2!\,\ldots n_m!}
\enddisplay
divided by $m^n$; hence the pgf for $B(h_1,\ldots,h_n)$ is
\begindisplay
B_n(z)=\sum\twoconditions{n_1,n_2,\ldots,n_m\ge0}{n_1+n_2+\cdots+n_m=n}
{n\choose n_1,n_2,\ldots,n_m}\,z^{{n_1\choose2}+{n_2\choose2}+\cdots+
{n_m\choose 2}}\,m^{-n}\,.
\enddisplay
This sum looks a bit scary to inexperienced eyes,
but our experiences in Chapter 7 have taught us to recognize it as
an $m$-fold convolution. Indeed, if we consider the exponential
super-generating function
\begindisplay
G(w,z)=\sum_{n\ge0}B_n(z) {m^n@w^n\over n!}\,,
\enddisplay
we can readily verify that $G(w,z)$ is simply an $m$th power:
\begindisplay
G(w,z)=\biggl(\sum_{k\ge0}z^{k\choose2}\,{w^k\over k!}\biggr)^{\!m}\,.
\enddisplay
As a check, we can try setting $z=1$; we get $G(w,1)=(e^w)^m$, so the
coefficient of $m^n@w^n\!/n!$ is $B_n(1)=1$.

If we knew the values of $B_n'(1)$ and $B_n''(1)$, we would be able to
calculate $\Var(B_n)$. So we take partial derivatives of $G(w,z)$
with respect to~$z$:
\begindisplay \openup5pt
{\partial\over\partial z}G(w,z)&=\sum_{n\ge0}B'_n(z){m^n@w^n\over n!}\cr
&=m\biggl(\sum_{k\ge0}z^{k\choose2}\,{w^k\over k!}\biggr)^{\!m-1}
 \sum_{k\ge0}{k\choose2}z^{{k\choose2}-1}\,{w^k\over k!}\,;\cr
{\partial^2\over\partial z^2}G(w,z)&=\sum_{n\ge0}B''_n(z){m^n@w^n\over n!}\cr
&=m(m{-}1)\biggl(\sum_{k\ge0}z^{k\choose2}\,{w^k\over k!}\biggr)^{\!m-2}
\biggl( \sum_{k\ge0}{k\choose2}z^{{k\choose2}{-}1}\,{w^k\over k!}\biggr)^{\!2}\cr
&\qquad+m\biggl(\sum_{k\ge0}z^{k\choose2}\,{w^k\over k!}\biggr)^{\!m-1}
 \sum_{k\ge0}{k\choose2}\biggl({k\choose2}\mskip-2mu-\mskip-2mu1\biggr)
 z^{{k\choose2}-2}\,{w^k\over k!}\,.
\enddisplay
Complicated, yes; but everything simplifies greatly when we set $z=1$.
For example, we have
\begindisplay \openup3pt
&\sum_{n\ge0}B_n'(1){m^nw^n\over n!}=me^{(m-1)w}\sum_{k\ge2}{w^k\over2@(k-2)!}\cr
&\qquad=me^{(m-1)w}\sum_{k\ge0}{w^{k+2}\over 2@k!}\cr
&\qquad={mw^2e^{(m-1)w}\over2}e^w
  =\sum_{n\ge0}{(mw)^{n+2}\over2m\,n!}
  =\sum_{n\ge0}{n(n{-}1)m^n@w^n\over2m\,n!}\,,\cr
\enddisplay
and it follows that
\begindisplay
B'_n(1)={n\choose2}{1\over m}\,.
\eqno
\enddisplay
The expression for $EA$ in \eq(|a-reduced-to-b|) now gives
$EA=1+(n-1)/2m$, in agreement with \eq(|e-unif-success|).

The formula for $B_n''(1)$ involves the similar sum
\begindisplay
&\sum_{k\ge0}{k\choose2}\biggl({k\choose2}{-}1\biggr) {w^k\over k!}
={1\over4}\sum_{k\ge0}{(k+1)k(k-1)(k-2)w^k\over k!}\cr
&\qquad={1\over4}\sum_{k\ge3}{(k+1)w^k\over (k-3)!}
={1\over4}\sum_{k\ge0}{(k+4)w^{k+3}\over k!}
 =\bigl(\textstyle{1\over4}w^4+w^3\bigr)e^w\,;
\enddisplay
hence we find that
\begindisplay
\sum_{n\ge0}B_n''(1){m^n@w^n\over n!}
&=m(m{-}1)e^{w(m-2)}\bigl(\textstyle\half w^2e^w\bigr)^{\!2}
 {+}me^{w(m-1)}\bigl(\textstyle{1\over4}w^4{+}w^3\bigr)e^w\cr
&=me^{wm}\bigl(\textstyle{1\over4}mw^4+w^3\bigr)\,;\cr
\noalign{\medskip}
B_n''(1)&={n\choose2}\biggl({n\choose2} - 1\biggr) {1\over m^2}\,.
\eqno
\enddisplay
Now we can put all the pieces together and evaluate the desired variance~$VA$.
Massive cancellation occurs, and the result is surprisingly simple:
\begindisplay \openup5pt
VA={VB\over n^2}&={B_n''(1)+B_n'(1)-B'_n(1)^2\over n^2}\cr
&={n(n-1)\over m^2n^2}\biggl({(n+1)(n-2)\over4}+{m\over2}-{n(n-1)\over4}\biggr)\cr
&={(m-1)(n-1)\over2m^2n}\,.
\eqno\eqref|unif-var-ave|
\enddisplay

When such ``coincidences'' occur, we suspect that there's a mathematical
reason; there might be another way to attack the problem, explaining why the
answer has such a simple form. And indeed, there is another approach
(in exercise~|prove-gen-va|), which shows that the variance of the
average successful search has the general form
\begindisplay
VA={m-1\over m^2}\sum_{k=1}^n s_k^2(k-1)
\eqno\eqref|gen-va|
\enddisplay
when $s_k$ is the probability that the $k$th-inserted element is
being sought. Equation \eq(|unif-var-ave|) is the special case
$s_k=1/n$ for $1\le k\le n$.

Besides the variance of the average, we might also consider the
average of the variance. In other words, each sequence
$(h_1,\ldots,h_n)$ that defines a hash table also defines a
probability distribution for successful searching, and the variance
of this probability distribution tells how spread out the number
of probes will be in different successful searches. For example,
let's go back to the case where we inserted $n=16$ things into
$m=10$ lists:
\g Where have I seen that pattern before?\medskip
  Where have I seen that graffito before?\medskip
 $I\eta\nu P_\pi$."!pizza"\g
\begindisplay \let\preamble=\tablepreamble \let\quad=\enspace
&\omit&(h_1,\ldots,h_{16})&=&3&1&4&1&5&9&2&6&5&3&5&8&9&7&9&3\cr
&\omit&(P_1,\ldots,P_{16})&=&1&1&1&2&1&1&1&1&2&2&3&1&2&1&3&3\cr
\enddisplay
A successful search in the resulting hash table has the pgf
\begindisplay
G(3,1,4,1,\ldots,3)
&=\sum_{k=1}^{16} s_kz^{P(3,1,4,1,\ldots,3;k)}\cr
&=s_1z+s_2z+s_3z+s_4z^2+\cdots+s_{16}z^3\,.
\enddisplay
We have just considered the average number of probes in a successful
search of this table, namely $A(3,1,4,1,\ldots,3)
 =\Mean\bigl(G(3,1,4,1,\ldots,3)\bigr)$.
We can also consider the variance,
\begindisplay
&s_1\cdt1^2+s_2\cdt1^2+s_3\cdt1^2+s_4\cdt2^2+\cdots+s_{16}\cdt3^2\cr
&\qquad-(s_1\cdt1+s_2\cdt1+s_3\cdt1+s_4\cdt2+\cdots+s_{16}\cdt3)^2\,.
\enddisplay
This variance is a random variable, depending on $(h_1,\ldots,h_n)$,
so it is natural to consider its average value.

In other words, there are three natural kinds of variance that we may
wish to know, in order to understand the behavior of a successful
search: The {\it overall "variance"\/} of the number of probes, taken over
all $(h_1,\ldots,h_n)$ and~$k$; the {\it variance of the average\/} number
of probes, where the average is taken over all~$k$ and the variance is then
taken over all~$(h_1,\ldots,h_n)$; and the {\it average of the variance\/}
"!average variance"
of the number of the probes, where the variance is taken over all~$k$
and the average is then taken over all $(h_1,\ldots,h_n)$. In symbols,
the overall variance is
\begindisplay \openup4pt
VP&=\sum_{1\le h_1,\ldots,h_n\le m}\,\sum_{k=1}^n{s_k\over m^n}P(h_1,\ldots,h_n;k)^2\cr
&\qquad-\biggl(\sum_{1\le h_1,\ldots,h_n\le m}\,
 \sum_{k=1}^n{s_k\over m^n}P(h_1,\ldots,h_n;k)\biggr)^{\!2}\,;
\enddisplay
the variance of the average is
\begindisplay \openup4pt
VA&=\sum_{1\le h_1,\ldots,h_n\le m}{1\over m^n}
 \biggl(\sum_{k=1}^n s_k P(h_1,\ldots,h_n;k)\biggr)^{\!2}\cr
&\qquad-\biggl(\sum_{1\le h_1,\ldots,h_n\le m}{1\over m^n}
 \sum_{k=1}^n s_kP(h_1,\ldots,h_n;k)\biggr)^{\!2}\,;
\enddisplay
and the average of the variance is
\begindisplay \openup4pt
AV&=\sum_{1\le h_1,\ldots,h_n\le m}{1\over m^n}
 \Biggl(\sum_{k=1}^n s_k P(h_1,\ldots,h_n;k)^2\cr
&\hskip11em  -\biggl(\sum_{k=1}^n s_kP(h_1,\ldots,h_n;k)\biggr)^{\!2}\Biggr)\,.
\enddisplay

It turns out that these three quantities are interrelated in a simple way:
\begindisplay
VP=VA+AV\,.
\eqno\eqref|vp:va+av|
\enddisplay
In fact, conditional probability distributions always satisfy the identity
\begindisplay
VX=V\bigl(E(X\given Y)\bigr)+E\bigl(V(X\given Y)\bigr)
\eqno\eqref|v:ev+ve|
\enddisplay
if $X$ and $Y$ are random variables in any probability space and if $X$
takes real values. (This identity is proved in exercise |prove-v:ev+ve|.)
Equation \eq(|vp:va+av|) is the special case where
$X$ is the number of probes in a successful search and $Y$ is
the sequence of hash values $(h_1,\ldots,h_n)$.

The general equation
\eq(|v:ev+ve|) needs to be understood carefully, because the notation
tends to conceal the different random variables and probability spaces
in which expectations and variances are being calculated. For each $y$
"!expectation, \string\see expected value"
in the range of\/~$Y$, we have defined the random variable $X\given y$
in \eq(|pr-x-given-y|), and this random variable has an expected value
$E(X\given y)$ depending on~$y$. Now $E(X\given Y)$ denotes the random
variable whose values are $E(X\given y)$ as $y$ ranges over all possible
values of\/~$Y$, and $V\bigl(E(X\given Y)\bigr)$ is the variance of
this random variable with respect to the probability distribution of\/~$Y$.
\g(Now is a good time to do warmup exercise~|conditional-warmup|.)\g
Similarly, $E\bigl(V(X\given Y)\bigr)$ is the average of the random
variables $V(X\given y)$ as $y$~varies. On the left of \eq(|v:ev+ve|)
is $VX$, the unconditional variance of\/~$X$. Since variances are
nonnegative, we always have
\begindisplay
VX\ge V\bigl(E(X\given Y)\bigr)\And VX\ge E\bigl(V(X\given Y)\bigr)\,.
\eqno
\enddisplay

\subhead Case 1, again: Unsuccessful search revisited.

Let's bring our microscopic examination of hashing to a close by
doing one more calculation typical of algorithmic analysis.
This time we'll look more
closely at the {\it total "running time"\/} associated with an unsuccessful search,
assuming that the computer will insert the previously unknown key
into its memory.

The insertion process in \eq(|insert-into-hash|) has two cases, depending
on whether $j$ is negative or zero. We have $j<0$ if and only if $P=0$,
\g$P$ is still the number of probes.\g
since a negative value comes from the {\tt FIRST} entry of an empty
list. Thus, if the list was previously empty, we have $P=0$ and
we must set $\array FIRST[h_{n+1}]:=n+1$. (The new record will be
inserted into position $n+1$.) Otherwise we have
$P>0$ and we must set a {\tt LINK} entry to $n+1$. These two cases
may take different amounts of time; therefore the total running time
for an unsuccessful search has the form
\begindisplay
T=\alpha+\beta P+\delta\[P=0]\,,
\eqno
\enddisplay
where $\alpha$, $\beta$, and $\delta$ are constants that depend on the
computer being used and on the way in which hashing is encoded in
that machine's internal language. It would be nice to know the
mean and variance of $T$, since such information is more relevant
in practice than the mean and variance of~$P$.

So far we have used probability generating functions only in connection
with random variables that take nonnegative integer values. But it
turns out that we can deal in essentially the same way with
\begindisplay
G_X(z)=\sum_{\omega\in\Omega}\Pr(\omega)z^{X(\omega)}
\enddisplay
when $X$ is any real-valued random variable, because the essential
characteristics of $X$ depend only on the behavior of $G_X$ near
$z=1$, where powers of~$z$ are well defined. For example, the
running time \thiseq\ of an unsuccessful search is a random variable,
defined on the probability space of equally likely hash values
$(h_1,\ldots,h_n,h_{n+1})$ with $1\le h_j\le m$; we can consider the series
\begindisplay
G_T(z)={1\over m^{n+1}}\sum_{h_1=1}^m\!\!
 \cdots\!\sum_{h_n=1}^m\sum_{h_{n+1}=1}^m\!\!z^{\alpha+\beta P(h_1,\ldots,
 h_{n+1})+\delta[P(h_1,\ldots,h_{n+1})=0]}
\enddisplay
to be a pgf
even when $\alpha$, $\beta$, and $\delta$ are not integers. (In fact,
the parameters $\alpha$, $\beta$, $\delta$ are physical quantities
that have dimensions of time; they aren't even pure numbers! Yet we
can use them in the exponent of~$z$.) We can still calculate the
mean and variance of\/~$T$, by evaluating $G_T'(1)$ and $G_T''(1)$
and combining these values in the usual way.

The generating function for $P$ instead of $T$ is
\begindisplay
P(z)=\Bigl({m-1+z\over m}\Bigr)^{\!n}=\sum_{p\ge0}\Pr\prp(P=p)z^p\,.
\enddisplay
Therefore we have
\begindisplay \openup3pt
G_T(z)&=\sum_{p\ge0}\Pr\prp(P=p)z^{\alpha+\beta p+\delta[p=0]}\cr
&=z^\alpha\Bigl((z^\delta-1)\Pr\prp(P=0)+\sum_{p\ge0}\Pr\prp(P=p)z^{\beta p}\Bigr)\cr
&=z^\alpha\biggl((z^\delta-1)\Bigl({m-1\over m}\Bigr)^{\!n}
 +\Bigl({m-1+z^\beta\over m}\Bigr)^{\!n}\biggr)\,.\cr
\enddisplay
The determination of $\Mean(G_T)$ and $\Var(G_T)$ is now routine:
\begindisplay \openup5pt
\Mean(G_T)&=G_T'(1)=\alpha+\beta{n\over m}
 +\delta\Bigl({m-1\over m}\Bigr)^{\!n}\,;\eqno\cr
G_T''(1)&=\alpha(\alpha-1)+2\alpha\beta{n\over m}+\beta(\beta-1){n\over m}
 +\beta^2{n(n-1)\over m^2}\cr
&\hskip7em+2\alpha\delta\Bigl({m-1\over m}\Bigr)^{\!n}
 +\delta(\delta-1)\Bigl({m-1\over m}\Bigr)^{\!n}\,;\cr
\Var(G_T)&=G_T''(1)+G_T'(1)-G_T'(1)^2\cr
&=\beta^2{n(m-1)\over m^2}
 -2\beta\delta\Bigl({m-1\over m}\Bigr)^{\!n}{n\over m}\cr
&\hskip7em+\delta^2\biggl( \Bigl({m-1\over m}\Bigr)^{\!n}
 -\Bigl({m-1\over m}\Bigr)^{\!2n}\biggr)\,.\eqno
\enddisplay

In Chapter 9 we will learn how to estimate quantities like this
when $m$ and~$n$ are large. If, for example, $m=n$ and $n\to\infty$,
the techniques of Chapter~9 will show that the mean and variance
of~$T$ are respectively $\alpha+\beta+\delta e^{-1}+O(n^{-1})$
and $\beta^2-2\beta\delta e^{-1}+\delta^2(e^{-1}-e^{-2})+O(n^{-1})$.
If $m=n/\!@\ln n$ and $n\to\infty$ the corresponding results are
\begindisplay
\Mean(G_T)&=\beta\ln n+\alpha+\delta/n+O\bigl((\log n)^2\!/n^2\bigr)\,;\cr
\Var(G_T)&=\beta^2\ln n-\bigl((\beta\ln n)^2+2\beta\delta\ln n-\delta^2\bigr)/n
 +O\bigl((\log n)^3\!/n^2\bigr)\,.
\enddisplay

\beginexercises

\subhead \kern-.05em Warmups

\ex:
What's the probability of doubles in the probability distribution $\Pr_{01}$
of~\eq(|pr2-def|), when one die is fair and the other is loaded?
What's the probability that $S=7$ is rolled?
\answer ${1\over24}+{1\over48}+{1\over48}+{1\over48}+{1\over48}+{1\over24}
={1\over6}$. (In fact, we {\it always\/} get doubles with probability~$1\over6$
when at least one of the dice is fair.)
Any two faces whose sum is~$7$ have the same
probability in distribution
 $\Pr_1$, so $S=7$ has the same probability as doubles.\par
\def\dieone{\die{\diedot22}}
\def\dietwo{\die{\diedot11\diedot33}}
\def\diethree{\die{\diedot11\diedot22\diedot33}}
\def\diefour{\die{\diedot11\diedot33\diedot13\diedot31}}
\def\diefive{\die{\diedot11\diedot22\diedot33\diedot13\diedot31}}
\def\diesix{\die{\diedot11\diedot12\diedot13\diedot31\diedot32\diedot33}}
\def\dieseven{\die{\diedot11\diedot12\diedot13\diedot31\diedot32\diedot33\diedot22}}
\def\dieeight{\die{\diedot11\diedot12\diedot13\diedot31\diedot32\diedot33%
 \diedot21\diedot23}}
\def\die#1{{\unitlength=2.5pt\beginpicture(5,4)(-.5,1)
 \put(0,0){\line(0,1)4}
 \put(0,0){\line(1,0)4}
 \put(0,4){\line(1,0)4}
 \put(4,0){\line(0,1)4}#1\endpicture}}
\def\diedot#1#2{\put(#1,#2){\disk1}}

\ex:
What's the probability that the top and bottom cards of a randomly
shuffled deck are both aces? (All $52!$ permutations have probability $1/52!$.)
\answer There are $12$ ways to specify the top and bottom cards and
$50!$ ways to arrange the others; so the probability is $12\cdt50!/52!=
12/(51\cdt52)={1\over17\cdt13}={1\over221}$.

\ex:
"Stanford"'s Concrete Math students were asked in 1979 to flip coins
until they got heads twice in succession, and to report the number
\g Why only ten\par numbers?
\medskip The other students either weren't empiricists
or they were just too flipped out.\g
of flips required. The answers were
\begindisplay
3,\,2,\,3,\,5,\,10,\,2,\,6,\,6,\,9,\,2\,.
\enddisplay
"Princeton"'s Concrete Math students were asked in 1987 to do a similar
thing, with the following results:
\begindisplay
10,\,2,\,10,\,7,\,5,\,2,\,10,\,6,\,10,\,2\,.
\enddisplay
Estimate the mean and variance, based on
(a) the Stanford sample; (b)~the Princeton sample.
\answer ${1\over10}(3+2+\cdots+9+2)=4.8$;
${1\over9}(3^2+2^2+\cdots+9^2+2^2-10(4.8)^2)={388\over45}$, which is
approximately $8.6$.
The true mean and variance with a fair coin are $6$ and~$22$, so Stanford
had an unusually heads-up class.
The corresponding Princeton figures are $6.4$ and ${562\over45}\approx 12.5$.
(This distribution has $\kappa_4=2974$, which is rather large. Hence
the standard deviation of this variance estimate when $n=10$ is also
rather large,
$\sqrt{\mathstrut2974/10+2(22)^{2\mathstrut}\!/9}\approx20.1$ according to
exercise~|var-hatv|. One cannot complain that the students cheated.)

\ex:\exref|divide-gfs|%
Let $H(z)=F(z)/G(z)$, where $F(1)=G(1)=1$. Prove that
\begindisplay
\Mean(H)&=\Mean(F)-\Mean(G)\,,\cr
\Var(H)&=\Var(F)-\Var(G)\,,\cr
\enddisplay
in analogy with \eq(|ind-sum1|) and \eq(|ind-sum2|), if the
indicated derivatives exist at $z=1$.
\answer This follows from \equ(8.|ind-sum1|) and \equ(8.|ind-sum2|), because
$F(z)=G(z)H(z)$. (A similar formula holds for all the cumulants,
even though $F(z)$ and $G(z)$ may have negative coefficients.)

\ex:
Suppose Alice and Bill play the game \eq(|hht:htt|) with a biased coin
that comes up heads with probability~$p$. Is there a value of~$p$ for which
the game becomes fair?
\answer Replace $\tt H$ by $p$ and $\tt T$ by $q=1-p$. If $S_A=S_B=\half$
we have $p^2qN=\half$ and $pq^2N=\half q+\half$; the solution is
$p=1/\phi^2$, $q=1/\phi$.

\ex:\exref|conditional-warmup|%
What does the conditional variance law \eq(|v:ev+ve|) reduce to,
when $X$ and $Y$ are independent random variables?
\answer In this case $X\given y$ has the same distribution as $X$,
for all~$y$, hence $E(X\given Y)=EX$ is constant and
$V\bigl(E(X\given Y)\bigr)=0$. Also $V(X\given Y)$ is constant and
equal to its expected value.

\subhead Basics

\ex:
Show that if two "dice" are loaded with the same probability distribution,
the probability of doubles is always at least $1\over6$.
\answer We have $1=(p_1+p_2+\cdots+p_6)^2\le6(p_1^2+p_2^2+\cdots+p_6^2)$
by Chebyshev's monotonic inequality of Chapter~2.

\ex:
Let $A$ and $B$ be events such that $A\cup B=\Omega$. Prove that
\begindisplay
\Pr\prp(\omega\in A\cap B)=
 \Pr\prp(\omega\in A)\Pr\prp(\omega\in B)
 -\Pr\prp(\omega\notin A)\Pr\prp(\omega\notin B)\,.
\enddisplay
\answer Let $p=\Pr\prp(\omega\in A\cap B)$,
$q=\Pr\prp(\omega\notin A)$, and $r=\Pr\prp(\omega\notin B)$. Then $p+q+r=1$, and
the identity to be proved is $p=(p+r)(p+q)-qr$.

\ex:
Prove or disprove: If $X$ and $Y$ are independent random variables,
then so are $F(X)$ and~$G(Y)$, when $F$ and $G$ are any functions.
\answer This is true (subject to the obvious proviso that $F$ and $G$
are defined on the respective ranges of\/~$X$ and $Y$), because
\begindisplay \openup6pt
\Pr\pbigi(F(X)=f\ {\rm and}\ G(Y)=g\bigr)
&=\sum\twoconditions{x\in F^{-1}(f)}{y\in G^{-1}(g)}\Pr\prp(X=x\ {\rm and}\ Y=y)\cr
&=\sum\twoconditions{x\in F^{-1}(f)}{y\in G^{-1}(g)}\Pr\prp(X=x)\cdot\Pr\prp(Y=y)\cr
&=\Pr\pbigi(F(X)=f\bigr)\cdot\Pr\pbigi(G(y)=g\bigr)\,.
\enddisplay

\ex:
What's the maximum number of elements that can be medians of a random
variable~$X$, according to definition \eq(|median1|)?
\answer Two. Let $x_1<x_2$ be medians; then $1\le\Pr\prp(X\le x_1)+
\Pr\prp(X\ge x_2)\le 1$, hence equality holds. (Some discrete distributions
have no median elements. For example, let $\Omega$ be the set of all
fractions of the form $\pm1/n$, with $\Pr(+1/n)=\Pr(-1/n)={\pi^2\over12}n^{-2}$.)

\ex:
Construct a random variable that has finite mean and infinite variance.
\answer For example, let $K=k$ with probability $4/(k+1)(k+2)(k+3)$,
for all integers $k\ge0$. Then $EK=1$, but $E(K^2)=\infty$.
(Similarly we can construct random variables with finite cumulants
through $\kappa_m$ but with $\kappa_{m+1}=\infty$.)

\ex:\exref|tail-ineq|%
\hangindent2\parindent \textindent{a}% \item combined with \itemitem
If $P(z)$ is the pgf for the random variable $X$, prove that
\begindisplay
\Pr(\hbox{$X\le r$})&\le x^{-r}P(x)\qquad\hbox{for $0<x\le1$;}\cr
\Pr(\hbox{$X\ge r$})&\le x^{-r}P(x)\qquad\hbox{for $x\ge1$.}\cr
\enddisplay
(These important relations are called the {\sl "tail inequalities".})
\itemitem{b}In the special case $P(z)=(1+z)^n\!/2^n$, use the first tail
inequality to prove that $\sum_{k\le\alpha n}{n\choose k}\le1/\alpha^{\alpha n}
(1-\alpha)^{(1-\alpha)n}$ when $0<\alpha<\half$.
\answer (a) Let $p_k=\Pr\prp(X=k)$. If $0<x\le1$, we have $\Pr\prp(X\le r)=
\sum_{k\le r}p_k\le\sum_{k\le r}x^{k-r}p_k\le\sum_k x^{k-r}p_k=x^{-r}P(x)$.
The other inequality has a similar proof. (b)~Let $x=\alpha/(1-\alpha)$ to
minimize the right-hand side.
(A more precise estimate for the given sum
is obtained in exercise 9.|binomial-tail|.)

\ex:
If $X_1$, \dots, $X_{2n}$ are independent random variables with the same
distribution, and if $\alpha$ is any real number whatsoever, prove that
\begindisplay
\Pr\biggl(\Bigl\vert{X_1+\cdots+X_{2n}\over2n}-\alpha\Bigr\vert\le
 \Bigl\vert{X_1+\cdots+X_n\over n}-\alpha\Bigr\vert\biggr)\;\ge\;\half\,.
\enddisplay
\answer (Solution by Boris "Pittel".) Let us set $Y=(X_1+\cdots+X_n)/n$ and
$Z=(X_{n+1}+\cdots+X_{2n})/n$. Then
\begindisplay
&\Pr\biggl(\Bigl\vert{Y+Z\over2}-\alpha\Bigr\vert\le\bigl\vert Y
 -\alpha\bigr\vert\biggr)\cr
&\qquad\ge\;\Pr\biggl(\Bigl\vert{Y-\alpha\over2}\Bigr\vert+
 \Bigl\vert{Z-\alpha\over2}\Bigr\vert\le\bigl\vert Y-\alpha\bigr\vert\biggr)\cr
&\qquad=\;\Pr\bigl(@\vert Z-\alpha\vert\le\vert Y-\alpha\vert@\bigr)\;\ge\;
\textstyle\half\,.\cr
\enddisplay
The last inequality is, in fact, `$>$' in any discrete probability
distribution, because $\Pr\prp(Y=Z)>0$.
\source{Thomas M. "Cover".*}

\ex:
Let $F(z)$ and $G(z)$ be probability generating functions, and let
\begindisplay
H(z)=p\,F(z)+q\,G(z)
\enddisplay
where $p+q=1$. (This is called a {\it"mixture"\/} of\/ $F$ and~$G$;
it corresponds to flipping a coin and choosing probability distribution
$F$ or~$G$ depending on whether the coin comes up heads or tails.)
Find the mean and variance of\/~$H$ in terms of $p$, $q$, and the mean and
variance of $F$ and~$G$.
\answer $\Mean(H)=p\Mean(F)+q\Mean(G)$;
$\Var(H)=p\Var(F)+q\Var(G)+pq\bigl(\Mean(F)-\Mean(G)\bigr){}^2$.
(A mixture is actually a special case of conditional probabilities:
Let $Y$ be the coin, let $X\given@\tt H$ be generated by $F(z)$,
and let $X\given@\tt T$ be generated by $G(z)$. Then
$VX=EV(X\given Y)+VE(X\given Y)$, where $EV(X\given Y)=
pV(X\given@{\tt H})+qV(X\given@{\tt T})$
and $VE(X\given Y)$
is the variance of $pz^{\Mean(F)}+qz^{\Mean(G)}$.)

\ex:\exref|pgf-composition|%
If $F(z)$ and $G(z)$ are probability generating functions, we can define
another pgf\/~$H(z)$ by ``"composition"'':
\begindisplay
H(z)=F\bigl(G(z)\bigr)\,.
\enddisplay
Express $\Mean(H)$ and $\Var(H)$ in terms of $\Mean(F)$, $\Var(F)$,
$\Mean(G)$, and $\Var(G)$. (Equation \eq(|p:s-of-binomial|) is a special case.)
\answer By the chain rule, $H'(z)=G'(z)@F'\bigl(G(z)\bigr)$;
$H''(z)=G''(z)@F'\bigl(G(z)\bigr)+G'(z)^2F''\bigl(G(z)\bigr)$. Hence
\begindisplay
\Mean(H)&=\Mean(F)\Mean(G)\,;\cr
\Var(H)&=\Var(F)\Mean(G)^2+\Mean(F)\Var(G)\,.\cr
\enddisplay
(The random variable corresponding to probability distribution~$H$ can
be understood as follows: Determine a nonnegative integer~$n$ by
distribution~$F@$; then add the values of~$n$ independent random variables
that have distribution~$G$. The identity for variance in this exercise
is a special case of \equ(8.|v:ev+ve|), when $X$ has distribution~$H$
and $Y$ has distribution~$F$.)
\source{[|knuth1|, exercise 1.2.10--17].}

\ex:
Find a closed form for the super generating function
$\sum_{n\ge0}F_n(z)@w^n$, when $F_n(z)$ is the football-fixation generating
function defined in \eq(|football-pgf|).
\answer $e^{w(z-1)}\!/(1-w)$.

\ex:
Let $X_{n,p}$ and $Y_{n,p}$ have the binomial and negative binomial
distributions, respectively, with parameters $(n,p)$. (These distributions
are defined in \eq(|binom-pgf|) and \eq(|negbinom-pgf|).) Prove that
$\Pr\prp(Y_{n,p}\le m)=\Pr\prp(X_{m+n,p}\ge n)$. What identity in
binomial coefficients does this imply?
\answer $\Pr\prp(Y_{n,p}\le m)=\Pr\prp(Y_{n,p}+n\le m+n)={}$probability
that we need $\le m+n$ tosses to obtain $n$~heads${}={}$probability
that $m+n$ tosses yield $\ge n$ heads${}=\Pr\prp(X_{m+n,p}\ge n)$. Thus
\begindisplay
\sum_{k\le m}{n+k-1\choose k}p^nq^k
&=\sum_{k\ge n}{m+n\choose k}p^kq^{m+n-k}\cr
&=\sum_{k\le m}{m+n\choose k}p^{m+n-k}q^k\,;\cr
\enddisplay
and this is \equ(5.|partial-binomial|) with $n=r$, $x=q$, $y=p$.
\source{"Patil" [|patil|].}

\ex:
A random variable $X$ is said
\g The distribution of fish per unit volume of water.\g
 to have the {\it "Poisson distribution"\/} with
mean~$\mu$ if $\Pr\prp(X=k)=e^{-\mu}\mu^k\!/k!$ for all $k\ge0$.
\itemitem{a}What is the pgf of such a random variable?
\itemitem{b}What are its mean, variance, and other cumulants?
\answer (a) $G_X(z)=e^{\mu(z-1)}$. (b) The $m$th cumulant is~$\mu$,
 for all~$m\ge1$.
(The case $\mu=1$ is called $F_\infty$ in \equ(8.|football-infty|).)

\ex:
Continuing the previous exercise, let $X_1$ be a random Poisson variable
with mean $\mu_1$, and let $X_2$ be a random Poisson variable with
mean $\mu_2$, independent of\/~$X_1$.
\itemitem{a} What is the probability that $X_1+X_2=n$?
\itemitem{b} What are the mean, variance, and other cumulants of $2X_1+3X_2$?
\answer (a) $G_{X_1+X_2}(z)=G_{X_1}(z)@G_{X_2}(z)=e^{(\mu_1+\mu_2)(z-1)}$.
Hence the probability is $e^{-\mu_1-\mu_2}(\mu_1+\mu_2)^n\!/n!$; the sum
of independent Poisson variables is Poisson.
(b)~In general, if $K_mX$ denotes the $m$th cumulant of a random variable~$X$,
we have $K_m(aX_1+bX_2)=a^m(K_mX_1)+b^m(K_mX_2)$, when $a,b\ge0$.
Hence the answer is $2^m\mu_1+3^m\mu_2$.

\ex:\exref|prove-aflips|%
Prove \eq(|aflip-e|) and \eq(|aflip-v|), the general formulas for
mean and variance of the time needed to wait for a given pattern
of heads and tails.
\answer The general pgf will be $G(z)=z^m\!/F(z)$, where
\begindisplay\openup4pt
F(z)&=z^m+(1-z)\sum_{k=1}^m{\widetilde A}_{(k)}\[A^{(k)}=A_{(k)}]@z^{m-k}\,,\cr
F'(1)&=m-\sum_{k=1}^m{\widetilde A}_{(k)}\[A^{(k)}=A_{(k)}]\,,\cr
F''(1)&=m(m-1)-2\sum_{k=1}^m(m-k){\widetilde A}_{(k)}\[A^{(k)}=A_{(k)}]\,.\cr
\enddisplay

\ex:\exref|n-meaning|%
What does the value of $N$ represent, if $\tt H$ and $\tt T$ are both set equal
to~$\half$ in \eq(|n-hht:htt|)?
\answer This is $\sum_{n\ge0}q_n$, where $q_n$ is the probability that the
game between Alice and Bill is still incomplete after $n$~flips. Let
$p_n$ be the probability that the game ends at the $n$th flip; then
$p_n+q_n=q_{n-1}$. Hence the average time to play the game is
$\sum_{n\ge1}np_n=(q_0-q_1)+2(q_1-q_2)+3(q_2-q_3)+\cdots=q_0+q_1+q_2+\cdots
=N$, since $\lim_{n\to\infty}nq_n=0$.
\par Another way to establish this answer is to replace $\tt H$ and
$\tt T$ by $\half z$. Then the derivative of the first equation in
\equ(8.|hht:htt|) tells us that $N(1)+N'(1)=N'(1)+S_A'(1)+S_B'(1)$.
\par By the way, $N={16\over3}$.

\ex:\exref|prove-v:ev+ve|%
Prove \eq(|v:ev+ve|), the law of conditional expectations and variances.
\answer By definition we have $V(X\given Y)=E(X^2\given Y)-\bigl(E(X\given Y)
\bigr){}^2$ and
$V\bigl(E(X\given Y)\bigr)=E\bigl((E(X\given Y))^2\bigr)
 -\bigl(E\bigl(E(X\given Y)\bigr)\bigr){}^2$;
hence $E\bigl(V(X\given Y)\bigr)+V\bigl(E(X\given Y)\bigr)=
E\bigl(E(X^2\given Y)\bigr)
 -\bigl(E\bigl(E(X\given Y)\bigr)\bigr){}^2$.
But $E\bigl(E(X\given Y)\bigr)=\sum_y\Pr\prp(Y=y)E(x\given y)=
\sum_{x,y}\Pr\prp(Y=y)\*\Pr\pbigi((X\given y)=x\bigr)=EX$ and
$E\bigl(E(X^2\given Y)\bigr)=E(X^2)$, so the result is just $VX$.

\subhead Homework exercises

\ex:
Let $\Pr_{00}$ be the probability distribution of two fair dice, and
let $\Pr_{11}$ be the probability distribution of two loaded dice as
given in~\eq(|pr1-def|). Find all events~$A$ such that $\Pr_{00}(A)=\Pr_{11}(A)$.
Which of these events depend only on the random variable~$S$?
(A probability space with $\Omega=D^2$ has $2^{36}$ events; only
$2^{11}$ of those events depend on $S$ alone.)
\answer Let $\Omega_0=\{\,\dieone,\,\diesix\,\}^2$ and
$\Omega_1=\{\,\dietwo,\,\diethree,\,\diefour,\,\diefive\,\}^2$;
and let $\Omega_2$ be the other 16 elements of~$\Omega$. Then
$\Pr_{11}(\omega)-\Pr_{00}(\omega)={+20\over576}$, ${-7\over576}$, ${+2\over576}$
according as $\omega\in\Omega_0$, $\Omega_1$, $\Omega_2$. The events~$A$
must therefore be chosen with $k_j$ elements from $\Omega_j$, where
$(k_0,k_1,k_2)$ is one of the following:
$(0,0,0)$,
$(0,2,7)$,
$(0,4,14)$,
$(1,4,4)$,
$(1,6,11)$,
$(2,6,1)$,
$(2,8,8)$,
$(2,10,15)$,
$(3,10,5)$,
$(3,12,12)$,
$(4,12,2)$,
$(4,14,9)$,
$(4,16,16)$.
For example, there are ${4\choose2}{16\choose6}{16\choose1}$ events of
type~$(2,6,1)$. The total number of such events is
$[z^0](1+z^{20})^4(1+z^{-7})^{16}(1+z^2)^{16}$, which turns out to be
$1304872090$. If we restrict ourselves to events that depend on $S$ only,
we get 40 solutions $S\in A$, where $A=\emptyset$,
$\{{2\atop12},{4\atop10},{6\atop8}\}$,
$\{{2\atop12},5,9\}$,
$\{2,12,{4\atop10},{6\atop8},5,9\}$,
$\{2,4,6,8,10,12\}$,
$\{{3\atop11},7,{5\atop9},4,10\}$,
and the complements of these sets.
(Here the notation `$2\atop12$' means either $2$ or $12$ but not both.)

\ex:
Player $J$ rolls $2n+1$ fair dice and removes those that come up \diesix.
Player $K$ then calls a number between $1$ and~$6$, rolls the remaining
dice, and removes those that show the number called. This process is repeated
until no dice remain. The player who has removed the most total dice ($n+1$
or~more) is the winner.
\itemitem{a}What are the mean and variance of the total number of dice
that $J$~removes? \Hint: The dice are independent.
\itemitem{b}What's the probability that $J$ wins, when $n=2$?
\answer (a) Any one of the dice ends up in $J$'s possession with
probability $p={1\over6}+\bigl({5\over6}\bigr){}^2p$; hence $p={6\over11}$.
Let $q={5\over11}$. Then the pgf for $J$'s total holdings is $(q+pz)^{2n+1}$,
with mean $(2n+1)p$ and variance $(2n+1)pq$, by \equ(8.|v-binom|).
(b)~${5\choose3}p^3q^2+{5\choose4}p^4q+{5\choose5}p^5={94176\over161051}
\approx .585$.
\source{John Knuth (age 4) and DEK; 1975 final."!Knuth, John""!Knuth, Don"}

\ex:
Consider a gambling game in which you stake a given amount~$A$ and you
roll a fair die. If $k$ spots turn up, you multiply your stake by
$2(k-1)/5$. (In particular, you double the stake whenever you roll
\diesix, but you lose everything if you roll \dieone.) You can stop
at any time and reclaim the current stake. What are the mean and
variance of your stake after $n$~rolls? (Ignore any effects of
rounding to integer amounts of currency.)
\answer The pgf for the current stake after $n$
rolls is $G_n(z)$, where
\begindisplay \openup3pt
G_0(z)&=z^A\,;\cr
G_n(z)&=\textstyle\sum_{k=1}^6G_{n-1}(z^{2(k-1)/5})/6\,,
\qquad\hbox{for $n>0$}.
\enddisplay
(The noninteger exponents cause no trouble.) It follows that
\g \vskip-20pt
This problem can perhaps be solved more easily without generating
functions than with them.\g
$\Mean(G_n)=\Mean(G_{n-1})$, and $\Var(G_n)+\Mean(G_n)^2=
{22\over15}(\Var(G_{n-1})+\Mean(G_{n-1})^2)$. So the mean is
always~$A$, but the variance grows
to $\bigl(\bigl({22\over15}\bigr){}^n-1\bigr)A^2$.

\ex:
Find the mean and variance of the number of $l$-cycles in a random
permutation of $n$ elements. (The football victory problem discussed
in \eq(|e-fixedpts|), \eq(|mu2-fixedpts|), and
\eq(|football-pgf|) is the special case $l=1$.)
\answer The pgf\/ $F_{l,n}(z)$ satisfies $F'_{l,n}(z)=F_{l,n-l}(z)/l$;
hence $\Mean(F_{l,n})=F'_{l,n}(1)=\[n\ge l]/l$ and $F''_{l,n}(1)
=\[n\ge2l]/l^2$; the variance is easily computed. (In fact, we have
\begindisplay
F_{l,n}(z)=\sum_{0\le k\le n/l}{1\over k!}\Bigl({z-1\over l}\Bigr)^{\!k}\,,
\enddisplay
which approaches a "Poisson distribution" with mean $1/l$ as $n\to\infty$.)
\source{[|knuth1|, exercise 1.3.3--18].}

\ex:\exref|estimate-kappa3|%
Let $X_1$, $X_2$, \dots, $X_n$ be independent samples of the
random variable~$X$. Equations \eq(|hat-e|) and \eq(|hat-v|) explain
how to estimate the mean and variance of\/~$X$ on the basis of these
observations; give an analogous formula for estimating the third
"cumulant"~$\kappa_3$. (Your formula should be an ``unbiased'' estimate,
"!unbiased estimate"
in the sense that its expected value should be~$\kappa_3$.)
\answer $(n^2\Sigma_3-3n\Sigma_2\Sigma_1
+2\Sigma_1^3)/n(n-1)(n-2)$ has the desired mean, where $\Sigma_k=
X_1^k+\cdots+X_n^k$. This follows from the identities
\begindisplay
E\Sigma_3&=n\mu_3\,;\cr
E(\Sigma_2\Sigma_1)&=n\mu_3+n(n-1)\mu_2\mu_1\,;\cr
E(\Sigma_1^3)&=n\mu_3+3n(n-1)\mu_2\mu_1+n(n-1)(n-2)\mu_1^3\,.\cr
\enddisplay
Incidentally, the third cumulant is $\kappa_3=E\bigl((X-EX)^3\bigr)$,
but the fourth cumulant does not have such a simple expression;
we have $\kappa_4=E\bigl((X-EX)^4\bigr)-3(VX)^2$.
\source{"Fisher" [|ra-fisher|].}

\ex:
What is the average length of the coin-flipping game \eq(|hht:htt|),
\itemitem{a}given that Alice wins?
\itemitem{b}given that Bill wins?
\answer (The exercise implicitly calls for $p=q=\half$, but the general
answer is given here for completeness.) Replace $\tt H$ by $pz$ and $\tt T$
by $qz$, getting $S_A(z)=p^2qz^3\!/(1-pz)(1-qz)(1-pqz^2)$ and
$S_B(z)=pq^2z^3\!/(1-qz)(1-pqz^2)$. The pgf for the conditional probability
that Alice wins at the $n$th flip, given that she wins the game, is
\begindisplay
{S_A(z)\over S_A(1)}=z^3\cdot
{q\over 1-pz}\cdot{p\over 1-qz}\cdot{1-pq\over 1-pqz^2}\,.
\enddisplay
This is a product of pseudo-pgf's, whose mean is $3+p/q+q/p+2pq/(1-pq)$.
The formulas for Bill are the same but without the factor $q/(1-pz)$, so
Bill's mean is $3+q/p+2pq/(1-pq)$. When $p=q=\half$, the answer in
case~(a) is $17\over3$; in case~(b) it is $14\over3$. Bill wins only
half as often, but when he does win he tends to win sooner. The overall
average number of flips is ${2\over3}\cdt{17\over3}+{1\over3}\cdt{14\over3}
={16\over3}$, agreeing with exercise~|n-meaning|. The solitaire game for
each pattern has a waiting time of~$8$.

\ex:
Alice, Bill, and Computer flip a fair coin until one of the respective
patterns $A=\tt HHTH$, $B=\tt HTHH$, or $C=\tt THHH$ appears for the
first time. (If only two of these patterns were involved,
we know from \eq(|alice-odds|) that
$A$ would probably beat~$B$, that $B$ would probably beat~$C$,
and that $C$ would probably beat~$A$; but all three patterns
are simultaneously
in the game.) What are each player's chances of winning?
\answer Set ${\tt H}={\tt T}=\half$ in
\begindisplay
1+N(@{\tt H+T}@)&=N+S_A+S_B+S_C\cr
N\,{\tt HHTH}&=S_A({\tt HTH}+1)+S_B({\tt HTH+TH})+S_C({\tt HTH+TH})\cr
N\,{\tt HTHH}&=S_A({\tt THH+H})+S_B({\tt THH}+1)+S_C({\tt THH})\cr
N\,{\tt THHH}&=S_A({\tt HH})+S_B({\tt H})+S_C\cr
\enddisplay
to get the winning probabilities. In general we will have
$S_A+S_B+S_C=1$ and
\begindisplay
S_A(A{:}A)+S_B(B{:}A)+S_C(C{:}A)
&=S_A(A{:}B)+S_B(B{:}B)+S_C(C{:}B)\cr
&=S_A(A{:}B)+S_B(B{:}C)+S_C(C{:}C)\,.
\enddisplay
In particular, the equations $9S_A+3S_B+3S_C=5S_A+9S_B+S_C=2S_A+4S_B+8S_C$
imply that $S_A={16\over52}$, $S_B={17\over52}$, $S_C={19\over52}$.
\source{"Guibas" and "Odlyzko" [|guibas-odlyzko|].}

\ex:\exref|more-variances|%
The text considers three kinds of variances associated with successful search
in a hash table. Actually there are two more: We can consider the average
(over~$k$) of the variances (over $h_1$, \dots,~$h_n$) of $P(h_1,\ldots,h_n;k)$;
and we can consider the variance (over~$k$) of the averages
(over $h_1$, \dots,~$h_n$). Evaluate these quantities.
\answer The variance of $P(h_1,\ldots,h_n;k)\given k$ is the variance of
the shifted binomial distribution
$\bigl((m-1+z)/m\bigr){}^{k-1}z$, which is $(k-1)({1\over m})(1-{1\over m})$
by \equ(8.|v-binom|). Hence the average of the variance is $\Mean(S)(m-1)/m^2$.
The variance of the average is the variance of $(k-1)/m$, namely
$\Var(S)/m^2$. According to \equ(8.|v:ev+ve|), the sum of these two
quantities should be $VP$, and it is. Indeed, we have just replayed
the derivation of \equ(8.|v-success|) in slight disguise. (See exercise~%
|pgf-composition|.)

\ex:
An apple is located at vertex $A$ of "pentagon" $ABCDE$, and a worm is located
"!worm and apple"
two vertices away, at~$C$. Every day the worm crawls with equal probability
\g "Schr\"odinger"'s worm.\g
to one of the two adjacent vertices. Thus after one day the worm is at
vertex~$B$ or vertex~$D$, each with probability~$\half$.
After two days, the worm might be back at~$C$ again, because it has no
memory of previous positions. When it reaches vertex~$A$, it stops to dine.
\itemitem{a}What are the mean and variance of the number of days until dinner?
\itemitem{b}Let $p$ be the probability that the number of days is $100$ or more.
What does "Chebyshev's inequality" say about $p$?
\itemitem{c}What do the "tail inequalities" (exercise |tail-ineq|) tell us
about $p$?
\answer (a) A brute force solution would set up five equations in five unknowns:
\begindisplay
&\textstyle
A=\half zB+\half zE\,;\quad B=\half zC\,;\quad C=1+\half zB+\half zD\,;\cr
&\textstyle
D=\half zC +\half zE\,;\quad E=\half zD\,.
\enddisplay
 But positions $C$ and~$D$ are equidistant from
the goal, as are $B$ and~$E$, so we can lump them together. If $X=B+E$
and $Y=C+D$, there are now three equations:
\begindisplay
\textstyle A=\half zX\,;\quad X=\half zY\,;\quad Y=1+\half zX+\half zY\,.
\enddisplay
Hence $A=z^2\!/(4-2z-z^2)$; we have $\Mean(A)=6$ and $\Var(A)=22$. (Rings
a bell? In fact,
this problem is equivalent to flipping a fair coin until getting heads
twice in a row: Heads means ``advance toward the apple'' and tails
means ``go back.'')
(b)~Chebyshev's inequality says that $\Pr\prp(S\ge100)=
\Pr\pbigi((S-6)^2\ge94^2\bigr)\le 22/94^2\approx.0025$.
(c)~The second tail inequality says that $\Pr\prp(S\ge100)\le1/x^{98}(4-2x-x^2)$
for all $x\ge1$, and we get the upper bound $0.00000005$ when $x=(\sqrt{49001}
-99)/100$. (The actual probability is approximately
$0.0000000009$, according to exercise |fib-heads|.)

\ex:
Alice and Bill are in the military, stationed in one of the five states
Kansas, Nebraska, Missouri, Oklahoma, or Colorado. Initially Alice is
in Nebraska and Bill is in Oklahoma. Every month each person is reassigned
to an adjacent state, each adjacent state being equally likely. (Here's
a diagram of the adjacencies:
\begindisplay
\unitlength=2pt
\beginpicture(100,39)(-50,-20)
	\put(0,0){\disk{1.5}}
	\put(0,15){\disk{1.5}}
	\put(25,0){\disk{1.5}}
	\put(0,-15){\disk{1.5}}
	\put(-25,0){\disk{1.5}}
	\put(0,15){\circle{4.5}}
	\put(0,-15){\circle{4.5}}
	\put(0,0){\line(0,1){15}}
	\put(0,0){\line(1,0){25}}
	\put(0,0){\line(0,-1){15}}
	\put(0,0){\line(-1,0){25}}
	\put(0,15){\line(5,-3){25}}
	\put(25,0){\line(-5,-3){25}}
	\put(0,-15){\line(-5,3){25}}
	\put(-25,0){\line(5,3){25}}
	\put(9,3){\makebox(0,0){Kansas}}
	\put(13,17){\makebox(0,0){Nebraska}}
	\put(34,3){\makebox(0,0){Missouri}}
	\put(14,-17){\makebox(0,0){Oklahoma}}
	\put(-35,3){\makebox(0,0){Colorado}}
\endpicture
\enddisplay
The initial states are circled.)
\g Definitely a finite-state situation.\g
For example, Alice is restationed after the first month to Colorado,
Kansas, or Missouri, each with probability $1/3$. Find the mean and
variance of the number of months it takes Alice and Bill to find
each other. (You may wish to enlist a computer's help.)
\answer By symmetry, we can reduce each month's situation to one
of four possibilities:
\g \noindent\llap{``}"Toto", I have a feeling we're not in Kansas anymore.''\par
\hfill\dash---Dorothy"!Gale""!Baum"\g
\begindisplay \openup-2pt
&D,\ &\hbox{the states are diagonally opposite;}\cr
&A,\ &\hbox{the states are adjacent and not Kansas;}\cr
&K,\ &\hbox{the states are Kansas and one other;}\cr
&S,\ &\hbox{the states are the same.}
\enddisplay
Considering the Markovian transitions, we get four equations
\begindisplay \openup2pt
D&=\textstyle 1+z({2\over9}D+{2\over12}K)\cr
A&=\textstyle z({4\over9}A+{4\over12}K)\cr
K&=\textstyle z({4\over9}D+{4\over9}A+{4\over12}K)\cr
S&=\textstyle z({3\over9}D+{1\over9}A+{2\over12}K)\cr
\enddisplay
whose sum is $D+K+A+S=1+z(D+A+K)$. The solution is
\begindisplay
S={81z-45z^2-4z^3\over243-243z+24z^2+8z^3}\,,
\enddisplay
but the simplest way to find the mean and variance may be to
write $z=1+w$ and expand in powers of $w$, ignoring multiples of~$w^2$:
\begindisplay
D&=\textstyle{27\over16}+{1593\over512}w+\cdots\,;\cr
A&=\textstyle{9\over8}+{2115\over256}w+\cdots\,;\cr
K&=\textstyle{15\over8}+{2661\over256}w+\cdots\,.\cr
\enddisplay
Now $S'(1)={27\over16}+{9\over8}+{15\over8}={75\over16}$, and
$\half S''(1)={1593\over512}+{2115\over256}+{2661\over256}={11145\over512}$.
The mean is $75^{\mathstrut}\over16$ and the variance is $105\over4$. (Is there
a simpler way?)
\source{1977 final exam.}

\ex:
Are the random variables $X_1$ and $X_2$ in \eq(|succ-p|) independent?
\answer First answer: "Clearly" yes, because the hash values $h_1$, \dots,~%
$h_n$ are independent. Second answer: Certainly no, even though the
hash values $h_1$, \dots,~$h_n$ are independent. We have
$\Pr\prp(X_j=0)=\sum_{k=1}^ns_k\bigl(\[j\ne k](m-1)/m\bigr)=(1-s_j)(m-1)/m$,
but $\Pr\prp(X_1=X_2=0)=\sum_{k=1}^ns_k\[k>2](m-1)^2\!/m^2=(1-s_1-s_2)(m-1)^2\!/m^2
\ne \Pr\prp(X_1=0)@\Pr\prp(X_2=0)$.

\ex:
Gina is a "golf"er who has probability $p=.05$ on each stroke of making
a ``supershot'' that gains a stroke over par, probability~$q=.91$ of making
an ordinary shot, and probability~$r=.04$ of making a ``subshot'' that costs
her a stroke with respect to par.
(Non-golfers: At
each turn she advances $2$, $1$, or $0$ steps toward her goal,
with probability $p$, $q$, or~$r$, respectively. 
On a par-$m$ hole, her score is the minimum $n$ such that she has advanced
$m$ or~more steps after taking $n$ turns. A low score is better than a high
score.) \g(Use a calculator for the numerical work on this problem.)\g
\itemitem{a} Show that Gina wins a par-$4$ hole more often than she loses,
when she plays against a player who shoots par. (In other words, the
probability that her score is less than~$4$ is greater than the probability
 that her score is greater than~$4$.)
\itemitem{b} Show that her average score on a par-$4$ hole is greater than~$4$.
(Therefore she tends to lose against a ``steady'' player on total points,
although she would tend to win in match play by holes.)
\answer Let $[z^n]\,S_m(z)$ be the probability that Gina has advanced
$<m$ steps after taking $n$~turns. Then $S_m(1)$ is her average score on
a par-$m$ hole; $[z^m]\,S_m(z)$ is the probability that she loses such a hole
against a steady player;
and $1-[z^{m-1}]\,S_m(z)$ is the probability that she wins it.
We have the recurrence
\begindisplay
S_0(z)&=0\,;\cr
S_m(z)&=\bigl(1+pzS_{m-2}(z)+qzS_{m-1}(z)\bigr)/(1-rz)\,,\qquad\hbox{for $m>0$}.
\enddisplay
To solve part (a), it suffices to compute the coefficients for $m,n\le4$;
it is convenient to replace $z$ by $100w$ so that the computations involve
nothing but integers. We obtain the following tableau of coefficients:
\begindisplay \def\preamble{&\hfill$##$\quad} \openup-1pt
S_0&0&0&0&0&0\cr
S_1&1&4&16&64&256\cr
S_2&1&95&744&4432&23552\cr
S_3&1&100&9065&104044&819808\cr
S_4&1&100&9975&868535&12964304\cr
\enddisplay
Therefore Gina wins with probability $1-.868535=.131465$; she loses
with probability $.12964304$. (b)~To find the mean number of strokes, we
compute
\begindisplay
\textstyle S_1(1)={25\over24}\,;\enspace
S_2(1)={4675\over2304}\,;\enspace
S_3(1)={667825\over221184}\,;\enspace
S_4(1)={85134475\over21233664}\,.
\enddisplay
(Incidentally, $S_5(1)\approx4.9995$; she wins with respect to both holes
and strokes on a par-$5$ hole, but loses either way when par is~$3$.)
\source{"Hardy" [|hardy-golf|] has an incorrect analysis leading to the opposite
conclusion.}

\subhead Exam problems

\ex:
A die has been loaded with the probability distribution
\begindisplay \openup3pt
\Pr(\dieone)=p_1\,;\qquad
\Pr(\dietwo)=p_2\,;\qquad\ldots\,;\qquad
\Pr(\diesix)=p_6\,.
\enddisplay
Let $S_n$ be the sum of the spots after this die has been rolled $n$~times.
Find a necessary and sufficient
condition on the ``loading distribution'' such that
the two random variables $\,S_n\bmod2\,$ and $\,S_n\bmod3\,$ are independent
of each other, for all~$n$.
\answer The condition will be true for all $n$ if and only if it is
true for~$n=1$, by the Chinese remainder theorem. One necessary and
sufficient condition is the polynomial identity
\begindisplay
&\bigl(p_2{+}p_4{+}p_6+(p_1{+}p_3{+}p_5)w\bigr)
\bigl(p_3{+}p_6+(p_1{+}p_4)z+(p_2{+}p_5)z^2\bigr)\cr
&\qquad=(p_1wz+p_2z^2+p_3w+p_4z+p_5wz^2+p_6)\,,
\enddisplay
but that just more-or-less restates the problem. A simpler characterization is
\begindisplay
(p_2+p_4+p_6)(p_3+p_6)=p_6\,,\qquad
(p_1+p_3+p_5)(p_2+p_5)=p_5\,,
\enddisplay
which checks only two of the coefficients in the former product. The general
solution has three degrees of freedom: Let $a_0+a_1=b_0+b_1+b_2=1$, and
put $p_1=a_1b_1$, \ $p_2=a_0b_2$, \ $p_3=a_1b_0$, \ $p_4=a_0b_1$, \ $p_5=a_1b_2$,
\ $p_6=a_0b_0$.
\source{1981 final exam.}

\ex:
The six faces of a certain die contain the spot patterns
\begindisplay
\dieone\qquad\diethree\qquad\diefour\qquad\diefive\qquad\diesix\qquad\dieeight
\enddisplay
instead of the usual \dieone\ through \diesix\thinspace.
\par\nobreak\smallskip
\itemitem{a}Show that there is a way to assign spots to the six faces of
another die so that, when these two dice are thrown, the sum of spots
has the same probability distribution as the sum of spots on two ordinary dice.
(Assume that all 36 face pairs are equally likely.)
\itemitem{b}Generalizing, find all ways to assign spots to the $6n$ faces of
$n$~dice so that the distribution of spot sums will be the same as the
distribution of spot sums on $n$~ordinary
dice. (Each face should receive a positive integer number of spots.)
\answer (a) \dieone\quad\dietwo\quad\dietwo\quad\diethree\quad\diethree\quad
\diefour\thinspace. (b)~If the $k$th die has faces with $s_1$, \dots,~$s_6$
spots, let $p_k(z)=z^{s_1}+\cdots+z^{s_6}$. We want to find such polynomials
with $p_1(z)\ldots p_n(z)=(z+z^2+z^3+z^4+z^5+z^6)^n$. The irreducible
factors of this polynomial with rational coefficients
are $z^n{(z+1)^n}\*{(z^2+z+1)^n}{(z^2-z+1)^n}$; hence $p_k(z)$ must
be of the form $z^{a_k}{(z+1)^{b_k}}\*{(z^2+z+1)^{c_k}}{(z^2-z+1)^{d_k}}$.
We must have $a_k\ge1$, since $p_k(0)=0$; and in fact $a_k=1$, since
$a_1+\cdots+a_n=n$. Furthermore the condition $p_k(1)=6$ implies that
$b_k=c_k=1$. It is now easy to see that $0\le d_k\le2$, since $d_k>2$ gives
negative coefficients. When $d=0$ and $d=2$,
we get the two dice in part~(a); therefore the only solutions have $k$~pairs
of dice as in~(a), plus $n-2k$~ordinary dice, for some $k\le\half n$.
\source{"Gardner" [|gardner-dice|] credits George "Sicherman".}

\ex:\exref|fib-heads|%
Let $p_n$ be the probability that exactly $n$ tosses of a fair coin
are needed before heads are seen twice in a row, and let $q_n=\sum_{k\ge n}p_k$.
Find closed forms for both $p_n$ and $q_n$ in terms of Fibonacci numbers.
\answer The number of coin-toss sequences of length~$n$ is $F_{n-1}$,
for all $n>0$, because of the relation between domino tilings
and coin flips. Therefore the probability that exactly $n$ tosses are needed
is $F_{n-1}/2^n$, when the coin is fair. Also $q_n=F_{n+1}/2^{n-1}$,
since $\sum_{k\ge n}F_kz^k=(F_nz^n+F_{n-1}z^{n+1})/(1-z-z^2)$.
(A~systematic solution via generating functions is, of course, also possible.)

\ex:
What is the probability generating function for the number of times you need
to roll a fair die until all six faces have turned up? Generalize to
$m$-sided fair dice: Give closed forms for the mean and variance of the
number of rolls needed to see $l$ of the $m$~faces. What is the probability
that this number will be exactly $n$?
\answer When $k$ faces have been seen, the task of rolling a new
one is equivalent to flipping coins with success probability
$p_k=(m-k)/m$. Hence the pgf is $\prod_{k=0}^{l-1}p_kz/(1-q_kz)=
\prod_{k=0}^{l-1}(m-k)z/(m-kz)$.
The mean is $\sum_{k=0}^{l-1}p_k^{-1}=
m(H_m-H_{m-l})$; the variance is $m^2\bigl(H_m^{(2)}-H_{m-l}^{(2)}\bigr)
-m(H_m-H_{m-l})$; and equation \equ(7.|stirl2-gf|) provides a closed form
for the requested probability, namely $m^{-n}m!{n-1\brace l-1}
/(m-l)!$. (The problem discussed in this exercise is
traditionally called ``"coupon collecting".\qback'')
\source{[|knuth2|, exercise 3.3.2--10].}

\ex:
A {\it "Dirichlet probability generating function"\/} has the form
\begindisplay
P(z)=\sum_{n\ge1}{p_n\over n^z}\,.
\enddisplay
Thus $P(0)=1$. If $X$ is a random variable with $\Pr\prp(X=n)=p_n$,
express $E(X)$, $V(X)$, and $E(\ln X)$ in terms of
$P(z)$ and its derivatives.
\answer $E(X)=P(-1)$; \ $V(X)=P(-2)-P(-1)^2$; \ $E(\ln X)=-P'(0)$.
\source{[|mariage-stables|, exercise 4.3(a)].}

\ex:
The $m$th cumulant $\kappa_m$ of the "binomial distribution" \eq(|binom-pgf|)
has the form $nf_m(p)$, where $f_m$ is a polynomial of degree~$m$.
(For example, $f_1(p)=p$ and $f_2(p)=p-p^2$, because the mean and
variance are $np$ and $npq$.)
\itemitem{a}Find a closed form for the coefficient of $p^k$ in $f_m(p)$.
\itemitem{b}Prove that $f_m(\half)=(2^m-1)B_m/m+\[m=1]$, where $B_m$
is the $m$th Bernoulli number.
\answer (a) We have $\kappa_m=n\bigl(0!{m\brace1}p-1!{m\brace2}p^2+2!{m\brace3}p^3
-\cdots\,\bigr)$, by \equ(7.|exp-power-gf|). Incidentally, the third cumulant is
$npq(q-p)$ and the fourth is $npq(1-6pq)$. The identity
$q+pe^t=(p+qe^{-t})e^t$ shows that $f_m(p)=(-1)^mf_m(q)+\[m=1]$;
hence we can write $f_m(p)=g_m(pq)(q-p)^{[m\,\,\rm odd]}$, where $g_m$
is a polynomial of degree $\lfloor m/2\rfloor$, whenever $m>1$.
(b)~Let $p=\half$ and
$F(t)=\ln(\half+\half e^t)$. Then $\sum_{m\ge1}\kappa_m t^{m-1}\!/(m-1)!
=F'(t)=1-1/(e^t+1)$, and we can use exercise 6.|z/ez+1|.

\ex:
Let the random variable $X_n$ be the number of flips of a fair coin until
heads have turned up a total of $n$~times.
 Show that $E(X_{n+1}^{-1})=(-1)^n(\ln2
+H_{\lfloor n/2\rfloor}-H_n)$. Use the methods of Chapter~9 to
"!average of a reciprocal"
estimate this value with an absolute error of $O(n^{-3})$.
\answer If $G(z)$ is the pgf for a random variable $X$ that assumes only
positive integer values, then $\int_0^1 G(z)\,dz/z=\sum_{k\ge1}\Pr\prp(X=k)/k
=E(X^{-1})$. If\/ $X$ is the distribution of the number of flips to obtain
$n+1$ heads, we have $G(z)=\bigl(pz/(1-qz)\bigr)^{\!n+1}$ by
\equ(8.|bernoulli-trials|), and the integral is
\begindisplay
\int_0^1\biggl({pz\over1-qz}\biggr)^{\!n+1}{dz\over z}=
\int_0^1{w^n\,dw\over 1+(q/p)w}
\enddisplay
if we substitute $w=pz/(1-qz)$. When $p=q$ the integrand can be written
$(-1)^n\bigl((1+w)^{-1}-1+w-w^2+\cdots+(-1)^n w^{n-1}\bigr)$, so the
integral is $(-1)^n\bigl(\ln2-1+\half-{1\over3}+\cdots+(-1)^n\!/n\bigr)$.
We have $H_{2n}-H_n=\ln2-{1\over4}n^{-1}+{1\over16}n^{-2}+O(n^{-4})$
by \equ(9.|o-harmonic|), and it follows that $E(X_{n+1}^{-1})=
\half n^{-1}-{1\over4}n^{-2}+O(n^{-4})$.
\source{"Feller" [|feller|, exercise IX.33].}

\ex:
A certain man has a problem finding work. If he is unemployed on any given
morning, there's constant probability $p_h$ (independent of past history)
that he will be hired before that evening; but if he's got a job when
\g Does \TeX\ choose optimal line breaks?\g
the day begins, there's constant probability $p_f$ that he'll be laid
off by nightfall. Find the average number of evenings on which he will have
a job lined up, assuming that he is initially employed and that this
process goes on for $n$~days. (For example, if $n=1$ the answer is $1-p_f$.)
\answer Let $F_n(z)$ and $G_n(z)$ be pgf's for the number
of employed evenings, if the man is initially unemployed or employed,
respectively. Let $q_h=1-p_h$ and $q_f=1-p_f$. Then $F_0(z)=G_0(z)=1$, and
\begindisplay
F_n(z)&=p_hzG_{n-1}(z)+q_hF_{n-1}(z)\,;\cr
G_n(z)&=p_f@F_{n-1}(z)+q_f@zG_{n-1}(z)\,.
\enddisplay
The solution is given by the super generating function
\begindisplay
G(w,z)=\sum_{n\ge0}
G_n(z)@w^n=A(w)/\bigl(1-zB(w)\bigr)\,,
\enddisplay
where $B(w)=w\bigl(q_f-(q_f-p_h)w\bigr)/
(1-q_hw)$ and $A(w)=\bigl(1-B(w)\bigr)/(1-w)$. Now
$\sum_{n\ge0}G'_n(1)@w^n=\alpha w/(1-w)^2+\beta/(1-w)
-\beta/\bigl(1-(q_f-p_h)w\bigr)$ where
\begindisplay
\alpha={p_h\over p_h+p_f}\,,\qquad
\beta={p_f(q_f-p_h)\over(p_h+p_f)^2}\,;
\enddisplay
hence $G_n'(1)=\alpha n+\beta\bigl(1-(q_f-p_h)^n\bigr)$. (Similarly
$G_n''(1)=\alpha^2 n^2+O(n)$, so the variance is $O(n)$.)

\ex:
Find a closed form for the pgf\/ $G_n(z)=\sum_{k\ge0}p_{k,n}z^k$, where
$p_{k,n}$ is the probability that a random permutation of $n$ objects
has exactly $k$~cycles. What are the mean and standard deviation
of the number of cycles?
\answer $G_n(z)=\sum_{k\ge0}{n\brack k}z^k\!/n!=z\_^n\!/n!$, by
\equ(6.|expand-rising-to-ord|). This is a product of binomial pgf's,
$\prod_{k=1}^n\bigl((k-1+z)/k\bigr)$, where the $k$th has mean $1/k$
and variance $(k-1)/k^2$; hence $\Mean(G_n)=H_n$ and $\Var(G_n)=
H_n-H_n^{(2)}$.
\source{[|knuth1|, sections 1.2.10 and 1.3.3].}

\ex:
The athletic department runs an intramural ``"knockout tournament"'' for
$2^n$ "tennis" players as follows. In the first round, the players are
"!sports, \string\see tennis, golf, football, frisbee, dice, baseball"
paired off randomly, with each pairing equally likely, and $2^{n-1}$
matches are played. The winners advance to the second round, where
the same process produces $2^{n-2}$ winners. And so on;
the $k$th round has $2^{n-k}$ randomly chosen matches between the
$2^{n-k+1}$ players who are still undefeated. The $n$th round
produces the champion. Unbeknownst to the "tournament" organizers,
there is actually an ordering among the players, so that $x_1$~is best,
$x_2$ is second best, \dots,~$x_{2^n}$~is worst. When $x_j$ plays $x_k$
and $j<k$, the winner is $x_j$ with probability~$p$ and $x_k$ with
\g A peculiar set of tennis players.\g
probability~$1-p$, independent of the other matches. We assume that
the same probability~$p$ applies to all $j$ and~$k$.
\itemitem{a}What's the probability that $x_1$ wins the tournament?
\itemitem{b}What's the probability that the $n$th round (the final
match) is between the top two players, $x_1$ and $x_2$?
\itemitem{c}What's the probability that the best $2^k$ players are
the competitors in the $k$th-to-last round? (The previous questions
were the cases $k=0$ and $k=1$.)
\itemitem{d}Let $N(n)$ be the number of essentially different
tournament results; two tournaments are essentially the same if the
matches take place between the same players and have the same winners.
Prove that $N(n)=2^n!$.
\itemitem{e}What's the probability that $x_2$ wins the tournament?
\itemitem{f}Prove that if $\half<p<1$, the probability that $x_j$
wins is strictly greater than the probability that $x_{j+1}$ wins,
for $1\le j<2^n$.
\answer (a) The champion must be undefeated in $n$ rounds, so the answer
is~$p^n$. (b,c)~Players $x_1$, \dots,~$x_{2^k}$ must be ``seeded'' (by
chance) in distinct subtournaments and they must win all $2^k(n-k)$
of their matches. The $2^n$ leaves of the tournament tree can be
filled in $2^n!$ ways; to seed it we have $2^k!(2^{n-k})^{2^k}$ ways
to place the top $2^k$ players, and $(2^n-2^k)!$ ways to place the
others. Hence the probability is $(2p)^{2^k(n-k)}\big/{2^n\choose2^k}$.
When $k=1$ this simplifies to $(2p^2)^{n-1}\!/(2^n-1)$.
(d)~Each tournament outcome corresponds to a permutation of the players:
Let $y_1$ be the champ; let $y_2$ be the other finalist; let
$y_3$ and $y_4$ be the players who lost to $y_1$ and $y_2$ in the
semifinals; let $(y_5,\ldots,y_8)$ be those who lost respectively
to $(y_1,\ldots,y_4)$ in the quarterfinals; etc. (Another proof shows
that the first round has $2^n!/2^{n-1}!$ essentially different outcomes;
the second round has $2^{n-1}!/2^{n-2}!$; and so on.) (e)~Let $S_k$
be the set of $2^{k-1}$ potential opponents of $x_2$ in the $k$th round.
The conditional probability that $x_2$ wins, given that $x_1$ belongs to~$S_k$, is
\begindisplay
\Pr(\hbox{$x_1$ plays $x_2$})\cdt p^{n-1}(1-p)\,&+\,
\Pr(\hbox{$x_1$ doesn't play $x_2$})\cdt p^n\cr
=p^{k-1}p^{n-1}(1-p)\,&+\,(1-p^{k-1})p^n\,.
\enddisplay
The chance that $x_1\in S_k$ is $2^{k-1}\!/(2^n-1)$; summing on $k$ gives
the answer:
\begindisplay\tightplus
\sum_{k=1}^n{2^{k-1}\over2^n-1}\bigl(p^{k-1}p^{n-1}(1{-}p)+(1{-}p^{k-1})p^n\bigr)
=p^n-{(2p)^n-1\over2^n-1}\,p^{n-1}\,.
\enddisplay
(f) Each of the $2^n!$ tournament outcomes has a certain probability of
occurring, and the probability that $x_j$ wins is the sum of these probabilities
over all ${(2^n-1)!}$ tournament outcomes in which $x_j$ is champion.
Consider interchanging $x_j$ with $x_{j+1}$ in all those outcomes;
this change doesn't affect the probability if $x_j$ and $x_{j+1}$ never meet,
but it multiplies the probability by $(1-p)/p<1$ if they do meet.
\source{1984 final exam.}

\ex:
True "sherry" is made in Spain according to a multistage system called
``Solera.\qback'' For simplicity we'll assume that the winemaker
has only three barrels, called $A$, $B$, and~$C$. Every year a third
\g\noindent\llap{``}A fast arithmetic computation shows that the sherry is always at
least three years old. Taking computation further gives the vertigo.''\par
\hfill\dash---Revue du vin de\par\hfill France (Nov 1984)\g
of the "wine" from barrel~$C$ is bottled and replaced by wine from~$B$;
then $B$ is topped off with a third of the wine from~$A$; finally
$A$~is topped off with new wine. Let $A(z)$, $B(z)$, $C(z)$ be
probability generating functions, where the coefficient of~$z^n$ is
the fraction of $n$-year-old wine in the corresponding barrel just after the
transfers have been made.
\itemitem{a}Assume that the operation has been going on since time
immemorial, so that we have a steady state in which
$A(z)$, $B(z)$, and $C(z)$ are the same at the beginning of each year.
Find closed forms for these generating functions.
\itemitem{b}Find the mean and standard deviation of the age of the wine in
each barrel, under the same assumptions. What is the average age of
the sherry when it is bottled? How much of it is exactly 25~years old?
\itemitem{c}Now take the finiteness of time
into account: Suppose that all three barrels contained new wine at the beginning
of year~$0$. What is the average age of the sherry that is bottled
at the beginning of year~$n$?
\answer (a) $A(z)=1/(3-2z)$; $B(z)=zA(z)^2$; $C(z)=z^2A(z)^3$. The pgf
for sherry when it's bottled is $z^3A(z)^3$, which is $z^3$ times
a negative binomial distribution with parameters $n=3$, $p={1\over3}$.
(b)~$\Mean(A)=2$, $\Var(A)=6$; $\Mean(B)=5$, $\Var(B)=2\Var(A)=12$;
$\Mean(C)=8$, $\Var(C)=18$. The sherry is nine years old, on the average.
The fraction that's 25~years old is ${-3\choose22}(-2)^{22}3^{-25}=
{24\choose22}2^{22}3^{-25}=23\cdt({2\over3})^{24}\approx.00137$.
(c)~Let the coefficient of $w^n$ be the pgf for the beginning of year~$n$.
Then
\begindisplay \openup3pt
A&=\textstyle\bigl(1+{1\over3}w/(1-w)\bigr)/(1-{2\over3}zw)\,;\cr
B&=\textstyle\bigl(1+{1\over3}zwA\bigr)/(1-{2\over3}zw)\,;\cr
C&=\textstyle\bigl(1+{1\over3}zwB\bigr)/(1-{2\over3}zw)\,.\cr
\enddisplay
Differentiate with respect to~$z$ and set $z=1$; this makes
\begindisplay
C'={8\over1-w}-{1/2\over(1-{2\over3}w)^3}
-{3/2\over(1-{2\over3}w)^2}
-{6\over1-{2\over3}w}\,.
\enddisplay
The average age of bottled sherry $n$ years after the process started is
$1$~greater than the coefficient of~$w^{n-1}$, namely
$9-({2\over3})^n(3n^2+21n+72)/8$. (This already exceeds~$8$ when $n=11$.)

\ex:
Stefan "Banach"
used to carry two boxes of matches, each containing $n$~matches
initially. Whenever he needed a light he chose a box at random,
each with probability~$\half$, independent of his previous choices.
After taking out a match he'd put the box back in its pocket (even
if the box became empty\dash---all famous mathematicians used to do this).
When his chosen box was empty he'd throw it away and reach for the
other box.
\itemitem{a} Once he found that the other box was empty too. What's the
probability that this occurs? (For $n=1$ it happens half the time
and for $n=2$ it happens $3/8$ of the time.) To answer this part,
find a closed form for the generating function $P(w,z)=\sum_{m,n}p_{m,n}
w^mz^n$, where $p_{m,n}$ is the probability that, starting with $m$~matches
in one box and $n$ in the other, both boxes are empty when an empty box
is first chosen. Then find a closed form for~$p_{n,n}$.
\itemitem{b} Generalizing your answer to part (a), find a closed form
for the probability that exactly~$k$ matches are in the other box when
an empty one is first thrown away.
\itemitem{c} Find a closed form for the average number of matches
\g And for the number in the empty box.\g
in that other box.
\answer (a) $P(w,z)=1+\half\bigl(wP(w,z)+zP(w,z)\bigr)=\bigl(1-\half(w+z)\bigr)
{}^{-1}$, hence $p_{mn}=2^{-m-n}{m+n\choose n}$. (b)~$P_k(w,z)=
\half(w^k+z^k)P(w,z)$; hence
\begindisplay
p_{k,m,n}=2^{k-1-m-n}\biggl({m+n-k\choose m}+
{m+n-k\choose n}\biggr)\,.
\enddisplay
 (c)~$\sum_k kp_{k,n,n}=\sum_{k=0}^n
k2^{k-2n}{2n-k\choose n}=\sum_{k=0}^n(n-k)2^{-n-k}{n+k\choose n}$;
this can be summed using \equ(5.|half/2|):
\begindisplay
&\sum_{k=0}^n2^{-n-k}\biggl((2n+1){n+k\choose n}-(n+1){n+1+k\choose n+1}\biggr)\cr
&\qquad=(2n+1)-(n+1)2^{-n}\biggl(2^{n+1}-2^{-n-1}{2n+2\choose n+1}\biggr)\cr
&\qquad={2n+1\over 2^{2n}}{2n\choose n}-1\,.
\enddisplay
(The methods of Chapter 9 show that this is $2\sqrt{n/\pi}-1+O(n^{-1/2})$.)
\source{"Feller" [|feller|] credits Hugo "Steinhaus".}

\ex:\exref|phages|%
Some physicians, collaborating with some physicists, recently discovered
a pair of microbes that reproduce in a peculiar way. The male microbe,
called a {\it"diphage"}, has two receptors on its surface; the female microbe,
"!phages"
called a {\it"triphage"}, has three:
\begindisplay
\unitlength=4pt
\setbox0=\hbox{\beginpicture(0,0)(0,0) % receptor
	\put(-.75,-.75){\line(1,0){1.5}}
	\put(-.75,-.75){\line(0,1){1.5}}
	\put(.75,.75){\line(-1,0){1.5}}
	\put(.75,.75){\line(0,-1){1.5}}
	\endpicture}
\hbox{diphage: \ }
\lower17pt\hbox{\beginpicture(12,12)(-3,-3)
	\put(0,0){\circle{6}}
	\put(7.121,7.121){\line(-1,-1){5}}
	\put(7.121,7.121){\line(-1,0){4}}
	\put(7.121,7.121){\line(0,-1){4}}
	\put(2.371,7.121){\copy0}
	\put(7.121,2.371){\copy0}
	\endpicture}
\qquad\hbox{triphage: \ }
\lower18pt\hbox{\beginpicture(10,12)(-3,-9)
	\put(0,0){\circle{6}}
	\put(0,-7.5){\line(0,1){4.5}}
	\put(-1.5,-5.25){\line(1,0){3}}
	\put(0,-8.25){\copy0}
	\put(-2.25,-5.25){\copy0}
	\put(2.25,-5.25){\copy0}
	\endpicture}
\qquad\hbox{receptor: \ }
\raise2pt\box0
\enddisplay
When a culture of diphages and triphages is irradiated with a psi-particle,
exactly one of the receptors on one of the phages absorbs the particle;
each receptor is equally likely. If it was a diphage receptor, that diphage
changes to a triphage; if it was a triphage receptor, that triphage splits
into two diphages. Thus if an experiment starts with one diphage, the
first psi-particle changes it to a triphage, the second particle splits
the triphage into two diphages, and the third particle changes one of the
diphages to a triphage. The fourth particle hits either the diphage
or the triphage; then there are either two triphages
(probability $2\over5$)
or three diphages (probability $3\over5$). Find a closed form for the
average number of diphages present, if we begin with a single diphage
and irradiate the culture $n$ times with single psi-particles.
\answer After $n$ irradiations there are $n+2$ equally likely receptors.
Let the random variable $X_n$ denote the number of diphages present;
then $X_{n+1}=X_n+Y_n$, where $Y_n=-1$ if the $(n+1)$st particle hits
a diphage receptor (conditional probability $2X_n/(n+2)$) and $Y_n=+2$ otherwise.
Hence
\begindisplay
EX_{n+1}=EX_n+EY_n=EX_n-2EX_n/(n{+}2)+2\bigl(1-2EX_n/(n{+}2)\bigr)\,.
\enddisplay
The recurrence $(n+2)EX_{n+1}=(n-4)EX_n+2n+4$ can be solved if we multiply
both sides by the summation factor $(n+1)\_5$; or we can guess the
answer and prove it by induction: $EX_n=(2n+4)/7$ for all $n>4$.
(Incidentally, there are always two diphages and one triphage after five
steps, regardless of the configuration after four.)
\source{1974 final, suggested by ``fringe analysis'' of 2-3 trees.}

\ex:\exref|frisbees|%
Five people stand at the vertices of a "pentagon", throwing "frisbees" to
\g Or, "!war"
 if this pentagon is in Arlington, throwing missiles at each other.\g
each other.
\begindisplay \advance\abovedisplayskip-8pt
\unitlength=1.5pt
\beginpicture(50,50)(-25,-30)
	\put(0,20){\disk{2}}
	\put(24,0){\disk{2}}
	\put(12,-24){\disk{2}}
	\put(-12,-24){\disk{2}}
	\put(-24,0){\disk{2}}
	\put(-12,-24){\circle{5}}
	\put(12,-24){\circle{5}}
	\put(0,20){\line(6,-5){24}}
	\put(24,0){\line(-1,-2){12}}
	\put(12,-24){\line(-1,0){24}}
	\put(-12,-24){\line(-1,2){12}}
	\put(-24,0){\line(6,5){24}}
	\put(-17,-23){\vector(-1,2){4}}
	\put(17,-23){\vector(1,2){4}}
	\put(-10,-28){\vector(1,0){8}}
	\put(10,-28){\vector(-1,0){8}}
\endpicture
\enddisplay
They have two frisbees, initially at adjacent vertices as shown. In each
time interval, each frisbee is thrown either to the left or to the right
(along an edge of the pentagon) with equal probability. This process continues
until one person is the target of two frisbees simultaneously; then the
\g Frisbee is a trademark of Wham-O Manufacturing Company.\g
game stops. (All throws are independent of past history.)
\itemitem{a} Find the mean and variance of the number of pairs of throws.
\itemitem{b} Find a closed form for the probability that the game
lasts more than 100 steps, in terms of Fibonacci numbers.
\answer (a) The distance between frisbees (measured so as to make it an
even number) is either $0$, $2$, or~$4$ units, initially~$4$.
The corresponding generating functions $A$, $B$, $C$ (where, say,
$[z^n]\,C$ is the probability of distance~$4$ after $n$ throws)
satisfy
\begindisplay
\textstyle A={1\over4}zB\,,\quad B=\half zB+{1\over4}zC\,,\quad
C=1+{1\over4}zB+{3\over4}zC\,.
\enddisplay
It follows that $A=z^2\!/(16-20z+5z^2)=z^2\!/F(z)$, and we have $\Mean(A)=2-
\Mean(F)=12$, $\Var(A)=-\Var(F)=100$. (A more difficult but more amusing
solution factors $A$ as follows:
\begindisplay
A={p_1z\over 1-q_1z}\cdot{p_2z\over 1-q_2z}=
{p_2\over p_2-p_1}{p_1z\over1-q_1z}\,+\,
{p_1\over p_1-p_2}{p_2z\over1-q_2z}\,,
\enddisplay
where $p_1=\phi^2\!/4=(3+\sqrt5\,)/8$,
$p_2=\phihat^2\!/4=(3-\sqrt5\,)/8$,
and $p_1+q_1=p_2+q_2=1$. Thus, the game is equivalent to having two
biased coins whose heads probabilities are $p_1$ and $p_2$; flip
the coins one at a time until they have both come up heads, and the
total number of flips will have the same distribution as the number of
frisbee throws. The mean
and variance of the waiting times for these two coins are
respectively $6\mp2\sqrt5$ and $50\mp22\sqrt5$, hence the total
mean and variance are $12$ and $100$ as before.)
\par(b) Expanding the generating function in partial fractions
makes it possible to sum the probabilities. (Note that $\sqrt 5/(4\phi)+
\phi^2\!/4=1$, so the answer can be stated in terms of powers of $\phi$.)
 The game will last more
than $n$~steps with probability $5^{(n-1)/2}4^{-n}(\phi^{n+2}-\phi^{-n-2})$;
when $n$ is even this is $5^{n/2}4^{-n}F_{n+2}$.
So the answer is $5^{50}4^{-100}F_{102}\approx.00006$.
\source{1979 final exam.}

\ex:\exref|snowwalker|%
Luke "Snowwalker" spends winter vacations at his mountain cabin. The front
porch has $m$~pairs of boots and the back porch has $n$~pairs. Every
time he goes for a walk he flips a (fair) coin to decide whether to
leave from the front porch or the back, and he puts on a pair of boots
at that porch and heads off. There's a 50/50 chance that he returns to
each porch, independent of his starting point, and he leaves the boots
at the porch he returns to. Thus after one walk there will be $m+
[-1,0,{\rm@or\,}{+1}]$ pairs on the front porch and $n-
[-1,0,{\rm@or\,}{+1}]$ pairs on the back porch. If all the boots pile up
on one porch and if he decides to leave from the other, he goes
without boots and gets frostbite, ending his vacation. Assuming
that he continues his walks until the bitter end, let $P_N(m,n)$
be the probability that he completes exactly $N$ nonfrostbitten
trips, starting with $m$~pairs on the front porch and $n$~on the back.
Thus, if both $m$ and~$n$ are positive,
\begindisplay \openup5pt \let\displaystyle=\textstyle
P_N(m,n)&={1\over4}P_{N-1}(m-1,n+1)
 +\half P_{N-1}(m,n)\cr
&\hskip11em +{1\over4}P_{N-1}(m+1,n-1)\,;\cr
\enddisplay
this follows because this first trip is either front/back, front/front,
back/\allowbreak back, or back/front, each with probability $1\over4$, and $N-1$
trips remain.
\itemitem{a}Complete the recurrence for $P_N(m,n)$ by finding
formulas that hold when $m=0$ or $n=0$. Use the recurrence to
obtain equations that hold among the probability generating functions
\begindisplay
g_{m,n}(z)=\sum_{N\ge0}P_N(m,n)z^N\,.
\enddisplay
\itemitem{b}Differentiate your equations and set $z=1$, thereby obtaining
relations among the quantities $g'_{m,n}(1)$. Solve these equations,
thereby determining the mean number of trips before frostbite.
\itemitem{c}Show that $g_{m,n}$ has a closed form if we substitute
$z=1/{\cos^2\theta}$:
\begindisplay
g_{m,n}\Bigl({1\over\cos^2\theta}\Bigr)={\sin(2m+1)\theta
+\sin(2n+1)\theta\over\sin(2m+2n+2)\theta}\,\cos\theta\,.
\enddisplay
\answer (a) If $n>0$, $P_N(0,n)=\half\[N=0]+{1\over4}P_{N-1}(0,n)
+{1\over4}P_{N-1}(1,n{-}1)$; $P_N(m,0)$ is similar; $P_N(0,0)=\[N=0]$.
Hence
\begindisplay \openup2pt
g_{m,n}&=\textstyle{1\over4}zg_{m-1,n+1}+\half zg_{m,n}+{1\over4}zg_{m+1,n-1}\,;\cr
g_{0,n}&=\textstyle\half+{1\over4}zg_{0,n}+{1\over4}g_{1,n-1}\,;\quad\rm etc.\cr
\enddisplay
(b) $g'_{m,n}=1+{1\over4}g'_{m-1,n+1}+\half g'_{m,n}+{1\over4}g'_{m+1,n-1}$;
$g'_{0,n}=\half+{1\over4}g'_{0,n}+{1\over4}g'_{1,n-1}$; etc. By induction
on~$m$, we have $g'_{m,n}=(2m+1)g'_{0,m+n}-2m^2$ for all $m,n\ge0$. And
since $g'_{m,0}=g'_{0,m}$, we must have $g'_{m,n}=m+n+2mn$. (c)~The recurrence
is satisfied when $mn>0$, because
\begindisplay \openup4pt
\sin(2m+1)\theta&={1\over\cos^2\theta}\biggl({\sin(2m-1)\theta\over4}\cr
&\hskip7em +{\sin(2m+1)\theta\over2}+{\sin(2m+3)\theta\over4}\biggr)\,;
\enddisplay
this is a consequence of the identity $\sin(x-y)+\sin(x+y)=2\sin x\cos y$.
So all that remains is to check the boundary conditions.
\source{"Blom" [|blom|]; 1984 final exam.}

\ex:
Consider the function
\begindisplay
H(z)=1+{1-z\over2z}\bigl(z-3+\sqrt{(1-z)(9-z)}\,\bigr)\,.
\enddisplay
The purpose of this problem is to prove that $H(z)=\sum_{k\ge0}h_kz^k$
is a probability generating function, and to obtain some basic facts
about it.
\itemitem{a}Let $(1-z)^{3/2}(9-z)^{1/2}=\sum_{k\ge0}c_kz^k$. Prove that
$c_0=3$, $c_1=-14/3$, $c_2=37/27$, and $c_{3+l}=3\sum_k{l\choose k}
{1/2@\choose3+k}\bigl({8\over9}\bigr){}^{k+3}$ for all $l\ge0$.
\Hint: Use the identity
\begindisplay
\textstyle(9-z)^{1/2}=3(1-z)^{1/2}\bigl(1+{8\over9}z/(1-z)\bigr){}^{1/2}
\enddisplay
and expand the last factor in powers of $z/(1-z)$.
\itemitem{b}Use part (a) and exercise 5.|bc-inequality| to show that
the coefficients of $H(z)$ are all positive.
\itemitem{c}Prove the amazing identity
\begindisplay
\sqrt{9-H(z)\over1-H(z)}=\sqrt{\,{\mathstrut9-z\over\mathstrut1-z}}+2\,.
\enddisplay
\itemitem{d}What are the mean and variance of $H$?
\answer (a) Using the hint, we get
\begindisplay
&3(1-z)^2\sum_k{1/2\choose k}\biggl({8\over9}z\biggr)^{\!k}(1-z)^{2-k}\cr
&\qquad=3(1-z)^2\sum_k{1/2\choose k}\biggl({8\over9}\biggr)^{\!k}
 \sum_j{k+j-3\choose j}z^{j+k}\,;
\enddisplay
now look at the coefficient of $z^{3+l}$.
(b)~$H(z)={2\over3}+{5\over27}z+\half\sum_{l\ge0}c_{3+l}z^{2+l}$.
(c)~Let $r=\sqrt{(1-z)(9-z)}$. One can show that $(z-3+r)(z-3-r)=4z$,
and hence that $\bigl(r/(1-z)+2\bigr){}^2=(13-5z+4r)/(1-z)=
\bigl(9-H(z)\bigr)/\bigl(1-H(z)\bigr)$. (d)~Evaluating the first
derivative at $z=1$ shows that $\Mean(H)=1$. The second derivative
diverges at $z=1$, so the variance is infinite.
\source{1986 final exam.}

\ex:
The state "lottery" in El Dorado uses the payoff distribution $H$
defined in the previous problem. Each lottery ticket costs $1$~doubloon,
and the payoff is $k$~doubloons with probability~$h_k$. Your chance
of winning with each ticket is completely independent of your chance with
other tickets; in other words, winning or losing with one ticket does not
affect your probability of winning with any other ticket you might
have purchased in the same lottery.
\itemitem{a}Suppose you start with one doubloon and play this game.
If you win $k$~doubloons, you buy $k$ tickets in the second game;
then you take the total winnings in the second game and apply all of
them to the third; and so on. If none of your tickets is a winner, you're
broke and you have to stop gambling. Prove that the pgf
of your current holdings after $n$ rounds of such play is
\begindisplay
1-{4\over\raise2ex\hbox{}{\sqrt{(9-z)/(1-z)}+2n-1}}
 +{4\over\raise2ex\hbox{}{\sqrt{(9-z)/(1-z)}+2n+1}}\,.
\enddisplay
\itemitem{b}Let $g_n$ be the probability that you lose all your money
for the first time on the $n$th game, and let $G(z)=g_1z+g_2z^2+\cdots\,$.
Prove that $G(1)=1$. (This means that you're bound to lose sooner or later,
with probability~$1$, although you might have fun playing in the meantime.)
What are the mean and the variance of~$G$?
\itemitem{c}What is the average total number of tickets you buy, if
you continue to play until going broke?
\itemitem{d}What is the average number of games until you lose everything
if you start with two doubloons instead of just one?
\g A doubledoubloon.\g
\answer (a) Let $H_n(z)$ be the pgf for your holdings after $n$ rounds of play,
with $H_0(z)=z$. The distribution for $n$ rounds is
\begindisplay
H_{n+1}(z)=H_n\bigl(H(z)\bigr)\,,
\enddisplay
so the result is true by induction (using the amazing identity of
the preceding problem). (b)~$g_n=H_n(0)-H_{n-1}(0)=4/n(n+1)(n+2)
=4(n-1)\_{-3}$. The mean is $2$, and the variance is infinite.
(c)~The expected number of tickets you buy on the $n$th round
is $\Mean(H_n)=1$, by exercise |pgf-composition|. So the total expected
number of tickets is infinite. (Thus, you almost surely
lose eventually, and you expect to
lose after the second game, yet you also expect to buy an infinite
number of tickets.) (d)~Now the pgf after $n$ games is $H_n(z)^2$,
and the method of part~(b) yields a mean of $16-{4\over3}\pi^2\approx 2.8$.
(The sum $\sum_{k\ge1}1/k^2=\pi^2\!/6$ shows up here.)
\source{1986 final exam.}

\subhead Bonus problems

\ex:
Show that the text's definitions of "median" and "mode" for random variables
correspond in some meaningful sense to the definitions of median and mode
for sequences, when the probability space is finite.
\answer If $\omega$ and $\omega'$ are events with
$\Pr(\omega)>\Pr(\omega')$, then a sequence of $n$~independent experiments
will encounter $\omega$ more often than $\omega'$, with high probability,
because $\omega$ will occur very nearly $n\Pr(\omega)$ times. Consequently,
as $n\to\infty$, the probability approaches~$1$ that the median or mode
of the values of~$X$ in a sequence of independent trials
 will be a median or mode of the random variable~$X$.

\ex:
Prove or disprove: If\/ $X$, $Y$, and $Z$ are random variables with
the property that all three pairs $(X,Y)$, $(X,Z)$ and $(Y,Z)$ are
independent, then $X+Y$ is independent of\/~$Z$.
\answer We can disprove the statement, even in the special case that
each variable is $0$ or~$1$.
Let $p_0=\Pr\prp(X=Y=Z=0)$, $p_1=\Pr\prp(X=Y=\overline Z=0)$,
\dots, $p_7=\Pr\prp(\overline X=\overline Y=\overline Z=0)$,
where $\overline X=1-X$. Then $p_0+p_1+\cdots+p_7=1$, and
the variables are independent in pairs if and only if we have
\begindisplay
(p_4+p_5+p_6+p_7)(p_2+p_3+p_6+p_7)=p_6+p_7\,,\cr
(p_4+p_5+p_6+p_7)(p_1+p_3+p_5+p_7)=p_5+p_7\,,\cr
(p_2+p_3+p_6+p_7)(p_1+p_3+p_5+p_7)=p_3+p_7\,.\cr
\enddisplay
But $\Pr\prp(X+Y=Z=0)\ne\Pr\prp(X+Y=0)\Pr\prp(Z=0)\iff
p_0\ne(p_0+p_1)(p_0+p_2+p_4+p_6)$. One solution is
\begindisplay
p_0=p_3=p_5=p_6=1/4\,;\qquad p_1=p_2=p_4=p_7=0\,.
\enddisplay
This is equivalent to flipping two fair coins and letting
$X=($the first coin is heads), $Y=($the second coin is heads),
$Z=($the coins differ). Another example, with all probabilities nonzero, is
\begindisplay
&p_0=4/64\,,\quad p_1=p_2=p_4=5/64\,,\cr
&p_3=p_5=p_6=10/64\,,\quad p_7=15/64\,.
\enddisplay
For this reason we say that $n$ variables $X_1$, \dots, $X_n$ are independent if
\begindisplay
\Pr\prp(X_1=x_1\ {\rm and\ \cdots\ and}\ X_n=x_n)=
\Pr\prp(X_1=x_1)\ldots\Pr\prp(X_n=x_n)\,;
\enddisplay
pairwise independence isn't enough to guarantee this.
\source{"Feller" [|feller|] credits S.\thinspace N. "Bernstein".}

\ex:\exref|var-hatv|%
Equation \eq(|hat-v|) proves that the average value of $\widehat VX$
is $VX$. What is the {\it variance\/} of $\widehat VX$?
\answer (See exercise |estimate-kappa3| for notation.) We have
\begindisplay \openup3pt
E(\Sigma_2^2)&=n\mu_4+n(n{-}1)\mu_2^2\,;\cr
E(\Sigma_2\Sigma_1^2)
 &=n\mu_4+2n(n{-}1)\mu_3\mu_1+n(n{-}1)\mu_2^2+n(n{-}1)(n{-}2)\mu_2\mu_1^2\,;\cr
E(\Sigma_1^4)&=n\mu_4+4n(n{-}1)\mu_3\mu_1+3n(n{-}1)\mu_2^2\cr
&\hskip5em+6n(n{-}1)(n{-}2)\mu_2\mu_1^2+n(n{-}1)(n{-}2)(n{-}3)\mu_1^4\,;\cr
\enddisplay
it follows that $V(\widehat VX)=\kappa_4/n+2\kappa_2^2/(n-1)$.

\ex:
A normal deck of playing cards contains 52 cards, four each with
face values in the set $\tt\{A,2,3,4,5,6,7,8,9,10,J,Q,K\}$. Let $X$ and~$Y$
denote the respective face values of the top and bottom cards, and consider
the following algorithm for shuffling:
\itemitem{S1}Permute the deck randomly so that each arrangement occurs with
probability $1/52!$.
\itemitem{S2}If $X\ne Y$, flip a biased coin that comes up heads with
probability~$p$, and go back to step~S1 if heads turns up. Otherwise stop.
\item{}Each coin flip and each permutation is assumed to be independent of
all the other randomizations.
What value of~$p$ will make $X$ and $Y$ independent random variables
after this procedure stops?
\answer There are $A={1\over17}\cdt52!$ permutations with $X=Y$, and
$B={16\over17}\cdt52!$ permutations with $X\ne Y$. After the stated
procedure, each permutation with $X=Y$ occurs with probability
${1\over17}\big/\bigl((1-{16\over17}p)A\bigr)$, because we return
to step~S1 with probability ${16\over17}p$. Similarly,
each permutation with $X\ne Y$ occurs with probability
${16\over17}(1-p)\big/\bigl((1-{16\over17}p)B\bigr)$. Choosing $p={1\over4}$
makes $\Pr\prp(X=x\,{\rm and}\,Y=y)=
{1\over169}$ for all $x$ and~$y$. (We could therefore make two flips of
a fair coin and go back to~S1 if both come up heads.)

\ex:
Generalize the "frisbee" problem of exercise |frisbees| from a pentagon
to an \hbox{$m$-gon}. What are the mean and variance of the number of
collision-free throws in general, when the frisbees are initially
at adjacent vertices? Show that, if $m$ is odd, the
pgf for the number of throws can be written as a product of
coin-flipping distributions:
\begindisplay \openup3pt
G_m(z)&=\prod_{k=1}^{(m-1)/2}{p_kz\over 1-q_kz}\,,\cr
{\rm where\ \ }p_k&=\sin^2{(2k-1)\pi\over2m}\,,\quad
 q_k=\cos^2{(2k-1)\pi\over2m}\,.
\enddisplay
\Hint: Try the substitution $z=1/{\cos^2\theta}$.
\answer If $m$ is even, the frisbees always stay an odd distance apart
and the game lasts forever. If $m=2l+1$, the relevant generating
functions are
\begindisplay \let\displaystyle=\textstyle
G_m&={1\over4}z A_1\,;\cr
A_1&=\half zA_1+{1\over4}zA_2\,,\cr
A_k&={1\over4}zA_{k-1}+\half zA_k+{1\over4}zA_{k+1}\,,
 \qquad\hbox{for $1<k<l$},\cr
A_l&={1\over4}zA_{l-1}+{3\over4}zA_l+1\,.
\enddisplay
(The coefficient $[z^n]\,A_k$ is the probability that the distance between
frisbees is $2k$ after $n$ throws.) Taking a clue from the similar
equations in exercise~|snowwalker|, we set $z=1/{\cos^2\theta}$
and $A_1=X\sin2\theta$, where $X$ is to be determined.
It follows by induction (not using the equation for $A_l$)
that $A_k=X\sin2k\theta$. Therefore we want to choose $X$ such that
\begindisplay
\Bigl(1-{3\over4\cos^2\theta}\Bigr)\,X\,\sin2l\theta=
 1+{1\over4\cos^2\theta}\,X\,\sin(2l-2)\theta\,.
\enddisplay
It turns out that $X=2\cos^2\theta/\sin\theta\cos(2l+1)\theta$,
hence
\begindisplay
G_m={\cos\theta\over\cos m\theta}\,.
\enddisplay
The denominator vanishes when $\theta$ is an odd multiple of $\pi/(2m)$;
thus $1-q_kz$ is a root of the denominator for $1\le k\le l$, and the
stated product representation must hold. \g Trigonometry wins again.
Is there a connection with pitching pennies along the angles
of the $m$-gon?\g
To find the mean and variance we can write
\begindisplay \let\displaystyle=\textstyle \openup3pt
G_m&=(1-\half\theta^2+{1\over24}\theta^4-\cdots\,)/
  (1-\half m^2\theta^2+{1\over24}m^4\theta^4-\cdots\,)\cr
&=1+\half(m^2-1)\theta^2+{1\over24}(5m^4-6m^2+1)\theta^4+\cdots\cr
&=1+\half(m^2-1)(\tan\theta)^2+{1\over24}(5m^4-14m^2+9)(\tan\theta)^4+\cdots\cr
&=1+G_m'(1)(\tan\theta)^2+\half G_m''(1)(\tan\theta)^4+\cdots\,,\cr
\enddisplay
because $\tan^2\theta=z-1$ and $\tan\theta=\theta+{1\over3}\theta^3+\cdots\,$.
So we have $\Mean(G_m)=\half(m^2-1)$ and $\Var(G_m)={1\over6}m^2(m^2-1)$.
(Note that this implies the identities
\begindisplay\openup5pt
{m^2-1\over2}&=\sum_{k=1}^{(m-1)/2}{1\over p_k}=\sum_{k=1}^{(m-1)/2}
 \Bigl(1\Big/\sin{(2k-1)\pi\over2m}\Bigr)^{\!2}\,;\cr
{m^2(m^2-1)\over6}%&=\sum_{k=1}^{(m-1)/2}{q_k\over p_k^2}\cr
&=\sum_{k=1}^{(m-1)/2}
 \Bigl(\cot{(2k-1)\pi\over2m}\Big/\sin{(2k-1)\pi\over2m}\Bigr)^{\!2}\,.\cr
\enddisplay
The third cumulant of this distribution is ${1\over30}m^2(m^2-1)(4m^2-1)$;
but the pattern of nice cumulant factorizations stops there.
There's a much simpler way to derive the mean: We have $G_m+A_1+\cdots+A_l
=z(A_1+\cdots+A_l)+1$, hence when $z=1$ we have $G_m'=A_1+\cdots+A_l$.
Since $G_m=1$ when $z=1$, an easy induction shows that $A_k=4k$.)

\ex:\exref|second-player-win|%
Prove that the "Penney-ante"
pattern $\tau_1\tau_2\ldots \tau_{l-1}\tau_l$ is always inferior to
the pattern ${\overline\tau}_2\tau_1\tau_2\ldots \tau_{l-1}$ when a fair coin is
flipped, if $l\ge3$.
\answer We have $A{:}A\ge2^{l-1}$ and $B{:}B<2^{l-1}+2^{l-3}$ and
$B{:}A\ge2^{l-2}$, hence $B{:}B-B{:}A\ge A{:}A-A{:}B$ is possible only
if $A{:}B>2^{l-3}$. This means that ${\overline\tau}_2=\tau_3$,
$\tau_1=\tau_4$, $\tau_2=\tau_5$, \dots, $\tau_{l-3}=\tau_l$. But then
$A{:}A\approx2^{l-1}+2^{l-4}+\cdots\,$,
$A{:}B\approx2^{l-3}+2^{l-6}+\cdots\,$,
$B{:}A\approx2^{l-2}+2^{l-5}+\cdots\,$, and
$B{:}B\approx2^{l-1}+2^{l-4}+\cdots\,$;
hence $B{:}B-B{:}A$ is less than $A{:}A-A{:}B$ after all. (Sharper
results have been obtained by "Guibas" and "Odlyzko"~[|guibas-odlyzko|],
who show that Bill's chances are always maximized with one of the
two patterns ${\tt H}@ \tau_1\ldots \tau_{l-1}$ or
${\tt T}@ \tau_1\ldots \tau_{l-1}$. Bill's winning strategy is,
in fact, unique; see the following exercise.)
\source{Lyle "Ramshaw".*}

\ex:
Is there any sequence $A=\tau_1\tau_2\ldots\tau_{l-1}\tau_l$
of $l\ge3$ heads and tails such that the sequences
${\tt H}@\tau_1\tau_2\ldots\tau_{l-1}$ and
${\tt T}@\tau_1\tau_2\ldots\tau_{l-1}$ both perform equally well
against~$A$ in the game of "Penney ante"?
\answer (Solution by J. "Csirik".) If $A$ is ${\tt H}^l$ or ${\tt T}^l$,
one of the two sequences matches~$A$ and cannot be used. Otherwise let
$\hat A=\tau_1\ldots\tau_{l-1}$, $H={\tt H}\hat A$, and $T={\tt T}\hat A$.
It is not difficult to verify that $H{:}A=T{:}A=\hat A{:}\hat A$, $H{:}H
+T{:}T=2^{l-1}+2(\hat A{:}\hat A)+1$, and $A{:}H+A{:}T=1+2(A{:}A)-2^l$.
Therefore the equation
\begindisplay
{H{:}H-H{:}A\over A{:}A-A{:}H}=
{T{:}T-T{:}A\over A{:}A-A{:}T}
\enddisplay
implies that both fractions equal
\begindisplay
{H{:}H-H{:}A+T{:}T-T{:}A\over A{:}A-A{:}H+A{:}A-A{:}T}
={2^{l-1}+1\over 2^l-1}\,.
\enddisplay
Then we can rearrange the original fractions to show that
\begindisplay
{H{:}H-H{:}A\over T{:}T-T{:}A}={A{:}A-A{:}H\over A{:}A-A{:}T}={p\over q}\,,
\enddisplay
where $p\rp q$. And $(p+1)\divides\gcd(2^{l-1}+1,2^l-1)=\gcd(3,2^l-1)$; so we
may assume that $l$ is even and that $p=1$, $q=2$. It follows that
$A{:}A-A{:}H=({2^l-1})/3$ and $A{:}A-A{:}T=(2^{l+1}-2)/3$, hence
$A{:}H-A{:}T=({2^l-1})/3\ge2^{l-2}$. We have $A{:}H\ge2^{l-2}$ if and only if
$A=({\tt TH})^{l/2}$. But then $H{:}H-H{:}A=A{:}A-A{:}H$, so $2^{l-1}+1
=2^l-1$ and $l=2$.\par
(Csirik [|csirik|] goes on to show that, when $l\ge4$, Alice can do no
better than to play ${\tt HT}^{l-3}{\tt H}^2$. But even with this strategy,
Bill wins with probability nearly $2\over3$.)
\source{"Guibas" and "Odlyzko" [|guibas-odlyzko|].}

\ex:
Are there patterns $A$ and $B$ of heads and tails such that
$A$ is longer than~$B$, yet $A$ appears before~$B$ more than half the time
when a fair coin is being flipped?
\answer According to \equ(8.|alice-odds|), we want $B{:}B-B{:}A>A{:}A-A{:}B$.
One solution is $A=\tt TTHH$, $B=\tt HHH$.

\ex:
Let $k$ and $n$ be fixed positive integers with $k<n$.
\itemitem{a}Find a closed form for the probability generating function
\begindisplay
G(w,z)={1\over m^n}\sum_{h_1=1}^m\cdots\sum_{h_n=1}^m
 w^{P(h_1,\ldots,h_n;k)} z^{P(h_1,\ldots,h_n;n)}
\enddisplay
for the joint distribution of the numbers of probes needed to find the
$k$th and $n$th items that have been inserted into a hash table with
$m$~lists.
\itemitem{b}Although the random variables $P(h_1,\ldots,h_n;k)$ and
$P(h_1,\ldots,h_n;n)$ are dependent, show that they are somewhat independent:
\begindisplay
&E\bigl(P(h_1,\ldots,h_n;k)P(h_1,\ldots,h_n;n)\bigr)\cr
&\qquad=
\bigl(EP(h_1,\ldots,h_n;k)\bigr)\bigl(EP(h_1,\ldots,h_n;n)\bigr)\,.
\enddisplay
\answer (a) Two cases arise depending on whether $h_k\ne h_n$ or $h_k=h_n$:
\begindisplay \openup4pt
G(w,z)&={m-1\over m}\Bigl({m-2+w+z\over m}\Bigr)^{\!k-1}w\,
 \Bigl({m-1+z\over m}\Bigr)^{\!n-k-1}z\cr
&\qquad+{1\over m}\Bigl({m-1+wz\over m}\Bigr)^{\!k-1}wz\,
 \Bigl({m-1+z\over m}\Bigr)^{\!n-k-1}z\,.
\enddisplay
(b) We can either argue algebraically, taking partial derivatives of $G(w,z)$
with respect to $w$ and~$z$ and setting $w=z=1$; or we can argue
combinatorially: Whatever the values of $h_1$, \dots,~$h_{n-1}$, the
expected value of $P(h_1,\ldots,h_{n-1},h_n;n)$ is the same (averaged
over $h_n$), because the hash sequence $(h_1,\ldots,h_{n-1})$ determines
a sequence of list sizes $(n_1,n_2,\ldots,n_m)$ such that the stated
expected value is $\bigl((n_1{+}1)+(n_2{+}1)+\cdots+(n_m{+}1)\bigr)/m=
(n-1+m)/m$. Therefore the random variable $EP(h_1,\ldots,h_n;n)$ is
independent of $(h_1,\ldots,h_{n-1})$, hence independent of
$P(h_1,\ldots,h_n;k)$.

\ex:\exref|prove-gen-va|%
Use the result of the previous exercise to prove \eq(|gen-va|).
\answer If $1\le k<l\le n$, the previous exercise shows that the coefficient
of $s_ks_l$ in the variance of the average is zero. Therefore we need
only consider the coefficient of $s_k^2$, which is
\begindisplay
\sum_{1\le h_1,\ldots,h_n\le m}\hskip-.7em{P(h_1,\ldots,h_n;k)^2\over m^n}
 \;-\;\biggl(@\sum_{1\le h_1,\ldots,h_n\le m}\hskip-.7em
  {P(h_1,\ldots,h_n;k)\over m^n}\biggr)^{\!2}\,,
\enddisplay
the variance of $\bigl((m-1+z)/m\bigr){}^{k-1}z$; and this is $(k-1)(m-1)/m^2$
as in exercise |more-variances|.

\ex:
Continuing exercise |phages|, find the {\it variance\/}
of the number of "diphages" after $n$ irradiations.
\answer The pgf\/ $D_n(z)$ satisfies the recurrence
\begindisplay
D_0(z)&=z\,;\cr
D_n(z)&=z^2D_{n-1}(z)\,+\,2(1-z^3)D_{n-1}'(z)/(n+1)\,,\qquad\hbox{for $n>0$}.
\enddisplay
We can now derive the recurrence
\begindisplay
D''_n(1)=(n-11)D''_{n-1}(1)/(n+1)+(8n-2)/7\,,
\enddisplay
which has the solution ${2\over637}(n+2)(26n+15)$ for all
$n\ge11$ (regardless of initial conditions). Hence the variance
comes to ${108\over637}(n+2)$ for $n\ge11$.

\subhead Research problem

\ex:
The {\it"normal distribution"\/} is a non-discrete probability
distribution characterized by having all its cumulants zero
except the mean and the variance. Is there an easy way to
tell if a given sequence of cumulants
$\<\kappa_1,\kappa_2,\kappa_3,\ldots\,\>$ comes from a
{\it"discrete"\/} distribution? (All the probabilities must be ``atomic''
in a discrete distribution.)
\answer (Another question asks if a given sequence of purported
cumulants comes from any
distribution whatever; for example, $\kappa_2$ must be nonnegative,
and $\kappa_4+3\kappa_2^2=E\bigl((X-\mu)^4\bigr)$ must be
at least $\bigl(E\bigl((X-\mu)^2\bigr)\bigr){}^2=\kappa_2^2$, etc.
A necessary and sufficient condition for this other problem was found by
"Hamburger" [|akhiezer|],\thinspace[|hamburger|].)


