% Preface to Concrete Mathematics
% (c) Addison-Wesley, all rights reserved.
\input gkpmac
\refin bib
\refin chap1
\refin chap2
\refin chap3
\refin chap4
\refin chap5
\refin chap6
\refin chap7
\refin chap8
\refin chap9

\pageno=-5

\beginchapter {} Preface

THIS BOOK IS BASED on a course of the same name
\g\noindent\llap{``}Audience, level, and treatment\dash---a description of such matters
is what prefaces are supposed to be about.''\par
\hfill\kern-10pt\dash---P.\thinspace R.\thinspace "Halmos"
 [|halmos-auto|]\g % p232
that has been taught annually
at "Stanford University" since 1970. About fifty students have taken it each
year\dash---juniors and seniors, but mostly graduate students\dash---and
alumni of these classes have begun to spawn similar courses elsewhere.
Thus the time seems ripe to present the
material to a wider audience (including sophomores).

It was a dark and stormy decade when Concrete Mathematics was born.
% Line breaks in the next paragraph are for Amy! -- OP
% (here's what he started with and revised:)
%People everywhere were questioning
%everything about university education. John "Hammersley" wrote an influential
%article ``On the enfeeblement of mathematical skills by `Modern Mathematics'
%and by similar soft intellectual trash in schools and
%universities''~[|hammersley|]; other mathematicians were asking
%``Can mathematics be saved?''~[|spohn|]. One of the present authors (DEK)
%had just published the first volume of a planned series of books on
%{\sl The Art of Computer Programming\/}~[|knuth1|], and he had found
%that the mathematics required for a comprehensive understanding of
%computer programs was quite different from what he'd learned as a
%mathematics major in college. Therefore he decided to introduce a new
Long-held values were constantly being questioned during those turbulent
years; college campuses were hotbeds of controversy. The college curriculum
itself was challenged, and mathematics did not escape scrutiny. John Ham%
mersley had just written a thought-provoking article ``On the enfeeblement of
mathematical skills by `Modern Mathematics' and by similar soft intellectual
trash in schools and universities''\thinspace[|hammersley|]; %
other worried mathematicians [|spohn|]
even asked, ``Can mathematics be saved?''
\g\noindent\llap{``}People do acquire a little brief authority by equipping themselves with
jargon: they can pontificate and air a superficial expertise. But what we should
%ask of the educated mathematician is not what he can speechify about, nor
%even what he knows about the existing corpus of mathematical knowledge,
%but rather what can he now do with his learning and whether he can actually solve
% [changes approved by Hammersley, May 88]
ask of educated mathematicians is not what they can speechify about, nor
even what they know about the existing corpus of mathematical knowledge,
but rather what can they now do with their learning and whether
they can actually solve
mathematical problems arising in practice. In short, we look for deeds not words.''
\par\hfill\kern-15pt\dash---J.\thinspace Hammersley~[|hammersley|]\g
 One of the present authors had
embarked on a series of books called {\sl The Art of Computer
 \null Programming}, and
in writing the first volume he (DEK) had found that there were
 \null mathematical
tools missing from his repertoire; the mathematics he needed for a thorough,
well-grounded understanding of computer programs was quite different from
what he'd learned as a mathematics major in college. So he introduced a new
%"!Knuth, Don" "!philosophy"
course, teaching what he wished somebody had taught him.

\looseness=-1
The course title ``Concrete Mathematics'' was originally intended
as an antidote to ``Abstract Mathematics,\qback'' since concrete
classical results were rap\-idly being swept out of the modern mathematical
curriculum by a new wave of abstract ideas popularly called the
``New Math.\qback'' Abstract mathematics is a wonderful subject, and
there's nothing wrong with it: It's beautiful, general, and useful.
But its adherents had become deluded that the rest of mathematics was
inferior and no longer worthy of attention. The goal of generalization
had become so fashionable that a generation of mathematicians had
become unable to relish beauty in the particular, to enjoy the challenge
of solving quantitative problems, or to appreciate the value of technique.
Abstract mathematics was becoming inbred and losing touch with
reality; mathematical education needed a concrete counterweight
in order to restore a healthy balance.

When DEK taught Concrete Mathematics at Stanford for the first time,
he explained the somewhat strange title by saying that it was his attempt
"!Knuth, Don"
to teach a math course that was hard instead of soft. He announced that,
contrary to the expectations of some of his colleagues, he was
{\it not\/} going to teach the Theory of Aggregates, nor "Stone"'s "!Cech"
Embedding Theorem, nor even the Stone--\v Cech compactification.
\g\noindent\llap{``}The heart of mathematics consists of concrete examples and
concrete problems.''\par\hfill\kern-10pt\dash---P.\thinspace R.\thinspace
 "Halmos" [|halmos-write|]\g % p129
"!Portland cement, \string\see Concrete (in another book)"
(Several students from the civil engineering department got up
and quietly left the room.)

Although Concrete Mathematics began as a reaction against other trends,
the main reasons for its existence were positive instead of negative.
And as the course continued its popular place in the curriculum,
its subject matter ``solidified'' and proved to be valuable in a variety
of new applications. Meanwhile, independent confirmation for the
appropriateness of the name came from another direction, when
\g\noindent\llap{``}It is downright sinful to teach the abstract before the concrete.''\par
\hfill\kern-10pt\dash---Z.\thinspace A.\thinspace Melzak [|melzak|]\g
Z.\thinspace A. "Melzak" published two volumes entitled
{\sl Companion to Concrete Mathematics\/}~[|melzak|].

The material of concrete mathematics may seem at first to be a
disparate bag of tricks, but practice makes it into a disciplined
set of tools. Indeed, the techniques have an underlying unity
and a strong appeal for many people. When another one of the authors
(RLG) first taught the course in 1979, the students had such fun that
"!Graham, Ron"
they decided to hold a class reunion a year later.

But what exactly is Concrete Mathematics? It is a blend
"!Concrete math, defined"
\g Concrete Mathematics is a bridge to~abstract mathematics.\g
of {\sc con}tinuous and dis{\sc crete} mathematics. More
concretely, it is the controlled manipulation of mathematical
formulas, using a collection of techniques for solving problems.
Once you, the reader, have learned the material in this book,
all you will need is a cool head, a large sheet of paper, and
fairly decent handwriting in order to evaluate horrendous-looking
sums, to solve complex recurrence relations, and to discover
subtle patterns in data. You will be so fluent in algebraic techniques
that you will often find it easier to obtain exact results than to
settle for approximate answers that are valid only in a limiting sense.

The major topics treated in this book include
\g\noindent\llap{``}The advanced reader who skips parts that appear too elementary
may miss more than the less advanced reader who skips parts
that appear too complex.''\par\hfill\dash---G. "P\'olya" [|polya|]\g
 sums, recurrences,
elementary number theory, binomial coefficients,
generating functions, discrete probability, and asymptotic methods.
The emphasis is on manipulative technique rather than on existence
theorems or combinatorial reasoning;
the goal is for each reader to become as familiar with discrete operations
(like the greatest-integer function and finite summation) as a
student of calculus is familiar with continuous operations
(like the absolute-value function and infinite integration).

Notice that this list of topics is quite different from what is
usually taught nowadays in undergraduate courses entitled ``Discrete
Mathematics.\qback'' Therefore the subject needs a distinctive name,
and ``Concrete Mathematics'' has proved to be as suitable as any other.
\g (We're not bold enough to try Distinuous Mathematics.)\g

The original textbook for Stanford's course on concrete mathematics
was the ``Mathematical Preliminaries'' section in {\sl
The Art of Computer Programming\/}~[|knuth1|].
But the presentation in those 110 pages is quite terse, so another
author~(OP) was inspired to draft a lengthy set of supplementary
"!Patashnik, Oren"
notes. The present book is an outgrowth of those notes; it is
an expansion of,
and a more leisurely introduction to, the material of Mathematical
Preliminaries. Some of the more advanced
parts have been omitted; on the other hand, several
topics not found there have been included here so that the story
will be complete.

The authors have enjoyed putting this book together because the subject
began to jell and to take on a life of its own before our eyes; this
\g\noindent\llap{``}\dots\ a concrete life~preserver thrown to students
sinking in a sea of abstraction.''\par\hfill\dash---W. "Gottschalk"\g
book almost seemed to write itself. Moreover, the somewhat unconventional
approaches we have adopted in several places have seemed to fit together
so well, after these years of experience,
that we can't help feeling that this book is a kind of manifesto
about our favorite way to do mathematics. "!philosophy"
So we think the book has turned
out to be a tale of mathematical beauty and surprise, and we hope that
our readers will share at least $\epsilon$ of the pleasure we had
while writing it.

Since this book was born in a university setting, we have tried to
capture the spirit of a contemporary classroom by
adopting an informal style. Some people think that mathematics
is a serious business that must always be cold and dry; but we think
mathematics is fun, and we aren't ashamed to admit the fact.
Why should a strict boundary line be drawn between work and play?
Concrete mathematics is full of appealing patterns; the manipulations
are not always easy, but the answers can be astonishingly
attractive. The joys and sorrows of mathematical work are reflected
explicitly in this book because they are part of our lives.

Students always know better than their teachers, so we have asked
the first students of this material to contribute their
\g Math graffiti:\par\smallskip
"Kilroy" wasn't "Haar".\par
Free the group.\par
Nuke the kernel.\par
Power to the~$n$.\par
$N{=}1\Rightarrow P{=}NP$.\g
frank opinions, as ``graffiti'' in the margins. Some of these marginal
markings are merely corny, some are profound; some of them warn
about ambiguities or obscurities, others are typical comments made
by wise guys in the back row; some are positive, some are negative,
some are zero. But they all are real indications of feelings that
should make the text material easier to assimilate. (The inspiration
for such marginal notes comes from a student handbook entitled
{\sl Approaching Stanford}, where the official university line is
counterbalanced by the remarks of outgoing students. For example,
\g I have only a marginal interest in this subject.\g
"Stanford" says, ``There are a few things you cannot miss in this amorphous
shape which is Stanford''; the margin says, ``Amorphous \dots\ what
the h{\tt***} does that mean? Typical of the pseudo-intellectualism
around here.'' \ Stanford: ``There is no end to the potential
of a group of students living together.'' \ Graffito: ``Stanford dorms
are like zoos without a keeper.'')

The margins also include
\g This was the most enjoyable course I've ever had. But it might be nice
to summarize the material as you go~along.\g
direct quotations from famous mathematicians of past generations,
giving the actual words in which they announced some of their
fundamental discoveries. Somehow it seems appropriate to
mix the words of "Leibniz", "Euler", "Gauss", and others with those of
the people who will be continuing the work. Mathematics is an
ongoing endeavor for people everywhere; many strands are being woven
into one rich fabric.

This book contains more than 500 "exercises", divided into six
\g I see:\par Concrete mathematics means drilling.\g
"!exercises, levels of"
categories:

\def\\#1{\smallskip\item{$\bullet$}{\subsubtitle#1\enspace}\ignorespaces}
\\{\kern-.05em Warmups} are exercises that {\sc every reader} should try to
	do when first reading the material.
\\{Basics} are exercises to develop facts that are best learned by trying
\g The homework was tough but I learned a lot. It was worth every hour.\g
	one's own derivation rather than by reading somebody else's.
\\{Homework exercises} are problems intended to deepen an understanding
	of material in the current chapter.
\\{Exam problems} typically involve ideas from two or more chapters
	simultaneously; they are generally intended for use in take-home
\g Take-home exams are vital\dash---keep them.\g
	exams (not for in-class exams under time pressure).
\\{Bonus problems} go beyond what an average student of concrete
	mathematics is expected to handle while taking a course based
\g Exams were harder than the homework led me to expect.\g
	on this book; they extend the text in interesting ways.
\\{Research problems} may or may not be humanly solvable, but the ones
	presented here seem to be worth a try (without time pressure).
\smallskip\noindent
"Answers" to all the exercises appear in Appendix A, often with
additional information about related results. (Of course, the ``answers''
to research problems are incomplete; but even in these cases,
partial results or hints
are given that might prove to be helpful.) Readers are encouraged to
look at the answers, especially the answers to the warmup problems,
but only {\sc after} making a serious attempt to solve the problem
\g Cheaters may pass this course by just copying the answers, but they're
only "cheating" themselves.\g
without peeking.

We have tried in Appendix C to give proper credit to the sources of
each exercise, since a great deal of creativity and/or luck often goes into the
design of an instructive problem. Mathematicians have unfortunately
developed a tradition of borrowing exercises without any acknowledgment;
we believe that the opposite tradition, practiced for example by books
and magazines about chess (where names, dates, and locations of original chess
problems are routinely specified) is far superior. However, we have not
\g Difficult exams don't take into account students who have
other classes to~prepare~for.\g
been able to pin down the sources of many problems that have become
part of the folklore. If any reader knows the origin of an exercise for
which our citation is missing or inaccurate, we would be glad to learn
the details so that we can correct the omission in subsequent
editions of this book.

The "typeface" used for mathematics throughout this book is a new
design by Hermann "Zapf"~[|knuth-zapf|],
commissioned by the "American Mathematical
Society" and developed with the help of a committee that included
B.~"Beeton", R.\thinspace P. "Boas", L.\thinspace K. "Durst",
D.\thinspace E. "Knuth",
P.~"Murdock", R.\thinspace S. "Palais", P.~"Renz",
 E.~"Swanson", S.\thinspace B. "Whidden",
and W.\thinspace B. "Woolf". The underlying philosophy of Zapf's design is
to capture the flavor of mathematics as it might be written by a
mathematician with excellent handwriting. A handwritten rather than
mechanical style is appropriate because people generally create mathematics
with pen, pencil, or chalk. (For example, one of the trademarks of
the new design is the symbol for zero, `$0$', which is slightly pointed
at the top because a handwritten zero rarely closes together smoothly
\g I'm unaccustomed to this face.\g
when the curve returns to its starting point.) The letters are upright,
not italic, so that subscripts, superscripts, and accents are more easily
fitted with ordinary symbols. This new type family has been named
{\sl "AMS~Euler"}, after the great Swiss mathematician Leonhard "Euler"
(1707--1783) who discovered so much of mathematics as we know it today.
The alphabets include Euler Text ($Aa\,Bb\,Cc$ through $Xx\,Yy\,Zz$),
Euler Fraktur ($\frak Aa\,Bb\,Cc$ through $\frak Xx\,Yy\,Zz$), and
Euler Script Capitals ($\scr A\,B\,C$ through $\scr X\,Y\,Z$), as
well as Euler Greek ($A\alpha\,B\beta\,\Gamma\gamma$ through
$X\chi\,\Psi\psi\,\Omega\omega$) and special symbols such as
$\wp$ and~$\aleph$. We are especially pleased to be able to inaugurate the
Euler family of typefaces in this book, because Leonhard Euler's spirit
truly lives on every page: Concrete mathematics is Eulerian mathematics.

The authors are extremely grateful to
\g Dear prof: Thanks for (1)~the "pun"s, (2)~the subject matter.\g
Andrei "Broder", Ernst~"Mayr",
Andrew "Yao", and Frances "Yao", who contributed
greatly to this book during the years that they taught Concrete
Mathematics at Stanford.
 Furthermore we offer 1024 thanks to the
teaching assistants who creatively transcribed what took place in class
each year and who helped to design the examination questions; their
names are listed in Appendix~C\null. This book, which is essentially a
compendium of sixteen years' worth of lecture notes, would have
been impossible without their first-rate work.

\looseness=-1
Many other people have helped to make this book a reality. For example,
\g I don't see how what I've learned will ever help me.\g
we wish to commend the students at "Brown", "Columbia", "CUNY",
"Princeton", "Rice", and "Stanford"
who contributed the choice graffiti and helped to debug our first drafts.
Our contacts at "Addison-Wesley" were especially efficient and helpful;
in particular, we wish to thank our publisher (Peter "Gordon"), production
supervisor (Bette "Aaronson"), designer (Roy "Brown"), and copy editor
(Lyn "Dupr\'e"). The "National Science Foundation" and the "Office of Naval
Research" have given invaluable support.
"Cheryl Graham" was tremendously helpful as we prepared the index.
 And above all, we wish to thank our wives (Fan, Jill, and Amy)
"!Knuth, Jill" "!Chung, Fan" "!Patashnik, Amy"
\g I had a lot of trouble in this class, but I know it sharpened my
math skills and my thinking skills.\g
for their patience, support, encouragement, and ideas.

This second edition features a new Section 5.8, which describes some
important ideas that Doron "Zeilberger" discovered shortly after the first
edition went to press. Additional improvements to the first printing can
also be found on almost every page.

We have tried to produce a perfect book, but we are imperfect authors.
Therefore we solicit help in correcting any mistakes that we've made.
A~"reward" of \$2.56 will gratefully be paid to the first finder of
any error, whether it is mathematical, historical, or typographical.
\g I would advise the casual student to stay away from this course.\g

\smallskip
{\advance\baselineskip-1pt
\halign to\hsize{\sl#\hfil\tabskip=0pt plus 1fil&\hfil#\tabskip=0pt\cr
Murray Hill, New Jersey&\dash---RLG\cr
"!Knuth, Don" "!Graham, Ron" "!Patashnik, Oren"
and Stanford, California&DEK\cr
May 1988 and October 1993&OP\cr}
}
\vfill\eject
\tracingpages=1
\beginchapter {} A Note on Notation

SOME OF THE SYMBOLISM in this book has not (yet?) become standard.
Here is a list of "notation"s that might be unfamiliar to readers
who have learned similar material from other books,
together with the page numbers where these notations are explained.
(See the general index, at the end of the book, for references to more
standard notations.)

\smallskip
\begingroup \advance\baselineskip 7pt plus 1pt \advance\lineskip 7pt plus 1pt
\advance\lineskiplimit7pt
\halign to\hsize{\displaymath#$\hfill\tabskip10pt plus20pt&
 #\hfil&\hfil#\tabskip0pt\cr
\openup5pt
\hbox{\sl Notation}&\sl Name&\sl Page\cr
\noalign{\vskip1pt}
\ln x&natural logarithm: $\log_e x$&|nn:ln|\cr
\lg x&binary logarithm: $\log_2 x$&|nn:lg|\cr
\log x&common logarithm: $\log_{10}x$&|nn:log|\cr
\lfloor x\rfloor&floor: $\max@@
 \{\,n\mid n\le x,\hbox{ integer $n\,\}$}$&|nn:floor|\cr
\lceil x\rceil&ceiling: $\min@@
 \{\,n\mid n\ge x,\hbox{ integer $n\,\}$}$&|nn:ceil|\cr
x\bmod y&remainder: $x-y\lfloor x/y\rfloor$&|nn:mod|\cr
\{x\}&fractional part: $x\bmod 1$&|nn:fracpt|\cr
\noalign{\vskip2pt}
\sum f(x)\,\delta x&indefinite summation&|nn:indef-sum|\cr
\sum\nolimits_a^b f(x)\,\delta x&definite summation&|nn:def-sum|\cr
\noalign{\vskip2pt}
x\_n&falling factorial power: $x!/(x-n)!$&|nn:fall|,\thinspace|nn:gfall|\cr
x\_^n&rising factorial power: $\Gamma(x+n)/\Gamma(x)$&|nn:rise|,\thinspace
                                                      |nn:grise|\cr
n\?&subfactorial: $n!/0!-n!/1!+\cdots+(-1)^nn!/n!$&|nn:subfact|\cr
\Re z&real part: $x$, if $z=x+iy$&|nn:realpart|\cr
\noalign{\g \vskip-15.5pt If you don't under\-stand what the x denotes at the
 bottom of this page, try asking your Latin professor
 instead of your math professor.\g}
\Im z&imaginary part: $y$, if $z=x+iy$&|nn:realpart|\cr
H_n&harmonic number: $1/1+\cdots+1/n$&|nn:hn|\cr
H_n^{(x)}&generalized harmonic number: $1/1^x+\cdots+1/n^x$&|nn:hn-gen|\cr
f^{(m)}(z)&$m$th derivative of $f$ at $z$&|nn:mth-deriv|\cr
{n\brack@m@}&Stirling cycle number (the ``first kind'')&|nn:brack|\cr
{n\brace m}&Stirling subset number (the ``second kind'')&|nn:brace|\cr
{n\euler m}&Eulerian number&|nn:euler|\cr
{\Euler nm}&Second-order Eulerian number&|nn:euler2|\cr
\noalign{\g Prestressed \hbox{concrete} mathematics is
concrete mathe\-matics that's preceded by
 a bewildering list of~notations.\g\vskip6pt}
(a_m\ldots a_0)_b&radix notation for $\sum_{k=0}^m a_kb^k$&|nn:radix|\cr
K(a_1,\ldots,a_n)&continuant polynomial&|nn:continuant|\cr
\hyp{a,b}cz&hypergeometric function&|nn:hyp|\cr
\#A&cardinality: number of elements in the set $A$&|nn:card|\cr
[z^n]\,f(z)&coefficient of $z^n$ in $f(z)$&|nn:coeff-brack|\cr
[\alpha\dts\beta]&closed interval: the set
 $\{x\mid \alpha\le x\le\beta\}$&|nn:interval|\cr
\[m=n]&$1\,$ if $m=n$, otherwise $0\,$*&|nn:iverson|\cr
\[m\divides n]&$1\,$ if $m$ divides $n$, otherwise $0\,$*&|nn:div|\cr
\[m\edivides n]&$1\,$ if $m$ exactly divides $n$, otherwise $0\,$*&|nn:ediv|\cr
\[m\rp n]&$1\,$ if $m$ is relatively prime to $n$, otherwise $0\,$*&|nn:rp|\cr}
\endgroup\bigskip\noindent
\llap{*}In general, if $S$ is any statement that can be true or false, the
bracketed notation
$\[S]$ stands for $1$~if $S$~is true, $0$ otherwise.

Throughout this text, we use single-quote marks (`\dots') to delimit
text as it is {\it written},
double-quote marks (``\dots'')
"!quotation marks"
 for a phrase as it is {\it spoken}. Thus, the string of letters `string'
\g Also `nonstring' is a~string.\g
is sometimes called a ``string.\qback''

An expression of the form `$a/bc$' means the same as `$a/(bc)$'. Moreover,
"!parenthesis conventions"
$\log x/\!@\log y=(\log x)/(\log y)$ and $2@n!=2(n!)$.

\bye
