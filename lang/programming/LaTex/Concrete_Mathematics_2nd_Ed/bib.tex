% Bibliography for Concrete Mathematics
% (c) Addison-Wesley, all rights reserved.
\input gkpmac
\font\logos=logosl10
\pageno=604

\tolerance=1000

\beginchapter B Bibliography

HERE ARE THE WORKS cited in this book. Numbers in the margin specify the
page numbers where citations occur.
\g\noindent\llap{``}This paper fills a much-needed gap in~the literature.''\par
\hfill\dash---Math.\thinspace Reviews\g

References to published problems are generally made to the places where
solutions can be found, instead of to the original problem statements.

Wherever possible, names and titles are spelled here as they appeared in
the original publication.

\hyphenation{re-printed Leip-zig Mathe-matik}
\advance\parindent2pt
\newcount\hits \newcount\nextpage \newif\ifdiff
\newread\bnxi
\gdef\nextref{}
\def\bib{\ex: \bibgrf\ref{\string\cite\number\excount.}}
\def\newbib{\newex: \newbibgrf\ref{\string\newcite\number\excount.}}
\def\newex:{\par{\advance\medskipamount-1pt\medbr}%
  \item{\number\excount\kern-1pt$'$}}
\openin\bnxi=bnx % bnx.tex should contain all the .bnx files
% namely pref.bnx, chap*.nbx, ans.bnx, bib.bnx, cred.bnx
% and it should be numerically sorted and followed by the line `[999] 0.'
\def\primepump{\read\bnxi to\next \expandafter\ppp\next}
\def\ppp[#1] #2. {\difffalse \gdef\testref{#1}%
 \ifx\testref\nextref\else\difftrue\fi \ifnum#2=\nextpage\else\difftrue\fi
 \ifdiff\global\let\nextref=\testref \global\nextpage=#2
  \let\next\relax\else\let\next\primepump\fi \next}
\primepump % if an error occurs, bnx.tex is bad!
\def\bibgrf{\g\hits=0 \xdef\exid{\number\excount}%
  \docites \ifnum\hits>0 .\fi\g}
\def\newbibgrf{\g\hits=0 \xdef\exid{\number\excount'}%
  \docites \ifnum\hits>0 .\fi\g}
\def\docites{\ifx\nextref\exid\let\next\doonecite
 \else\let\next\relax\fi \next}
\def\doonecite{\ifnum\hits>0 , \fi \advance\hits 1
  \ifnum\nextpage<0 \romannumeral-\nextpage \else\number\nextpage\fi
  \primepump\docites}

\def\=#1{{\accent22 #1}}
\def\|{\kern.04em}
\def\?{\hskip0pt minus1pt\relax}
\let\,=\thinspace
\catcode`\@=\active \def@#1{{\char#1}}
\def\spacefix#1{\kern#1.5pt} \catcode`\#=\active \let #=\spacefix
\frenchspacing

\bib|abel|%
N.\,H. "Abel", letter to B. "Holmboe" (1823), in his {\sl \OE uvres
Com\-pl\`etes}, first edition, 1839, volume~2, 264--265.
Reprinted in the second edition, 1881, volume~2, 254--255.

\bib|abramowitz-stegun|%
Milton "Abramowitz" and Irene A. "Stegun", editors, {\sl Handbook of
Mathematical Functions}. \
 United States Government Printing Office, 1964.
 Reprinted by Dover, 1965.

\bib|adams-davison|%
William W. "Adams" and J.\,L. "Davison",
 ``A remarkable class of continued fractions,''
{\sl Proceedings of the American Mathematical Society\/ \bf65}
(1977), 194--198.

\bib|aho-sloane|%
A.\,V. "Aho" and N.\,J.\,A. "Sloane",
 ``Some doubly exponential sequences,''
{\sl Fibonacci Quarterly\/ \bf11} (1973), 429--437.

\bib|ahrens|%
W. "Ahrens", {\sl Mathematische Unterhaltungen und Spiele}. \
Teubner, Leipzig, 1901.
Second edition, in two volumes, 1910 and 1918.

\bib|akhiezer|%
Naum Il'ich "Akhiezer", {\sl Klassicheska\t\i a Problema Momentov i Nekotorye
Voprosy Analiza, Sv\t\i azannye s Ne\t\i u}. \ Moscow, 1961.
English translation, {\sl The Classical Moment Problem and Some Related Questions
in Analysis},
Hafner, 1965.

\bib|allardice|%
R.\,E. "Allardice" and A.\,Y. "Fraser",
 ``La Tour d'Hano\"\i,''
{\sl Proceedings of the Edinburgh Mathematical Society\/ \bf2} (1884), 50--53.

\bib|andre|%
D\'esir\'e "Andr\'e", ``Sur les permutations altern\'ees,''
{\sl Journal de Math\'ema\-tiques pures et appliqu\'ees}, series 3, {\bf7}
(1881), 167--184.

\bib|andrews-siam|%
George E. "Andrews", ``Applications of basic hypergeometric functions,''
{\sl SIAM Review\/ \bf16} (1974), 441--484.

\bib|andrews-saalschutz|%
George E. "Andrews", ``On sorting two ordered sets,''
{\sl Discrete Mathematics\/ \bf11} (1975), 97--106.

\bib|andrews-partitions|%
George E. "Andrews", {\sl The Theory of Partitions}. \ Addison-Wesley, 1976.

\bib|andrews-euler|%
George E. "Andrews", ``Euler's `exemplum memorabile inductionis
fallacis' and $q$-trinomial coefficients,'' {\sl Journal of the
American Mathematical Society\/ \bf3} (1990), 653--669.

\bib|andrews-uchimura|%
George E. "Andrews" and K. "Uchimura", ``Identities in combinatorics IV:
Differentiation and harmonic numbers,'' {\sl Utilitas Mathematica\/ \bf28}
(1985), 265--269.

\bib|apery|%
Roger "Ap\'ery", ``Interpolation de fractions continues et irrationalit\'e
de certaines constantes,'' in {\sl Math\'ematiques}, Minist\`ere
des universit\'es (France), Comit\'e
des travaux historiques et scientifiques, Section des sciences,
{\sl Bulletin de la Section des Sciences\/ \bf3} (1981), 37--53.

\bib|arnold|%
V.\,I. "Arnold", ``Bernoulli-Euler updown numbers associated with function
singularities, their combinatorics and arithmetics,'' {\sl Duke
Mathematical Journal\/ \bf63} (1991), 537--555.

\bib|atkinson|%
M.\,D. "Atkinson", ``The cyclic towers of Hanoi,''
{\sl Information Processing Letters\/ \bf13} (1981), 118--119.

\bib|bachmann|%
Paul "Bachmann", {\sl Die analytische Zahlentheorie}. \
Teubner, Leipzig, 1894.

\bib|bailey|%
W.\,N. "Bailey", {\sl Generalized Hypergeometric Series.} \
Cambridge University Press, 1935; second edition, 1964.

\bib|bailey-jacobi|%
W.\,N. "Bailey", ``The generating function for "Jacobi polynomials",''
{\sl Journal of the London Mathematical Society\/ \bf13} (1938), 243--246.

\bib|rouse-ball|%
W.\,W. Rouse "Ball" and H.\,S.\,M. "Coxeter",
{\sl Mathematical Recreations and Essays}, twelfth edition. \
University of Toronto Press, 1974.
\ (A revision of Ball's {\sl Mathematical Recreations and Problems},
first published by Macmillan, 1892.)

\bib|barlow|%
P. "Barlow", ``Demonstration of a curious numerical proposition,''
{\sl Journal of Natural Philosophy, Chemistry, and the Arts\/ \bf27}
(1810), 193--205.

\bib|beatty|%
Samuel "Beatty", ``Problem 3177,'' {\sl American Mathematical Monthly\/
\bf 34} (1927), 159--160.

\bib|bell-gf|%
E.\,T. "Bell", ``"Euler" algebra,''
{\sl Transactions of the American Mathematical Society\/ \bf25} (1923),
 135--154.

\bib|bell-numbers|%
E.\,T. "Bell", ``Exponential numbers,''
{\sl American Mathematical Monthly\/ \bf41} (1934), 411--419.

\bib|bender|%
Edward A. "Bender", ``Asymptotic methods in enumeration,''
{\sl SIAM Review\/ \bf16} (1974), 485--515.

\bib|bernoulli-ars|%
Jacobi "Bernoulli", {\sl Ars Conjectandi}, opus posthumum.
Basel, 1713. Reprinted in {\sl Die Werke von Jakob Bernoulli}, volume~3,
107--286.

\bib|bertrand|%
J. "Bertrand", ``M\'emoire sur le nombre de valeurs que peut prendre une
fonction quand on y permute les lettres qu'elle renferme,''
{\sl Journal de l'\'Ecole Royale Polytechnique\/ \bf18}, cahier 30 (1845),
123--140.

\bib|crc-tables|%
William H. "Beyer", editor, {\sl CRC Standard Mathematical Tables and
Formulae}, 29th
edition. \ CRC Press, Boca Raton, Florida, 1991.

\bib|bienayme|%
J. "Bienaym\'e", ``Consid\'erations \`a l'appui de la d\'ecouverte de "Laplace"
sur la loi de probabilit\'e dans la m\'ethode des moindres carr\'es,''
{\sl Comptes Rendus hebdomadaires des s\'eances de l'Acad\'emie des Sciences\/}
(Paris) {\bf37} (1853), 309--324.

\bib|binet-det|%
J. "Binet", ``M\'emoire sur un syst\`eme de Formules analytiques, et leur
application \`a des consid\'erations g\'eom\'etriques,''
{\sl Journal de l'\'Ecole Polytechnique\/ \bf9}, cahier 16 (1812),
280--354.

\bib|binet|%
J. "Binet", ``M\'emoire sur l'int\'egration des \'equations lin\'eaires
aux diff\'e\-ren\-ces finies, d'un ordre quelconque, \`a coefficients
variables,'' {\sl Comptes Rendus hebdomadaires des s\'eances
 de l'Acad\'emie des  Sciences\/} (Paris) {\bf17} (1843), 559--567.

\bib|blom|%
Gunnar "Blom", ``Problem E\,3043: Random walk until no shoes,''
{\sl American Mathematical Monthly\/ \bf94} (1987), 78--79.

\bib|boas-wrench|%
R. P. "Boas", Jr.\ and J. W. "Wrench", Jr., ``Partial sums of the harmonic
series,'' {\sl American Mathematical Monthly\/ \bf78} (1971), 864--870.

\bib|bohl|%
P. "Bohl", ``\"Uber ein in der Theorie der s\"akularen St\"orungen vorkommendes
Problem,'' {\sl Journal f\"ur die reine und angewandte Mathematik\/ \bf135}
(1909), 189--283.

\bib|borel|%
\'Emile "Borel", {\sl Le\c cons sur les s\'eries \`a termes positifs}. \
Gauthier-Villars, 1902.

\bib|borweins|%
Jonathan M. "Borwein" and Peter B. "Borwein", {\sl Pi and the AGM}. \
Wiley, 1987.

\bib|brent-gaps|%
Richard P. "Brent", ``The first occurrence of large gaps between
successive primes,'' {\sl Mathematics of Computation\/ \bf27} (1973), 959--963.

\bib|brent-gamma|%
Richard P. "Brent", ``Computation of the regular continued fraction for
Euler's constant,''
{\sl Mathematics of Computation\/ \bf31} (1977), 771--777.

\bib|brillhart|%
John "Brillhart", ``Some miscellaneous factorizations,''
{\sl Mathematics of Computation\/ \bf17} (1963), 447--450.

\bib|brocot|%
Achille "Brocot", ``Calcul des rouages par approximation, nouvelle m\'e\-thode,''
{\sl Revue Chronom\'etrique\/ \bf6} (1860), 186--194.
\ (He also published a 97-page monograph with the same title in 1862.)

\bib|brooke|%
Maxey "Brooke" and C.\,R. "Wall", ``Problem B-14: A little surprise,''
{\sl Fibonacci Quarterly\/ \bf1},\,3 (1963), 80.

\bib|bro-alfred|%
Brother U. "Alfred" ["Brousseau"], ``A mathematician's progress,''
{\sl Mathematics Teacher\/ \bf59} (1966), 722--727.

\bib|mort-brown|%
Morton "Brown", ``Problem 6439: A periodic sequence,''
{\sl American Mathematical Monthly\/ \bf92} (1985), 218.

\bib|not-cited-here|%
T. "Brown", ``Infinite multi-variable subpolynormal Woffles which
\g \vskip-6pt (Such papers aren't cited in this book.)\g
do not satisfy the lower regular $Q$-property (Piffles),''
in {\sl A Collection of 250 Papers on Woffle Theory Dedicated to
R.\,S. "Green" on His 23rd Birthday}.
Cited in A.\,K. "Austin", ``Modern research in mathematics,''
{\sl The Mathematical Gazette\/ \bf51} (1967), 149--150.

\bib|t-c-brown|%
Thomas C. "Brown", ``Problem E\,2619: Squares in a recursive
sequence,'' {\sl American Mathematical Monthly\/ \bf85} (1978), 52--53.

\bib|catalan-history|%
William G. "Brown", ``Historical note on a recurrent combinatorial problem,''
{\sl American Mathematical Monthly\/ \bf72} (1965), 973--977.

\bib|burr-fib|%
S.\,A. "Burr", ``On moduli for which the Fibonacci sequence contains a complete
system of residues,'' {\sl Fibonacci Quarterly\/ \bf9} (1971), 497--504.

\bib|canfield|%
E. Rodney "Canfield", ``On the location of the maximum Stirling number(s)
of the second kind,'' {\sl Studies in Applied Mathematics\/ \bf59}
(1978), 83--93.

\bib|carlitz-max|%
L. "Carlitz", ``The generating function for $\max(n_1,n_2,\cdots,n_k)$,
{\sl Portugaliae Mathematica\/ \bf 21} (1962), 201--207.

\bib|alice-looking|%
Lewis "Carroll" [pseudonym of C.\,L. "Dodgson"],
 {\sl Through the Looking Glass
and What Alice Found There}. Macmillan, 1871.

\bib|cassini|%
Jean-Dominique "Cassini", ``Une nouvelle progression de nombres,''
{\sl Histoire de l'Acad\'emie Royale des Sciences}, Paris, volume~1, 201.
\ (Cassini's work is summarized here as one of the mathematical results
presented to the academy in 1680. This volume was published in 1733.)

\bib|catalan-paper|%
E. "Catalan", ``Note sur une \'Equation aux diff\'erences finies,''
{\sl Journal de Math\'ematiques pures et appliqu\'ees\/ \bf3}
(1838), 508--516.

\bib|cauchy-cours|%
Augustin-Louis "Cauchy",
{\sl Cours d'analyse de l'\'Ecole Royale Polytechnique}. \
Imprimerie Royale, Paris, 1821.
Reprinted in his {\sl \OE uvres Com\-pl\`etes}, series~2, volume~3.

\bib|rhind|%
Arnold Buffum "Chace", {\sl The Rhind Mathematical Papyrus}, volume~1. \
Mathematical Association of America, 1927. \ (Includes an excellent bibliography
of "Egyptian mathematics" by R.\,C. "Archibald".)

\bib|chaimovich-et-al|%
M. "Chaimovich", G. "Freiman", and J. "Sch\"onheim", ``On exceptions
to "Szegedy"'s theorem,'' {\sl Acta Arithmetica\/ \bf49} (1987), 107--112.

\bib|chebyshev|%
P.\,L. Tchebichef ["Chebyshev"], ``M\'emoire sur les nombres premiers,''
{\sl Journal de Math\'e\-ma\-tiques pures et appliqu\'ees\/ \bf17}
(1852), 366--390. Reprinted in his {\sl \OE uvres}, volume~1, 51--70. \
Russian translation, ``O prostykh chislakh,'' in his
{\sl Polnoe Sobranie Sochineni\u\i}, volume~1, 191--207.

\bib|chebyshev-ineq|%
P.\,L. "Chebyshev", ``O srednikh velichinakh,''
{\sl Matematicheski\u\i\ Sbornik'\/ \bf2} (1867), 1--9. Reprinted in his
{\sl Polnoe Sobranie Sochineni\u\i}, volume~2, 431--437. \ French translation,
``Des valeurs moyennes,'' {\sl Journal de Math\'e\-ma\-tiques
pures et appliqu\'ees}, series 2, {\bf12} (1867), 177--184; reprinted
in his {\sl \OE uvres}, volume~1, 685--694.

\bib|chebyshev-mono|%
P.\,L. "Chebyshev", ``O priblizhennykh vyrazheni\t{\i}akh odnikh integralov
cherez drugie, vz\t{\i}atye v tex zhe predelakh,''
{\sl Soobshchen\={\i}\t{\i}a i protokoly zas\t{\i}edan\={\i}\u{\i}
matematicheskago obshchestva pri Imperatorskum\H{} Khar'kovskom\H{}
Universitet\t{\i}e\/ \bf4},2 (1882), 93--98. Reprinted in his
{\sl Polnoe Sobranie Sochineni\u\i}, volume~3, 128--131. \ French translation,
``Sur les expressions approximatives des int\'egrales d\'efinies par les autres
prises entre les m\^emes limites,'' 
in his {\sl \OE uvres}, volume~2, 716--719.

\bib|chung-graham|%
Fan "Chung" and Ron "Graham", ``On digraph polynomials,''
submitted for publication, 1993.

\bib|clausen1|%
Th.\ "Clausen", ``Ueber die F\"alle, wenn die Reihe von der Form
\begindisplay \def\\{\mskip1mu.\mskip1mu}
y=1+{\alpha\over1}\cdt{\beta\over\gamma}\,x+
    {\alpha\\\alpha+1\over1\\2}\cdt{\beta\\\beta+1\over\gamma\\\gamma+1}\,x^2
   +\vcenter{\hbox{$\scriptstyle\rm etc.$}}
\enddisplay
ein Quadrat von der Form
\begindisplay \def\\{\mskip1mu.\mskip1mu}
z=1+{\alpha'\over1}\cdt{\beta'\over\gamma'}\cdt{\delta'\over\epsilon'}\,x+
    {\alpha'\!\\\alpha'{+}1\over1\\2}\cdt
      {\beta'\!\\\beta'{+}1\over\gamma'\!\\\gamma'{+}1}
    \cdt{\delta'\!\\\delta'{+}1\over\epsilon'\!\\\epsilon'{+}1}\,x^2
   +\vcenter{\hbox{$\scriptstyle\rm etc.$}}\hbox{ \ hat,''}
\enddisplay
{\sl Journal f\"ur die reine und angewandte Mathematik\/ \bf3} (1828), 89--91.

\bib|clausen2|%
Th.\ "Clausen", ``Beitrag zur Theorie der Reihen,''
{\sl Journal f\"ur die reine und angewandte Mathematik\/ \bf3} (1828), 92--95.

\bib|clausen3|%
Th.\ "Clausen", ``Theorem,'' {\sl Astronomische Nachrichten\/ \bf17}
(1840), col\-umns 351--352.

\bib|carroll-pictures|%
Stuart Dodgson "Collingwood", {\sl The Lewis "Carroll" Picture Book}. \
T. Fisher Unwin, 1899. Reprinted by Dover, 1961, with the new title
{\sl Diversions and Digressions of Lewis Carroll.}

\bib|comtet|%
Louis "Comtet", {\sl Advanced Combinatorics}. \
Dordrecht, Reidel, 1974.

\bib|conway-graham|%
J.\,H. "Conway" and R.\,L. "Graham", ``Problem E\,2567: A periodic
recurrence,'' {\sl American Mathematical Monthly\/ \bf84} (1977), 570--571.

\bib|cramer|%
Harald "Cram\'er", ``On the order of magnitude of the difference
between consecutive prime numbers,''
{\sl Acta Arithmetica\/ \bf2} (1937), 23--46.

\bib|crelle|%
A.\,L. "Crelle", ``D\'emonstration \'el\'ementaire du th\'eor\`eme de
"Wilson" g\'en\'e\-ralis\'e,''
{\sl Journal f\"ur die reine und angewandte Mathematik\/ \bf20} (1840), 29--56.

\bib|crowe|%
D.\,W. "Crowe", ``The $n$-dimensional cube and the Tower of Hanoi,''
{\sl American Mathematical Monthly\/ \bf63} (1956), 29--30.

\bib|csirik|%
J\'anos A. "Csirik", ``Optimal strategy for the first player in the Penney
ante game,'' {\sl Combinatorics, Probability and Computing\/ \bf1} (1992),
311--321.

\bib|curtiss|%
D.\,R. "Curtiss", ``On "Kellogg"'s Diophantine problem,''
{\sl American Mathematical Monthly\/ \bf29} (1922), 380--387.

\bib|david-barton|%
F.\,N. "David" and D.\,E. "Barton", {\sl Combinatorial Chance}. \
Hafner, 1962.

\bib|davis|%
Philip J. "Davis", ``Leonhard "Euler"'s integral: A historical profile
of the "Gamma function",'' {\sl American Mathematical Monthly\/ \bf66}
(1959), 849--869.

\bib|davison|%
J.\,L. "Davison", ``A series and its associated continued fraction,''
{\sl Proceedings of the American Mathematical Society\/ \bf63}
(1977), 29--32.

\bib|de-bruijn|%
N.\,G. "de Bruijn", {\sl Asymptotic Methods in Analysis}. \
North-Holland, 1958; third edition, 1970. Reprinted by Dover, 1981.

\bib|de-br-prob9|%
N.\,G. "de Bruijn", ``Problem 9,'' {\sl Nieuw Archief voor Wiskunde},
series 3, {\bf12} (1964), 68.

\bib|de-moivre|%
Abraham "de Moivre", {\sl Miscellanea analytica de seriebus et quadraturis}. \
London, 1730.

\bib|dedekind|%
R. "Dedekind", ``Abri\ss\ einer Theorie der h\"oheren Congruenzen in
Bezug auf einen reellen Primzahl-Modulus,''
{\sl Journal f\"ur die reine und angewandte Mathematik\/ \bf54} (1857),
1--26. Reprinted in his {\sl Gesammelte mathematische Werke}, volume~1, 40--67.

\bib|dickson|%
Leonard Eugene "Dickson", {\sl History of the Theory of Numbers}. \
Carnegie Institution of Washington, volume~1, 1919; volume~2, 1920;
volume~3, 1923. Reprinted by Stechert, 1934, and by Chelsea, 1952, 1971.

\bib|dijkstra|%
Edsger W. "Dijkstra", {\sl Selected Writings on Computing: A Personal
Perspective}. \ Springer-Verlag, 1982.

\bib|dirichlet-schubfach|%
G. Lejeune "Dirichlet", ``Verallgemeinerung eines Satzes aus der Lehre
von den Kettenbr\"uchen nebst einigen Anwendungen auf die Theorie der
Zahlen,'' {\sl Bericht \"uber die Verhandlungen der K\"oniglich-Preu\ss ischen
Akademie der Wissenschaften zu Berlin\/} (1842), 93--95. Reprinted in
his {\sl Werke}, volume~1, 635--638.

\bib|dixon|%
A.\,C. "Dixon", ``On the sum of the cubes of the coefficients in a
certain expansion by the binomial theorem,'' {\sl Messenger of
Mathematics\/ \bf20} (1891), 79--80.

\bib|dougall|%
John "Dougall", ``On Vandermonde's theorem, and some more general expansions,''
{\sl Proceedings of the Edinburgh Mathematical Society\/ \bf25} (1907),
114--132.

\bib|holmes-four|%
A. Conan "Doyle", ``The sign of the four; or, The problem of the Sholtos,''
{\sl Lippincott's Monthly Magazine\/} (Philadelphia) {\bf45} (1890),
147--223.

\bib|holmes-final|%
A. Conan "Doyle", ``The adventure of the final problem,''
{\sl The Strand Magazine\/ \bf6} (1893), 558--570.

\bib|du-bois|%
P. du "Bois-Reymond", ``Sur la grandeur relative des infinis des fonctions,''
{\sl Annali di Matematica pura ed applicata}, series 2, {\bf4} (1871), 338--353.

\bib|dubner|%
Harvey "Dubner", ``Generalized repunit primes,''
{\sl Mathematics of Computation\/ \bf61} (1993), 927--930.

\bib|dudeney|%
Henry Ernest "Dudeney", {\sl The Canterbury Puzzles and Other Curious Problems}. \
E.\,P. Dutton, New~York, 1908; 4th edition, Dover, 1958.
\ (Dudeney had first considered the generalized Tower of Hanoi in {\sl The
Weekly Dispatch}, on 25 May 1902 and 15 March 1903.)

\bib|gauss-bio|%
G. Waldo "Dunnington", {\sl Carl Friedrich "Gauss": Titan of Science}. \
 Exposition Press, New York, 1955.

\bib|dyson|%
F.\,J. "Dyson", ``Some guesses in the theory of partitions,'' {\sl Eureka\/
\bf8} (1944), 10--15.

\bib|edwards|%
A.\,W.\,F. "Edwards", {\sl Pascal's Arithmetical Triangle}. \
Oxford University Press, 1987.

\bib|eisenstein-exp|%
G. "Eisenstein", ``Entwicklung von $\smash{
 \alpha^{\alpha^{\alpha^{^{.^{.^.}}}}}}\!$,''
{\sl Journal f\"ur die reine und angewandte Mathematik\/ \bf28}
(1844), 49--52. Reprinted in his {\sl Mathematische Werke\/ \bf1}, 122--125.

\bib|elkies|%
Noam D. "Elkies", ``On $A^4+B^4+C^4=D^4$,'' {\sl Mathematics of Computation\/
\bf 51} (1988), 825--835.

\bib|erdos-curtiss|%
"Erd\H os" P\'al, ``Az $\displaystyle{1\over x_1}
+{1\over x_2}+\cdots+{1\over x_{n_{\mathstrut}}}={a\over b}$
 egyenlet eg\'esz sz\'am\'u
meg\-old\'asair\'ol,'' % that journal did hyphenate after the g!
{\sl Matematikai Lapok\/ \bf1} (1950), 192--209. English abstract on page~210.

\bib|erdos-scottish|%
Paul "Erd\H os", ``My Scottish Book `problems','' in {\sl The Scottish Book:
Mathematics from the Scottish Caf\'e\/},
edited by R.~Daniel "Mauldin", 1981, \hbox{35--45}.

\bib|erdos-graham|%
P. "Erd\H os" and R.\,L. "Graham", {\sl Old and New Problems and Results in
Combinatorial Number Theory}. \ Universit\'e de Gen\`eve, L'Enseignement
Math\'ematique, 1980.

\bib|ruzsa-et-al|%
P. "Erd\H os", R.\,L. "Graham", I.\,Z. "Ruzsa", and E.\,G. "Straus", ``On the
prime factors of $2n\choose n$,''
{\sl Mathematics of Computation\/ \bf29} (1975), 83--92.

\bib|esw-levine|%
Arulappah "Eswarathasan" and Eugene "Levine", ``$p$-integral harmonic sums,''
{\sl Discrete Mathematics\/ \bf91} (1991), 249--257.

\bib|euclid-elts|% 0Gmm 1Dlt 2Tht 3Lmb 4Xi 5Pi 6Sgm 7Ups 8Phi 9Psi 10Omg
"Euclid", {\sl @6TOI#+XEIA}.
 \ Ancient manuscript first printed in Basel, 1533.
Scholarly edition (Greek and Latin) by J.\,L. "Heiberg" in five volumes,
Teubner, Leipzig, 1883--1888.

\bib|euler-factorial|% E788 (Enestrom index no., JhrsbchtDMV 1910, Erg_band 4)
Leonhard "Euler", letter to Christian "Goldbach" (13~October 1729), in
{\sl Correspondance Math\'ematique et Physique de Quelques C\'el\`ebres
G\'eom\`etres du XVIII\kern1pt\raise.2ex\hbox{\`eme} Si\`ecle},
 edited by P.\,H. "Fuss", St.~Petersburg, 1843, volume~1, 3--7.
% this letter is number 715 in the index volume for Euler's series 4

\bib|euler-gamma|% E19
L. "Euler"o, ``De progressionibus transcendentibus seu quarum termini
generales algebraice dari nequeunt,''
{\sl Commentarii academ\-i{\ae} scientiarum imperialis Petropolitan{\ae}
\bf5} (1730), 36--57.
Re\-printed in his {\sl Opera Omnia}, series~1, volume~14, 1--24.

\bib|euler-summation|% E25
Leonh.~"Euler"o, ``Methodus generalis summandi progressiones,''
{\sl Commentarii academ\-i{\ae} scientiarum imperialis Petropolitan{\ae}
\bf6} (1732), 68--97.
Re\-printed in his {\sl Opera Omnia}, series~1, volume~14, 42--72.

\bib|euler-f5|% E26
Leonh.~"Euler"o, ``Observationes de theoremate quodam Fermatiano, aliisque
ad numeros primos spectantibus,''
{\sl Commentarii academ\-i{\ae} scientiarum imperialis Petropolitan{\ae}
\bf6} (1732), 103--107.
Re\-printed in his {\sl Opera Omnia}, series~1, volume~2, 1--5.
Reprinted in his {\sl Commentationes arithmetic{\ae} collect{\ae}},
volume~1, 1--3.

\bib|euler-harmonics|% E43
Leonh.~"Euler"o, ``De progressionibus harmonicis observationes,''
{\sl Commentarii academ\-i{\ae} scientiarum imperialis Petropolitan{\ae}
\bf7} (1734), 150--161.
Re\-printed in his {\sl Opera Omnia}, series~1, volume~14, 87--100.

\bib|euler-eulerian|% E55
Leonh.~"Euler"o, ``Methodus universalis series summandi ulterius promota,''
{\sl Commentarii academ\-i{\ae} scientiarum imperialis Petropolitan{\ae}
\bf8} (1736), 147--158.
Re\-printed in his {\sl Opera Omnia}, series~1, volume~14, 124--137.

\bib|euler-e-cf|% E71 (not Eulero this time)
Leonh.~"Euler", ``De fractionibus continuis, Dissertatio,''
{\sl Commentarii academ\-i{\ae} scientiarum imperialis Petropolitan{\ae}
\bf9} (1737), 98--137.
Reprinted in his {\sl Opera Omnia}, series~1, volume~14, 187--215.

\bib|euler-goof|% E72
Leonh.~"Euler", ``Vari{\ae} observationes circa series infinitas,''
{\sl Commentarii academ\-i{\ae} scientiarum imperialis Petropolitan{\ae}
\bf9} (1737), 160--188.
Reprinted in his {\sl Opera Omnia}, series~1, volume~14, 216--244.

\bib|euler-fourier|% E788
Leonhard "Euler", letter to Christian "Goldbach" (4~July 1744), in
{\sl Correspondance Math\'ematique et Physique de Quelques C\'el\`ebres
G\'eom\`etres du XVIII\kern1pt\raise.2ex\hbox{\`eme} Si\`ecle},
 edited by P.\,H. "Fuss", St.~Petersburg, 1843, volume~1, 278--293.
% this letter is number 794 in the index volume for Euler's series 4

\bib|euler-intro-anal|% E101
Leonhardo "Euler"o, {\sl Introductio in Analysin Infinitorum}. \
Tomus primus, Lausanne, 1748.
Reprinted in his {\sl Opera Omnia}, series~1, volume 8.
Translated into French, 1786; German, 1788; Russian, 1936.

\bib|euler-partitions|% E191
L. "Euler"o, ``De partitione numerorum,''
{\sl Novi commentarii academ\-i{\ae} scientiarum imperialis Petropolitan{\ae}
\bf3} (1750), 125--169.
Reprinted in his {\sl Commentationes arithmetic{\ae} collect{\ae}},
volume~1, 73--101.
Reprinted in his {\sl Opera Omnia}, series~1, volume~2, 254--294.

\bib|euler-diff-calc|% E212
Leonhardo "Euler"o, {\sl Institutiones Calculi Differentialis cum eius usu
in An\-al\-ysi Finitorum ac Doctrina Serierum}. \
Berlin, Academi{\ae} Imperialis Scientiarum Petropolitan\ae, 1755.
Reprinted in his {\sl Opera Omnia}, series~1, volume~10.
Translated into German, 1790.

\bib|euler-totient|% E271
L. "Euler"o, ``Theoremata arithmetica nova methodo demonstrata,''
{\sl Novi commentarii academi{\ae} scientiarum imperialis Petropolitan{\ae}
\bf8} (1760), 74--104.
(Also presented in 1758 to the Berlin Academy.)
Reprinted in his {\sl Commentationes arithmetic{\ae} collect{\ae}},
volume~1, 274--286.
Reprinted in his {\sl Opera Omnia}, series~1, volume~2, 531--555.

\bib|euler-continuants|% E281
L. "Euler"o, ``Specimen algorithmi singularis,''
{\sl Novi commentarii academ\-i{\ae} scientiarum imperialis Petropolitan{\ae}
\bf9} (1762), 53--69.
(Also presented in 1757 to the Berlin Academy.)
Reprinted in his {\sl Opera Omnia}, series~1, volume~15, 31--49.

\bib|euler-middle|% E326
L. "Euler"o, ``Observationes analytic\ae,''
{\sl Novi commentarii academi{\ae} scientiarum imperialis
Petropolitan{\ae}
\bf11} (1765), 124--143.
Reprinted in his {\sl Opera Omnia}, series~1, volume~15, 50--69.

\bib|euler-algebra|% E387
Leonhard "Euler", {\sl Vollst\"andige Anleitung zur Algebra.
Erster Theil. Von den verschiedenen Rechnungs-Arten,
Verh\"altnissen und Proportionen}. \
St.~Petersburg, 1770.
Reprinted in his {\sl Opera Omnia}, series~1, volume~1.
Translated into Russian, 1768;
Dutch, 1773; French, 1774; Latin, 1790; English, 1797.

\bib|euler-conject|% E428
L. "Euler"o, ``Observationes circa bina biquadrata quorum summam in duo
alia biquadrata resolvere liceat,''
{\sl Novi commentarii academi{\ae} scientiarum imperialis Petropolitan{\ae}
\bf17} (1772), 64--69.
Reprinted in his {\sl Commentationes arithmetic{\ae} collect{\ae}},
volume~1, 473--476.
Reprinted in his {\sl Opera Omnia}, series~1, volume~3, 211--217.

\bib|euler-josephus|% E476
L. "Euler"o, ``Observationes circa novum et singulare progressionum
genus,''
{\sl Novi commentarii academi{\ae} scientiarum imperialis Petropolitan{\ae}
\bf20} (1775), 123--139.
Reprinted in his {\sl Opera Omnia}, series~1, volume~7, 246--261.

\bib|euler-lambert|% E532
L. "Euler"o, ``De serie "Lambert"ina, plurimisque eius insignibus
proprietatibus,'' {\sl Acta academi{\ae} scientiarum imperialis
Petropolitan{\ae}\/ \bf3},2 (1779), 29--51.
Reprinted in his {\sl Opera Omnia}, series~1, volume~6, 350--369.

\bib|euler-hyp|% E710
L. "Euler"o, ``Specimen transformationis singularis serierum,''
{\sl Nova acta academi{\ae} scientiarum imperialis Petropolitan{\ae}
\bf12} (1794), 58--70.
Submitted for publication in 1778.
Reprinted in his {\sl Opera Omnia}, series~1, volume~16(2), 41--55.

\bib|faulhaber|%
Johann "Faulhaber"n, {\sl Academia Algebr\ae}, Darinnen die miraculosische
Inventiones zu den h\"ochsten Cossen weiters {\it continuirt\/} und
{\it profitiert\/} werden, \dots~bi\ss\ auff die regulierte
{\it Zensicubiccubic\/} Co\ss\ durch offnen Truck {\it publiciert\/} worden. \
Augsburg, 1631.

\bib|feller|%
William "Feller", {\sl An Introduction to Probability Theory and
Its Applications}, volume~1. \
Wiley, 1950; second edition, 1957; third edition, 1968.

\bib|fermat|%
Pierre de "Fermat", letter to Marin "Mersenne" (25~December 1640), in
{\sl \OE uvres de Fermat}, volume~2, 212--217.

\bib|fibonacci|%
Leonardo "Fibonacci" ["Pisano"], {\sl Liber Abaci}. \
First edition, 1202 (now lost); second edition, 1228. Reprinted in
{\sl Scritti di Leonardo Pisano}, edited by Baldassarre "Boncompagni",
1857, volume~1.

\bib|de-finetti|%
Bruno de "Finetti", {\sl Teoria delle Probabilit\`a.} \ Turin, 1970.
English translation, {\sl Theory of Probability}, Wiley, 1974--1975.

\bib|fisher|%
Michael E. "Fisher", ``Statistical mechanics of dimers on a plane
lattice,'' {\sl Physical Review\/ \bf124} (1961), 1664--1672.

\bib|ra-fisher|%
R.\,A. "Fisher", ``Moments and product moments of sampling distributions,''
{\sl Proceedings of the London Mathematical Society}, series 2,
{\bf30} (1929), 199--238.

\bib|forcadel|%
Pierre "Forcadel", {\sl L'arithmeticque}. \ Paris, 1557.

\bib|fourier|%
J. "Fourier", ``Refroidissement s\'eculaire du globe terrestre,''
{\sl Bulletin des Sciences par la Soci\'et\'e philomathique de Paris}, series 3,
{\bf7} (1820), 58--70. Reprinted in {\sl \OE uvres de Fourier}, volume~2,
271--288.

\bib|fraenkel-cover|%
Aviezri S. "Fraenkel", ``Complementing and exactly covering sequences,''
{\sl Journal of Combinatorial Theory}, series~A, {\bf14} (1973), 8--20.

\bib|fraenkel-wyt|%
Aviezri S. "Fraenkel", ``How to beat your "Wythoff" games' opponent on three
fronts,'' {\sl American Mathematical Monthly\/ \bf89} (1982), 353--361.

\bib|amm-hanoi|%
J.\,S. "Frame", B.\,M. "Stewart", and Otto "Dunkel",
``Partial solution to problem 3918,''
{\sl American Mathematical Monthly\/ \bf48} (1941), 216--219.

\bib|pacioli|%
Piero della "Francesca", {\sl Libellus de
quinque corporibus regularibus.} \ Vatican Library, manuscript Urbinas~632.
Translated into Italian by Luca "Pacioli", as part~3 of
Pacioli's {\sl Diuine Proportione}, Venice, 1509.

\bib|franel|%
J. "Franel", Solutions to questions 42 and 170, in
{\sl L'Interm\'ediare des Math\'ematiciens\/ \bf1} (1894),
45--47; {\bf2} (1895), 33--35.

\bib|samplesort|%
W.\,D. "Frazer" and A.\,C. "McKellar", ``Samplesort: A sampling approach
to minimal storage tree sorting,''
{\sl Journal of the ACM\/ \bf27} (1970), 496--507.

\bib|fredman-thesis|%
Michael Lawrence "Fredman", {\sl Growth Properties of a Class of
Recursively Defined Functions}. \
Ph.D. thesis, Stanford University, Computer Science Department, 1972.

\bib|fuss|%
Nikolao "Fuss", ``Solutio qu\ae stionis, quot modis polygonum $n$
laterum in polygona $m\mskip1mu$ laterum, per diagonales resolvi qu{\ae}at,''
{\sl Nova acta academi{\ae} scientiarum imperialis Petropolitan{\ae}\/ \bf9}
(1791), 243--251.

\bib|gardner-phi|%
Martin "Gardner", ``About phi, an irrational number that has some
remarkable geometrical expressions,''
{\sl Scientific American\/ \bf201},\,2 (August 1959), 128--134. Reprinted
with additions
in his book {\sl The 2nd Scientific American Book of Mathematical Puzzles
\& Diversions}, 1961, 89--103.

\bib|gardner-coins|%
Martin "Gardner", ``On the paradoxical situations that arise from nontransitive
relations,'' {\sl Scientific American\/ \bf231},\,4
(October 1974), 120--124. Reprinted with additions
in his book {\sl Time Travel
and Other Mathematical Bewilderments}, 1988, 55--69.

\bib|gardner-rope|%
Martin "Gardner", ``From rubber ropes to rolling cubes, a miscellany of
refreshing problems,'' {\sl Scientific American\/ \bf232},\,3
(March 1975), 112--114; {\bf232},\,4 (April 1975), 130, 133.
 Reprinted with additions in his book {\sl Time Travel
and Other Mathematical Bewilderments}, 1988, 111--124.

\bib|gardner-dice|%
Martin "Gardner", ``On checker jumping, the amazon game, weird dice, card
tricks and other playful pastimes,'' {\sl Scientific American\/ \bf238},\,2
(February 1978), 19, 22, 24, 25, 30, 32.
Reprinted with additions in his book {\sl Penrose Tiles to Trapdoor Ciphers},
1989, 265--280.

\bib|garfunkel|%
J. "Garfunkel", ``Problem E\,1816: An inequality related to Stirling's
formula,'' {\sl American Mathematical Monthly\/ \bf74} (1967), 202.

\bib|gasper-rahman|%
George "Gasper" and Mizan "Rahman", {\sl Basic Hypergeometric Series}. \
Cambridge University Press, 1990.

\bib|gauss-disq|%
C.\,F. "Gauss", {\sl Disquisitiones Arithmetic\ae}. \
Leipzig, 1801. Reprinted in his {\sl Werke}, volume~1.

\bib|gauss-hyp|%
Carolo Friderico "Gauss", ``Disquisitiones generales circa seriem
infinitam
\begindisplay \openup2pt
&1+{\alpha\beta\over1\,.\,\gamma}\,x
 +{\alpha(\alpha+1)\beta(\beta+1)\over1\ .\ 2\ .\ \gamma(\gamma+1)}\,xx\cr
&\qquad +{\alpha(\alpha+1)(\alpha+2)\beta(\beta+1)(\beta+2)\over
    1\ .\ 2\ .\ 3\ .\ \gamma(\gamma+1)(\gamma+2)}x^3+\rm etc.
\enddisplay
Pars prior,'' {\sl Commentationes societatis regi\ae\ scientiarum Gottingensis
recentiores\/ \bf2} (1813). (Thesis delivered to the Royal Society in
G\"ottingen, 20~January 1812.)
 Reprinted in his {\sl Werke}, volume~3, 123--163,
together with an unpublished sequel on pages 207--229.

\bib|gauss-penta|%
C.\,F. "Gauss", ``Pentagramma mirificum,'' written prior to 1836.
Published posthumously in his {\sl Werke}, volume~3, 480--490.

\bib|genocchi|%
A. "Genocchi", ``Intorno all' expressioni generali di numeri Bernoulliani,''
{\sl Annali di Scienze Matematiche e Fisiche\/ \bf3} (1852), 395--405.

\bib|gessel-apery|%
Ira "Gessel", ``Some congruences for Ap\'ery numbers,''
{\sl Journal of Number Theory\/ \bf14} (1982), 362--368.

\bib|gessel-stanley|%
Ira "Gessel" and Richard P. "Stanley", ``Stirling polynomials,''
{\sl Journal of Combinatorial Theory}, series~A, {\bf24} (1978), 24--33.

\bib|ginsburg|%
Jekuthiel "Ginsburg", ``Note on Stirling's numbers,''
{\sl American Mathematical Monthly\/ \bf35} (1928), 77--80.

\bib|glaisher|%
J.\,W.\,L. "Glaisher", ``On the product
$1^1.2^2.3^3\ldots n^n$, {\sl Messenger of Mathematics\/ \bf7} (1877),
43--47.

\bib|golomb-self|%
Solomon W. "Golomb", ``Problem 5407: A nondecreasing indicator function,''
{\sl American Mathematical Monthly\/ \bf74} (1967), 740--743.

\bib|sales-tax|%
Solomon W. "Golomb", ``The `Sales Tax' theorem,'' {\sl Mathematics
Magazine\/ \bf49} (1976), 187--189.

\bib|golomb-sum|%
Solomon W. "Golomb", ``Problem E\,2529: An application of $\psi(x)$,''
{\sl American Mathematical Monthly\/ \bf83} (1976), 487--488.

\bib|good|%
I.\,J. "Good", ``Short proof of a conjecture by "Dyson",'' {\sl Journal
of Mathematical Physics\/ \bf11} (1970), 1884.

\bib|gosper|%
R. William "Gosper", Jr., ``Decision procedure for indefinite hypergeometric
summation,'' {\sl Proceedings of the National Academy of Sciences of the
United States of America\/
\bf75} (1978), 40--42.

\bib|young-ron|%
R.\,L. "Graham", ``On a theorem of "Uspensky",''
{\sl American Mathematical Monthly\/ \bf70} (1963), 407--409.

\bib|graham-fib-composites|%
R.\,L. "Graham", ``A Fibonacci-like sequence of composite numbers,''
{\sl Mathematics Magazine\/ \bf37} (1964), 322--324.

\bib|graham-amm|%
R.\,L. "Graham", ``Problem 5749,''
{\sl American Mathematical Monthly\/ \bf77} (1970), 775.

\bib|graham-cover|%
Ronald L. "Graham", ``Covering the positive integers by disjoint sets of the
form $\{\,[n\alpha+\beta]:n=1,2,\ldots\,\,\}$,''
{\sl Journal of Combinatorial Theory}, series A, {\bf15} (1973), 354--358.

\bib|graham-squareprod|%
R.\,L. "Graham", ``Problem 1242: Bijection between integers and composites,''
{\sl Mathematics Magazine\/ \bf60} (1987), 180.

\bib|graham-knuth|%
R.\,L. "Graham" and D.\,E. "Knuth", ``Problem E\,2982: A double
infinite sum for $\vert x\vert$,''
{\sl American Mathematical Monthly\/ \bf96} (1989), 525--526.

\bib|conc-math|%
Ronald L. "Graham", Donald E. "Knuth", and Oren "Patashnik", {\sl Concrete
Mathematics: A Foundation for Computer Science}. \
"!self-referential"
Addison-Wesley, 1989; second edition, 1994.

\bib|graham-pollak|%
R.\,L. "Graham" and H.\,O. "Pollak", ``Note on a nonlinear recurrence related
to $\sqrt2$,'' {\sl Mathematics Magazine\/ \bf43} (1970), 143--145.

\bib|grandi|%
Guido "Grandi", letter to "Leibniz" (July 1713), in
{\sl Leibnizens mathematische Schriften}, volume~4, 215--217.

\bib|greene-knuth|%
Daniel H. "Greene" and Donald E. "Knuth", {\sl Mathematics for the
Analysis of Algorithms}. \
Birkh\"auser, Boston, 1981; third edition, 1990.

\bib|olympiads1|%
Samuel L. "Greitzer", {\sl International Mathematical Olympiads, 1959--1977}. \
Mathematical Association of America, 1978.

\bib|gross|%
Oliver A. "Gross", ``Preferential arrangements,''
{\sl American Mathematical Monthly\/ \bf69} (1962), 4--8.

\bib|gruenbaum|%
Branko "Gr\"unbaum", ``Venn diagrams and independent families of sets,''
{\sl Mathematics Magazine\/ \bf48} (1975), 12--23.

\bib|guibas-odlyzko|%
L.\,J. "Guibas" and A.\,M. "Odlyzko", ``String overlaps, pattern matching,
and nontransitive games,'' {\sl Journal of Combinatorial Theory},
series~A, {\bf30} (1981), 183--208.

\bib|guy|%
Richard K. "Guy", {\sl Unsolved Problems in Number Theory}. \
Springer-Verlag, 1981.

\bib|haaland-knuth|%
Inger "H{\aa}land" and Donald E. "Knuth", ``Polynomials involving the
floor function,'' submitted for publication, 1993.

\bib|hall-groups|%
Marshall "Hall", Jr., {\sl The Theory of Groups}. \
Macmillan, 1959.

\bib|halmos-write|%
P.\,R. "Halmos", ``How to write mathematics,''
{\sl L'Enseignement math\'e\-ma\-tique\/ \bf16} (1970), 123--152.
Reprinted in {\sl How to Write Mathematics}, American Mathematical
Society, 1973, 19--48.

\bib|halmos-auto|%
Paul R. "Halmos", {\sl I Want to Be a Mathematician: An Automathography}. \
Springer-Verlag, 1985. Reprinted by Mathematical Association of America, 1988.

\bib|halphen|%
G.\,H. "Halphen", ``Sur des suites de fractions analogues \`a la suite
 de "Farey",''
{\sl Bulletin de la Soci\'et\'e math\'ematique de France\/ \bf5} (1876),
170--175. Reprinted in his {\sl \OE uvres}, volume~2, 102--107.

\bib|hamburger|%
Hans "Hamburger", ``\"Uber eine Erweiterung des "Stieltjes"schen
 Momentenproblems,''
{\sl Mathematische Annalen\/ \bf81} (1920), 235--319; {\bf82} (1921),
120--164, 168--187.

\bib|hammersley|%
J.\,M. "Hammersley", ``On the enfeeblement of mathematical skills
by `Modern Mathematics'
and by similar soft intellectual trash in schools and universities,''
{\sl Bulletin of the Institute of Mathematics and its Applications\/
\bf4},\,4 (October 1968), 66--85.

\bib|hammersley-undergrad|%
J.\,M. "Hammersley", ``An undergraduate exercise in manipulation,''
{\sl The Mathematical Scientist\/ \bf14} (1989), 1--23.

\bib|hansen|%
Eldon R. "Hansen", {\sl A Table of Series and Products}. \
Prentice-Hall, 1975.

\bib|hardy-tract|%
G.\,H. "Hardy", {\sl Orders of Infinity: The `Infinit\"arcalc\"ul' of
Paul du "Bois-Reymond"}. Cambridge University Press, 1910;
second edition, 1924.

\bib|hardy-golf|%
G.\,H. "Hardy", ``A mathematical theorem about golf,''
{\sl The Mathematical Gazette\/ \bf29} (1944), 226--227. Reprinted
in his {\sl Collected Papers}, volume~7, 488.

\bib|hardy-wright|%
G.\,H. "Hardy" and E.\,M. "Wright", {\sl An Introduction to the Theory of
Numbers}. \ Clarendon Press, Oxford, 1938; fifth edition, 1979.

\bib|henrici|%
Peter "Henrici", {\sl Applied and Computational Complex Analysis}.
Wiley, volume~1, 1974; volume~2, 1977; volume~3, 1986.

\bib|henrici-bieberbach|%
Peter "Henrici", ``"De Branges"' proof of the "Bieberbach" conjecture:
A view from computational analysis,'' {\sl Sitzungsberichte der
Berliner Mathematischen Gesellschaft\/} (1987), 105--121.

\bib|hermite-staudt|%
Charles "Hermite", letter to C.\,W. "Borchardt" (8 September 1875), in
{\sl Journal f\"ur die reine und angewandte Mathematik\/ \bf81}
 (1876), 93--95. Reprinted in his
{\sl \OE uvres}, volume~3, 211--214.

\bib|hermite-cours|%
Charles "Hermite", {\sl Cours de M. Hermite}. \
Facult\'e des Sciences de Paris, 1882. Third edition, 1887; fourth edition, 1891.
% capital S is correct in spite of the library card

\bib|hermite|%
Charles "Hermite", letter to S. "Pincherle" (10~May 1900), in
{\sl Annali di Matematica pura ed applicata}, series 3, {\bf5}
(1901), 57--60. Reprinted in his
{\sl \OE uvres}, volume~4, 529--531.

\bib|herstein-kaplansky|%
I.\,N. "Herstein" and I. "Kaplansky", {\sl Matters Mathematical}. \
Harper \& Row, 1974.

\bib|hillman-hoggatt|%
A.\,P. "Hillman" and V.\,E. "Hoggatt", Jr., ``A proof of Gould's Pascal
hexagon conjecture,'' {\sl Fibonacci Quarterly\/ \bf10} (1972), 565--568, 598.

\bib|hoare|%
C.\,A.\,R. "Hoare", ``Quicksort,''
{\sl The Computer Journal\/ \bf5} (1962), 10--15.

\bib|hsu-identity|%
L.\,C. "Hsu", ``Note on a combinatorial algebraic identity and its application,''
{\sl Fibonacci Quarterly\/ \bf11} (1973), 480--484.

%\bib|inkeri|%
%K. "Inkeri", ``Absch\"atzungen f\"ur eventuelle L\"osungen der Gleichung
%im Fermatschen Problem,'' {\sl Annales Universitatis Turkuensis}, series A,
%{\bf 16}, 1 (1953), 3--9.

\bib|APL|%
Kenneth E. "Iverson", {\sl A Programming Language}. \
Wiley, 1962.

\bib|jacobi|%
C.\,G.\,J. "Jacobi", {\sl Fundamenta nova theori\ae\ functionum ellipticarum}. \
K\"onigsberg, Borntr\"ager, 1829. Reprinted in his {\sl Gesammelte Werke},
volume~1, 49--239.

\bib|bgc|%
Svante "Janson", Donald~E. "Knuth", Tomasz "{\L}uczak", and Boris "Pittel",
``The birth of the giant component,'' {\sl Random Structures and
Algorithms\/ \bf4} (1993), 233--358.

\bib|jarden|%
Dov "Jarden" and Theodor "Motzkin", ``The product of sequences with a common linear
recursion formula of order~2,'' {\sl Riveon Lematematika\/ \bf3} (1949),
25--27, 38 (Hebrew with English summary). English version reprinted in
Dov Jarden, {\sl Recurring Sequences}, Jerusalem, 1958, 42--45; second edition,
1966, 30--33.

\bib|jonassen-knuth|%
Arne "Jonassen" and Donald E. "Knuth", ``A trivial algorithm whose analysis
isn't,'' {\sl Journal of Computer and System Sciences\/ \bf16} (1978),
301--322.

\bib|jones|%
Bush "Jones", ``Note on internal merging,'' {\sl Software\dash---Practice
and Experience\/ \bf2} (1972), 241--243.

\bib|josephus-source|% 0Gmm 1Dlt 2Tht 3Lmb 4Xi 5Pi 6Sgm 7Ups 8Phi 9Psi 10Omg
Flavius "Josephus", {\sl I@6TOPIA IO#-@7#-#-@1#-A\"IKO#-@7#-\ @5O@3EMO#-@7#-\
 @5PO@6 P@{10}\-MAI#+O#-@7#-@6}. \
English translation, {\sl History of the Jewish War against the Romans},
by H.\,St.\,J. "Thackeray",
 in the Loeb Classical Library edition of Josephus's
works, volumes 2 and~3,
Heinemann, London, 1927--1928. \ (The ``Josephus problem'' may be based
on an early manuscript now preserved only in the Slavonic version; see volume~2,
page~xi, and volume~3, page 654.)

\bib|jungen|%
R. "Jungen", ``Sur les s\'eries de Taylor n'ayant que des singularit\'es
alg\'ebrico-logarithmiques sur leur cercle de convergence,''
{\sl Commentarii Mathematici Helvetici\/ \bf3} (1931), 266--306.

\bib|karamata|%
J. "Karamata", ``Th\'eor\`emes sur la sommabilit\'e exponentielle et
d'autres sommabilit\'es rattachant,'' {\sl Mathematica\/} (Cluj) {\bf9}
(1935), 164--178.

\bib|kaucky|%
I. "Kauck\'y", ``Problem E\,2257: A harmonic identity,''
{\sl American Mathematical Monthly\/ \bf78} (1971), 908.

%\bib|kaucky-book|%
%Josef "Kauck\'y", {\sl Kombinatorick\'e Identity.} \
%Bratislava, 1975.

\bib|keiper|%
J.\,B. "Keiper", ``Power series expansions of Riemann's $\xi$ function,''
{\sl Mathematics of Computation\/ \bf58} (1992), 765--773.

\bib|olympiads2|%
Murray S. "Klamkin", {\sl International Mathematical Olympiads, 1978--1985, and
Forty Supplementary Problems}. \ Mathematical Association of America, 1986.

\bib|knoebel|%
R. Arthur "Knoebel", ``Exponentials reiterated,'' {\sl American
Mathematical Monthly\/ \bf88} (1981), 235--252.

\bib|knopp|%
Konrad "Knopp", {\sl Theorie und Anwendung der unendlichen Reihen}. \
Julius Springer, Berlin, 1922; second edition, 1924.
Reprinted by Dover, 1945. Fourth edition, 1947; fifth edition, 1964. English
translation, {\sl Theory and Application of Infinite Series},
1928; second edition, 1951.

\bib|knuth-gamma|%
Donald E. "Knuth", ``Euler's constant to 1271 places,''
{\sl Mathematics of Computation\/ \bf16} (1962), 275--281.

\bib|knuth-trans|%
Donald "Knuth", ``Transcendental numbers based on the Fibonacci
sequence,'' {\sl Fibonacci Quarterly\/ \bf2} (1964), 43--44, 52.

\bib|knuth1|%
Donald E. "Knuth", {\sl The Art of Computer Programming},
volume~1: {\sl Fundamental Algorithms}. \
Addison-Wesley, 1968; second edition, 1973.

\bib|knuth2|%
Donald E. "Knuth", {\sl The Art of Computer Programming},
volume~2: {\sl Semi\-numerical Algorithms}. \
Addison-Wesley, 1969; second edition, 1981.

\bib|knuth3|%
Donald E. "Knuth", {\sl The Art of Computer Programming},
volume~3: {\sl Sorting and Searching}. \
Addison-Wesley, 1973; second printing, 1975.

%\bib|knuth-fib|%
%Donald E. "Knuth", ``Letter to the editor,'' {\sl Fibonacci Quarterly\/ \bf12}
%(1974), 46, 79, 82.

\bib|some-sum|%
Donald E. "Knuth", ``Problem E\,2492: Some sum,''
{\sl American Mathematical Monthly\/ \bf82} (1975), 855.

\bib|mariage-stables|%
Donald E. "Knuth", {\sl Mariages stables et leurs relations avec d'autres
probl\`emes combinatoires}. \
Les Presses de l'Universit\'e de Montr\'eal, 1976. Revised and corrected
edition, 1980.

\bib|texbook|%
Donald E. "Knuth", {\sl The \TeX book}. \ Addison-Wesley, 1984.
Reprinted as volume~A of {\sl Computers \& Typesetting}, 1986.

\bib|knuth-caching|%
Donald E. "Knuth", ``An analysis of optimum caching,''
{\sl Journal of Algorithms\/ \bf6} (1985), 181--199.

\bib|knuthd|%
Donald E. "Knuth", {\sl Computers \& Typesetting}, volume~D:
{\logos METAFONT\kern1pt\sl: The Program}. \
Addison-Wesley, 1986.

\bib|knuth-polya|%
Donald E. "Knuth", ``Problem 1280: Floor function identity,''
{\sl Mathematics Magazine\/ \bf61} (1988), 319--320.

\bib|amm-phi-problem|%
Donald E. "Knuth", ``Problem E\,3106:
A new sum for $n^2$,''
 {\sl American Mathematical Monthly\/ \bf94} (1987), 795--797.

\bib|knuth-fib-mult|%
Donald E. "Knuth", ``Fibonacci multiplication,''
{\sl Applied Mathematics Letters\/ \bf1} (1988), 57--60.

\bib|knuth-fib-composites|%
Donald E. "Knuth", ``A Fibonacci-like sequence of composite numbers,''
 {\sl Mathematics  Magazine\/ \bf63} (1990), 21--25.

\bib|knuth-bci|%
Donald E. "Knuth", ``Problem E\,3309: A binomial coefficient
inequality,''
{\sl American Mathematical Monthly\/ \bf97} (1990), 614.

\bib|knuth-tnn|%
Donald E. "Knuth", ``Two notes on notation,'' {\sl American
Mathematical Monthly\/ \bf99} (1992), 403--422.

\bib|knuth-cp|%
Donald E. "Knuth", ``Convolution polynomials,'' {\sl The Mathematica
Journal\/ \bf2},4 (Fall 1992), 67--78.

\bib|knuth-faul|%
Donald E. "Knuth", ``Johann "Faulhaber" and sums of powers,'' {\sl
Mathematics of Computation\/ \bf 61} (1993), 277--294.

\bib|knuth-bn|%
Donald E. "Knuth", ``Bracket notation for the coefficient-of operator,''
in {\sl A Classical Mind}, essays in honour of C.\,A.\,R.
"Hoare", edited by A.\,W. "Roscoe", Prentice-Hall, 1994, 247--258.

\bib|knuth-buckholtz|%
Donald E. "Knuth" and Thomas J. "Buckholtz", ``Computation of Tangent,
Euler, and Bernoulli numbers,'' "!Euler numbers"
{\sl Mathematics of Computation\/ \bf21} (1967), 663--688.

\bib|knuth-vardi|%
Donald E. "Knuth" and Ilan "Vardi", ``Problem 6581: The asymptotic
expansion of the middle binomial coefficient,'' {\sl American Mathematical
Monthly\/ \bf97} (1990), 626--630.

\bib|kw-carry|%
Donald E. "Knuth" and Herbert S. "Wilf", ``The power of a prime that
divides a generalized binomial coefficient,'' 
{\sl Journal f\"ur die reine und angewandte Mathematik\/ \bf396} (1989),
212--219.

\bib|knuth-zapf|%
Donald E. "Knuth" and Hermann "Zapf", ``AMS Euler\dash---A new typeface
for mathematics,'' {\sl Scholarly Publishing\/ \bf20} (1989), 131--157.

\bib|kramp|%
C. "Kramp", {\sl \'El\'emens d'arithm\'etique universelle}. \
Cologne, 1808.

\bib|kummer|%
E.\,E. "Kummer", ``Ueber die hypergeometrische Reihe
\begindisplay \def\\{\,.\,} \openup2pt
&1+{\alpha\beta\over1\\\gamma}\,x
 +{\alpha(\alpha+1)\beta(\beta+1)\over1\\2\\\gamma(\gamma+1)}\,xx\cr
&\qquad +{\alpha(\alpha+1)(\alpha+2)\beta(\beta+1)(\beta+2)\over
    1\\2\\3\\\gamma(\gamma+1)(\gamma+2)}\,x^3+\ldots\,\hbox{,''}
\enddisplay
{\sl Journal f\"ur die reine und angewandte Mathematik\/ \bf15}
 (1836), 39--83, 127--172. Reprinted in his {\sl Collected Papers},
volume~2, 75--166.

\bib|kummer-carry|%
E.\,E. "Kummer", ``\"Uber die Erg\"anzungss\"atze zu den allgemeinen
Reciprocit\"atsgesetzen,''
{\sl Journal f\"ur die reine und angewandte Mathematik\/ \bf44}
 (1852), 93--146. Reprinted in his {\sl Collected Papers},
 volume~1, 485--538.

\bib|kurshan|%
R.\,P. "Kurshan" and B. "Gopinath", ``Recursively generated periodic sequences,''
{\sl Canadian Journal of Mathematics\/ \bf26} (1974), 1356--1371.

\bib|lagny|%
Thomas Fantet de "Lagny", {\sl Analyse g\'en\'erale ou M\'ethodes nouvelles
pour r\'esoudre les probl\`emes de tous les genres et de tous les
degr\'es \`a l'infini}. \ Published as volume~11 of {\sl M\'emoires
de l'Acad\'emie Royale des Sciences}, Paris, 1733.

\bib|lagrange-wilson|%
de la Grange ["Lagrange"],
``D\'emonstration d'un th\'eor\`eme nouveau concernant
les nombres premiers,''
{\sl Nouveaux M\'emoires de l'Acad\'emie royale des Sciences et
Belles-Lettres}, Berlin (1771), 125--137.
Reprinted in his {\sl \OE uvres}, volume~3, 425--438.

\bib|lagrange-euler|%
de la Grange ["Lagrange"], ``Sur une nouvelle esp\`ece de calcul
 r\'elatif % sic, in the original
\`a la diff\'erentia\-tion \& \`a l'int\'egration des quantit\'es variables,''
{\sl Nouveaux M\'emoires de l'Acad\'emie royale des Sciences et
Belles-Lettres}, Berlin (1772), 185--221.
Reprinted in his {\sl \OE uvres}, volume~3, 441--476.

\bib|lah|%
I. "Lah", ``Eine neue Art von Zahlen, ihre Eigenschaften und Anwendung
in der mathematischen Statistik,'' {\sl Mitteilungsblatt f\"ur
Mathematische Statistik\/ \bf7} (1955), 203--212.

\bib|lamb1|%
I.\,H. "Lambert", ``Observationes vari\ae\ in Mathesin puram,''
{\sl Acta Helvetica\/ \bf3} (1758), 128--168. Reprinted in his
{\sl Opera Mathematica}, volume~1, \hbox{16--51}.

\bib|lamb2|%
"Lambert", ``Observations analytiques,''
{\sl Nouveaux M\'emoires de l'Acad\-\'emie royale des Sciences et
Belles-Lettres}, Berlin (1770), 225--244. Reprinted in his
{\sl Opera Mathematica}, volume~2, 270--290.

\bib|landau-primes|%
Edmund "Landau", {\sl Handbuch der Lehre von der Verteilung der Prim\-zahlen},
two volumes. \
Teubner, Leipzig, 1909.

\bib|landau-vorlesungen|%
Edmund "Landau", {\sl Vorlesungen \"uber Zahlentheorie}, three volumes. \
\kern-.7pt Hirzel, Leipzig, 1927.

\bib|laplace|%
P.\,S. de la Place ["Laplace"], ``M\'emoire sur les approximations
des Formules
qui sont fonctions de tr\`es-grands nombres,'' {\sl M\'emoires de l'Academie
royale des Sciences de Paris\/} (1782), 1--88. Reprinted in his
{\sl \OE uvres Compl\`etes\/ \bf10}, 207--291.

\bib|legendre-theorie|%
Adrien-Marie "Legendre", {\sl Essai sur la Th\'eorie des Nombres}. \
Paris, 1798; second edition, 1808. Third edition (retitled {\sl Th\'eorie
des Nombres}, in two volumes), 1830; fourth edition, Blanchard, 1955.

\bib|lehmer-primality|%
D.\,H. "Lehmer", ``Tests for primality by the converse of Fermat's theorem,''
{\sl Bulletin of the American Mathematical Society}, series~2,
{\bf33} (1927), 327--340. Reprinted in his {\sl Selected Papers}, volume~1,
69--82.

\bib|lehmer-sp|%
D.\,H. "Lehmer", ``On Stern's diatomic series,''
{\sl American Mathematical Monthly\/ \bf36} (1929), 59--67.
% not in his Selected Papers

\bib|lehmer-conj|%
D.\,H. "Lehmer", ``On Euler's totient function,''
{\sl Bulletin of the American Mathematical Society}, series~2,
{\bf38} (1932), 745--751. Reprinted in his {\sl Selected Papers}, volume~1,
319--325.

\bib|leibniz|%
G.\,W. "Leibniz", letter to Johann "Bernoulli" (May 1695), in
{\sl Leibnizens mathematische Schriften}, volume~3, 174--179.

\bib|zeck1|%
C.\,G. "Lekkerkerker", ``Voorstelling van natuurlijke getallen door een som
van getallen van Fibonacci,'' {\sl Simon Stevin\/ \bf29} (1952), 190--195.

\bib|world-series|%
Tam\'as "Lengyel", ``A combinatorial identity and the world series,''
"!baseball"
{\sl SIAM Review\/ \bf35} (1993), 294--297.

\bib|lengyel|%
Tam\'as "Lengyel", ``On some properties of the series $\sum_{k=0}^\infty
k^nx^k$ and the Stirling numbers of the second kind,''
submitted for publication, 1993.

\bib|li-shan-lan|%
"Li" Shan-Lan, {\sl Du\`o J\=\i\ B\u{\i} L\`ei\/} [Sums of Piles Obtained
Inductively]. \ In his {\sl Z\'eg\u{u}x\={\i} Zha\={\i} Su\`anxu\'e\/}
[Classically inspired meditations on mathematics], Nanjing, 1867.

\bib|lieb|%
Elliott H. "Lieb", ``Residual entropy of square ice,''
{\sl Physical Review\/ \bf162} (1967), 162--172.

\bib|liouv|%
J. "Liouville", ``Sur l'expression $\varphi(n)$, qui marque combien la suite
$1$, $2$, $3$, \dots,~$n$ contient de nombres premiers \`a~$n$,''
{\sl Journal de Math\'e\-ma\-tiques pures et appliqu\'ees}, series 2, {\bf2}
(1857), 110--112.

\bib|logan-orth|%
B.\,F. "Logan", ``The recovery of orthogonal polynomials from a sum of squares,''
{\sl SIAM Journal on Mathematical Analysis\/ \bf21} (1990), 1031--1050.

\bib|logan-stirl|%
B.\,F. "Logan", ``Polynomials related to the Stirling numbers,'' AT\&T Bell
Laboratories internal technical memorandum, August 10, 1987.

\bib|long-hoggatt|%
Calvin T. "Long" and Verner E. "Hoggatt", Jr., ``Sets of binomial coefficients
with equal products,'' {\sl Fibonacci Quarterly\/ \bf12} (1974), 71--79.

\bib|lou-yao|%
Shituo "Lou" and Qi "Yao", ``A Chebychev's % sic
type of prime number theorem in a short interval-II,''
{\sl Hardy-Ramanujan Journal\/ \bf15} (1992), 1--33.
%``On upper bound of difference between
%consecutive primes,'' {\sl Kexue Tongbao [=K'o Hs\"ueh T'ung Pao]\/ \bf30},8
%(1985), Foreign language edition, 1127--1128.

\bib|loyd-cyclopedia|%
Sam "Loyd", {\sl Cyclopedia of Puzzles}. \
% Green cloth cover says ``Sam Loyd's Cyclopedia of 5,000 Puzzles, Tricks,
% and Conundrums, with Answers'', but the title page reads as cited.
Franklin Bigelow Corporation, Morningside Press, New York, 1914.

\bib|lucas-gcd|%
E. "Lucas", ``Sur les rapports qui existent entre la th\'eorie des nombres
et le Calcul int\'egral,''
 {\sl Comptes Rendus hebdomadaires des s\'eances de l'Acad\'emie des
 Sciences\/} (Paris) {\bf82} (1876), 1303--1305.

\bib|lucas-bc-mod|%
\'Edouard "Lucas", ``Sur les congruences des nombres eul\'eriens et des
coefficients diff\'erentiels des fonctions trigonom\'etriques, suivant un
module premier,''
{\sl Bulletin de la Soci\'et\'e math\'ematique de
France\/ \bf6} (1878), \hbox{49--54}.

\bib|lucas-theorie|%
Edouard "Lucas", {\sl Th\'eorie des Nombres}, volume~1. \ % sic, no accent
Gauthier-Villars, Paris, 1891.

\bib|lucas-recr|%
\'Edouard "Lucas", {\sl R\'ecr\'eations math\'ematiques}, four volumes. \
Gauthier-Villars, Paris, 1891--1894. Reprinted by Albert Blanchard, Paris, 1960.
\ (The Tower of Hanoi is discussed in volume~3, pages 55--59.)

%\bib|lyness1|%
%R.\,C. "Lyness", ``Cycles,'' {\sl The Mathematical Gazette\/ \bf26} (1942),
% 62.

\bib|lyness2|%
R.\,C. "Lyness", ``Cycles,'' {\sl The Mathematical Gazette\/ \bf29} (1945),
231--233.

\bib|lyness3|%
R.\,C. "Lyness", ``Cycles,'' {\sl The Mathematical Gazette\/ \bf45} (1961),
207--209.

\bib|maclaurin|%
Colin "Maclaurin", {\sl Collected Letters}, edited by Stella Mills. \
Shiva Publishing, Nantwich, Cheshire, 1982.

\bib|macmahon|%
P.\,A. "MacMahon", ``Application of a theory of permutations in circular
procession to the theory of numbers,''
{\sl Proceedings of the London Mathematical Society\/ \bf23} (1892), 305--313.

\bib|martzloff|%
J.-C. "Martzloff", {\sl Histoire des Math\'ematiques Chinoises}. \
Paris, 1988.

\bib|mat-ich|%
\t Iu.\,V. Mati{\t\i}asevich, "!Matijasevich"
``Diofantovost' perechislimykh mnozhestv,''
{\sl Doklady Akademii Nauk SSSR\/ \bf191} (1970), 279--282. English translation,
with amendments by the author, ``Enumerable sets are diophantine,''
{\sl Soviet Mathematics\/ \bf11} (1970), 354--357.

\bib|melzak|%
Z.\,A. "Melzak", {\sl Companion to Concrete Mathematics}. \
Volume~1, {\sl Mathematical Techniques and Various Applications}, Wiley,
1973; volume~2, {\sl Mathematical Ideas, Modeling \& Applications}, Wiley, 1976.

\bib|mendelsohn|%
N.\,S. "Mendelsohn", ``Problem E\,2227:
Divisors of binomial coefficients,''
{\sl American Mathematical Monthly\/ \bf78} (1971), 201.

\bib|mersenne|%
Marini "Mersenni", {\sl Cogitata Physico-Mathematica}. \ Paris, 1644.

\bib|mertens-phi|%
F. "Mertens", ``Ueber einige asymptotische Gesetze der Zahlentheorie,''
{\sl Journal f\"ur die reine und angewandte Mathematik\/ \bf77} (1874),
289--338.

\bib|mertens-M|%
"Mertens", ``Ein Beitrag zur analytischen Zahlentheorie,''
{\sl Journal f\"ur die reine und angewandte Mathematik\/ \bf78} (1874),
46--62.

\bib|mills|%
W.\,H. "Mills", ``A prime representing function,''
{\sl Bulletin of the American Mathematical Society}, series~2,
{\bf53} (1947), 604.

\bib|moeb|%
A.\,F. "M\"obius", ``\"Uber eine besondere Art von Umkehrung
der Reihen,''
{\sl Journal f\"ur die reine und angewandte Mathematik\/ \bf9} (1832),
105--123. Reprinted in his {\sl Gesammelte Werke}, volume~4, 589--612.

\bib|moessner|%
A. "Moessner", ``Eine Bemerkung \"uber die Potenzen der nat\"urlichen Zahlen,''
{\sl Sitzungsberichte der Mathematisch\hskip1ptplus1pt-\hskip1ptplus1pt
Naturwissenschaftliche\break Klasse der
Bayerischen Akademie der Wissenschaften}, 1951, Heft~3, 29.{\parfillskip=0pt\par}

\bib|monty-phi|%
Hugh L "Montgomery", ``Fluctuations in the mean of "Euler"'s phi function,''
{\sl Proceedings of the Indian Academy of Sciences}, Mathematical Sciences,
{\bf97} (1987), 239--245.

\bib|montgomery|%
Peter L. "Montgomery", ``Problem E\,2686:
LCM of binomial coefficients,''
{\sl American Mathematical Monthly\/ \bf86} (1979), 131.

\bib|moser-reflections|%
Leo "Moser", ``Problem B-6: Some reflections,''
{\sl Fibonacci Quarterly\/ \bf1},\,4 (1963), 75--76.

\bib|motzkin-straus|%
T.\,S. "Motzkin" and E.\,G. "Straus", ``Some combinatorial extremum problems,''
{\sl Proceedings of the American Mathematical Society\/ \bf7} (1956), 1014--1021.

%\bib|mozzochi|%
%C.\,J. "Mozzochi", ``On the difference between consecutive primes,''
%{\sl Journal of Number Theory\/ \bf24} (1986), 181--187.

\bib|myers|%
B.\,R. "Myers", ``Problem 5795: The spanning trees of an $n$-wheel,''
{\sl American Mathematical Monthly\/ \bf79} (1972), 914--915.

\bib|newton-harm|%
Isaac "Newton", letter to John "Collins" (18~February 1670),
in {\sl The Correspondence of Isaac Newton},
volume~1, 27. % edited by H.\,W. Turnbull Camb U press 1959
Excerpted in {\sl The Mathematical Papers of Isaac Newton},
volume~3, 563. %edited by D.\,T. Whiteside; Camb U press 1969

\bib|niven|%
Ivan "Niven", {\sl Diophantine Approximations}. \
Interscience, 1963.

\bib|niven-gf|%
Ivan "Niven", ``Formal power series,''
{\sl American Mathematical Monthly\/ \bf76} (1969), 871--889.

\bib|odl-wilf|%
Andrew M. "Odlyzko" and Herbert S. "Wilf", ``Functional iteration and the
Josephus problem,'' {\sl Glasgow Mathematical Journal\/ \bf33} (1991),
235--240.

\bib|pascal-digits|%
Blaise "Pascal", ``De numeris multiplicibus,'' presented to Acad\'emie
 Pari\-si\-enne in 1654 and published with his {\sl Trait\'e du
triangle arithm\'etique\/}~\kern-.2pt[|pascal-treatise|].
 Reprinted in {\sl \OE uvres de Blaise Pascal}, volume~3, 314--339.
%\expandafter\write\expandafter\bnx\expandafter{\expandafter[\number\excount]
%\number\pageno.}%

\bib|pascal-treatise|%
Blaise "Pascal", ``Trait\'e du triangle arithmetique,'' in his
{\sl Trait\'e du Triangle Arithmetique, avec quelques autres petits
% sic --- accents are missing in original, including on contents page
% where caps and lowercase are mixed; "trait\'e" has the only accent
traitez sur la mesme matiere}, Paris, 1665. Reprinted
in {\sl \OE uvres de Blaise Pascal\/} (Hachette, 1904--1914), volume~3,
445--503; Latin editions from 1654 in volume~11, 366--390.

\bib|patil|%
G.\,P. "Patil", ``On the evaluation of the negative binomial distribution with
examples,'' {\sl Technometrics\/ \bf2} (1960), 501--505.

\bib|peirce-seq|%
C.\,S. "Peirce", letter to E.\,S. "Holden" (January 1901).
In {\sl The New Elements of Mathematics}, edited by Carolyn "Eisele",
Mouton, The Hague, 1976, volume~1, 247--253. \ (See also page~211.)

\bib|peirce-tree|%
C.\,S. "Peirce", letter to Henry B. "Fine" (17~July 1903).
In {\sl The New Elements of Mathematics}, edited by Carolyn "Eisele",
Mouton, The Hague, 1976, volume~3, 781--784. \ (See also ``Ordinals,''
an unpublished manuscript from circa 1905, in {\sl Collected Papers
of Charles Sanders Peirce}, %edited by Charles "Hartshorne" and Paul "Weiss",
volume~4, 268--280.)

\bib|penney|%
Walter "Penney", ``Problem 95: Penney-Ante,''
{\sl Journal of Recreational Mathematics\/ \bf7} (1974), 321.

\bib|percus|%
J.\,K. "Percus", {\sl Combinatorial Methods}. \ Springer-Verlag, 1971.

\bib|petk|%
Marko "Petkov\v{s}ek", ``Hypergeometric solutions of linear recurrences
with polynomial coefficients,'' {\sl Journal of Symbolic Computation\/ \bf14}
(1992), 243--264.

\bib|pfaff|%
J.\,F. "Pfaff", ``Observationes analytic\ae\ ad {\it L. Euleri\/} institutiones
calculi integralis, Vol.~IV, Supplem.~II \&~IV,''
{\sl Nova acta academi{\ae} scientiarum imperialis Petropolitan{\ae}\/ \bf11},
Histoire section, 37--57. \ (This volume, printed in 1798, contains
mostly proceedings from 1793, although Pfaff's memoir was actually
received in 1797.)

\bib|pochhammer|%
L. "Pochhammer", ``Ueber hypergeometrische Functionen $n^{\rm ter}$
Ordnung,'' 
{\sl Journal f\"ur die reine und angewandte Mathematik\/ \bf71} (1870),
316--352.

\bib|poincare|%
H. "Poincar\'e", ``Sur les fonctions \`a espaces lacunaires,''
{\sl American Journal of Mathematics\/ \bf14} (1892), 201--221.

\bib|poisson|%
S.\,D. "Poisson", ``M\'emoire sur le calcul num\'erique des int\'egrales
d\'efinies,'' {\sl M\'emoires de l'Acad\'emie Royale des Sciences de
l'Institut de France}, series~2, {\bf6} (1823), 571--602.

\bib|polya-counting|%
G. "P\'olya", \kern-2.51pt``Kombinatorische Anzahlbestimmungen f\"ur Gruppen,
Graphen und chemische Verbindungen,'' {\sl Acta Mathematica\/ \bf68}
(1937), 145--254. English translation, with commentary by Ronald~C. "Read",
{\sl Combinatorial Enumeration of Groups, Graphs, and Chemical Compounds},
Springer-Verlag, 1987.

\bib|polya|%
George "P\'olya", {\sl Induction and Analogy in Mathematics}. \
Princeton University Press, 1954.

\bib|polya-pictures|%
G. "P\'olya", ``On picture-writing,''
{\sl American Mathematical Monthly\/ \bf63} (1956), 689--697.

\bib|polya-szego|%
G. "P\'olya" and G. "Szeg\"o", % [sic] from original title page
 {\sl Aufgaben und Lehrs\"atze
aus der Analy\-sis}, two volumes. \ Julius Springer, Berlin, 1925;
fourth edition, 1970 and 1971.
English translation, {\sl Problems and Theorems in Analysis}, 1972 and~1976.

%\bib|poonen|%
%Bjorn "Poonen", ``Josephus sets.''
%Unpublished manuscript, 1987.

%\bib|qian|%
%"Qi\'an" B\u{a}o-Zh\=ong, {\sl Zh\=ong-G\'uo Sh\`u X\'ue Sh\u{\i}\/} [Chinese
%mathematics history]. \ Beijing, 1964. [Sometimes romanized as
%"Ch'ien" Pao-Ts'ung, {\sl Chung-Kuo She Hs\"ueh Shih}.]

\bib|rado|%
R. "Rado", ``A note on the Bernoullian numbers,''
{\sl Journal of the London Mathematical Society\/ \bf9} (1934), 88--90.

\bib|rainville|%
Earl D. "Rainville", ``The contiguous function relations for $_pF_q$ with
applications to "Bateman"'s $J_n^{u,v}$ and "Rice"'s $H_n(\zeta,p,v)$,''
{\sl Bulletin of the American Mathematical Society}, series~2,
{\bf51} (1945), 714--723.

\bib|raney|%
George N. "Raney", ``Functional composition patterns and power series
reversion,'' {\sl Transactions of the American Mathematical Society\/ \bf94}
(1960), 441--451.

\bib|d-r-rao|%
D. Rameswar "Rao", ``Problem E\,2208: A divisibility problem,''
{\sl American Mathematical Monthly\/ \bf78} (1971), 78--79.

\bib|rayleigh|%
John William "Strutt", Third Baron "Rayleigh", {\sl The Theory of Sound}. \
First edition, 1877; second edition, 1894. \ (The cited material about
irrational spectra is from section 92a of the second edition.)

\bib|recorde-wit|%
Robert "Recorde", {\sl The Whetstone of Witte}. \
London, 1557.

\bib|asymptotic-query|%
Simeon "Reich", ``Problem 6056: Truncated exponential-type series,''
{\sl American Mathematical Monthly\/ \bf84} (1977), 494--495.

\bib|rham|%
Georges de "Rham", ``Un peu de math\'ematiques \`a propos d'une courbe plane,''
{\sl Elemente der Mathematik\/ \bf2} (1947), 73--76, 89--97.
Reprinted in his {\sl \OE uvres Math\'ematiques}, 678--689.

\bib|ribenboim|%
Paolo "Ribenboim", {\sl 13 Lectures on Fermat's Last Theorem}. \
Springer-Verlag, 1979.

\bib|riemann-sums|%
Bernhard "Riemann", ``Ueber die Darstellbarkeit einer Function durch eine
trigonometrische Reihe,'' Habilitations\-schrift, G\"ottingen, 1854.
Published in {\sl Abhandlungen der mathematischen Classe der
K\"oniglichen Gesellschaft der Wissenschaften zu G\"ottingen\/ \bf13}
(1868), 87--132. Reprinted in his {\sl Gesammelte Mathematische Werke},
227--264.

\bib|roberts|%
Samuel "Roberts", ``On the figures formed by the intercepts of a system
of straight lines in a plane, and on analogous relations in space of
three dimensions,''
{\sl Proceedings of the London Mathematical Society\/ \bf19} (1889), 405--422.

\bib|roedseth|%
\O ystein "R\o dseth", ``Problem E\,2273:
Telescoping Vandermonde convolutions,''
{\sl American Mathematical Monthly\/ \bf79} (1972), 88--89.

\bib|rosser-schoenfeld|%
J. Barkley "Rosser" and Lowell "Schoenfeld", ``Approximate formulas for
some functions of prime numbers,'' {\sl Illinois Journal of Mathematics\/
\bf 6} (1962), 64--94.

\bib|rota-mobius|%
Gian-Carlo "Rota", ``On the foundations of combinatorial theory.~I.~
Theory of M\"obius functions,'' {\sl Zeitschrift f\"ur
Wahrscheinlichkeitstheorie und verwandte Gebiete\/ \bf2} (1964), 340--368.

\bib|roy|%
Ranjan "Roy", ``Binomial identities and hypergeometric series,''
{\sl American Mathematical Monthly\/ \bf94} (1987), 36--46.

\bib|saalschutz|%
Louis "Saalsch\"utz", ``Eine Summationsformel,''
{\sl Zeitschrift f\"ur Mathematik und Physik\/ \bf35} (1890), 186--188.

\bib|salty-phi|%
A.\,I. "Saltykov", ``O funktsii \'E\u{\i}lera,''
{\sl Vestnik Moskovskogo Universiteta}, series 1, Matematika, Mekhanika
(1960), number~6, 34--50.

\bib|sarkozy|%
A. "S\'ark\"ozy", ``On divisors of binomial coefficients, I,''
{\sl Journal of Number Theory\/ \bf20} (1985), 70--80.

\bib|sawyer|%
W.\,W. "Sawyer", {\sl Prelude to Mathematics}. \
Baltimore, Penguin, 1955.

\bib|schlomilch|%
O. "Schl\"omilch", ``Ein geometrisches Paradoxon,''
{\sl Zeitschrift f\"ur Mathematik und Physik\/ \bf13} (1868), 162.

\bib|schroeder|%
Ernst "Schr\"oder", ``Vier combinatorische Probleme,''
{\sl Zeitschrift f\"ur Mathematik und Physik\/ \bf15} (1870), 361--376.

\bib|schroeter|%
Heinrich "Schr\"oter", ``Ableitung der Partialbruch- und Produkt-Entwicke\-lungen %sic
f\"ur die trigonometrischen Funktionen,''
{\sl Zeitschrift f\"ur Mathematik und Physik\/ \bf13} (1868), 254--259.

\bib|scorer|%
R.\,S. "Scorer," P.\,M. "Grundy", and C.\,A.\,B. "Smith", ``Some binary games,''
{\sl The Mathematical Gazette\/ \bf28} (1944), 96--103.

\bib|sedlacek|%
J. "Sedl\'a\v cek", ``On the skeletons of a graph or digraph,''
in {\sl Combinatorial Structures and their Applications}, Gordon and
Breach, 1970, 387--391.
\ (This volume contains
proceedings of the Calgary International Conference of Combinatorial
Structures and their Applications, 1969.)

\bib|shallit-log|%
J.\,O. "Shallit", ``Problem 6450: Two series,'' {\sl American
Mathematical Monthly\/ \bf92} (1985), 513--514.

\bib|sharp-stack|%
R.\,T. "Sharp", ``Problem 52: Overhanging dominoes,''
{\sl Pi Mu Epsilon Journal\/ \bf1},\,10 (1954), 411--412.

\bib|sierpinski|%
W. "Sierpi\'nski", ``Sur la valeur asymptotique d'une certaine somme,''
{\sl Bulletin International Acad\'emie Polonaise des Sciences et des
Lettres\/} (Cracovie), series A (1910), 9--11.

\bib|sierpinski-2|%
W. "Sierpi\'nski", ``Sur les nombres dont la somme de diviseurs est
une puissance du nombre~2,'' {\sl Calcutta Mathematical Society
Golden Jubilee Commemorative Volume\/} (1958--1959), part~1, 7--9.

\bib|sierpinski-probs|%
Wac\l aw "Sierpi\'nski", {\sl A Selection of Problems in the Theory
of Numbers}. \ Macmillan, 1964.

\bib|silverman-links|%
David L. "Silverman", ``Problematical Recreations 447: Numerical links,''
{\sl Aviation Week \& Space Technology\/ \bf89},\,10 (1~September 1968), 71.
Reprinted as Problem~147 in {\sl Second Book of Mathematical
Bafflers}, edited by Angela Fox "Dunn", Dover, 1983.

%\bib|slater|%
%Lucy Joan "Slater", {\sl Generalized Hypergeometric Series.} \
%Cambridge University Press, 1966.

\bib|sloane|%
N.\,J.\,A. "Sloane", {\sl A Handbook of Integer Sequences}. \
Academic Press, 1973. Sequel, {\sl The New Book of Integer Sequences},
Springer, 1994.

\bib|soloviev|%
A.\,D. "Solov'ev", ``Odno kombinatornoe tozhdestvo i ego primenenie
k zadache o pervom nastuplenii redkogo sobyti\t\i a,''
{\sl Teori\t\i a vero\t\i atnoste\u\i\ i e\"e primeneni\t\i a\/ \bf11}
(1966), 313--320. English translation, ``A combinatorial identity
and its application to the problem concerning the first occurrence of
a rare event,'' {\sl Theory of Probability and its Applications\/ \bf11}
(1966), 276--282.

\bib|spohn|%
William G. "Spohn", Jr., ``Can mathematics be saved?''
{\sl Notices of the American Mathematical Society\/ \bf16} (1969),
890--894.

\bib|stanley-d-finite|%
Richard P. "Stanley", ``Differentiably finite power series,''
{\sl European Journal of Combinatorics\/ \bf1} (1980), 175--188.

\bib|stanley-dimers|%
Richard P. "Stanley", ``On dimer coverings of rectangles of fixed width,''
{\sl Discrete Applied Mathematics\/ \bf12} (1985), 81--87.

\bib|stanley|%
Richard P. "Stanley", {\sl Enumerative Combinatorics}, volume~1. \
Wadsworth \& Brooks/Cole, 1986.

\bib|staudt|%
K.\,G.\,C. von "Staudt", ``Beweis eines Lehrsatzes, die
B\|e\|r\|n\|o\|u\|l\|l\|i\|schen Zahlen betreffend,''
{\sl Journal f\"ur die reine und angewandte Mathematik\/ \bf21}
(1840), 372--374.

\bib|hackers-dict|%
Guy L. "Steele Jr.", Donald R. "Woods", Raphael~A. "Finkel",
Mark~R. "Crispin", Richard~M. "Stallman", and Geoffrey~S. "Goodfellow",
{\sl The Hacker's Dictionary: A Guide to the World of Computer
Wizards}. \ Harper \& Row, 1983.

\bib|steiner|%
J. "Steiner", ``Einige Gesetze \"uber die Theilung der Ebene und des Raumes,''
{\sl Journal f\"ur die reine und angewandte Mathematik\/ \bf1} (1826), 349--364.
Reprinted in his {\sl Gesammelte Werke}, volume~1, 77--94.

\bib|stern|%
M.\,A. "Stern", ``Ueber eine zahlentheoretische Funktion,''
{\sl Journal f\"ur die reine und angewandte Mathematik\/ \bf55} (1858), 193--220.

\bib|stickelberger|%
L. "Stickelberger", ``Ueber eine Verallgemeinerung der Kreis\-theilung,''
{\sl Mathematische Annalen\/ \bf37} (1890), 321--367.

\bib|stieltjes|%
T.\,J. "Stieltjes", letters to "Hermite" (June 1885), in
{\sl Correspondance d'Her\-mite et de Stieltjes}, volume~1, 146--159.

\bib|stieltjes-table|%
T.\,J. "Stieltjes", ``Table des valeurs des sommes $S_k=
\sum_1^\infty n^{-k}$,'' {\sl Acta Mathematica\/ \bf10} (1887), 299--302.
Reprinted in his {\sl \OE uvres Compl\`etes}, volume~2, 100--103.

\bib|stirling-method|%
James "Stirling", {\sl Methodus Differentialis}. \ London, 1730. English
translation, {\sl The Differential Method}, 1749.

\bib|strehl|%
Volker "Strehl", ``Binomial identities\dash---combinatorial and algorithmic
aspects,'' {\sl Discrete Mathematics}, to appear in 1994.

\bib|sweeney-gamma|%
Dura W. "Sweeney", ``On the computation of Euler's constant,''
{\sl Mathematics of Computation\/ \bf17} (1963), 170--178.

\bib|sylvester|%
J.\,J. "Sylvester", ``Problem 6919,''
{\sl Mathematical Questions with their Solutions from the `Educational
Times'\/ \bf37} (1882), 42--43, 80.

\bib|sylvester-totient|%
J.\,J. "Sylvester", ``On the number of fractions contained in any `Farey series'
of which the limiting number is given,''
{\sl The London, Edinburgh and Dublin Philosophical
Magazine and Journal of Science}, series~5, {\bf15} (1883), 251--257.
Reprinted in his {\sl Collected Mathematical Papers}, volume~4, 101--109.

\bib|szegedy|%
M. "Szegedy", ``The solution of "Graham"'s greatest common divisor
problem,'' {\sl Combinatorica\/ \bf6} (1986), 67--71.

%\bib|wagstaff|%
%Jonathan W. "Tanner" and
%%Samuel S. Wagstaff, Jr., ``The irregular primes to 125000,''
%%{\sl Mathematics of Computation\/ \bf32} (1978), 583--591.
%Samuel S. "Wagstaff", Jr., ``New congruences for the Bernoulli numbers,''
%{\sl Mathematics of Computation\/ \bf48} (1987), 341--350.

\bib|tanny|%
S. "Tanny", ``A probabilistic interpretation of Eulerian numbers,''
{\sl Duke Mathematical Journal\/ \bf40} (1973), 717--722.

\bib|theisinger|%
L. "Theisinger", ``Bemerkung \"uber die harmonische Reihe,''
{\sl Monatshefte f\"ur Mathematik und Physik\/ \bf26} (1915), 132--134.

\bib|thiele|%
T.\,N. "Thiele", {\sl The Theory of Observations}. \
Charles \& Edwin Layton, London, 1903. Reprinted in {\sl The
Annals of Mathematical Statistics\/ \bf2} (1931), 165--308.

\bib|zeta-function|%
E.\,C. "Titchmarsh", {\sl The Theory of the Riemann Zeta-Function}. \
Clarendon Press, Oxford, 1951; second edition, revised by D.\,R.
"Heath-Brown", 1986.

\bib|tricomi-erdelyi|%
F.\,G. "Tricomi" and A. "Erd\'elyi", ``The asymptotic expansion of a ratio
of gamma functions,'' {\sl Pacific Journal of Mathematics\/ \bf1} (1951),
133--142.

\bib|ungar|%
Peter "Ungar", ``Problem E\,3052: A sum involving Stirling numbers,''
{\sl American Mathematical Monthly\/ \bf94} (1987), 185--186.

\bib|uspensky|%
J.\,V. "Uspensky", ``On a problem arising out of the theory of a certain game,''
{\sl American Mathematical Monthly\/ \bf34} (1927), 516--521.

\bib|vdP|%
Alfred "van der Poorten", ``A proof that "Euler" missed \dots\ "Ap\'ery"'s
proof of the irrationality of $\zeta(3)$, an informal report,''
{\sl The Mathematical Intelligencer\/ \bf1} (1979), 195--203.

\bib|vandermonde|%
A. "Vandermonde", ``M\'emoire sur des irrationnelles de diff\'erens ordres
avec une application au cercle,''
{\sl Histoire de l'Acad\'emie Royale des Sciences\/} (1772), part~1, 71--72;
{\sl M\'emoires de Math\'ematique et de Physique, Tir\'es des
Registres de l'Acad\'emie Royale des Sciences\/} (1772), 489--498.

\bib|vardi-self|%
Ilan "Vardi", ``The error term in "Golomb"'s sequence,'' {\sl Journal of
Number Theory\/ \bf40} (1992), 1--11.

\bib|venn|%
J. "Venn", ``On the diagrammatic and mechanical representation of propositions
and reasonings,'' {\sl The London, Edinburgh and Dublin Philosophical
Magazine and Journal of Science}, series~5, {\bf9} (1880), 1--18.

\bib|wallis-phi|%
John "Wallis", {\sl A Treatise of Angular Sections}. \
Oxford, 1684.

\bib|waring|%
Edward "Waring", {\sl Meditationes Algebra{\"\i}c\ae}.\
Cambridge, 1770; third edition, 1782.

\bib|waterhouse|%
William C. "Waterhouse", ``Problem E\,3117: Even odder than we
thought,'' {\sl American Mathematical Monthly\/ \bf94} (1987), 691--692.

\bib|rt-3|%
Frederick V. "Waugh" and Margaret W. "Maxfield", ``Side-and-diagonal numbers,''
{\sl Mathematics Magazine\/ \bf40} (1967), 74--83.

\bib|weaver|%
Warren "Weaver", ``Lewis "Carroll" and a geometrical paradox,''
{\sl American Mathematical Monthly\/ \bf45} (1938), 234--236.

\bib|weber|%
H. "Weber", ``Leopold Kronecker,'' {\sl Jahresbericht der Deutschen
Mathe\-matiker-Vereinigung\/ \bf2} (1892), 5--31. Reprinted in
{\sl Mathematische Annalen\/ \bf43} (1893), 1--25.

\bib|weisner|%
Louis "Weisner", ``Abstract theory of inversion of finite series,''
{\sl Transactions of the American Mathematical Society\/ \bf38} (1935),
474--484.

\bib|wermuth|%
Edgar M.\,E. "Wermuth", ``Die erste Fourierreihe,'' {\sl Mathematische
Se\-mes\-ter\-berichte\/ \bf40} (1993), 133--145.

\bib|weyl|%
Hermann "Weyl", ``\"Uber die "Gibbs"'sche Erscheinung und
verwandte Konvergenzph\"anomene,'' {\sl Rendiconti del Circolo Matematico di
Palermo\/ \bf30} (1910), 377--407.

\bib|whipple|%
F.\,J.\,W. "Whipple", ``Some transformations of generalized hypergeometric
series,'' {\sl Proceedings of the London Mathematical Society}, series 2,
{\bf26} (1927), 257--272.

\bib|whitehead-intro|%
Alfred North "Whitehead", {\sl An Introduction to Mathematics.} \
London and New York, 1911.

\bib|whitehead-aims|%
Alfred North "Whitehead", ``Technical education and its relation to
science and literature,'' chapter~2 in {\sl The Organization of Thought,
Educational and Scientific},
London and New York, 1917. Reprinted as chapter~4 of % Williams and Norgate
 {\sl The Aims of Education and Other Essays}, New York, 1929. % Macmillan

\bib|whitehead-science|%
Alfred North "Whitehead", {\sl Science and the Modern World.} \
New York, 1925. Chapter~2 reprinted in {\sl The World of Mathematics},
edited by James~R. "Newman", 1956, volume~1, 402--416.

\bib|wilfology|%
Herbert S. "Wilf", {\sl generatingfunctionology.} \ Academic Press, 1990.

\bib|wilf-zeil|%
Herbert S. "Wilf" and Doron "Zeilberger", ``An algorithmic proof theory for
hypergeometric (ordinary and `$q$') multisum/integral identities,''
{\sl Inventiones Mathematicae\/ \bf108} (1992), 575--633.

\bib|williams-dubner|%
H.\,C. "Williams" and Harvey "Dubner", ``The primality of R1031,''
{\sl Mathematics of Computation\/ \bf47} (1986), 703--711.

\bib|wolstenholme|%
J. "Wolstenholme", ``On certain properties of prime numbers,''
{\sl Quarterly Journal of Pure and Applied Mathematics\/ \bf5}
(1862), 35--39.

\bib|wood|%
Derick "Wood", ``The Towers of Brahma and Hanoi revisited,''
{\sl Journal of Recreational Mathematics \bf14} (1981), 17--24.

\bib|worpitzky|%
J. \?"Worpitzky", \?``Studien \?\"uber \?die \?\?{\it Bernoulli\/}schen \?und 
\?\?{\it Euler\/}schen
\?Zahlen,'' {\sl Journal f\"ur die reine und angewandte Mathematik\/ \bf94}
(1883), 203--232.

\bib|wright-primes|%
E.\,M. "Wright", ``A prime-representing function,''
{\sl American Mathematical Monthly\/ \bf58} (1951), 616--618;
errata in {\bf59} (1952), 99.

%\bib|zapf|%
%Hermann "Zapf", collected works, entitled {\sl Hermann Zapf \& His Design
%Philosophy}. \ Society of Typographic Arts, Chicago, 1987.
%\ (The AMS~Euler typeface is mentioned on pages 97 and~136.)

\bib|zave|%
Derek A. "Zave", ``A series expansion involving the harmonic numbers,''
{\sl Information Processing Letters\/ \bf5} (1976), 75--77.

\bib|zeck2|%
E. "Zeckendorf", ``Repr\'esentation des nombres naturels par une somme de
nombres de Fibonacci ou de nombres de Lucas,''
{\sl Bulletin de la Soci\'et\'e Royale des Sciences de Li\`ege\/ \bf41}
(1972), 179--182.

\bib|zeil-hol|%
Doron "Zeilberger", ``A holonomic systems approach to special functions
identities,'' {\sl Journal of Computational and Applied Mathematics\/ \bf32}
(1990), 321--368.

\bib|creat|%
Doron "Zeilberger", ``The method of creative telescoping,''
{\sl Journal of Symbolic Computation\/ \bf11} (1991), 195--204.


\bye
